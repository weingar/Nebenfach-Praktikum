
%%%%%%%%%%%%%%%%%%%%%%%%%%%%%%%%%%%%%%%%%%%%%%%%%%%%%%%%%%%%%%%%%%%

\chapter{Literatur f"ur das Praktikum}

%%%%%%%%%%%%%%%%%%%%%%%%%%%%%%%%%%%%%%%%%%%%%%%%%%%%%%%%%%%%%%%%%%%

Angesichts der sehr unterschiedlichen Vorkenntnisse in Physik und der gro{\ss}en Auswahl physikalischer Lehrb"ucher ist es leider nicht m"oglich, einen einzigen Text als verbindlich zu erkl"aren. Zu Beginn jeder Versuchsanleitung finden Sie eine kurze Einf"uhrung in die zugrundeliegende Physik, welche allerdings keine vollst"andige Darstellung der physikalischen Zusammenh"ange sein kann. Bitte eignen Sie sich daher selbst"andig die zugeh"orige Physik durch Nachlesen in mehreren B"uchern (zur Not auch in Vorg"angerprotokollen\index{Vorg"angerprotokolle}, aber auf Richtigkeit achten!) tiefgehender an. Die unterschiedlichen Darstellungsweisen f"ordern das Verst"andnis. Die nachfolgend  aufgef"uhrten Literaturhinweise sollen dazu  dienen, Ihnen die Suche nach weiterf"uhrendem Material zu erleichtern. Die Aufz"ahlung erhebt weder einen Anspruch auf Vollst"andigkeit, noch stellt sie eine Wertung dar. Die meisten B"ucher sind in der Bereichsbibliothek Physik \textsc{BBP} ausleih- oder einsehbar. Es ist nicht wichtig, von jedem Buch unbedingt die letzte Auflage zu bekommen, da die Physik, die im Praktikum behandelt wird, sich in den letzten 100 Jahren nicht mehr ver"andert hat!

\noindent
Wir empfehlen  Ihnen, die Textstellen nicht isoliert herauszugreifen (Physiklexika sind f"ur die meisten Anf"anger erfahrungsgem"ass nicht  brauchbar!),  sondern Zusammenh"ange herzustellen.   \\
Auch Wikipedia kann nur Anhaltspunkte liefern...


%Als begleitende Literatur\index{Literatur} f"ur das Physikalische
%Praktikum sind prinzipiell alle Physikb"ucher geeignet. Insbesondere
%sind die folgenden B"ucher\index{B"ucher} zu nennen. Die Aufz"ahlung
%erhebt weder einen Anspruch auf Vollst"andigkeit, noch stellt sie
%eine Wertung dar. Welches Buch f"ur Sie pers"onlich das Beste ist,
%k"onnen nur Sie selbst entscheiden. Schauen Sie sich die B"ucher an,
%vergleichen Sie dabei beispielsweise direkt die unterschiedlichen
%Darstellungen eines bestimmten engen Gebietes. W"ahlen Sie dann
%dasjenige Buch aus, welches Ihnen am besten liegt. Neben dieser
%Aufz"ahlung finden sich diese und weitere B"ucher auch im
%Literaturverzeichnis wieder. Die Abk"urzungen der B"ucher werden zum
%Teil auch bei den Versuchen zur Angabe vertiefender Literatur
%benutzt.

%W"ahlen Sie anhand der Sachverzeichnisse\index{Sachverzeichnis} und
%der Stichworte\index{Stichworte} in den Anleitungen die geeignete
%Literatur zum jeweiligen Versuch aus. Zu einigen Versuchen wird
%spezielle Literatur angegeben. Die meisten B"ucher sind in der
%Bereichsbibliothek Physik \acro{BBP} ausleih- oder einsehbar. Wir
%sind bem"uht, die wichtigsten physikalischen Grundlagen in diese
%Anleitung aufzunehmen. Dies kann aber keine vollst"andige Darstellung der den Versuchen zugrunde liegenden Physik sein. 
%Bitte eignen Sie sich daher selbst"andig die zugeh"orige Physik
%durch Nachlesen in mehreren B"uchern (zur Not auch in
%Vorg"angerprotokollen\index{Vorg"angerprotokolle}, aber auf
%Richtigkeit achten!) tiefgehender an. Die unterschiedlichen
%Darstellungsweisen f"ordern das Verst"andnis.


\section{Spezielle Praktikumsb"ucher}

Tabelle~\ref{t:litprakt} enth"alt eine Aufz"ahlung von B"uchern, die
speziell f"ur Physikalische Praktika gedacht sind
(Praktikumsb"ucher\index{B"ucher!Praktikumsb"ucher}) und somit auch
Methodisches und Handlungshinweise enthalten.
%
\begin{table}[h!]
  \centering
  \caption[Praktikumsb"ucher]{\label{t:litprakt}Dedizierte Praktikumsb"ucher}
  \begin{tabular}{lp{11cm}}
    \hline
    K"urzel & Autor, Titel, Verlag, Jahr, Referenz \\
    \hline
    NPP & \person{Eichler, Kronfeld, Sahm}, Das Neue Physikalische Grundpraktikum, Springer, 2001 \\
    Wal & \person{Walcher}, Praktikum der Physik, Teubner, 2004 \\
    Wes & \person{Westphal}, Praktikum der Physik, Springer, 1984 (vergriffen)  \\
    SK & \person{Stuart, Klages}, Kurzes Lehrbuch der Physik, 2009\\
%    Geschke & \person{Geschke}, Physikalisches Praktikum, Teubner, 2001  \cite{geschke} \\
%    BeJo & \person{Becker, Jodl}, Physikalisches Praktikum, VDI-Verlag, 1983 \cite{bejo} \\
%    CIP & \person{Diemer, Basel, Jodl}, Computer im Praktikum  Springer, 1999 \cite{cip} \\
%    Paus & \person{Paus}, Physik in Experimenten und Beispielen,Hanser \cite{paus} \\
    \hline
  \end{tabular}
\end{table}



\section{Allgemeine Physikb"ucher}

Folgende, in Tabelle~\ref{t:litallg} aufgef"uhrte, allgemeine
Physikb"ucher\index{B"ucher!Allgemeine Physik} sind f"ur das
Praktikum und das Studium insgesamt n"utzlich.
%
\begin{table}[h!]
  \centering
  \caption[Allgemeine Physikb"ucher]{\label{t:litallg}Allgemeine Physikb"ucher, die f"ur das Praktikum n"utzlich sind.}
  \begin{tabular}{lp{11cm}}
    \hline
    K"urzel & Autor, Titel, Verlag, Jahr, Referenz \\
    \hline
 BS 1-8 & \person{Bergmann-Schaefer}, Experimentalphysik 1-8, DeGruyter, 2000 \\
 Dem 1-4 & \person{W. Demtr"oder}, Experimentalphysik 1-4, Springer, 2002\\
 Gerthsen & \person{Meschede, Vogel, Gerthsen}, Gerthsen: Physik, Springer, 2003\\
% Halliday & \person{Halliday}, Physik, Springer, 2003 \cite{halli}\\
% Feyn & \person{Feynman} Physics Lectures \cite{feyn1,feyn2,feyn3}\\
% Kohlr 1-3 & \person{Kohlrausch}, Praktische Physik 1-3, Teubner, 2002 \cite{kohlr1,kohlr2,kohlr3} \\
% Wesp & \person{Westphal}, Physik, Springer  \cite{wesp} \\
% Pohl & \person{L"uders-Pohl}, Pohls Einf"uhrung in die Physik, Springer 2004 \cite{pohl}\\
 %Grim 1-4 & \person{Grimsehl}, Lehrbuch der Physik 1-4, Teubner \cite{grim1,grim2,grim3,grim4}\\
% Alonso & \person{Alonso, Finn}, Physik, Oldenbourg, 2000 \cite{alonso} \\
% L"usch 1-3 & \person{L"uscher}, Experimentalphysik 1-3, Piper \cite{luesch1,luesch2,luesch3} \\
% St"ocker & \person{St"ocker}, Taschenbuch der Physik, Harri Deutsch \cite{tabphys} \\
 Tipler & \person{Tipler}, Physik, Spektrum, 1994\\
 Kuhn & \person{Kuhn}, Physik 2, Westermann, 2000\\
 Dorn & \person{Dorn, Bader}, Physik Grundkursband 12/13, Hermann Schroedel Verlag, 1976\\
 Harten & \person{Harten}, Physik f"ur Mediziner: Eine Einf"uhrung, Springer, 2011\\
 Metzler & \person{Grehn, Krause}, Metzler Physik, Schroedel, 2007\\
 \hline
  \end{tabular}
\end{table}


\section{Handb"ucher und Nachschlagewerke}

N"utzliche Hinweise zur Auswertung und Fehlerrechnung, sowie eine Vielzahl von Werten und Materialdaten, findet man in den in Tabelle~\ref{t:litnach} aufgef"uhrten\index{B"ucher!Handb"ucher}
Nachschlagewerken\index{B"ucher!Nachschlagewerke}.
%
\begin{table}[h!]
  \centering
  \caption[Handb"ucher und Nachschlagewerke]{\label{t:litnach}Handb"ucher und Nachschlagewerke f"ur das Praktikum}
  \begin{tabular}{lp{11cm}}
    \hline
    K"urzel & Autor, Titel, Verlag, Jahr, Referenz \\
    \hline
 Bron & \person{Bronstein-Semendajev}, Taschenbuch der Mathematik, H.~Deutsch \\
 TBMathe & \person{St"ocker}, Taschenbuch mathematischer Formeln u.~moderner Verfahren, H.~Deutsch  \\
 TBPhys & \person{St"ocker}, Taschenbuch der Physik, H.~Deutsch \\
% TBRegel & \person{Lutz, Wendt}, Taschenbuch der Regelungstechnik, H.~Deutsch \cite{tbregel} \\
 TBStat & \person{Rinne}, Taschenbuch der Statistik, H.~Deutsch \\
% TBElektro & \person{Kories, Schmidt-Walter}, Taschenbuch der Elektrotechnik, H.~Deutsch \cite{tbelektro} \\
% TBChem & \person{Schr"oter, Lautenschl"ager, Bibrack}, Taschenbuch d.~Chemie, H.~Deutsch \cite{tbchem} \\
 Kneu & \person{Kneub"uhl}, Repetitorium der Physik, Teubner \\
 Lichten & \person{Lichten}, Scriptum Fehlerrechnung, Springer \\
% Beving & \person{Bevington, Robinson}, Data reduction and error analysis for the physical sciences, McGraw-Hill, 1992 \cite{beving} \\
% Tab & \person{Berber, Kacher, Langer}, Physik in Formeln und Tabellen, Teubner \cite{tab} \\
 Messunsicher & \person{Weise, W"oger}, Messunsicherheit und Messdatenauswertung, Wiley-VCH, Weinheim, 1999  \\
 UmgUnsich & \person{Drosg}, Der Umgang mit Unsicherheiten, facultas, 2006 \\
 Kunze & \person{Kunze}, Physikalische Messmethoden, Teubner, 1986 \\
% Webelm & Web-Elements \url{www.webelements.com} \\
% LandB"orn & \person{Landolt-B"ornstein} \url{www.springeronline.de} \cite{landbern} \\
 NIST & \person{NIST} \url{www.nist.gov} \\
 \hline
  \end{tabular}
\end{table}
%
%

%\section{Verwendung im Praktikum}
%
%W"ahlen Sie anhand der Sachverzeichnisse\index{Sachverzeichnis} und
%der Stichworte\index{Stichworte} in den Anleitungen die geeignete
%Literatur zum jeweiligen Versuch aus. Zu einigen Versuchen wird
%spezielle Literatur angegeben. Die meisten B"ucher sind in der
%Bereichsbibliothek Physik \acro{BBP} ausleih- oder einsehbar. Wir
%sind bem"uht, die wichtigsten physikalischen Grundlagen in diese
%Anleitung aufzunehmen. Dies ist aber erst f"ur einige Versuche
%gelungen. Bitte eignen Sie sich selbst"andig die zugeh"orige Physik
%durch Nachlesen in mehreren B"uchern (zur Not auch in
%Vorg"angerprotokollen\index{Vorg"angerprotokolle}, aber auf
%Richtigkeit achten!) tiefgehender an. Die unterschiedlichen
%Darstellungsweisen f"ordern das Verst"andnis.

\newpage
\section{Fundamentalkonstanten}

Viele physikalische Fundamentalkonstanten werden im Praktikum f"ur
Berechnungen ben"otigt oder werden dort gemessen.
Tabelle~\ref{t:fundamentconst} gibt eine Auswahl aus der von der
\textsc{IUPAP} (\textit{International Union of Pure and Applied
Physics}) festgelegten Zusammenstellung \textsc{CODATA} \cite{codata}
wieder.
%
\begin{table}[hb]
  \centering
  \caption[Fundamentalkonstanten]{\label{t:fundamentconst} Wichtige
  physikalische Fundamentalkonstanten \cite{codata}. $\Delta x/x$
  ist die relative Unsicherheit (\textit{ppm} - parts per million, $\times 10^{-6}$).}%
  \begin{tabular}{p{5cm}ccccc}\hline
%
 Konstante & Symbol & Wert & $\Delta x/x$ [ppm]  \\ \hline
%
 Vakuumlichtgeschwindigkeit & $c_0$ & $\mathrm{299792458\,m \, s^{-1}}$ & exakt \\
%
 Permeabilit"at des Vakuums  & $\mu_0$ & $4\pi \cdot 10^{-7}\mathrm{\,N \, A^{-2}}$ & exakt \\
%
 Permittivit"at des Vakuums  & $\epsilon_0 $ & $\mathrm{8.854187817\cdot 10^{-12}\,F \, m^{-1}}$ & exakt \\
%
 Gravitationskonstante  & $G, \, \gamma$ & $\mathrm{6.67259\cdot 10^{-11}\,m^3 \, kg^{-1} \, s^{-2}}$ & 128 \\
%
 Planck Konstante  & $h$ & $\mathrm{6.6260755\cdot 10^{-34}\,J\, s}$ & $\mathrm{0.60}$ \\
                   & $h$ & $\mathrm{4.1356692\cdot 10^{-15}\,eV \, s}$ & $\mathrm{0.30}$ \\
%
 Elementarladung   & $e$ &$\mathrm{1.60217733\cdot 10^{-19}\,C}$ & $\mathrm{0.30}$ \\
%
 Hall-Widerstand & $R_{\rm H}$ & $\mathrm{25812.8056\,\Omega}$ & $\mathrm{0.045}$ \\
%
 Bohr Magneton & $\mu_{\rm B}=\nicefrac{e\hslash}{2m_e}$ & $\mathrm{5.78838263\cdot 10^{-5}\,eV/T}$ & $\mathrm{0.089}$ \\
%
 Feinstrukturkonstante & $\alpha=\nicefrac{\mu_0 ce^2}{2h}$ & $\mathrm{0.00729735308}$ & $\mathrm{0.045}$ \\
%
      & $\alpha^{-1}$ & $\mathrm{137.0359895}$ & $\mathrm{0.045}$ \\
%
%
 Rydberg Konstante & $R_\infty$ & $\mathrm{10973731.534\,m^{-1}}$ & $\mathrm{0.0012}$ \\
%
                   & $cR_\infty$ & $\mathrm{3.2898419499e15\,Hz}$ & $\mathrm{0.0012}$ \\
%
                   & $h c R_\infty$ & $\mathrm{13.6056981\,eV}$ & $\mathrm{0.30}$ \\
%
 Bohr Radius & $a_0=\nicefrac{\alpha}{4 \pi R_\infty}$ & $\mathrm{0.529177249\cdot 10^{-10}\,m}$ & $\mathrm{0.045}$ \\
%
 Elektronenmasse & $m_e$ & $\mathrm{9.1093897\cdot 10^{-31}\,kg}$ & $\mathrm{0.59}$ \\
%
 Avogadro Konstante & $N_{\rm A}$ & $\mathrm{6.0221367\cdot 10^{23}\,mol^{-1}}$ & $\mathrm{0.59}$ \\
%
 Atomare Masseneinheit & $m_u$ & $\mathrm{1.6605402\cdot 10^{-27}\,kg}$ & $\mathrm{0.59}$ \\
%
          & $m_u$ & $\mathrm{931.49432\,MeV}$ & $\mathrm{0.30}$ \\
%
 Faraday Konstante & $F$ & $\mathrm{96485.309\,C\, mol^{-1}}$ & $\mathrm{0.30}$ \\
%
 Molare Gaskonstante & $R$ & $\mathrm{8.314510\,J \, mol^{-1} \, K^{-1}}$ & $\mathrm{8.4}$ \\
%
 Boltzmann Konstante & $k_{\rm B}$ & $\mathrm{1.380658\cdot 10^{-23}\,J \, K^{-1}}$ & $\mathrm{8.5}$ \\
%
           & $k_{\rm B}$ & $\mathrm{8.617385\cdot 10^{-5}\,eV \, K^{-1}}$ & $\mathrm{8.4}$ \\
%
 Molvolumen Ideales Gas\footnote{Normalbedingungen} & $V_{\rm m}$ & $\mathrm{22414.10\,cm^3 \, mol^{-1}}$ & $\mathrm{8.4}$ \\
%
 Loschmidt Konstante & $n_0=\nicefrac{N_A}{V_m}$ & $\mathrm{2.686763\cdot 10^{25}\,m^{-3}}$ & $\mathrm{8.5}$ \\
%
 Stefan-Boltzmann Konstante & $\sigma$ & $\mathrm{5.67051\cdot 10^{-8}\,W\, m^{-2} \, K^{-4}}$ & $\mathrm{34}$ \\
%
 Wien Konstante & $b=\lambda_{\rm max}T$ & $\mathrm{0.002897756\,m \, K}$ & $\mathrm{8.4}$ \\
%
 \hline
  \end{tabular}
\end{table}



%%%%%% noch Tabelle mit Umrechnungen ?
