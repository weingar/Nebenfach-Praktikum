\chapter{Anfertigung eines Versuchsprotokolls}
\label{c:protokoll}

Das Protokoll ist nicht nur ein wichtiger Aspekt im Praktikum,
es ist auch eine "`Visitenkarte"' f"ur Ihre Arbeit im Praktikum. Das
Protokoll fasst Ihre Ergebnisse des Versuches zusammen und soll das
Nachvollziehen des Versuches erm"oglichen. Eine klare Gliederung und
eine pr"agnante Formulierung ist anzustreben. Die Fertigkeit, einen Versuch
gut darzustellen (mit Messprotokoll, Ergebnis, Diskussion) ist eine der wichtigen
Schl"usselkompetenzen Ihrer Ausbildung, die durch das Praktikum vermittelt werden.

Die "au"sere Form des Protokolles sollte zudem ein sauberes
Schriftbild und Seitenbild und einen Heft- und Korrekturrand
beinhalten. Das Protokoll ist zu heften.

%Die Muster f"ur ein Protokoll\index{Protokoll!Muster} (f"ur \LaTeX{}
%und \acro{WORD}) k"onnen von den Praktikumswebseiten herunter geladen
%werden. Wir schlagen folgenden schematischen Aufbau und Inhalt f"ur
%ein Versuchsprotokoll vor:

\vspace{1cm}

\noindent{\bf \Large Titel}

\noindent Versuchstitel und Nummer:  \hfill  Datum der
Durchf"uhrung:

\noindent Praktikant/-in: Name, E-Mail: e-mail
\hfill Mitarbeiter/in:

\noindent Assistent: Name des Assistenten  \hfill Platz f"ur
Stempel, Testat und Unterschrift


%\section{Einleitung}

%Was wird gemessen?  Was ist die Motivation? Hinweise auf Literatur.

%F"ur alle Teile des Protokolls (Messprotokoll, Auswertung, Diskussion) ist darauf zu achten, da{\ss} die benutzten/gemessenen Gr"o{\ss}en, benutzten Formeln, usw. klar ersichtlich sind. Bitte "uberlegen Sie, ob eine kleine Zeichnung zum Verst"andnis hilfreich w"are.

\section{Messprotokoll}

Das Messprotokoll wird schon w"ahrend der Durchf"uhrung des Versuchs angelegt, es bildet die Grundlage f"ur die folgende Auswertung des Versuchs und muss vom Assistenten abgezeichnet werden. Ab diesem Moment ist das Messprotokoll als offizielles Dokument anzusehen und darf nicht mehr ver"andert werden.

Zus"atzlich zu den Messwerten selbst, enth"alt das Messprotokoll eine Absch"atzung der systematischen Unsicherheit der Messfehler. Hierunter fallen Ungenauigkeiten bei der Ablesung eines Messwertes ebenso wie m"ogliche Schwierigkeiten mit dem Messaufbau selbst.

Achten Sie darauf, dass das Messprotokoll sauber geschrieben und der Auswertung beigeheftet wird. Werte im Messprotokoll werden mit Tinte (Kugelschreiber oder F"uller) geschrieben, nicht mit Bleistift. Sollten Sie einzelne Werte nachmessen oder haben Sie bei einer Messung einen Fehler gemacht, so wird der alte Wert im Messprotokoll durchgestrichen, auf keinen Fall ausradiert oder mit Tintenkiller oder TippEx unkenntlich gemacht.

\section{Auswertung}

Die Auswertung erfolgt nach dem Versuch zu Hause. Sie soll von der Zweiergruppe, die den Versuch durchgef"uhrt hat, gemeinsam erstellt werden.

Bei der Auswertung ist darauf zu achten, dass benutzte Formeln, Konstanten, etc. eindeutig erkennbar und leicht aufzufinden sind. Ausserdem m"ussen auch Rechnungen angegeben werden, anstatt nur eines Ergebnisses. Dies tr"agt zur allgemeinen Lesbarkeit des Protokolls bei und erleichtert dem Assistenten die Korrektur.

Bitte schreiben Sie in vollst"andigen S"atzen, anstatt nur Stichpunkte anzugeben. 

Wenn das Messprogramm abweichend von der Versuchsanleitung durchgef"uhrt wird, so ist dies unbedingt im Protokoll zu vermerken. Ebenso sind Skizzen des Versuchsaufbaus oder elektrischer Schaltungen Teil des Protokolls, sofern sie nicht in der Versuchsanleitung enthalten sind.

\begin{important}
	Jede Messung ist mit einer Unsicherheit behaftet und daher auch jedes Messergebnis. Auch wenn die Fehlerrechnung nicht sehr beliebt ist, so ist sie dennoch ein wichtiger Teil jeder Versuchsauswertung. Erst die Angabe eines 	Fehlers auf Ihr Messergebnis macht den Vergleich mit einem Literaturwert m"oglich.
\end{important}

Die Messergebnisse sind immer auf Plausibilit"at zu "uberpr"ufen. Wenn zum Beispiel bei der Messung des Durchmessers eines Haares etwas von der Gr"o{\ss}enordnung ein Meter rauskommt, ist Ihnen wohl ein Fehler unterlaufen. Ebenso gibt die Betrachtung der Einheiten eines Rechenergebnisses (\textit{Dimensionsanalyse}) Hinweise auf Fehler in der Rechnung. 

\section{Angabe von Messergebnissen}

Bei der Angabe von Zahlenwerten von Messergebnissen ist darauf zu achten, dass die Genauigkeit, i.e. die Anzahl der Nachkommastellen, sinnvoll gew"ahlt ist. Wenn Sie zum Beispiel mit einem Lineal mit Millimetereinteilung eine L"ange in der Gr"o{\ss}enordnung Zentimeter messen, macht es keinen Sinn, das Messergebnis mit mehr als zwei Nachkommastellen anzugeben. Dies w"urde n"amlich eine Messgenauigkeit von unter 0,1~mm andeuten, die mit dieser Methode einfach nicht zu erreichen ist.\\
In diesem Beispiel w"are also eine gute Angabe:
\begin{equation*}
	L = 24,34 \pm 0,05~\mathrm{cm}
\end{equation*}

Auch wenn Sie die Unsicherheit einer indirekt gemessene Gr"o{\ss}e berechnen, macht es wenig Sinn, alle Nachkommastellen anzugeben, die der Taschenrechner ausspuckt. In diesem Fall rundet man den Fehler auf, auf die gr"o{\ss}te signifikante Stelle, d.h. die erste Stelle (vor oder nach dem Komma), die nicht gleich Null ist. Am Beispiel der Fl"ache eines Rechtecks, gemessen mit eine Lineal mit Millimetereinteilung:
\begin{equation*}
	\sigma_F = 0,0236~\mathrm{cm^2} \approx 0,03~\mathrm{cm^2}
\end{equation*}
Wenn mit dem so berechneten Fehler noch weiter gerechnet wird, macht es Sinn, eine weitere Stelle anzugeben, in unserem Beispiel also $\sigma_F\approx 0,024~\mathrm{cm^2}$. So wird verhindert, dass die endg"ultige Unsicherheit durch zu grobes Runden k"unstlich vergr"o{\ss}ert wird.

In diesem Fall einer indirekt gemessenen Gr"o{\ss}e ergibt sich die Anzahl an Stellen (vor oder nach dem Komma), die beim Ergebnis angegeben werden, wieder aus der erreichten Genauigkeit, i.e. aus der Gr"o{\ss}enordnung des Fehlers. Beim Ergebnis werden n"amlich genausoviele Stellen angegeben, wie bei der Unsicherheit. Am Beispiel der Fl"ache eines Rechtecks:
\begin{equation*}
	F = 12,26746~\mathrm{cm^2} \approx 12,27~\mathrm{cm^2}
\end{equation*}

Das Ergebniss der Messung der Fl"ache des Rechtecks w"urde also lauten:
\begin{equation*}
	F = 12,27 \pm 0,03~\mathrm{cm^2}.
\end{equation*}

%\section{Anfertigung graphischer Darstellungen}

%Graphische Darstellungen von Messergebnissen sind grunds"atzlich von Hand auf Millimeterpapier auszuf"uhren, da diesen ein immenser p"adagogischer Wert zukommt. Eingescanntes Millimeterpapier steht auf StudIP zum Download bereit.

%Die Form von Diagrammen ist wohldefiniert und soll hier trainiert werden. Daher muss jedes Diagramm ordentlich beschriftet sein. Zur Beschriftung geh"oren sowohl die Angabe der dargestellten Gr"o{\ss}e mit Einheit, als auch die Einteilung der entsprechenden Achse.

\section{Checkliste}

Benutzen Sie die nachfolgende Checkliste, um zu überprüfen ob das Protokoll, welches Sie angefertigt haben vollständig ist. Die Betreuerinnen und Betreuer benutzen dieselbe Checkliste bei ihrer Korrektur des Protokolls.
\begin{itemize}[label={$\square$}]
	\item Deckblatt vollständig (Namen und Unterschrift der Teilnehmer, Name des Betreuers, Datum)
	%
	\item Ordentliche Form
	\begin{itemize}[label={$\square$}]
		\item Lesbare Handschrift
		%
		\item Sinnvolle Gleiderung
		%
		\item Ganze Sätze
	\end{itemize}
	%
	\item Wichtigste Punkte der Versuchsdurchführung beschrieben
	\begin{itemize}[label={$\square$}]
		\item Was wurde gemacht? Wieso?
		%
		\item Was hat nicht geklappt?
	\end{itemize}
	%
	\item Auswertung komplett und richtig?
		\begin{itemize}[label={$\square$}]
			\item Formeln zur Ergebnis- und Fehlerberechnung angegeben und richtig
			\begin{itemize}[label={$\square$}]
				\item Rechenschritte nachvollziehbar
				%
				\item Benutzte Variablen definiert, einheitlich
			\end{itemize}
			%
			\item Plots korrekt
			\begin{itemize}[label={$\square$}]
				\item Achsenbeschriftung, -einteilung, evtl. Legende
				%
				\item Fehlerbalken
				%
				\item Ausgleichsgerade, inkl. Steigungsdreiecken und Fehlergeraden
				%
				\item suaber, ordentlich auf Millimeterpapier gezeichnet
			\end{itemize}
			%
			\item Zwischenschritte nachvollziehbar
			%
			\item Ergebnis inkl. Fehler angegeben und kenntlich gemacht
			%
			\item Ergebnisdiskussion: Vergleich mit Literaturwert oder Erwartung, Plausibilität des Ergebnisses, Fehlerbetrachtung
		\end{itemize}
	%
	\item Protokoll links oben getackert
\end{itemize}