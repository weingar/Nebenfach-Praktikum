
%%%%%%%%%%%%%%%%%%%%%%%%%%%%%%%%%%%%%%%%%%%%%%%%%%%%%%%%%%%%%%%%%%%

\chapter{Computersimulation und Fehlerrechnung} \label{v:fehler}

%%%%%%%%%%%%%%%%%%%%%%%%%%%%%%%%%%%%%%%%%%%%%%%%%%%%%%%%%%%%%%%%%%%

Die experimentelle Physik arbeitet mit Apparaturen. Dem entsprechend
liegt das Schwergewicht des Physikalischen Praktikums für Anfänger auf
dem Umgang mit Apparaten und Messinstrumenten. Die allgemeine
Verfügbarkeit von leistungsfähigen Computern hat allerdings in den
letzten Jahren einen Strukturwandel bewirkt, indem die
Computersimulation und die Experimentsteuerung und Datennahme mit
Computern eine immer größere Bedeutung erlangen. Die
Computersimulation ermöglicht es, experimentelle Abläufe auf dem
Computer so genau nachzubilden, wie die Naturgesetze bekannt sind.
Dadurch können Versuchsanordnungen optimiert und neue physikalische
Erkenntnisse aus den Unterschieden zwischen dem simulierten Ergebnis
und dem gemessenen Ergebnis gewonnen werden. Die Experimentsteuerung
und Datennahme mit dem Computer ermöglicht die Handhabung immer
komplexerer Versuchsanordnungen. Als Beispiel mag dienen, daß bei
Rutherfords Experimenten zur Streuung von Alphateilchen einzelne
Lichtblitze auf Szintillationsschirmen mit dem Auge beobachtet werden
mußten, während in modernen Apparaturen der Kern- und Teilchenphysik
mehrere tausend Detektoren gleichzeitig betrieben werden und die
Meßergebnisse zeitgeordnet aufgeschrieben werden können. Die Daten
liegen dann in einer Form vor, daß sie unmittelbar mit dem Computer
weiterverarbeitet werden können.

Der vorliegende Praktikumsversuch führt in dieses Anwendungsgebiet der
Computer ein und vermittelt gleichzeitig Kenntnisse in der Fehler- und
Ausgleichsrechnung. Er kann anstelle von zwei Versuchen der
Serie~1--10 ausgeführt werden. Vorschläge werden in den entsprechenden
Anleitungen unterbreitet.





Hinweis: Es können auch die Programm-Pakete unter Linux "'xmgr"' oder
"'gnuplot"' benutzt werden, sowie die gängigen Pakete unter
DOS/WINDOWS. Hier kann man die entsprechenden Programme selbst
schreiben (siehe auch Einführungsversuch~\ref{v:computer}
"'Computer"').


\section{Stichworte}

Zufallszahlen, Normalverteilung, Varianz der Normalverteilung,
Bestwert, Fehler des Bestwertes, Fehlerfortpflanzung, Gauß'sche
Methode der kleinsten Quadrate (minimales $\chi^2$),
Computer-Betriebssystem MS-DOS 5.0

\section{Zubehör}

Computer und Peripherie im Praktikum.




\section{Grundlagen}

\subsection{Aufgabe des Versuchs}

Aufgabe des Versuchs ist eine Einführung in die Fehlerrechnung und die
Motivation zur Benutzung der CIP-Rechner (CIP = Computer Investitions
Programm). Die CIP-Rechner stehen allen Studenten der Physik zur
Benutzung zur Verfügung auch außerhalb der Praktikumsstunden und
während der vorlesungsfreien Zeit. Die geringe Anzahl der CIP-Rechner
macht allerdings eine gewisse Einschränkung des Zugangs unumgänglich.
Deshalb ist eine Absprache mit Ihrem Assistenten erforderlich.

\subsection{Vorbemerkung zur Fehlerrechnung}

Meßvorgänge liefern Meßergebnisse mit einem Fehler, der nach einer
verbindlichen Übereinkunft ein Maß für die Genauigkeit des
Meßergebnisses darstellt.

Ein Beispiel: Die Länge eines Stabes wird durch Anlegen eines Maßstabes
bestimmt. Dann sind zwei Arten von Fehlern möglich

\begin{enumerate}
  \item der systematische Fehler des Maßstabs, der sich durch genauen Vergleich mit dem Urmeter ermitteln läßt
  \item der zufällige Fehler, der sich durch Unsicherheiten beim Anlegen des
 Maßstabs ergibt.
\end{enumerate}

Wir beschäftigen uns im folgenden ausschließlich mit dem zufälligen
Fehler.

Ein mathematisches Modell für den Zufall wird durch Zufallszahlen
geliefert. Darunter versteht man Zahlen zwischen zwei Grenzen $G1$ und
$G2$, deren Werte möglichst gleichmäßig zwischen diesen Grenzen
verteilt sind. Ein Beispiel für einen Zufallsgenerator ist das
Roulette, das echte Zufallszahlen erzeugt. Rechner erzeugen
Pseudozufallszahlen nach einer Vorschrift, die garantiert, daß die
Gleichverteilung möglichst gut erfüllt ist. Wir verwenden in unserem
Versuch einen Zufallsgenerator, der im folgenden Abschnitt beschrieben
wird.

\subsection{Zufallsgenerator}

Die interne Uhr des Computers liefert den Zeitpunkt des Beginns des
Versuchs, zum Beispiel

\begin{equation} \label{e:uhr}
\mbox{ 14 Uhr 31 Minuten 48.57 Sekunden}
\end{equation}


Der Zufallsgenerator ist angewiesen, hieraus nach einem leicht
erkennbaren Bildungsgesetz als erste Zufallszahl

\begin{equation}   \label{e:zahl}
  N(1)= 0.57483114
\end{equation}

herzustellen. Alle weiteren Zufallszahlen werden nach der Rekursion

\begin{equation}  \label{e:zahl1}
 N(i+1) = \mbox{MOD} \left( N(i) \cdot  317 , 1 \right)
\end{equation}

erzeugt. Da

\begin{equation} \label{e:rech}
 0.57483114 \cdot 317 = 182.22147138
\end{equation}

beträgt, ergibt sich als nächste Zufallszahl $N(2) = 0.22147138$
usw. Der Multiplikator in (Gl.~\ref{e:rech}) ist dabei weitgehend
beliebig wählbar, aber nicht vollständig beliebig, da z.B. die Zahl
100 anstelle von 317 ungeeignet wäre (warum?). Alle so erzeugten
Zufallszahlen liegen zwischen 0 und 1. Zur Vereinfachung der
Darstellung multiplizieren wir alle Zahlen mit 100 und lassen die
Stellen hinter dem Komma fort. Die Zufallszahlen unseres Beispiels sind
dann $N(1) = 57$, $N(2) = 22$, usw. In dieser Darstellung liegen die
Zufallszahlen zwischen 0 und 99.

Man kann sich leicht vorstellen, daß nicht in allen Fällen ein Satz von
sinnvollen Zufallszahlen entsteht, z.B. dann nicht, wenn Sie pünktlich
um Mitternacht mit der Erzeugung der Zufallszahlen beginnen (warum?).
Es ist deshalb erforderlich, die "'Qualität"' der Zufallszahlen zu
prüfen.

Probe : Der Mittelwert der Gesamtmenge der Zufallszahlen und
jeder Teilmenge sollte 49.5 betragen oder von diesem
Erwartungswert oder wahren Wert um einen Fehler abweichen, der
mit den Gesetzen der Statistik verträglich ist. Berechnen Sie
also zur Kontrolle Mittelwerte mit dem Computer!

Dafür steht Ihnen ein in der Programmiersprache FORTRAN geschriebenes
Programm zur Verfügung, das aus folgenden Zeilen besteht

\vspace*{2ex}
\begin{tabular}{ll}
C & BERECHNUNG DES MITTELWERTES\\
10 & ARITHM = 0.\\
20 & DO 40 I = IA, IA+NA-1\\
30 & ARITHM = ARITHM + N(I)\\
40 & CONTINUE\\
50 & ARITHM = ARITHM/NA\\
\end{tabular}                                 \hfill (FC-1)
\vspace*{2ex}

Wenn Sie bereits die Programmiersprache BASIC kennen, werden Sie
Ähnlichkeiten mit dem FORTRAN-Text (FC-1) erkennen, aber auch
feststellen, daß FORTRAN eine kompaktere Formulierung der Anweisungen
ermöglicht. Die mit C eingeleitete erste Zeile enthält einen Kommentar,
der vom Computer überlesen wird. Ausführbare Anweisungen können
numeriert sein. Die Numerierung ist aber nicht unbedingt erforderlich.
Die Zeile 10 definiert einen mit ARITHM bezeichneten Speicherplatz für
eine Dezimalpunktkonstante und weist diesem Speicherplatz den
Zahlenwert 0.0 zu. In den Zeilen 20 bis 40 befindet sich eine
Zählschleife, die alle Zufallszahlen mit Nummern zwischen $i = IA$ und
$i = IA+NA-1$ addiert. Die Größe $IA$ ist die vom Programm erfragte
Nummer der ersten Zufallszahl und $NA$ die Anzahl der zu addierenden
Zufallszahlen. Die Zählschleife endet bei der hinter der DO-Anweisung
angegebenen Zeilennummer. Diese Zeile ist hier aus Gründen der
Übersichtlichkeit mit der leeren Anweisung CONTINUE belegt. Die
Zählschleife wird NA-mal durchlaufen. Bei jedem Durchlauf wird zum
Inhalt des Speicherplatzes ARITHM eine weitere Zufallszahl addiert. Ist
dieser Vorgang abgeschlossen, dann wird durch die Anweisung in Zeile 50
der Inhalt des Speicherplatzes ARITHM durch das gesuchte arithmetische
Mittel ersetzt. Nach dem hier beschriebenen Muster können Sie leicht
selbst einfache FORTRAN-Programme schreiben, wenn Sie beachten, daß mit
den Buchstaben I, K, L, M, N beginnende Speicherplatznamen für ganze
Zahlen reserviert sind. Im Programm (17.A) sind dies die Speicherplätze
I, IA, NA und N(I). Dabei vertritt N(I) die hundert mit Zufallszahlen
belegten Speicherplätze.


\subsection{Grundlagen der Fehlerrechnung: Bestwert und Fehler}

Wir betrachten im folgenden die im Abschnitt 3 berechneten Mittelwerte
als Meßergebnisse für eine physikalische Größe und bezeichnen diese mit
$M$. Für diese Größe $M$ kennen wir den im wahren Wert $M_W$, d.h.

\begin{equation}\label{a}
M_W = 49.5,
\end{equation}

der in einem wirklichen Experiment natürlich unbekannt ist, aber als
existent angenommen werden kann. Jeder Meßwert $M_i$ der Größe $M$
weicht vom wahren Wert um den absoluten Fehler $\Delta M_i$

\begin{equation} \label{b}
\Delta M_i = M_i - M_W
\end{equation}

ab.


Das Endergebnis einer n-mal wiederholten Bestimmung von $M$ soll
durch einen Bestwert $M_B$ beschrieben werden, der der Vorschrift\\

\begin{equation} \label{c}
\sum_{i=1}^{n}\, (M_{i} - M_{B})^{2} = f(M_{B}) = \mbox{Minimum}.
\end{equation}



genügt, die als Gauß'sche Methode der kleinsten Quadrate zur Bestimmung
des Bestwertes bezeichnet wird. Führt man die Bestimmung des Minimums
nach der Vorschrift\\

\begin{equation} \label{d}
\frac{d}{d M_B}\,\sum_{i=1}^{n}\,(M_{i} - M_{B})^{2}\,=\,0
\end{equation}

aus, so ergibt sich\\

\begin{equation} \label{e}
M_{B}\,=\,\frac{1}{n}\,\sum_{i=1}^{n}\,M_{i}\,=\,\bar M
\end{equation}

und damit der {\bf Satz: Der Bestwert ist gleich dem arithmetischen Mittel.}\\


Der in (\ref{b}) definierte absolute Fehler der Einzelmessung läßt sich
in der Praxis nicht ermitteln. Deshalb führen wir nach der Vorschrift\\

\begin{equation} \label{f}
 \Delta M\,= \left\{\frac{1}{n-1}\,\sum_{i=1}^{n}\,(M_{i} -
 M_{B})^{2}\right\}^{1/2}
\end{equation}

den mittleren quadratischen Fehler der Einzelmessung ein. Der in
(\ref{f}) eigentlich erwartete Gewichtsfaktor $1/n$ wurde durch
$1/(n-1)$ ersetzt, damit der Ausdruck für $n=1$ sinnvoll bleibt
(Begründung?).


\subsection{Die Normalverteilung}

Sie können sich an Hand der vom Computer gelieferten Daten leicht davon
überzeugen, daß die Meßdaten $M$ genähert in Form einer Glockenkurve um
den wahren Wert $M_W$ oder den Bestwert $M_B$ verteilt sind. Die
mathematische Form der Glockenkurve ist gegeben durch die Gauß'sche
Fehlerfunktion oder Normalverteilung\\

\begin{equation} \label{g}
P(v)
=\,\frac{1}{\sqrt{2\pi}\sigma}\,\mbox{exp}\,\left\{-v^{2}/(2\sigma^{2})\right\}
\end{equation}

wobei $v$ = $M - M_B$ gesetzt wurde und die Normierung\\

\begin{equation} \label{h}
\int_{-\infty}^{+\infty}\,P(v)dv = 1
\end{equation}

erfüllt ist. Man kann zeigen, daß \\

\begin{equation} \label{i}
\sigma^{2}\,=\,\int_{-\infty}^{+\infty}\,P(v)v^{2}dv
\end{equation}

ist. Das bedeutet, daß die Standardabweichung $\sigma$ mit dem
mittleren quadratischen Fehler $\Delta M$ übereinstimmt. Man kann
ferner zeigen, daß \\

\begin{equation} \label{j}
 W(\sigma)\,=\,\int_{-\sigma}^{+\sigma}\,P(v)dv = 0.68
\end{equation}

beträgt. Beziehung (\ref{j}) beinhaltet, daß die Angabe des mittleren
quadratischen Fehlers nicht bedeutet, daß für alle Meßwerte $M$ die
Abweichung vom Bestwert $M_B$ kleiner als $\sigma$ ist. Vielmehr
beträgt die relative Häufigkeit hierfür nur 68\%.\\

Im Teil 3 des Praktikumsversuchs benutzen wir die erzeugten
Zufallszahlen und die Normalverteilung (\ref{g}) zur Simulation der
Messung der Länge eines Stabes durch Anlegen eines Maßstabs. Dazu
bilden wir die Menge der Zufallszahlen $(N)$ nach der Vorschrift\\

\begin{equation} \label{k}
  | P(v)dv| = | P(N) dN |
\end{equation}

auf die Normalverteilung ab. Die Vorschrift (\ref{k}) besagt, daß die
Häufigkeitsverteilung der neuen Zufallszahlen der Normalverteilung
folgt, wobei die statistischen Schwankungen aber beibehalten werden. Im
Praktikumsversuch beträgt $P(N)$=1/100 für alle $N$. Ferner ist $dv$ =
0.1 mm für alle Längenintervalle auf dem Maßstab. Die Vorschrift
(\ref{k}) läßt sich deshalb ausführen, indem man den wahren Wert der
Stablänge $X_W$ = 1000 mm dem Mittelwert $\bar N$ =49,5 zuordnet und
$dN$ proportional $P(v)$ wählt. Die Zufallszahlen in dem so definierten
Intervall $dN$ werden dann dem entsprechenden Längenintervall $dv$ als
Meßwert für die Länge des Stabes zugeordnet.


\subsection{Der Bestwert einer Funktion $f(x,y,...)$ von verschiedenen
Meßgrößen $x,y,...$ und Fehlerfortpflanzungsgesetz}


Gegeben seien die Meßwerte\\

\begin{equation} \label{l}
x_{i},\quad i=1\ldots r;\hspace{2cm}
y_{k},\quad k=1\ldots s
\end{equation}

aus denen ein Endergebnis $f_{i,k}$=$f(x_{i},y_{k})$ berechnet wird.
Beispiel: Berechnung der Fläche {f} eines Rechtecks aus den
Kantenlängen $x$ und $y$. Es läßt sich zeigen, daß in guter Näherung
der Bestwert $\bar f$ von $f$ gegeben ist durch\\

\begin{equation} \label{m}
\bar
f\equiv\,r^{-1}\,\cdot\,s^{-1}\,\sum_{i=1}^{r}\,\sum_{k=1}^{s}\,f(x_{i},
y_{k})\,=\,f(\bar x, \bar y)
\end{equation}

(siehe Anhang)\\


Dieses Ergebnis gilt für beliebige Funktionen und beliebig viele
Variablen.\\


Wir bezeichnen jetzt die mittleren quadratischen Fehler von $f$, $x$
und $y$ mit $\sigma_f$, $\sigma_x$ bzw. $\sigma_y$. Dann läßt sich
unter Benutzung der Definitionen der mittleren quadratischen Fehler
dieser drei Größen zeigen, daß ein Fehlerfortpflanzungsgesetz in der
Form\\

\begin{equation} \label{n}
\sigma_{f}\,=\,\sqrt{\sigma_{x}^{2}\,\left (\frac{\partial f}{\partial
 x}\right )^{2}\,+\,\sigma_{y}^{2}\left (\frac{\partial f}{\partial
 y}\right )^{2}}
\end{equation}

gilt. Auch in (\ref{n}) sind beliebige Anzahlen von Variablen zugelassen.
Spezialfälle von (\ref{n}) sind\\

\begin{equation} \label{o}
\bar f = \bar x + \bar y
\end{equation}

mit\\

\begin{equation} \label{p}
\sigma_{f}\,=\,\sqrt{\sigma_{x}^{2}\,+\,\sigma_{y}^{2}}
\end{equation}

und

\begin{equation} \label{q}
\bar f = \bar x \cdot \bar y
\end{equation}

mit\\

\begin{equation} \label{r}
\frac{\sigma_{f}}{\bar f}\,=\,\sqrt{\left(\frac{\sigma_{x}}{\bar
 x}\right)^{2}\,+\,\left(\frac{\sigma_{y}}{\bar y}\right)^{2}}
\end{equation}

(siehe Anhang).\\



Die Beziehungen (\ref{m}) und (\ref{n}) können in dem Praktikumsversuch
für die Spezialfälle (\ref{o}) und (\ref{q}) überprüft werden. Dazu
benutze man zwei aus je 5, 10 oder 20 Meßwerten bestehende Gruppen für
die Länge des Stabes und bezeichne die Meßwerte der einen Gruppe mit
$x_i$, die der anderen mit $y_k$. Dann berechne man $\bar f$ nach
(\ref{m}) auf zwei verschiedene Weisen, indem man einmal den mittleren
Ausdruck und zum anderen den rechten Ausdruck der Doppelgleichung
benutzt. Zur Prüfung der Beziehung (\ref{m}) vergleiche man die beiden
Ergebnisse für $\bar f$ miteinander. Das Fehlerfortpflanzungsgesetz
(\ref{n}) läßt sich prüfen, indem man $\sigma_f$ einmal direkt aus den
Werten $f_{ik}$ und zum anderen nach der Beziehung (\ref{n}) berechnet
und die beiden Ergebnisse miteinander vergleicht. Für diese Aufgabe
steht ein Rechenprogramm zur Verfügung (Versuchsteil 5).\\



\subsection{Der mittlere quadratische Fehler des Bestwertes}

Die Beziehung (\ref{f}) gibt den mittleren quadratischen Fehler $\Delta
M_{i}$ der Einzelmessung $M_{i}$ an. Da aber nicht die Einzelmessung
sondern der Bestwert $M_B$ das Endergebnis darstellt, muß {} der Fehler
des Bestwertes $\Delta M_{B}$ bestimmt werden. Dazu fassen wir $M_B$
als Funktion der Größen $M_{i}$ auf, d.h.\\

\begin{equation} \label{s}
M_{B}\,=\,\frac{1}{n}\,\sum_{i=1}^{n}\,M_{i}\,=\,f(M_{1}, M_{2},...)
\end{equation}

und wenden hierauf das Fehlerfortpflanzungsgesetzt\\

\begin{equation} \label{t}
\Delta M_{B}\,=\,\sqrt{\sum_{i=1}^{n}\,\left(\sigma_{i}\,\frac{\partial
 f}{\partial M_{i}}\right)^{2}}
\end{equation}

an. Da alle $\sigma_i$ als gleich angenommen werden können , d.h. $\sigma_i$ =$\sigma$, und\\

\begin{equation} \label{u}
\frac{\partial f}{\partial M_{i}}\,=\,\frac{1}{n}
\end{equation}

ist, ergibt sich\\

\begin{equation} \label{v}
\Delta
M_{B}\,=\,\frac{\sigma}{\sqrt{n}}\,=\,\sqrt{\frac{\sum_{i=1}^{n}\,(\Delta
 M_{i})^{2}}{n(n-1)}}
\end{equation}



\subsection{Anpassung von Funktionen an Meßwerte nach der Methode der
 kleinsten Quadrate (minimales $\chi^2$)}

Ein häufig vorkommendes Problem ist die Anpassung einer glatten Kurve
an eine Folge von Meßpunkten, die zur Bestimmung der Kurvenparameter
dienen sollen. Als Beispiel benutzen wir das radioaktive
Zerfallsgesetz\\

\begin{equation} \label{w}
N(t)\,=\,N_{0}e^{-\lambda t},
\end{equation}


das die Anzahl $N$(t) der pro Zeiteinheit zerfallenden radioaktiven
Kerne als Funktion der Zeit $t$ beschreibt. In (\ref{w}) ist $N_0$ die
Zahl der Zerfälle zum Zeitpunkt $t$=0 und $\lambda$ die
Zerfallskonstante, die über die Beziehung

\begin{equation} \label{x}
\lambda \,=\,\frac{\ln 2}{T_{1/2}}
\end{equation}

mit der Halbwertszeit $T_{1/2}$ des radioaktiven Nuklids zusammenhängt.
Zur Vereinfachung setzen wir voraus, daß $N_0$ aus anderen Messungen
bekannt ist, so daß nur noch $T_{1/2}$ zu bestimmen ist. Wir erzeugen
zunächst nach der Vorschrift (\ref{k}) "'Meßwerte"' auf dem Computer
und stellen uns die Aufgabe, $T_{1/2}$ durch geeignete Anpassung der
Funktion (\ref{w}) an die Meßwerte zu bestimmen. Die "`Messung"'
liefert\\

\begin{equation} \label{y}
\mbox{Wertepaare} (N_i, t_i),
\end{equation}

d.h. zum Zeitpunk $t_i$ pro Zeiteinheit gemessene Anzahlen $N_i$
radioaktiver Zerfälle. Die Vorschrift für die Anpassung der Funktion
(\ref{w}) lautet\\

\begin{equation} \label{z}
\chi^{2}(T_{1/2})\,=\,\sum_{i=1}^{n}\frac{(N_{i}-N(t_{i}))^{2}}{\sigma_{i}^{2}}\,=\,\mbox{Minimum}.
\end{equation}

In (\ref{z}) bedeutet $\sigma_i$ den mittleren quadratischen Fehler von
N$_i$, der nach den Gesetzen der Statistik durch die Beziehung\\

\begin{equation} \label{aa}
\sigma_i^2 = N_i
\end{equation}

gegeben ist. Wir vergleichen in (\ref{z}) also die Abweichung jedes
Meßwertes $N_i$ von der gewählten Kurve mit dem Fehler des Meßwertes
und minimieren die Summe der mit dem reziproken Fehlerquadrat
gewichteten Abweichungsquadrate. Ausdrücke der Form (\ref{z}) werden
allgemein mit $\chi^2$ bezeichnet und beinhalten die Gauß'sche Methode
der kleinsten Quadrate. Die praktische Auswertung der Vorschrift
(\ref{z}) erfolgt, indem man $\chi^2$ für eine Anzahl geeignet
erscheinender Werte $T_{1/2}$ berechnet und das Minimum mit Hilfe einer
graphischen Darstellung der Funktion $\chi^2$($T_{1/2}$) bestimmt.\\



\section{Fragen}




\section{Durchführung}


\subsection{Versuchsvorbereitung}

\begin{enumerate}

\item Einschalten des Rechners durch den Assistenten (Vorsicht:
unbedingt folgende Reihenfolge einhalten: 0\,2\,4\,1\,3\,5).

\item Beantworten Sie die Aufforderung des (eingeschalteten)
Rechners zur Benutzeridentifizierung \verb|C:\USER>| mit dem Befehl
\verb|A-PRAKT<Return>|. Die \verb|<RETURN>| - Taste ist durch \verb|<>|
gekennzeichnet und übergibt die eingegebenen Informationen dem
Computer. Dadurch wird der Computer in einen definierten Anfangszustand
zurückversetzt und nicht mehr benötigte Programme werden gelöscht.

\end{enumerate}




\subsection{Durchführung}

\begin{enumerate}

\item Start und Erzeugen von Zufallszahlen \newline
Starten Sie das Programm durch die Eingabe \verb|FEHLER<RETURN>|. Das
Programm enthält einen Zufallsgenerator, der für Sie 100 Zufallzahlen
N(i), i=1...100 im Intervall [0,1] berechnet. Zur Berechnung einer
Reihe von Zufallszahlen wird folgende Methode benutzt:

\hspace{2cm} $N(i+1) = \mbox{Mod}(N(i)\,*\,317.1)$

Anschließend werden aus diesen Zufallszahlen 100 gleichverteilte
natürliche Zufallszahlen berechnet, die Werte zwischen 0 und 99
annehmen können und als Matrix auf dem Bildschirm dargestellt werden.\\
{\bf Anmerkung:} (i) Alle auf dem Bildschirm ausgegebenen Ergebnisse
werden zusätzlich in der Datei VERS17.DAT gespeichert. Sie können diese
Ergebnisse am Ende des Versuchs durch folgenden Befehl ausdrucken:\\

\hspace{2cm} \verb|PRINT VERS17.DAT<RETURN>|\\

und/oder auf eine Diskette kopieren.

(ii) Nach dem Beenden jedes Versuchsteils wird durch Betätigen der
\verb|<RETURN>| Taste automatisch zum nächsten Versuchsteil
übergeleitet.

\item Mittelwertberechnung\\
Führen Sie 5-mal eine Mittelwertberechnung für unterschiedliche
Anzahlen von Zufallszahlen durch.

\item Normalverteilung\\
Die 100 gleichverteilten Zufallszahlen werden in 100 normalverteilte
Zufallszahlen, die um den Mittelwert streuen, umgerechnet. Diese
normalverteilten Zahlen geben 100 Messungen einer Stablänge wieder.
Diese Zahlen werden in 5 Gruppen zu je 20 Zahlen auf dem Bildschirm
ausgegeben und als Histogramm graphisch dargestellt. Durch
Bestätigung der \verb|<RETURN>|-Taste wird aus dem Graphik-Modus in
den Text-Modus zurückgeschaltet.

\item Gauß'sche Methode der kleinsten Quadrate\\
Der Computer liefert 100 "'Meßwerte"' für die relative Aktivität des
Zerfalls von $^{108}$Ag (vergl. Versuch 34) - Halbwertszeit =
144.06~s.\\ Führen Sie einen Fit der ``Meßwerte'' durch, indem Sie eine
Halbwertzeit vorgeben. Der Computer berechnet dann nach dem
Zerfallsgesetz eine Kurve, die gemeinsam mit den ``Meßwerten''
graphisch dargestellt wird. Zusätzlich erhalten Sie das $\chi^{2}$ des
durchgeführten Fits. Wiederholen Sie diesen Fit für verschiedene
Halbwertzeiten in der Nähe der wahren Halbwertzeit von $^{108}$Ag.

\item Fehlerfortpflanzung\\
Sie sollen nun selbst ein kleines FORTRAN-Programm schreiben, um die
Formeln (\ref{o}) bis (\ref{r}) zu überprüfen. Als Daten dienen dazu
für die Meßwerte $x_{i}$ und $y_{i}$ jeweils 20 der normalverteilten
Zufallszahlen, die in 3. dieses Versuchs berechnet wurden. Also zum
Beispiel die ersten 20 Werte als $x_{i}$ und die zweiten zwanzig Werte
als $y_{i}$.\\ Zur Überprüfung des Bestwertes (Formeln (\ref{o}) und
(\ref{q})) berechnen Sie den Bestwert für \=f einmal aus den Bestwerten
für \=x und \=y nach den Formeln (\ref{o}) und (\ref{q}) und
vergleichen dieses Ergebnis mit dem Bestwert, den Sie für \=f nach
Formel (\ref{m}) erhalten.\\ Zur Überprüfung des
Fehlerfortpflanzungsgesetzes berechnen Sie den Fehler einmal nach den
Formeln (\ref{p}) und (\ref{r}) und vergleichen diese Ergebnisse mit
dem mittleren quadratischen Fehler (Formel (\ref{f})) der Werte
$f_{ik}$ = $f(x_{i},y_{k})$.\\

Warum sind die Fehler, die man auf diese Weise erhält, nicht exakt
gleich?\\

Für diese Rechnung steht Ihnen ein Rumpfprogramm mit dem Namen
GAUSS.FOR zur Verfügung, das Sie mit dem SEE-Editor oder EDIT
ergänzen können. Das Rumpfprogramm enthält bereits die
Einleseanweisung für die normalverteilten Zufallszahlen. Sie können
innerhalb Ihres Programms auf die Zahlen unter dem Namen STABL (I) für
I = 1,.....,100 zugreifen. Die Stelle, an der Sie Ihren eigenen
Programmtext einfügen müssen, ist gekennzeichnet.\\

Ein Editor wird zum Laden von Files (Dateien) benutzt, so daß man sie
lesen (und ändern) kann, um sie dann erneut auf der Festplatte zu
speichern (zu schreiben). Sie können durch den Befehl SEE FILENAME oder
EDIT FILENAME ein beliebiges File editieren (Vorsicht: Files, die mit
.OBJ oder .EXE enden, können nicht editiert werden.)\\

Zunächst einige Erklärungen zum SEE-Editor:\\

Aufruf: SEE FILENAME oder EDIT FILENAME \\

Bewegung des Cursers innerhalb des Files durch die Pfeiltasten des
Nummernblocks (Tasten 8, 4, 6, 2). Dabei wird der Cursor jeweils um
eine Zeile nach oben oder unten bzw. um einen Buchstaben nach links
oder rechts bewegt. Mit den Tasten Pfeil (Bild) des Nummernblocks
(Tasten 9, 3) kann man jeweils eine ganze Bildschirmseite
weiterblättern. Mit der grauen Pfeiltaste (Pfeil nach links) oberhalb
der RETURN-Taste kann man den Buchstaben links vom Cursor löschen.\\

Abspeichern des Files und Verlassen des Editors durch die Eingabe der
Tastenfolge: ``\verb|<|EING LÖSCH\verb|>|'', ``Q'',
``S''. ``\verb|<|EING LÖSCH\verb|>| ist auf dem Nummernblock zu finden.\\

Ein paar Worte zu FORTRAN:
\begin{itemize}
  \item Rangfolge der arithmetischen Operationen:\\
    Klammern, Sonderfunktionen (SIN, Wurzel, etc.), Potenzieren
    (Zeichen:**), Multiplikationen, Additionen.
  \item Ausgabe des Wertes der Variablen ``TEXT'' auf dem Bildschirm:\\
    WRITE (*,*) TEST
  \item Quadratwurzel von X: SQRT (X)
  \item Zur Programmierung einer Schleife siehe Programmbeispiel (17.A).
\end{itemize}

Haben Sie weitere Fragen zur Programmiersprache FORTRAN, so wenden Sie
sich bitte an Ihren Assistenten.\\

Anschließend müssen Sie Ihr Programm noch COMPILIEREN (übersetzen) und
LINKEN (zusammenbinden mit anderen zur Ausführung notwendigen
Programmteilen, die vom Rechner geliefert werden). Dafür steht für Sie
ein MAKE-File bereit. Zum COMPILIEREN und LINKEN des Files GAUSS.FOR
tippen Sie bitte folgenden Befehl ein: MAKE GAUSS.MK\\

Starten und testen Sie Ihr Programm nun durch: GAUSS\\

Anmerkung: Wollen Sie statt FORTRAN lieber BASIC, C oder PASCAL
benutzen, so ist dies selbstverständlich möglich. Auch für diese
Sprachen gibt es ein Rumpfprogramm und ein MAKE-File.

\begin{itemize}
  \item die Zufallszahlen stehen in dem File GAUSS.DAT
  \item Anleitungen zu den Compilern der anderen Programmiersprachen liegen
    hier im Terminalraum in einer Plastikhülle aus.
\end{itemize}

Bitte haben Sie Verständnis, daß Sie Ihr Assistent nicht in all diesen
Sprachen beraten kann.\\

\end{enumerate}



\section{Angaben}


\subsection{Ableitung der Beziehung (\protect\ref{m}) }


Wir entwickeln das Endergebnis $f(x_{i}$, $y_{k})$ einer Messung von
Teilergebnissen $x_{i}$, $y_{k}$ in eine Taylorreihe um die Bestwerte
$\bar x$, $\bar y$ und brechen nach dem ersten Glied ab:

\begin{equation} \label{ab}
f(x_{i}, y_{k})\,=\,f(\bar x, \bar y)\,+\,(x_{i}-\bar x)\,\frac{\partial
 f}{\partial x}\,+\,(y_{k}-\bar y)\,\frac{\partial f}{\partial y}.
\end{equation}


Der Bestwert $\bar f$ der Endergebnisse $f(x_{i}$, $y_{k})$ ist dann gegeben durch\\

\begin{equation} \label{ac}
\bar f\,=\,r^{-1}\cdot
s^{-1}\,\sum_{i=1}^{r}\,\sum_{k=1}^{s}\,f(x_{i}, y_{k})\\
 =\,f(\bar x, \bar y)\,+\,\frac{1}{r}\,\sum_{i=1}^{r}\,(x_{i}-\bar
x)\,\frac{\partial f}{\partial
 x}\,+\frac{1}{s}\,\sum_{k=1}^{s}\,(y_{k}-\bar y)\,\frac{\partial
 f}{\partial y}.
\end{equation}

Da der 2. und 3. Term auf der rechten Seite von (\ref{ac}) gemäß
(\ref{e}) verschwinden, ergibt sich (\ref{m}).\\


\subsection{Ableitung der Beziehung (\protect\ref{n})}


Wir gehen aus von der Definition des mittleren quadratischen Fehlers (Standard-Abweichung)\\

\begin{equation} \label{ad}
\sigma_{f}^{2}\,=\,(r-1)^{-1}\cdot
(s-1)^{-1}\,\left\{\sum_{i=1}^{r}\,\sum_{k=1}^{s}\,(f(x_{i},
 y_{k})-\bar f)^{2}\right\}
\end{equation}

und entwickeln wie in (\ref{ab}) in eine Taylorreihe. Dann ergibt sich
unter der Benutzung von (\ref{m})\\

\begin{eqnarray} \label{ae}
\sigma_{f}^{2}\,=\,(r-1)^{-1}\cdot
(s-1)^{-1}\,\left\{\,\sum_{i=1}^{r}\,\sum_{k=1}^{s}\,\left[(x_{i}-\bar
x)\,\frac{\partial f}{\partial x}\,+\,(y_{k}-\bar y)\,\frac{\partial
 f}{\partial y}\right]^{2}\right\}
=(r-1)^{-1}\cdot
(s-1)^{-1} \nonumber\\
\left\{\,\sum_{i=1}^{r}\,\sum_{k=1}^{s}\,\left[(x_{i}-\bar
x)^{2} \left(\frac{\partial f}{\partial x}\right)^{2}\,+\,(y_{k}-\bar
y)^{2}\,\left(\frac{\partial f}{\partial y}\right)^{2}\,+\,2(x_{i}-\bar
x)\,\frac{\partial f}{\partial x}\,(y_{k}-\bar y)\,\frac{\partial f}{\partial
  y}\right]\right\}.
\end{eqnarray}

Definieren wir nun die Fehler der Teilergebnisse $\sigma_x$
bzw. $\sigma_y$ nach der in (\ref{f}) gegebenen Vorschrift, so ergibt
sich\\

\begin{equation} \label{af}
\sigma_{f}^{2}\,=\,\sigma_{x}^{2}\left(\frac{\partial f}{\partial
 x}\right)^{2}\,+\,\sigma_{y}^{2}\left(\frac{\partial f}{\partial
 y}\right)^{2}
 \end{equation}

in Übereinstimmung mit (\ref{n}).\\


Wir wenden jetzt (\ref{ad}) auf die in (\ref{o}) und (\ref{q}) angegebenen
Spezialfälle an:\\



Aus $f = x+y$ ergibt sich $\frac{\partial f}{\partial x}$=$\frac{\partial
 f}{\partial y}$ = 1 und damit
$\sigma_{f}^{2}=\sigma_{x}^{2}+\sigma_{y}^{2}$ in Übereinstimmung mit
(\ref{p}). Aus $f=x\cdot y$ ergibt sich $\frac{\partial
  f}{\partial x}|_{y=\bar y}=\bar y$ und $\frac{\partial f}{\partial y}|_{x=\bar x}=\bar x$ und damit
$\sigma_{f}^{2}=\bar y^{2} \sigma_{x}^{2}\,+\bar x^{2}
\sigma_{y}^{2}$ oder wegen $\bar f = \bar x \cdot\bar y$ dann
$\frac{\sigma_{f}^{2}}{\bar f^{2}}\,=\frac{\sigma_{x}^{2}}{\bar
 x^{2}}\,+\,\frac{\sigma_{y}^{2}}{\bar y^{2}}$
in Übereinstimmung mit (\ref{r}) .\\





\section{Auswertung}

\begin{enumerate}

\item Zu 2.: Berechnung des Fehlers des Mittelwertes.
\item Zu 3.: Berechnung des Bestwertes und seines Fehlers für
je 20 Messungen der Stablänge.
\item Zu 4.: Auftragen von $\chi^{2}$ als Funktion der
Halbwertzeit. Bestimmung der ``gemessenen'' Halbwertzeit aus dem
minimalen $\chi^{2}$.
\item Zu 5.:\\
a) Prüfen Sie das Fehlerfortpflanzungsgesetz für die Summe zweier
Stablängen (Ein Stab der Länge 2 m wird durch zweimaliges Anlegen des
1~m-Maßstabs vermessen.)\\ b) Prüfen Sie das Fehlerfortpflanzungsgesetz
für das Produkt zweier Stablängen (Der Flächeninhalt eines Quadrats
wird durch Messung der Kantenlängen bestimmt.)

\end{enumerate}



\section{Bemerkungen}


Keine.
