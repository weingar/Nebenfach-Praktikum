\chapter*{Praktikumsordnung} \label{v:ordnung}

%**********************************************************************************************************************
%**********************************************************************************************************************
\section{Geltungsbereich}

Die nachstehende Praktikumsordnung gilt für alle Teilnehmende des Physikalischen Praktikums für Nebenfach Physik, Modul B.Phy-NF.7004, der Friedrich-Wilhelms-Universität Göttingen ab dem Sommersemester 2016.

%**********************************************************************************************************************
%**********************************************************************************************************************
\section{Ablauf des Praktikums}

\begin{enumerate}
	%
	\item Das Praktikum erstreckt sich über ein Semester. Wird das Praktikum nicht in diesem Semester abgeschlossen, so muss es erneut belegt werden. Versuche aus dem vorherigen Kurs können nicht angerechnet werden.
	%
	\item Das Praktikum umfasst 14 Versuche. Die Termine können dem Kursplan entnommen werden, der vor Beginn des Praktikums auf den Webseiten des Physikalischen Praktikums für Nebenfach Physik veröffentlicht wird.
	%
	\item Das Praktikum beginnt mit der obligatorischen Einführungsveranstaltung mit Sicherheitsbelehrung. Eine Teilnahme am Praktikum ohne Sicherheitsbelehrung ist nicht möglich. Sollte das Praktikum zum wiederholten Male belegt werden, ist die Einführungsveranstaltung erneut zu besuchen.
	%
	\item Voraussetzung für die Teilnahme am Praktikum ist die erfolgreich bestandene Klausur zur Vorlesung Experimentalpyhsik I, Modul B.Phy-NF.7001 oder B.Phy-NF.7002.
\end{enumerate}

%**********************************************************************************************************************
%**********************************************************************************************************************
\section{}