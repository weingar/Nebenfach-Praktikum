
%%%%%%%%%%%%%%%%%%%%%%%%%%%%%%%%%%%%%%%%%%%%%%%%%%%%%%%%%%%%%%%%%%%

\chapter*{Vorwort}

\addcontentsline{toc}{chapter}{Vorwort}


Herzlich Willkommen zum physikalischen Praktikum im Nebenfach an der Universit"at
G"ottingen. Diese Praktikumsanleitung enth"alt alle Informationen, die Sie ben"otigen, um das einsemestrige Praktikum erfolgreich zu absolvieren.
%wird Sie die n"achsten zwei bis drei Semester
%begleiten. Das Grundpraktikum f"ur das Fach Physik
%beinhaltet die Vorlesung \glqq{}Grundlagen des Experimentierens\grqq{} 
%(GdE) sowie 25~Versuche und erstreckt sich "uber zwei Semester. Es
%soll im Regelfall im 2.~und 3.~Semester absolviert werden. Das
%separate Projektpraktikum schlie"st sich im 4.~Semester an.

Das Praktikum (Modul B.Phy-NF.7004) wird ben"otigt f"ur den Bachelor in Chemie, Mathematik, Geologie, Biologie, Bio-Diversit"at, Mineralogie, Physiologie und Molekulare Medizin sowie f"ur den Zwei-Fach Bachelor in Biologie, Chemie, Geowissenschaften und Mathematik. Dieses Praktikum wird mit 4~SWS angerechnet und bringt 4~Credits.
%Physik und f"ur den 2-F"acher-Bachelor (Lehramt an Gymnasien) mit Physik
%als Fach. 
%Dieses Praktikum wird mit 4~SWS angerechnet und bringt 12~Credits -- 2~Credits f"ur
%die GdE als B.phy.410.1 und 10~Credits f"ur die 25~Versuche als
%B.phy.410.2. F"ur den Bachelor in Physik ist zus"atzlich auch
%das Projektpraktikum (Modul B.phy.604) erforderlich.\footnote{Das
%Projektpraktikum kann auch von Lehramtstudentinnen und -studenten
%durchgef"uhrt und als Studienleistung anerkannt werden. Bitte
%informieren Sie sich bei Ihrer Studienberaterin f"ur das
%Lehramt/Zweif"acher-Bachelor.} 


Das G"ottinger Nebenfach-Physikpraktikum f"uhrt anhand ausgew"ahlter, vorgefertigter Versuche
in einen weiten Bereich physikalischer Grundlagen, in den Umgang mit
Apparaturen und Messger"aten und in die Technik des physikalischen
Experimentierens ein und stellt damit einen wesentlichen Teil der
traditionellen Grundausbildung in Physik dar. Ziel ist hierbei auch 
eine Vertiefung des
in der Vorlesung ''Experimentalphysik I im Nebenfach'' erlernten Stoffes durch eigenes Umsetzen
und das Erfahren von Physik (\textit{learning by doing}). Sie erlernen
den Umgang mit verschiedensten Ger"aten und erfahren durch eigenes
Tun, wie eine physikalische Aufgabenstellung experimentell und
methodisch angegangen wird (\textit{hands on physics}). %Problem -
%Analyse - Bearbeitung - L"osung - Dokumentation, dies ist die
%Sequenz, die Sie in Ihrem ganzen "`Physik"=Leben"' begleiten wird.
Hierbei spielt auch Gruppen- oder Teamarbeit eine wichtige Rolle.
Nutzen Sie die Gelegenheit im Praktikum auch dies zu "uben, und
bringen Sie sich aktiv ein. Es wird sich auszahlen.

Dieses Handbuch beschreibt derzeit 20~Versuche, wovon nur 14
verpflichtend durchgef"uhrt werden m"ussen. %Zu Abschnitt 15 (Ferro-, Para-,
%Diamagnetismus) stehen zwei Versuche, zu Abschnitt 17 (Elektronik)
%stehen drei Versuche zur Auswahl, wovon jeweils einer durchgef"uhrt wird.
%Sprechen Sie die Auswahl bitte rechtzeitig mit Ihren Praktikumspartnern und
%Ihrem/r Betreuer/in ab.

%Mit dem seit 2002 eingef"uhrten Projektpraktikum, welches sich an die
%vorgefertigten Versuche anschlie"st, soll verst"arkt auch das
%eigenst"andige wissenschaftliche Arbeiten gef"ordert werden. Hierzu
%werden nur Themen (bzw. Aufgabenstellungen) vorgegeben oder auch von
%den Praktikantinnen und Praktikanten selbst ausgew"ahlt, die dann
%eigenst"andig in der Gruppe (max. 6~Personen) mit
%Hilfestellung eines Betreuers bearbeitet werden. Weitere Einzelheiten zum
%Projektpraktikum sind in diesem Handbuch im Teil \glqq
%Projektpraktikum\grqq{} angef"uhrt.

%Wir legen hiermit ein wiederum "uberarbeitetes Handbuch
%f"ur das G"ottinger Grundpraktikum Physik vor. 
%Wir m"ochten uns
%bei allen Praktikantinnen und Praktikanten, sowie allen Betreuerinnen
%und Betreuern bedanken, die durch ihre Hinweise und Vorschl"age geholfen
%haben, diese Anleitung zu verbessern. Ein besonderer Dank gilt den 
%Beteiligten des Lehrportals, die diese Anleitung mit einer online-Version
%und Videos unterst"utzen und mit verbesserten Abbildungen zu dieser
%Version des Handbuchs beigetragen haben. Trotz der Verbesserungen ist 
%es nur nat"urlich, dass sich auch wieder neue
%Fehler und Unzul"anglichkeiten in dieses Handbuch eingeschlichen haben.
%Wir w"aren dankbar, wenn Sie uns Fehler und auch Verbesserungsvorschl"age
%\emph{sofort} mitteilen w"urden (E-Mail: {\footnotesize
%\url{jgrosse1@uni-goettingen.de}}). Wir werden diese dann
%schnellstm"oglich beheben und auf den Web"=Seiten des Praktikums eine
%verbesserte Version der jeweiligen Anleitung zur Verf"ugung stellen.
%Noch sind nicht alle Versuche und ihre Anleitungen auf dem Stand, den
%Sie und wir gerne h"atten. Dies werden Sie sicherlich feststellen.
%Bedenken Sie, dass diese Anleitung und auch die "Uberarbeitung und
%Erneuerung der Versuche sehr viel Arbeit erfordert, und wir bei der
%derzeitigen Personal- und Betreuungssituation nicht alles umsetzen
%k"onnen, was Sie sich und wir uns w"unschen.

%Wir modernisieren das Grundpraktikum Physik weiterhin
%kontinuierlich durch Neuanschaffungen, Versuchsmodifikationen und
%Entwicklung neuer Versuche. Daneben gibt es Apparaturen, die etwas
%\glqq{}altmodischer\grqq{} aussehen, aber doch noch ganz ihrer
%(didaktischen) Aufgabe gerecht werden. Es kostet gro"se M"uhe, alle
%diese Apparaturen in einem einwandfreien Zustand zu erhalten.
%Sollten Sie dennoch Fehler feststellen, so geben Sie uns bitte
%sofort Bescheid. Nur dann k"onnen wir f"ur Abhilfe sorgen.

%Auch nach dem Druck dieser \glqq{}Praktikumsanleitung\grqq{} werden
%Versuche weiterentwickelt und verbessert, so dass es zu Abweichungen
%des aktuellen Versuches von dieser Anleitung kommen kann. Bedenken
%Sie bitte, dass zwischen Drucklegung und Ihrer Durchf"uhrung des
%Versuches schon eine lange Zeitspanne vergangen sein kann (im
%Extremfall "uber ein Jahr). Wir bem"uhen uns, Ihnen solche "Anderungen
%und Verbesserungen rechtzeitig mitzuteilen, hoffen aber auch, dass
%Sie diese Verbesserungen honorieren werden. Wir werden versuchen,
%auf den Webseiten immer aktuelle Versuchsanleitungen zur Verf"ugung
%zu stellen. Es lohnt sich also bestimmt, von Zeit zu Zeit auf den
%Web-Seiten des Praktikums {\footnotesize
%\url{http://www.praktikum.physik.uni-goettingen.de}} nachzusehen, da
%wir uns bem"uhen werden, dort immer die aktuellsten Informationen zu
%publizieren. Auf den Webseiten finden Sie auch wertvolle Hinweise,
%Kontaktadressen, Termine, Gruppeneinteilungen und eine E-Mail Liste.
%Die E-Mail-Liste ist zur Vermeidung von \acro{SPAM}-Mail 
%nur nach einem Login "uber das Benutzerkonto, welches Sie bei
%der online-Anmeldung angelegt haben, zu erreichen.

In diesem Handbuch finden Sie eine kleine Abhandlung "uber die
Grundlagen der Fehlerrechnung und Protokollerstellung. 
%W"ahrend der
%\glqq Grundlagen des Experimentierens\grqq{} werden Sie davon bereits
%profitiert haben und k"onnen das dort gelernte hier entsprechend anwenden.
%Sprechen Sie
%bitte Ihren Betreuer/Ihre Betreuerin an, damit diese/r an den ersten
%Versuchstagen die Fehlerrechnung und Protokollerstellung nochmal an
%einem konkreten Beispiel mit Ihnen und Ihrer Gruppe "ubt.

Bitte bedenken Sie auch immer, dass Ihre Betreuerinnen und Betreuer
f"ur \emph{Ihr} Praktikum, also f"ur \emph{Ihren} Lernerfolg, viel
Arbeit und Zeit investieren. Dies geschieht neben eigenem Studium
oder eigener Promotion und resultiert in einer Belastung, die weit
"uber das hinausgeht, was als Lehrverpflichtung von Betreuer(innen)
im Durchschnitt an der Fakult"at erbracht wird. Leider
stehen uns nicht so viele Betreuer(innen) zur Verf"ugung, wie wir dies aus
praktischen und didaktischen Erw"agungen f"ur sinnvoll erachten.
Erleichtern Sie deshalb bitte Ihren Betreuerinnen und Betreuern
diese Belastung durch Ihre engagierte, aktive, gut vorbereitete und
m"oglichst eigenst"andige Mitarbeit im Praktikum und
\glqq{}zahlen\grqq{} Sie deren Engagement mit Ihrem pers"onlichen
guten Lernerfolg zur"uck. Nur Ihr \emph{aktives und eigenst"andiges}
Arbeiten erzielt auch eine hohe Nachhaltigkeit des Erlernten und
schafft so das solide Wissensfundament, auf dem Sie Ihre Zukunft
aufbauen k"onnen.

Zusammenfassend w"unschen Ihnen alle Betreuerinnen und Betreuer des
Praktikums viel Spa"s im und einen guten Lernerfolg durch das
Praktikum. Wir alle, insbesondere Ihre Betreuerinnen und Betreuer,
bem"uhen uns, damit dies -- Ihre Mithilfe angenommen -- auch erreicht
werden kann.


%\signature{G"ottingen}{im Oktober 2015}{Dr. Jens Weingarten}
