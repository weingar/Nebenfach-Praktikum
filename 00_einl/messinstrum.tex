\chapter{Verwendung von Messinstrumenten}   \label{c:messinstr}

Ein Hauptpunkt im Praktikum ist das Erlernen des Umgangs mit
verschiedensten Messger"aten. Einige wichtige und h"aufige
Messger"ate werden hier etwas genauer beschrieben. Dennoch sollten
Sie sich die ausf"uhrlicheren Anleitungen zu den jeweiligen
Messger"aten mindestens einmal durchlesen. Dies erleichtert Ihnen
auch die Durchf"uhrung der entsprechenden Versuche.

\section{Allgemeines}

Sollte bei der Bedienung des Ger"ates Unsicherheit bestehen, so
fragen Sie bitte Ihre/n Betreuer/in. Die Anleitungen f"ur die
Ger"ate \index{Ger"ate} sind meist am Ger"at selbst oder in dessen
N"ahe zu finden. Falls nicht bitte einfach nachfragen. Ist ein
Ger"at defekt, bitte dies sofort dem/r Betreuer/in melden. Das
Ger"at bitte nicht zur"uck in den Schrank stellen, denn im Schrank
repariert es sich nicht von alleine.

Auf den Praktikums-Webseiten \index{Ger"ate!Webseiten} finden Sie
auch einen Punkt "`Ger"ate"', unter dem wir Bilder und detaillierte
Anleitungen mit Zusatzinformationen zusammengestellt haben. Sehen
Sie sich die Anleitungen zu den entsprechenden Ger"aten bitte vor
dem Versuch an. Im Praktikum wird auch ein Ordner mit einer
Sammlung aller Bedienungsanleitungen\index{Bedienungsanleitung} zu
finden sein.

Alle Messger"ate, also auch digitale Messger"ate\index{Messger"ate},
verursachen Messfehler\index{Messfehler}. Diese sind, wenn
m"oglich, herauszufinden\footnote{Diese stehen meist auf dem Ger"at
selbst oder in der entsprechenden Anleitung}, oder aber
abzusch"atzen und im Messprotokoll zu notieren.

\section{Multimeter}

Die im Praktikum wohl am h"aufigsten verwendeten Messger"ate sind
die Digital-Multimeter (\acro{DMM}) \acro{M2012} oder
\acro{MetraMax~12}\index{Multimeter}. Ersteres ist in
Bild~\ref{a:multimeter} dargestellt. Die Bedeutung der einzelnen
Anschl"usse und Kn"opfe ist in Tabelle~\ref{t:multimeter}
aufgef"uhrt.
%
{\samepage
\begin{figure}[t]
% \centering
% \setcapindent{1em}
 \begin{captionbeside}[Bild eines Digitalmultimeters]{\label{a:multimeter}Darstellung des im Praktikum verwendeten
  Digital-Multimeters \acro{M2012} (Messger"at f"ur elektrische Spannung,
  Strom und Widerstand).}
  \includegraphics[width=5.5cm]{00_einl/multimeter}
 \end{captionbeside}
% \caption
\end{figure}
%
\begin{table}[t]%
  \centering%
  \caption{\label{t:multimeter}Bedienelemente des Digital-Multimeters.}%
  \begin{tabular}{lp{10cm}}\toprule%
  1 & EIN-/AUS-Schalter  \\
  2 & Messbereichsumschalter  \\
  3 & Fl"ussigkristallanzeige (LCD)  \\
  4 & Anschlussbuchse f"ur alle Messbereiche, niedriges Potenzial ("`Erde"')\\
  5 & Anschlussbuchse f"ur alle Messgr"o"sen (ausgenommen Str"ome gr"o"ser als
  2~A)   \\
  6 & Anschlussbuchse f"ur Netzadapter   \\
  7 & Anschlussbuchse f"ur Strommessbereich 10~A  \\ \bottomrule
\end{tabular}
\end{table}
}
%
Das Multimeter wird an Knopf (1) ein- und auch ausgeschaltet. Der
Messbereich wird mit dem Drehschalter (2) gew"ahlt. Auf der
Schalterstellung mit dem Batteriezeichen kann der Ladezustand des
internen Akkus gepr"uft werden, die Spannung sollte mehr als
\numprint[V]{7} betragen. Vor der Messung ist darauf zu achten,
dass i) die richtige Messgr"o"se (Strom, Spannung, Widerstand, auch
AC oder DC!) und ii) zuerst der h"ochste Messbereich ausgew"ahlt
wird. Dann kann auf den optimalen Messbereich runtergeschaltet
werden.\footnote{Dies bitte nicht beim Versuch "`Transformator"',
da durch den Messbereichswechsel beim Amp\`{e}remeter der Strom
kurzfristig unterbrochen wird und somit gro"se Induktionsspannungen
erzeugt werden k"onnen, die andere Messger"ate besch"adigen.}

\emph{Achtung}: Bei einigen Versuchen k"onnen die Messleitungen
hohe Spannungen oder Str"ome f"uhren (z.B. Transformator), also die
Kabel vorsichtig an der Isolierung anfassen, nie am eigentlichen
Stecker. Auch nicht am Kabel ziehen. Prinzipiell sollten alle
Manipulationen an der Schaltung und den Messger"aten im spannungs-
und stromlosen Zustand get"atigt werden.

Das Multimeter hat eine so genannte 3$\nicefrac{1}{2}$-stellige
Anzeige, d.h.~die gr"o"ste darstellbare Ziffernanzeige ist
\numprint{1999}. Wird der Messbereich "uberschritten, dann
verschwinden die rechten 3 Ziffern und nur die linke 1 wird
angezeigt.

Das Ger"at ist mit einer Schutz-Sicherung gegen "Uberlast gesch"utzt.
Wurde diese Sicherung ausgel"ost, ist keine Messung mehr m"oglich,
das Ger"at bitte aussondern und den/die Betreuer/in
informieren.\footnote{Sie k"onnen das Ger"at testen, indem Sie den
Widerstand eines Kabels messen, ist dieser kleiner als
\numprint[\Omega]{1}, so ist die Sicherung in Ordnung, ist der
Widerstand $\infty$, ist die Sicherung defekt und das Ger"at muss
vom Praktikumstechniker repariert werden.} Der gesonderte
\numprint[A]{10} Eingang ist nicht gesichert, hier kann also das
Ger"at bei Fehlbedienung zerst"ort werden, also ist hier Vorsicht
angebracht (Insbesondere beim Schalten von Spulenstr"omen!).


Die Genauigkeits- oder auch G"uteklasse\index{G"uteklasse} des
Ger"ates ist \numprint{1.5}, daraus ergeben sich die in
Tabelle~\ref{t:multifehler} zusammengefassten maximalen
(systematischen) Messfehler. Der
Eingangswiderstand\index{Eingangswiderstand} betr"agt in allen
Spannungsmessbereichen \numprint[M\Omega]{10}. Die
Eingangswiderst"ande in den Strommessbereichen h"angen nat"urlich
stark vom Messbereich ab (warum?) und k"onnen mit einem weiteren
Multimeter im Widerstandsmessbetrieb gemessen werden (gew"unschten
Messbereich w"ahlen und Ger"ate einschalten).
%
{\small
\begin{table}[htb]
  \centering
  \caption[Messfehler des Digitalmultimeters]{\label{t:multifehler}Gr"o"stm"ogliche
    Messfehler des Digitalmultimeters in den verschiedenen Messbereichen.}
  \begin{tabular}{lll} \toprule
  % after \\ : \hline or \cline{col1-col2} \cline{col3-col4} ...
  Messbereich & DC & AC \\ \midrule
  alle Spannungen & $\pm$(\numprint[\%]{0.25} v. Max.-Wert + 1 Digit) &
                    $\pm$(\numprint[\%]{1} v. Max.-Wert + 3 Digits) \\
  alle Str"ome & $\pm$(\numprint[\%]{1} v. Max.-Wert + 1 Digit) &
                 $\pm$(\numprint[\%]{1.5} v. Max.-Wert + 3 Digits) \\
  Widerst"ande \numprint[k\Omega]{2}-\numprint[M\Omega]{2} &
    $\pm$(\numprint[\%]{0.5} v. Max.-Wert + 1 Digit) &
  \\
  Widerstand \numprint[M\Omega]{20} &
   $\pm$(\numprint[\%]{2} v. Max.-Wert + 1 Digit) &
   \\            \bottomrule
\end{tabular}

\end{table}
}

Die detaillierte Anleitung kann auf Wunsch eingesehen werden
(Assistenten fragen) und ist auf den Webseiten zu finden. Warum
sollte man immer im kleinstm"oglichen Messbereich messen?

Es gibt auch neuere Multimeter, die einen gr"o"seren Funktionsumfang
haben, und u.a. auch Kapazit"aten messen k"onnen. Informieren Sie
sich auf unseren Webseiten "uber diese.

{\bf Eine Bitte}: Beim Wegr"aumen Multimeter ausschalten, Messbereich
1000~V w"ahlen, defekte Ger"ate nicht wieder in den Schrank wegr"aumen.
Ihre Nachfolgerinnen "argern sich sonst "uber defekte Ger"ate und
irgendwann sind Sie selbst ein Nachfolger.



%%%%%%%%%%%%%%%%%%%%%%%%%%%%%%%%%%%%%%%%%%%%%%%%%%%%%%%%%%%%%%%%%%%%%%%

\section{Oszilloskop}

Mittlerweile werden im Praktikum nur noch die digitalen
Speicheroszilloskope\index{Oszilloskop} \acro{TDS210} und
\acro{TDS1012} von \new{Tektronix} verwendet. Die Bedienung von
allen Oszilloskope ist aber prinzipiell sehr "ahnlich.

Ein Oszilloskop ist das vielseitigste Messger"at im Bereich der
Elektrik und Elektronik. Jedes Oszilloskop erschreckt durch die
Vielzahl seiner Kn"opfe, die aber halbwegs logisch sortiert sind.
Eine Anleitung liegt im Praktikum aus, kann aber auch auf den
Webseiten eingesehen werden.

%Eine Abbildung der Frontseite, des im Praktikum verwendeten
%Oszilloskops \acro{HM604}, findet man in Bild~\ref{a:oszi}. Man
%erkennt drei wesentliche Bedienfelder. Links oben ist der Bildschirm
%mit einem unterlegten Netz, deren gro"se K"astchen 10~mm Kantenl"ange
%haben.
%%
%\begin{figure}[htb]%
% \centering%
% \includegraphics[width=9cm]{00_einl/oszi}%
% \caption{\label{a:oszi} Darstellung des im Praktikum verwendeten
%  Oszilloskops \acro{HM604}.}%
%\end{figure}%
%%
%Rechts oben im Feld befindet sich der Einschaltknopf (1), die
%anderen Kn"opfe bestimmen die Zeitablenkung des Oszilloskops,
%d.h.~die x-Achse des Bildes und die so genannte Triggerung, d.h.~den
%Beginn der Ablenkung des Elektronenstrahls von links nach rechts
%(Stichwort: stehendes Bild). Im unteren Feld werden die zu
%untersuchenden Spannungen angeschlossen. Dieses Ger"at kann zwei
%Signale gleichzeitig anzeigen, daher sind viele Funktionen doppelt.
%Hier regelt man auch die y-Achse. Im schmalen linken Feld unter dem
%Bildschirm gibt es die Helligkeits- (18) und Sch"arferegelung (20)
%des Bildes, sowie weitere Testfunktionen des Ger"ates.
%
%\emph{Betrieb}: Nach Einschalten (1) erscheinen eine oder zwei
%waagerechte Linien auf dem Bildschirm. Helligkeit (18) und Sch"arfe
%(20) werden geregelt. Die wei"sen Pfeile auf den drei roten Kn"opfen
%(13, 30, 34) m"ussen nach rechts zeigen (besser: eingerastet sein),
%denn nur in dieser Position gelten die Eichangaben (V/cm oder ms/cm)
%der x- und y-Ablenkung. Da wir nur auf Position~I messen, wird der
%Tastkopf bei (27) eingesteckt, die zugeh"orige Masseverbindung
%erfolgt an der Bananenbuchse daneben. Die zweite waagerechte Linie
%kann mit Drehknopf (39) aus dem Bild geschoben werden. Mit dem Knopf
%(28) wird die Empfindlichkeit der y-Achse eingestellt: Stellung 2~V
%bedeutet, dass jedes K"astchen auf dem Bildschirm 2~V hoch ist
%(Achtung: man beachte auch einen eventuellen Faktor des Tastkopfes,
%Untersetzung). F"ur Schalter (29) gibt es drei Stellungen: GD
%(=ground) bedeutet, dass das Oszi die Nulllinie $U$=0~V anzeigt. Man
%schiebt sie geschickterweise mit (25) auf die Mittellinie. Die
%Stellung AC (=alternating current) wird bei Wechselspannungen
%benutzt, das Signal bewegt sich um die Mittellinie. Die Stellung DC
%(=direct current) wird f"ur Gleichspannungen verwendet. Entsprechend
%muss der Schalter (10) im oberen rechten Feld auf AC oder DC stehen.
%
%Der Drehknopf (12) regelt die Empfindlichkeit der Zeitachse
%(x-Achse): Im kleinsten Bereich \numprint[\mu s]{0.05} ist die
%Ablenkung pro K"astchen 50~ns. Die L"ange eines Signales oder die
%Periode einer Schwingung kann also direkt vom Bildschirm abgelesen
%werden. Am Schalter (15) wird der TRIGGER eingestellt. Hier wird
%geregelt, wann das Oszi mit einer weiteren Zeitablenkung beginnt.
%Das ist sehr wichtig, um stehende Bilder zu erhalten. Im Praktikum
%sollte nach M"oglichkeit die Stellung AT (=automatic triggering)
%eingerastet sein. Schalter (31) regelt, ob ein Signal in Kanal~I
%(CH~I) oder in Kanal~II (CH~II) f"ur diese Triggerung verwendet
%wird. Alle Oszilloskope werden nach diesem Schema bedient. Die
%vielen Kn"opfe verf"uhren zum Spielen, und wir m"ochten alle
%ermutigen, diese auch alle auszuprobieren. Der/die Betreuer/in ist
%sicherlich behilflich, die f"ur eine richtige Messung notwendigen
%Einstellungen wieder zu finden. Also viel Spa"s damit!
%
%Hinweis: Die Genauigkeit des Ger"ates betr"agt etwa \numprint[\%]{1.5}
%vom Messbereichsendwert. Der Eingangswiderstand betr"agt etwa
%\numprint[M\Omega]{1}, die Eingangskapazit"at ist am Ger"at und in der
%Anleitung angegeben. Die genauen Werte sind am Ger"at abzulesen oder
%der Anleitung zu entnehmen.

%\subsection{Digitales Speicher-Oszilloskop}

Im Praktikum verwenden wir mittlerweile nur noch digitale
Speicheroszilloskope der Marke Tektronix \acro{TDS210} und
\acro{TDS1012}. Dies sind insbesondere die Versuche
"`Wechselstromwiderst"ande"', "`Messung grosser Widerst"ande"' und
"`Transistor"'. Der besondere Vorteil dieser Ger"ate ist, dass sie
eine Kurve speichern (im Gegensatz zu den herk"ommlichen) und "uber
eine Schnittstelle die Messkurve auf einen Drucker ausgeben k"onnen
oder die Daten an einen Computer zur weiteren Verarbeitung
weitergeleitet werden k"onnen. Die weiteren Eigenschaften sind jedoch
vollkommen "ahnlich zu den alten analogen
Elektronenstrahloszillographen.

Nutzen Sie die Gelegenheit sich mit diesem wichtigen Messinstrument
vertraut zu machen und spielen Sie damit.\footnote{Bei den
normalerweise im Praktikum vorkommenden Situation k"onnen Sie es
nicht besch"adigen.} Die Daten des Ger"ates und die Anleitung finden
Sie im Fach unter dem Oszilloskop. Detaillierte Anleitungen und
Handb"ucher finden Sie auch auf den Webseiten. Auf Wunsch kann das
Ger"at auch gerne als Erg"anzung f"ur andere Versuche eingesetzt
werden.

An die Eing"ange der Oszis geh"oren normalerweise so genannte BNC
Stecker (au"sen Masse und innen Signal). Die Massen der beiden
Eing"ange sind verbunden. Daneben gibt es noch einen
"`Trigger"'-Eingang. Den Oszis liegen Tastk"opfe\index{Tastkopf} bei.
Machen Sie sich mit der Funktion und dem Anschluss vertraut. Bitten
Sie hierzu einfach Ihre Betreuerin um Hilfestellung.


\section{Stromzange}

Im Praktikum steht ein "`Zangenstromwandler"'
(\new{Stromzange}\index{Stromzange}) f"ur die Messung hoher
Wechselstr"ome zur Verf"ugung.
%
\begin{figure}[htb]
% \centering
% \setcapindent{1em}
 \begin{captionbeside}{\label{a:stromzange}Abbildung der Stromzange C160.}
  \includegraphics[width=3cm]{00_einl/stromzange}
 \end{captionbeside}
%  \caption
\end{figure}
%
Das vorhandene Modell C160 ist in Bild~\ref{a:stromzange}
dargestellt und hat einen Nenn-Messbereich f"ur Wechselstr"ome von
\numprint[mA]{100} bis zu \numprint[A]{1000}. Es kann mit einem
\acro{BNC}-Kabel direkt an die Eingangsbuchse des Oszilloskops
angeschlossen werden, oder auch an Multimeter, Wattmeter und
Messwertschreiber. An der \acro{BNC}-Ausgangsbuchse steht eine
Wechselspannung mit Messbereichsumschaltung zur Verf"ugung.

Der oben auf den Zangenbacken eingepr"agte Pfeil zeigt die
Stromrichtung an. Man geht davon aus, dass der Strom vom
Stromerzeuger zum Stromverbraucher in positive Richtung flie"st.
Die Flu"srichtung ist nur bei Leistungsmessungen oder Detektion
einer Phasenverschiebung von Bedeutung.

Zur Verwendung wie folgt vorgehen:
\begin{enumerate*}
    \item Vor Anschluss der Stromzange pr"ufen, ob das
    angeschlossene Messger"at den richtigen Messbereich hat.
    \item Zangenbacken "offnen und Leiter mit dem zu messenden
    Strom umschlie"sen. Den Leiter in den Backen m"oglichst
    zentrieren und auf die Flu"srichtung achten, falls es die
    Messung erfordert.
    \item Beim Ablesen des Messwertes auf das Wandlungsverh"altnis
    und den Messbereich der Messger"ates (Oszi) achten.
    \item Den Bereich mit der bestm"oglichen Aufl"osung w"ahlen.
    \item Achten Sie darauf, dass keine anderen
    stromdurchflossenen Leiter in der N"ahe liegen und dass keine
    magnetischen Streufelder einkoppeln.
\end{enumerate*}

Die Eingangsimpedanz des C160 ist $\geq\numprint[M\Omega]{10}$ und
$\leq\numprint[pF]{100}$. Der Frequenzbereich geht von
\numprint[Hz]{10} bis \numprint[kHz]{100}.

Die Wandlungsverh"altnisse sind in Tabelle~\ref{t:wandlung}
angegeben.
%
\begin{table}[htb]
  \centering
  \caption{\label{t:wandlung}Wandlungsverh"altnisse der Stromzange C160 (jeweils
  AC).}
  \begin{tabular}{|l|l|l|} \toprule
    Nenn-Messbereich & Wandlerverh"altnis Ausg./Eing. & Fehler \\ \midrule
    \numprint[A]{1} - \numprint[A]{1000} & \numprint[mV]{1} / \numprint[A]{1}
      & $\leq$ 1\% + \numprint[mV]{1} \\
    \numprint[mA]{100} - \numprint[A]{100} & \numprint[mV]{10} / \numprint[A]{1}
      & $\leq$ 2\% + \numprint[mV]{5} \\
    \numprint[mA]{100} - \numprint[A]{10} & \numprint[mV]{100} / \numprint[A]{1}
     & $\leq$ 3\% + \numprint[mV]{10} \\ \bottomrule
  \end{tabular}
\end{table}



\section{Stoppuhr}

Die alten mechanischen (sehr genauen und teuren)
Stoppuhren\index{Stoppuhr} wurden mittlerweile durch neue
elektronische Stoppuhren (nicht genauer, aber billiger) ersetzt.
Eine Anleitung liegt im Messger"ateschrank aus und kann zudem auf den
Praktikums"=Webseiten abgerufen werden. Wie funktioniert eine
elektronische Uhr eigentlich? In Bild~\ref{a:stopuhr} ist die Uhr
dargestellt.
%
\begin{figure}[htb]
% \centering
% \setcapindent{1em}
 \begin{captionbeside}{\label{a:stopuhr}Die elektronische Stoppuhr.}
  \includegraphics[width=4.5cm]{00_einl/stoppuhr}
 \end{captionbeside}
% \caption
\end{figure}


\subsection{Bedienungsanleitung}

\begin{description*}

%\item Besonderheiten
%
%\begin{itemize*}
%  \item Elektronische Stoppuhr fur Sportmessungen
%  \item W"ahlbare "`Lap"'(Zwischenzeit) oder "`Split"' (Rundenzeit)
%  \item Messbereich 9 STD. 59 MIN. 59 SEK, 99/100 SEK
%\end{itemize*}

\item Einstellen der Funktionen

Dr"ucken Sie "`C"' und die Funktion schaltet von "`Lap"' (Rundenzeit)
in "`Split"' (Zwischenzeit) um.

\item Split (Zwischenzeit) Funktion

Sobald die Stoppuhr gestartet wurde, wird durch Dr"ucken von B die
verstrichene Zeit angezeigt. Durch nochmaliges Dr"ucken von B f"ahrt
die Uhr mit der Zeitmessung fort.

\item Lap (Rundenzeit) Funktion

Befindet sich die Uhr in der Lap (Zwischenzeit) Funktion, so wird
die verstrichene Zeit f"ur die vorhergehende Runde oder Etappe
angezeigt. Die Uhr stellt sich intern sofort auf Null und beginnt
mit der Messung f"ur die folgende Runde oder Etappe.

\item Der Gebrauch der Stoppuhr
%
\begin{itemize*}
  \item Dr"ucken Sie A und die Uhr beginnt sofort mit der
Zeitmessung.
  \item Dr"ucken Sie A nochmals, und die Uhr h"alt sofort an.
  \item Dr"ucken Sie B, um die Uhr
wieder in die Nullstellung zu bringen.

\end{itemize*}

\end{description*}



\section{Barometer}


Bei einigen Versuchen ist f"ur die Auswertung der aktuelle
Luftdruck\index{Luftdruck} zu bestimmen. Hierzu befinden sich in
den Praktikumsr"aumen Quecksilber-Barometer\index{Barometer}. Das
Aussehen ist in Bild~\ref{a:barometer} zu erkennen. Die Bedienung
ist in der dort beigef"ugten Beschreibung erkl"art. Wie kann der
Luftdruck noch gemessen werden?

\begin{figure}[htb]
% \centering
% \setcapindent{1em}
 \begin{captionbeside}{\label{a:barometer}Quecksilber-Barometer nach \person{Lambrecht}.}
 \includegraphics[width=4.5cm]{00_einl/barometer}
\end{captionbeside}
% \caption
\end{figure}

Die Messung des Luftdrucks erfolgt durch eine L"angenmessung der
Quecksilbers"aule, die dem augenblicklichen Luftdruck gegen"uber dem
\person{Toricelli}schen Vakuum das Gleichgewicht h"alt. Der
Nullpunkt der mm-Skala f"allt mit der Spitze des Ma"sstabes
zusammen. Der Nullpunkt des Nonius\index{Nonius} f"allt mit dem
unteren Rand des Ableserings zusammen.

\begin{enumerate}
    \item Mit der am unteren Ende der mm-Skala befindlichen R"andelmutter II
wird die Spitze so auf das Hg-Niveau im unteren Gef"a"s eingestellt,
dass sie mit Ihrem Spiegelbild ein X bildet.
    \item Mit der im oberen Teil der mm-Skala befindlichen R"andelmutter I
wird der am Nonius befestigte Ablesering so eingestellt, dass sein
unterer Rand mit der Kuppe der Hg-S"aule abschlie"st. F"ur das Auge
m"ussen die vor und hinter dem Glasrohr liegenden Kanten des
Metallringes zusammenfallen (parallaxenfrei) und die Hg-Kuppe
tangential ber"uhren.
\end{enumerate}


Die Einstellungen 1. und 2. sind einige Male zu wiederholen. Nach
beendeter Messung die Spitze bitte wieder mit R"andelmutter~II
ca.~10~mm oberhalb des Hg-Niveaus einstellen!


%\subsection{}

\section{Mobile Computer mit Oszilloskop}

F"ur das Praktikum stehen 6 mobile Computer (auf fahrbarem Tisch) zur
Verf"ugung. Sie sind mit WLAN ausgestattet, wodurch eine einfache
Daten"ubertragung auf den eigenen PC und auch ein Ausdrucken auf den
zentralen Druckern m"oglich ist. Ferner sind diese Computer mit einem
Digital-Oszilloskop verbunden, so dass man die Oszilloskop-Messungen
mit einem speziellen Programm auch direkt in den PC einlesen und als
Messdaten abspeichern kann.
%
\begin{figure}[htb]
% \centering
% \setcapindent{1em}
 \begin{captionbeside}{\label{a:pcmobil}Mobiler Computer mit WLAN und Oszilloskop.}
 \includegraphics[width=5cm]{00_einl/pcmobil}
\end{captionbeside}
% \caption
\end{figure}
%
Sie k"onnen einen solchen Computer gerne f"ur Ihren Versuch benutzen,
wenn es angebracht ist. Fragen Sie Ihren Betreuer rechtzeitig vor
dem Versuch danach.


\section{Diverse}

Dar"uberhinaus kommen auch weitere Instrumente zu Einsatz, z.B.
Schieblehre\index{Schieblehre},
Mikrometerschraube\index{Mikrometerschraube}, Waagen, Thermometer,
Mikroskope, Goniometer, Winkelskalen, Manometer, Hallsonde,
Stromintegratoren. Deren Einsatz ist meist selbstverst"andlich oder
selbsterkl"arend, bei Bedarf wird eine Betreuerin/ein Betreuer
gerne "uber die Verwendung eines Ger"ates informieren. Das
Galvanometer wird in einem eigenen Versuch behandelt, vereinzelt
kommen auch noch analoge Messger"ate f"ur Spannungen, Ladungen und
Str"ome zu Einsatz. Diese sind f"ur einzelne Versuche sogar
notwendig, da die Digital"=Multimeter dort "uberfordert w"aren.

\section{Anleitungen und Handb"ucher}

Wie bereits erw"ahnt, werden zu fast allen Ger"aten\index{Ger"ate}
detaillierte Anleitungen\index{Anleitungen} und
Handb"ucher\index{Handb"ucher} auf unseren
Webseiten\index{Webseiten}\index{Anleitungen!Webseiten} angeboten.
Teilweise findet man dort auch Schulungsb"ucher. Auch im
Handapparat des Praktikums sind alle verf"ugbaren Anleitungen
nochmals zusammengestellt. Falls Sie weitere Informationen
ben"otigen oder etwas nicht finden k"onnen, fragen Sie bitte Ihren
Betreuer/Ihre Betreuerin.
