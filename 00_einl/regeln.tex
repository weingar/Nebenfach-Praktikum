%%%%%%%%%%%%%%%%%%%%%%%%%%%%%%%%%%%%%%%%%%%%%%%%%%%%%%%%%%%%%%%%%%%

\chapter{Praktikumsordnung} \label{v:ordnung}

Das Physikalische Praktikum für Nebenfach Physik (B.Phy-NF.7004) besteht aus 14 Versuchen, die während der Vorlesungszeit eines Semesters durchgeführt werden. An einem Praktikumstag werden zwei Versuche durchgeführt, die Termine können dem Kursplan entnommen werden, der vor Beginn des Praktikums auf den Webseiten des Physikalischen Praktikums für Nebenfach Physik veröffentlicht wird. \\
\textbf{Versuche aus einem vorherigen Kurs können nicht angerechnet werden.}

Das Praktikum beginnt mit der obligatorischen Einführungsveranstaltung mit Sicherheitsbelehrung. Eine Teilnahme am Praktikum ohne Sicherheitsbelehrung ist nicht möglich. Sollte das Praktikum zum wiederholten Male belegt werden, ist die Einführungsveranstaltung erneut zu besuchen.

%**********************************************************************************************************************
%**********************************************************************************************************************

%\section{Ablauf des Praktikums}
%
%\begin{enumerate}
	%%
	%\item Das Praktikum erstreckt sich über ein Semester. Wird das Praktikum nicht in diesem Semester abgeschlossen, so muss es erneut belegt werden. Versuche aus dem vorherigen Kurs können nicht angerechnet werden.
	%%
	%\item Das Praktikum umfasst 14 Versuche. Die Termine können dem Kursplan entnommen werden, der vor Beginn des Praktikums auf den Webseiten des Physikalischen Praktikums für Nebenfach Physik veröffentlicht wird.
	%%
	%\item Wenn ein Versuch nicht testiert wurde, 
	%%
	%\item Das Praktikum beginnt mit der obligatorischen Einführungsveranstaltung mit Sicherheitsbelehrung. Eine Teilnahme am Praktikum ohne Sicherheitsbelehrung ist nicht möglich. Sollte das Praktikum zum wiederholten Male belegt werden, ist die Einführungsveranstaltung erneut zu besuchen.
	%%
	%\item Mit der Anmeldung zum Praktikum verpflichtet sich der/die Praktikant/in zur regelmäßigen Teilnahme.
%\end{enumerate}

%**********************************************************************************************************************
%**********************************************************************************************************************


%Es hat Vorteile, wenn eine Zweiergruppe alle Versuch des Praktikums gemeinsam durchf"uhrt. Da dies aufgrund terminlicher Bedingungen nicht immer m"oglich ist, ist es jedoch nicht zwingend.\\
%An jedem Arbeitstag werden von jeder Arbeitsgruppe je 2 Versuche durchgef"uhrt. Nach der Vorbesprechung w"ahlen die Zweiergruppen sich ihre 14 Versuche (7 Arbeitstage) aus den m"oglichen Terminen aus. Es m"ussen aus den Bereichen Mechanik (Versuche 1-10) und Elektrik (Versuche 11-20) jeweils mindestens 6 Versuche ausgesucht werden.

%\noindent
%Es gelten folgende organisatorische Regeln\index{Regeln} f"ur das Praktikum, die einen reibungslosen und effektiven Ablauf des Praktikums erm"oglichen sollen. Diese sind immer zu beachten.

\section{Voraussetzungen zur Teilnahme}

Voraussetzungen zur Teilnahme am Nebenfachpraktikum Physik sind die erfolgreiche Teilnahme an der Vorlesung "`Experimentalphysik I im Nebenfach"' (B.Phy-NF.7001 oder B.Phy-NF.7002), als auch die persönliche Teilnahme an der Einführungsveranstaltung des Praktikums. 

\section{Anmeldung und Gruppeneinteilung}

Zur Teilnahme am Praktikum melden Sie sich bitte auf StudIP für die Veranstaltung an.\\
Das Praktikum wird in festen Gruppen von bis zu 10 Studierenden durchgeführt. Bitte tragen Sie sich bei der Anmeldung gleich in eine der vordefinierten Gruppen ein.\\
Die Versuche werden in Zweiergruppen durchgeführt, welche zusammen ein Protokoll abgeben. Sollte eine/r der Teilnehmer/innen zur Versuchsdurchführung nicht erscheinen, so kann in Absprache mit dem Betreuer eine Dreiergruppe gebildet werden.

\section{Termine}

Der Termin für die Einführungsveranstaltung wird rechtzeitig in UniVZ und auf StudIP angekündigt. Typischerweise findet diese in der ersten Vorlesungswoche statt.

Die Versuche werden jeweils Mittwoch oder Freitag an 14~Uhr~c.t. durchgeführt. Ab etwa 13:30~Uhr sind die Tutoren in den Praktikumsräumen anzutreffen, damit Sie Ihre Protokolle abgeben oder den Versuch testiert bekommen können. 

\section{Versuchsvorbereitung}

Jede Praktikantin und jeder Praktikant muss sich genügend auf den durchzuführenden Versuch\index{Versuchsvorbereitung} vorbereiten. Die Durcharbeitung der Anleitung zum Praktikum und das Literaturstudium sind obligatorisch. \\
Der schriftliche Kurztest vor Versuchsbeginn (''Quicky'') dient dem Nachweis genügender fachlicher Grundkenntnisse und ausreichender Vorbereitung der Teilnehmer, er bildet die Grundlage für die Teilnahme an den Praktikumsversuchen.

Wer unvorbereitet zu einem Versuch kommt, riskiert, dass er/sie den Versuch an diesem Tag nicht durchf"uhren darf und einen Nachholtermin in Anspruch nehmen muss. Zur Hilfe bei der Vorbereitung sind in der jeweiligen Versuchsanleitung einige Fragen gestellt. 
In der ersten Stunde jeden Praktikumtages prüft der Assistent, ob Sie sich auf die beiden Versuche vorbereitet haben. Dazu k"onnen Sie in den ersten 30 Minuten Fragen zu möglichen Unklarheiten im Versuch an den Assistenten stellen. Anschließend werden Zettel ausgeteilt, auf denen jeweils auf einer Viertelseite 2 der Vorbereitungsfragen schriftlich beantwortetet werden müssen (Quickie). Die Zettel werden sofort durchgesehen und bleiben bei den Assistenten.\\ 
\textbf{Wer beide Fragen falsch beantwortet, wird vom Versuch ausgeschlossen.} Nur eine falsche Antwort führt nicht zum Ausschluss, wohl aber, wenn es schon das zweite Mal ist. 

%Die Betreuerinnen und Betreuer sind gehalten, vor jedem Versuch
%nochmals die Sicherheitsaspekte zum Versuch zu erl"autern und deren
%Verst"andnis zu "uberpr"ufen.

\section{Durchführung}

\textbf{Die Versuchsdurchführung beginnt um 14~Uhr~c.t. Erheblich verspätetes Erscheinen führt zum Ausschluss an der Durchführung des Versuchs.}\\
Vor dem Beginn der Messungen mache man sich mit den Apparaturen vertraut, d.h. wie sind welche Messgeräte anzuschließen, wie funktionieren sie, wie werden sie abgelesen, welche Fehler haben sie, bei welchen Apparaturen ist besondere Vorsicht geboten, usw. Insbesondere bei elektrischen Stromkreisen ist darauf zu achten, dass Strom und
Spannungen sehr gefährlich sein können! Messgeräte sind vor dem Gebrauch - sofern möglich - auf Funktionsfähigkeit zu testen und auf den richtigen Messbereich einzustellen. Manchmal ist es hilfreich, Schalter und Messgeräte durch Zettel oder ähnliches (z.B. "`Post-it"') zu beschriften, um Irrtümer zu vermeiden. Bei elektrischen Schaltungen ist nach dem Aufbau zunächst der Assistent zu benachrichtigen, erst mit dessen Zustimmung wird die Stromversorgung eingeschaltet! Messkurven sind während der Versuchsdurchführung grafisch darzustellen (Millimeterpapier nicht vergessen!). Jeder ist selbst dafür verantwortlich, dass alle benötigten Daten richtig und vollständig gemessen werden. Bitte denken Sie nach, ob die gemessenen Werte sinnvoll sind!

\textbf{Während der Versuchsdurchführung ist ein Messprotokoll\index{Messprotokoll} \emph{dokumentenecht} anzufertigen.} \\
Es darf also nur Kugelschreiber oder Tusche verwendet werden (kein Bleistift). Es wird nichts radiert, sondern nur gestrichen. Datum und Mitarbeiter angeben, Seiten nummerieren. Die Versuchsdurchführung muss nachvollziehbar sein. \\
Darauf müssen folgende Informationen zu finden sein:
\begin{itemize}
	\item Name des Versuchs
	\item Datum der Durchführung 
	\item Namen aller beteiligten Praktikanten 
	\item die gemessenen Werte mit Fehlerangabe.
\end{itemize} 
Das Protokoll muss leserlich sein und sollte übersichtlich gestaltet sein, z.B. durch einleitende Sätze, was mit den dann folgenden Messwerten bestimmt werden soll. Jeder Messwert muss eindeutig mit der gemessenen Größe in Verbindung gebracht werden können, ggf. sollten Skizzen angefertigt werden. Es müssen die tatsächlich gemessenen (direkt abgelesenen) Werte aufgeschrieben werden, zusätzlich ausgerechnete Werte (z.B. Differenzen) dürfen nur zusätzlich aufgeschrieben werden. Dies soll (Kopf-)Rechen- und Denkfehlern
vorbeugen. Zu jedem Messwert ist die Einheit zu notieren! Zu jedem Messwert ist ein Fehler zu notieren (Ablesefehler, Gerätefehler, Schwankungen).

\textbf{Am Ende des Versuchs wird das Messprotokoll vom Assistenten testiert, ansonsten ist es ungültig!}

Sinnigerweise sollte der Versuch erst hiernach abgebaut werden, da u.U. bestimmte Dinge erneut gemessen werden m"ussen oder Daten fehlen.\\ 
{\bf Jede(r) Student/in sollte ein eigenes testiertes Messprotokoll haben.}\\
Erst nachdem der/die Betreuer/in die Werte kontrolliert, das Versuchsprotokoll testiert (Versuchs"=Testat) \index{Versuchs-Testat} und dies in die Karteikarte eingetragen hat,
ist der Versuch abzubauen und alles aufzuräumen. Weiterhin erteilt der Assistent nach Abschluss eines Versuches auch auf der Karteikarte durch Unterschrift ein Vortestat. Dieses dient gleichzeitig als Teilnahme-Beweis.

Nach Beendigung eines Versuchstages\index{Versuchsende}\index{Aufräumen} sind alle Versuche, Geräte und Räume wieder in den ursprünglichen Zustand zu versetzen. Messgeräte, Kabel und Stoppuhren sind wieder an die vorgesehenen Stellen zu bringen. \textbf{Defekte sind sofort einem/r Betreuer/in zu melden.} Flaschen und sonstige Abfälle sind bitte zu entsorgen.


\section{Protokolle}

%Beachten Sie zum Protokoll auch die Hinweise in
%Kapitel~\ref{c:protokoll}. 
Im Protokollkopf müssen der Name des Versuchs und das Datum der Durchführung stehen. Bei einem in Eigenarbeit geschriebenen Protokoll\index{Protokoll} steht als "`Praktikant"' der Name des Praktikanten und unter "`Mitarbeiter"' die Namen der übrigen an der Versuchsdurchführung beteiligten Personen. Bitte auch den Namen des/der Betreuers/in und die eigene E-Mail Adresse im Kopf angeben. 
%Jeder
%Praktikant braucht f"ur die Unterschrift auf der Karteikarte ein
%eigenes Protokoll, welches auch unterschrieben werden
%soll. 

Das Protokoll muss leserlich und übersichtlich gestaltet sein. Es ist für sich eigenständig, also keine Verweise auf die Praktikumsanleitung.\footnote{Sätze wie "`Versuchsdurchführung s. Praktikumsanleitung"' sind überflüssig.} Aus dem Protokoll alleine muss klar verständlich sein, was das Ziel des Versuchs war, wie Sie den Versuch durchgeführt haben und was das Ergebnis des Versuchs ist. Es muss aus dem Protokoll ersichtlich sein, wie die Auswertung aufgebaut ist, d.h. welche Werte gemessen wurden und was man aus diesen Messwerten bestimmen möchte.

Im eigentlichen Teil der Auswertung sind deutlich und nachvollziehbar die einzelnen Auswertungsschritte aufzuschreiben. Einleitende Sätze, was gemessen wurde, und was daraus berechnet wird, sind obligatorisch. Ergebnisse sind deutlich zu kennzeichnen (Rahmen, farbiges Markieren, größere Schrift, usw.). Was sind Zwischen- oder Hilfsergebnisse, was sind Endergebnisse? Alle benutzten Formeln müssen beschrieben sein bzw. deren Herleitung klar werden. Es ist auf eine durchgehend eindeutige Variablendefinition und Variablenbenutzung im Protokoll zu achten. Alle Variablen in den Funktionen sind zu benennen bzw. zu definieren. Die Fehlerrechnung muss nachvollziehbar beschrieben werden (einfacher Mittelwert oder gewichtetes Mittel, ggf. Formel der Fehlerfortpflanzung angeben). Fehler sind sinnvoll anzugeben! Alle Werte haben Einheiten, alle Grafen eine Beschriftung! Bei Vergleich mit Literaturwerten: Woher kommen die Werte (Quellenangabe)? Eine Diskussion der Ergebnisse und der Fehler ist obligatorisch. Dazu muss man sich natürlich vorher die Frage stellen, ob das, was man berechnet hat, ein sinnvolles Ergebnis ist.

Bei der Abgabe des Protokolls muss das dazugehörige original unterschriebene Messprotokoll mit abgegeben werden. Protokolle müssen in geeigneter Form zusammengeheftet sein (einfache Mappe oder Heftung reichen vollkommen), lose Blätter werden nicht akzeptiert.
\xhintbox{Wird ein/e Praktikant/in auf die Auswertung seines/ihres eigenen Protokolls angesprochen und kann keine Auskunft zu den gemachten Rechnungen geben, so gilt das Protokoll als nicht selbständig erstellt und wird nicht testiert.} 
Protokolle mit einer Auswertung, die nicht auf den eigenen Messdaten basieren, bei denen die Messdaten nachträglich geändert wurden oder bei denen die Liste der am Versuch beteiligten Personen erweitert wurde, gelten als Täuschungsversuch/Urkundenfälschung und werden entsprechend geahndet.
%
%\begin{definition}[Abgabefrist der Protokolle]
% Die Abgabefrist f"ur ein Protokoll betr"agt 2~Wochen nach Durchf"uhrung
% des Versuchs.
%\end{definition}
%

\xhintbox{Ein vollständiges Protokoll muss zum nächsten Praktikumstermin, also innerhalb einer Woche nach der Versuchsdurchführung, abgegeben werden.} 
%Wird das Protokoll {\bf nicht innerhalb von 2~Wochen} abgegeben, {\bf verf"allt das Vortestat und der Versuch muss wiederholt werden.} 

Ein Protokoll gilt nur dann als vollständig, wenn es oben genannte Bedingungen erfüllt. Insbesondere gilt es als nicht vollständig, wenn es außer dem Endergebnis keine Zwischenergebnisse enthält, die den Rechenweg und die Werte nachvollziehbar machen. Sollte das Protokoll für Korrekturen ohne Testat zurückgegeben werden, so gilt erneut die 1~Wochen-Frist ab dem Tag der Rückgabe. Die Korrekturen sind (z.B. als Anhang) zusammen mit dem vollständigen ursprünglichen Protokoll abzugeben. Sie haben die Möglichkeit, jedes Protokoll genau einmal zu überarbeiten. Wird das Protokoll nach der Korrektur immer noch nicht testiert, so gilt der Versuch als nicht bestanden und es muss ein Versuch (nicht notwendigerweise derselbe) nachgeholt werden.

Auch für die Rückgabe der Protokolle durch die Assistenten soll die 1-Wochen-Frist eingehalten werden. Das bedeutet, dass nach spätestens vier Wochen feststeht, ob der Versuch testiert wird, oder nicht.

Es ist zu beachten, dass es für jedes Semester einen Termin gibt, ab dem alle bis zu diesem Tag nicht testierten Protokolle nicht mehr angenommen und testiert werden. Dies ist in der Regel der 30.04. für das vergangene Wintersemester und der 31.10. für das vergangene Sommersemester. Dies ist erforderlich, da auch die Betreuer Fluktuationen
unterworfen sind und so der/die Betreuer/in später eventuell Göttingen schon verlassen hat.

\section{Nachholtermine}

\textbf{Es stehen genau zwei Nachholtermine\index{Nachholtermin} für versäumte oder nicht testierte Versuche zur Verfügung.} Sollten Sie aus triftigen Gründen (schwere Erkrankung, o.Ä.) mehr als zwei Versuche versäumen, melden Sie sich bitte umgehend bei der Praktikumsleitung. \textbf{Wenn aus anderen Gründen mehr als zwei Versuche nicht durchgeführt oder testiert werden, gilt das Praktikum als nicht bestanden.}\\
Bitte sorgen Sie dafür, dass ggf. ein/e Partner/in für die Durchführung des Nachholversuchs zur Verfügung steht. Die Nachholtermine finden in den beiden Wochen nach Ende der regulären Versuche statt.
%Um einen Nachholtermin zu bekommen, melden Sie sich bitte z"ugig beim Praktikumsleiter. Es kann vorkommen, dass ein Nachholtermin erst im n"achsten Semster m"oglich ist.


\section{Karteikarte}

Während der Vorbesprechung zum Praktikum werden die Karteikarten\index{Karteikarte} ausgeteilt, die die Praktikantin oder der Praktikant bis zum Abschluss des Praktikums behält. Auf dieser werden dann die Versuchs-Durchführung und Protokoll-Testate vom Assistenten eingetragen und mit Unterschrift bestätigt. Diese Karteikarte ist der Nachweis für die
Gesamtleistung im Praktikum und damit Zulassung zum Noteneintrag in FlexNow, also bitte nicht verlieren.

Nach Abschluss des Praktikums geben Sie bitte Ihre vollständig ausgefüllte Karteikarte bei der Praktikumsleitung ab.


\section{Leistungsnachweis}

Die Gesamtleistung besteht aus 14 durchgeführten und komplett testierten Versuchen. \\
Das Praktikum ist unbenotet und wird nur mit Bestanden oder Nicht Bestanden in FlexNow eingetragen. In besonderen Fällen kann auch ein Schein ausgestellt werden.
