%%%%%%%%%%%%%%%%%%%%%%%%%%%%%%%%%%%%%%%%%%%%%%%%%%%%%%%%%%%%%%%%%%%

\chapter{Sicherheit im Praktikum} \label{v:sicherheit}

Nehmen Sie Ihre eigene Sicherheit\index{Sicherheit} und die Ihrer
Kommilitonen sehr wichtig. Auch im Praktikum gibt es viele
Gefahrenquellen (Spannung, Strom, Wasserdampf, Kochplatten, Radioaktivit"at, Druck, Vakuum, \emph{et
cetera}). Bitte machen Sie sich dies immer bewusst und handeln Sie
besonnen. \emph{Immer zuerst denken, dann handeln.} Sind Sie sich
"uber Gefahren, Prozeduren und Vorgehensweisen im Unklaren, wenden
Sie sich bitte zuerst an eine betreuende Person. Generell sind alle
Unfallverh"utungsvorschriften \textsc{UVV}\index{UVV} zu beachten.

Aus Sicherheitsgr"unden m"ussen w"ahrend des Aufenthaltes in den R"aumen
des Praktikums mindestens zwei Studierende anwesend und eine
betreuende Person in unmittelbarer N"ahe sein, damit bei einem Unfall
f"ur eine schnelle und wirksame Erste Hilfe\index{Erste Hilfe}
gesorgt werden kann. F"ur dringende Notf"alle\index{Notf"alle} sind bei
den Telefonen die Notrufnummern\index{Notrufnummern} 110 und 112
freigeschaltet.

Folgende Sicherheitsbestimmungen fassen die f"ur das Praktikum
wichtigsten Punkte zusammen und erheben keinen Anspruch auf
Vollst"andigkeit. Auf Wunsch k"onnen die einschl"agigen
Sicherheitsbestimmungen eingesehen werden.

In den Labors und Praktikumsr"aumen darf weder geraucht noch
gegessen oder getrunken werden.

Im Falle eines Feuers\index{Feuer} ist unverz"uglich eine
betreuende Person zu verst"andigen. Feuerl"oscher befinden sich in
den Fluren. Die Feuerwehr ist unter der Notrufnummer 112 zu
erreichen. Die Feuermelder sind im Notfall auf dem Weg aus dem
Geb"aude zu bet"atigen. Bei einem Feueralarm ist das Geb"aude auf den
gekennzeichneten Fluchtwegen z"ugig, aber ruhig zu verlassen. Man
muss sich am entsprechenden Sammelpunkt\index{Sammelpunkt} vor dem
Geb"aude einfinden (Mitte des Parkplatzes), damit festgestellt
werden kann, ob alle Personen das Praktikum verlassen haben. Es
gilt der generelle Grundsatz "`Personenschutz geht vor
Sachschutz"'.

Auch bei Unf"allen\index{Unfall} oder Verletzungen ist sofort eine
betreuende Person zu benachrichtigen. Ein Verbandskasten ist in
den Praktikumsr"aumen vorhanden. Die Notrufnummern 110 und 112 sind
freigeschaltet. Ein Notfallblatt mit entsprechenden Telefonnummern
ist an verschiedenen Stellen ausgeh"angt. Die betreuende Person
muss Unf"alle und Verletzungen sofort weitermelden.

Werden Sch"aden an einer Apparatur oder an einem Ger"at
festgestellt, d"urfen diese nicht weiter verwendet werden. Bitte
sofort eine betreuende Person benachrichtigen.

Bananenstecker geh"oren keinesfalls in Steckdosen! Bei Aufbau und
Arbeiten an elektrischen Schaltungen ist die Schaltung zuerst in
einen spannungsfreien Zustand zu bringen, d.h. Netzger"at nicht einschalten. Schaltungen sind vor deren
Einsatz durch eine betreuende Person zu kontrollieren.
Schwingkreise, Spulen und Kondensatoren k"onnen auch nach Abschalten
der Spannung noch eine l"angere Zeit Spannung f"uhren. Sollte ein
elektrischer Unfall passieren, ist sofort der NOT"=AUS
Schalter\footnote{Der NOT-AUS Schalter (Roter Knopf auf gelbem
Grund), welcher den ganzen Raum stromlos schaltet, befindet sich
immer direkt neben der Raumt"ur. Bei Gefahr einfach eindr"ucken.} zu
bet"atigen und dann Hilfe zu leisten. Danach sofort eine betreuende
Person verst"andigen oder weitere Hilfe veranlassen.

Beim Umgang mit Chemikalien und anderen
Gefahrstoffen\index{Gefahrstoffverordnung} sind die
Gefahrstoffverordnung und weitere Vorschriften zu beachten. Beim
Umgang mit radioaktiven Stoffen und ionisierender Strahlung ist die
Strahlenschutzverordnung\index{Strahlenschutzverordnung StrSchV}
(StrSchV) und R"ontgenverordnung\index{R"ontgenverordnung R"oV} (R"oV)
zu beachten. Beide liegen in den jeweiligen R"aumen aus. Da diese
Gesetze auch besondere Regeln f"ur Schwangere enthalten, m"ussen
Schwangerschaften dem Praktikumsleiter gemeldet werden. Am Versuch
"`Radioaktivit"at"' darf dann nicht teilgenommen werden.

%Der Umgang und das Hantieren mit tiefkalten Gasen (fl"ussiger
%Stickstoff) darf nur durch eine betreuende Person erfolgen.
%Fl"ussiger Stickstoff kann schwerwiegende Verbrennungen verursachen
%und durch die Verdr"angung des Sauerstoffs auch zu Sauerstoffmangel
%bis hin zum Ersticken f"uhren. Deshalb ist beim Umgang mit fl"ussigem
%Stickstoff immer f"ur ausreichende L"uftung zu sorgen.

Kochendes Wasser, Wasserdampf und hei"se Kochplatten stellen ein
Gefahrenpotenzial f"ur schwere Verbrennungen dar. Unter Druck
stehender Wasserdampf (Versuch "`Dampfdruck"') ist noch eine Stufe
gef"ahrlicher.

%Laserlicht\index{Laser} ist "au"serst intensiv und kann bei direkter
%Einstrahlung in das Auge zu Sch"adigungen, bis hin zur Erblindung,
%f"uhren. Im Praktikum werden Laser der Klasse~2 verwendet. Gehen
%Sie bitte entsprechend vorsichtig damit um.

%Druckgasflaschen stehen unter sehr hohem Druck, sie sind nur durch
%eine betreuende Person zu benutzen. Die Bet"atigung
%der zentralen Gasarmaturen in den Praktikumsr"aumen erfolgt
%ausschlie"slich durch den Praktikumstechniker.

Die Betreuerinnen und Betreuer sind gehalten, vor jedem Versuch
nochmals die Sicherheitsaspekte zum Versuch zu erl"autern und deren
Verst"andnis zu "uberpr"ufen.

\emph{Generell gilt: Alle Unf"alle und Verletzungen sind sofort einer
betreuenden Person zu melden, die dann das weitere veranlassen und
den Unfall weitermelden muss.}


%%%%%%%%%%%%%%%%%%%%%%%%%%%%%%%%%%%%%%%%%%%%%%%%%%%%%%%%%%%%%%%%%%%%
