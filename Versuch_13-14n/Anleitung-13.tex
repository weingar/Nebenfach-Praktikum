\chapter{Thermoelement}
\label{v:11}

Ein Thermoelement wandelt Wärme in elektrische Energie um. In diesem Versuch soll das Thermoelement zur Messung einer Temperatur benutzt werden.

%------------------------------------------------
\section{Stichworte}
%------------------------------------------------

Metallbindung; Austrittsarbeit; Fermienergie; Kontaktspannung; Thermospannung; Thermoelement; Wärmebad.
%
%------------------------------------------------
\section{Literatur}
%------------------------------------------------

Gehrtsen, Kapitel 6.6.1, 8.1.1, 14.1.5, 14.3.1/2\\
R. Pelster, R. Pieper, I. Hüttl, \textit{Thermospannungen - Viel genutzt und fast immer falsch erklärt!}, PhyDid 1/4 (2005)

%------------------------------------------------
\section{Anwendungsbeispiele}
%------------------------------------------------

Thermoelemente werden zur Temperaturmessung in vielen verschiedenen Umgebungen benutzt, z. Bsp. in Flüssigkeiten verschiedenster Ph-Werte, in industriellen Anwendungen mit extremen Temperaturen und Atmosphären Zusammensetzungen. Sie werden aber auch zur hochgenauen Temperaturmessung im Labor verwendet.

%------------------------------------------------
\section{Theoretischer Hintergrund}
%------------------------------------------------

\textit{Thomas Johann Seebeck} lötete 1821 zwei Drähte aus verschiedenen Metallen zu einem Ring zusammen und fand, dass eine Temperaturdifferenz zwischen den beiden Lötstellen einen Strom antreibt.\\
In vielen Lehrbüchern, auch für Experimentalphysiker, sowie in älteren Versionen dieses Skripts, wird dieses Phänomen fälschlicherweise durch die unterschiedliche Auslösearbeit der Elektronen in den beiden Metallen erklärt. In Wirklichkeit liefert diese jedoch nur einen kleinen Beitrag zum gesamten Effekt. Hier legen wir eine vereinfachte, jedoch komplette Behandlung, dar, wie sie von Pelster et al. vorgestellt wurde.

\subsection{Thermodiffusion}

Betrachten wir zunächst einen Metallstab bei konstanter Temperatur. Die freien Ladungsträger, in diesem Fall Elektronen, sind homogen im Stab verteilt und führen eine ungeordnete Wärmebewegung (s. Browne'sche Bewegung, Maxwell Verteilung). Der Betrag der mittleren Geschwindigkeit hängt dabei von der Temperatur ab: Je wärmer der Körper, desto schneller bewegen sich die Elektronen.\\
Haben die Stabenden nun unterschiedliche Temperaturen, so ist die Geschwindigkeit der Elektronen am heißen Ende höher als am kalten Ende. Dies führt zu einer gerichteten Netto-Bewegung der Elektronen vom warmen zum kalten Ende des Stabes (vgl. Diffusion). Passenderweise nennt man diese Bewegung Thermodiffusion. Sie führt dazu, dass sich das kalte Ende des Stabes gegenüber dem warmen negativ auflädt. Die Aufladung wächst solange an, bis das sich aufbauende elektrische Feld stark genug ist, die Diffusion zu kompensieren: Der Diffusionsstrom kommt zum Erliegen.\\
Die Spannung zwischen den beiden Stabenden, die sogenannte Thermodiffusionsspannung, ist in erster Näherung proportional zur Temperaturdifferenz zwischen den Stabenden:
\begin{equation}
	U_{TD} = -Q\cdot \Delta T
\end{equation}
Der Proportionalitätsfaktor $Q$, der Seebeck-Koeffizient, ist materialspezifisch, d.h. es ergeben sich unterschiedlich große Spannungen je nachdem, welches Metall (oder Halbleiter) man betrachtet.

\subsection{Thermoelement}

Fügt man zwei Teilstücke aus unterschiedlichen Materialien zu einem Ring zusammen, so ergibt sich eine Schleife mit einem thermoelektrischen Kreisstrom. Beim Thermoelement ist ein Voltmeter in die Schleife geschaltet, so dass praktisch kein Strom fließt. Das Voltmeter zeigt dann die Thermospannung, i.e. die Differenz der beiden Thermodiffusionsspannungen:
\begin{equation}
	U_{Thermo} = U^A_{TD} - U^B_{TD}.
\end{equation}
Entfernt man das Voltmeter und schließt den Kreis, so treibt diese Thermospannung einen Kreisstrom an. Dabei bestimmt das Material mit der größeren Thermodiffusionsspannung die Stromrichtung (ähnlich wie in einem Stromkreis mit zwei ungleichen gegeneinander geschalteten Batterien).

\begin{tutorhint}
%------------------------------------------------
\section{Fragen zur Vorbereitung}
%------------------------------------------------

\begin{enumerate}
 %
 \item Was soll heute im Praktikum gemessen werden? Warum?
 %
 \item Welche Kraft wirkt auf eine elektrische Ladung in einem elektrischen Feld? Wovon hängt die Kraft ab?
 %
 \item Welche Bedeutung hat die Austrittsarbeit? Ist sie bei allen Metallen gleich?
 %
 \item Warum tritt eine Kontaktspannung auf, wenn sich zwei unterschiedliche Metalle berühren?
 %
 \item Wie ist ein Thermoelement aufgebaut? Wie erzeugt es die Thermospannung?
 %
\end{enumerate}
\end{tutorhint}

%------------------------------------------------
\section{Durchführung} 
%------------------------------------------------

\begin{hint}
Bitte achten Sie darauf, dass keine der Kabel mit der heißen Kochplatte in Berührung kommen!
\end{hint}

\begin{enumerate}
 %
 \item Tauchen Sie die Lötstellen des Thermoelements in Eiswasser. Messen Sie die Thermospannung.
 %
 \item Erhitzen Sie nun das Wasserbad langsam bis das Wasser siedet. Messen Sie während des Erhitzens die Thermospannung f"ur Temperaturen in Schritten von $5^{\circ}$\,C.
 
 \noindent
 \textbf{Beachten Sie:} Die Herdplatte wird kurz auf Stufe 12 betrieben. Der komplette Versuch ist schon fertig aufgebaut, es muss nichts mehr angeschlossen werden!\\
 Stellen Sie das Thermometer nicht auf den Gef"a{\ss}boden! Die Kabel d"urfen die Herdplatte nicht ber"uhren.\\
 Eine langsamere Erw"armung des Wassers erlaubt eine genauere Messung.
 %
 \item Setzen Sie die Messung fort, während das Wasser wieder abkühlt. \\
 Der Abkühlungsprozess kann durch Zugabe von Eisstücken beschleunigt werden.
 
 \noindent
 \textbf{Beachten Sie:} Das Wasser muss zur gleichmäßigen Temperaturänderung mit einem Glasstab in Bewegung gehalten werden.
\end{enumerate}
\begin{figure}[h!]
	\centering
		%\includegraphics[width=0.15\textwidth]{Versuch_11-12/Abbildungen/Messung.JPG}
		\includegraphics[width=0.75\textwidth]{Abbildungen/Thermoelement_Aufbau2.JPG}
	\caption{Messaufbau}
	\label{fig:Messung}
\end{figure}
%------------------------------------------------
\section{Auswertung} 
%------------------------------------------------
\etodo{Musterauswertung}

\begin{hint}
	Bitte fertigen Sie die Graphen in der folgenden Auswertung per Hand auf Millimeterpapier an.
\end{hint}

\begin{enumerate}
 %
 \item Stellen Sie die Thermospannung in mV als Funktion der Temperatur $T$ grafisch dar. Tragen Sie die Kurven für Aufwärmung und Abkühlung des Wasserbades getrennt auf. \label{Aufg:a}
 %
 \item Lesen Sie aus der Auftragung die Empfindlichkeit des Thermoelementes ab. Schätzen Sie ihren Fehler aus den Grenzgeraden ab.\\
 Alle Geraden, auch die Grenzgeraden, müssen dabei durch den Ursprung gehen. Wieso?
	\etutorhint{Mit Empfindlichkeit des Thermoelementes ist natürlich die Spannungsänderung pro Temperaturänderung gemeint, also die Steigung der Kalibrationsgeraden.}
 %
 \item Berechnen Sie den gewichteten Mittelwert der Empfindlichkeit des Thermoelementes. Die Formeln finden Sie im Skript.
 %
 \item Warum liegen die Geraden aus Aufgabe \ref{Aufg:a} nicht aufeinander?
	\etodo{Eine mögliche Erklärung ist die thermische Kapazität des Thermoelements selbst, i.e. Temperatur des Thermoelements ist nicht gleich Wassertemperatur. \\
		Gibt es andere Erklärungen?}
 %
 \item Diskutieren Sie die Fehlerquellen in Ihrer Messung.
\end{enumerate}