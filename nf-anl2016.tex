%\pdfoutput=0
%%%%%%%%%%%%%%%%%%%%%%%%%%%%%%%%%%%%%%%%%%%%%%%%%%%%%%%%%%%%%%%%%%%
%%\documentclass[english,ngerman,headsepline,noparindent,greybox, chapterprefix=true]{bgteubner}
%
%\usepackage{graphicx}
%\usepackage{dcolumn}
%\usepackage[gen]{eurosym}
%%\usepackage{miller}
%\usepackage{contour}
%%\usepackage{fixme}
%\usepackage{nicefrac}
%\usepackage{url}
%%\usepackage{overpic}
%\usepackage{verbatim}
%%\usepackage{url}
%%\usepackage{tocloft}
%%\usepackage{cite}
%\usepackage{babelbib}
%\usepackage{psfrag}
%\usepackage[utf8]{inputenc}
%\usepackage{capt-of}
%\usepackage{ngerman}
%%\usepackage[ps2pdf,bookmarks=true,bookmarksopen=false,bookmarksnumbered=true]{hyperref}
%%\usepackage{bookmark}
%\usepackage{amsmath}
%\usepackage{xcolor}
%%**** tikz package, �bernommen von Johannes Agricola aus dem EP Skript********
%\usepackage{tikz}
%\usepackage{pgfplots}
%\usetikzlibrary{calc, babel, intersections}
%\usetikzlibrary{arrows}
%\usetikzlibrary{decorations.markings}
%\usetikzlibrary{decorations.pathreplacing}
%\usepackage[european resistors]{circuitikz}

%%%%%%%%%%%%%%%%%%%%%%%%%%%%%%%%%%%%%%%%%%%%%%%%%%%%%%%%%%%%%%%%%%
% �bernommen vom EP Skript
%%%%%%%%%%%%%%%%%%%%%%%%%%%%%%%%%%%%%%%%%%%%%%%%%%%%%%%%%%%%%%%%%%
\documentclass[11pt, twoside, a4paper]{book}

\usepackage{graphicx}
\usepackage[utf8]{inputenc}
\usepackage{ngerman}
%\usepackage{lineno}
\usepackage{verbatim}
%\usepackage{comment}
\usepackage[squaren]{SIunits}
\usepackage{amsmath}
\usepackage{amsfonts}
\usepackage{amssymb}
\usepackage{enumitem}
\usepackage{fancyhdr}
\usepackage{textcomp}
\usepackage{subcaption}
\usepackage[noadjust]{marginnote}
\usepackage{hyperref}
\usepackage{tikz}
\usepackage{pgfplots}
\usepackage{nicefrac}
\usepackage{framed}
\usetikzlibrary{calc,intersections}
\usetikzlibrary{arrows}
\usetikzlibrary{decorations.markings}
\usetikzlibrary{decorations.pathreplacing}
\include{tikz/shape-dff.tex}
\makeatletter

\pgfdeclareshape{halfadder}{
  % The 'minimum width' and 'minimum height' keys, not the content, determine
  % the size
  \savedanchor\northeast{%
    \pgfmathsetlength\pgf@x{\pgfshapeminwidth}%
    \pgfmathsetlength\pgf@y{\pgfshapeminheight}%
    \pgf@x=0.5\pgf@x
    \pgf@y=0.5\pgf@y
  }
  % This is redundant, but makes some things easier:
  \savedanchor\southwest{%
    \pgfmathsetlength\pgf@x{\pgfshapeminwidth}%
    \pgfmathsetlength\pgf@y{\pgfshapeminheight}%
    \pgf@x=-0.5\pgf@x
    \pgf@y=-0.5\pgf@y
  }
  % Inherit from rectangle
  \inheritanchorborder[from=rectangle]

  % Define same anchor a normal rectangle has
  \anchor{center}{\pgfpointorigin}
  \anchor{north}{\northeast \pgf@x=0pt}
  \anchor{east}{\northeast \pgf@y=0pt}
  \anchor{south}{\southwest \pgf@x=0pt}
  \anchor{west}{\southwest \pgf@y=0pt}
  \anchor{north east}{\northeast}
  \anchor{north west}{\northeast \pgf@x=-\pgf@x}
  \anchor{south west}{\southwest}
  \anchor{south east}{\southwest \pgf@x=-\pgf@x}
  \anchor{text}{
    \pgfpointorigin
    \advance\pgf@x by -.5\wd\pgfnodeparttextbox%
    \advance\pgf@y by -.5\ht\pgfnodeparttextbox%
    \advance\pgf@y by +.5\dp\pgfnodeparttextbox%
  }

  % Define anchors for signal ports
  \anchor{A}{
    \pgf@process{\northeast}%
    \pgf@x=-1\pgf@x%
    \pgf@y=.5\pgf@y%
  }
  \anchor{B}{
    \pgf@process{\northeast}%
    \pgf@x=-1\pgf@x%
    \pgf@y=-.5\pgf@y%
  }
  \anchor{S}{
    \pgf@process{\northeast}%
    \pgf@y=.5\pgf@y%
  }
  \anchor{C}{
    \pgf@process{\northeast}%
    \pgf@y=-0.5\pgf@y%
  }

  % Draw the rectangle box and the port labels
  \backgroundpath{
    % Rectangle box
    \pgfpathrectanglecorners{\southwest}{\northeast}
    % Angle (>) for clock input

    % Draw port labels
    \begingroup
    \pgf@anchor@halfadder@A
    \pgftext[left,base,at={\pgfpoint{\pgf@x}{\pgf@y}},x=\pgfshapeinnerxsep]{\raisebox{-0.75ex}{A}}
    
    \pgf@anchor@halfadder@B
    \pgftext[left,base,at={\pgfpoint{\pgf@x}{\pgf@y}},x=\pgfshapeinnerxsep]{\raisebox{-0.75ex}{B}}


    \pgf@anchor@halfadder@S
    \pgftext[right,base,at={\pgfpoint{\pgf@x}{\pgf@y}},x=-\pgfshapeinnerxsep]{\raisebox{-.75ex}{S}}

    \pgf@anchor@halfadder@C
    \pgftext[right,base,at={\pgfpoint{\pgf@x}{\pgf@y}},x=-\pgfshapeinnerxsep]{\raisebox{-.75ex}{C}}

    \endgroup
  }
}

% Define default style for this node
\tikzset{every halfadder node/.style={draw,minimum width=1.5cm,minimum 
  height=1.5cm,thick,inner sep=1mm,outer sep=0pt,cap=round}}

\makeatother

\makeatletter

\pgfdeclareshape{fulladder}{
  % The 'minimum width' and 'minimum height' keys, not the content, determine
  % the size
  \savedanchor\northeast{%
    \pgfmathsetlength\pgf@x{\pgfshapeminwidth}%
    \pgfmathsetlength\pgf@y{\pgfshapeminheight}%
    \pgf@x=0.5\pgf@x
    \pgf@y=0.5\pgf@y
  }
  % This is redundant, but makes some things easier:
  \savedanchor\southwest{%
    \pgfmathsetlength\pgf@x{\pgfshapeminwidth}%
    \pgfmathsetlength\pgf@y{\pgfshapeminheight}%
    \pgf@x=-0.5\pgf@x
    \pgf@y=-0.5\pgf@y
  }
  % Inherit from rectangle
  \inheritanchorborder[from=rectangle]

  % Define same anchor a normal rectangle has
  \anchor{center}{\pgfpointorigin}
  \anchor{north}{\northeast \pgf@x=0pt}
  \anchor{east}{\northeast \pgf@y=0pt}
  \anchor{south}{\southwest \pgf@x=0pt}
  \anchor{west}{\southwest \pgf@y=0pt}
  \anchor{north east}{\northeast}
  \anchor{north west}{\northeast \pgf@x=-\pgf@x}
  \anchor{south west}{\southwest}
  \anchor{south east}{\southwest \pgf@x=-\pgf@x}
  \anchor{text}{
    \pgfpointorigin
    \advance\pgf@x by -.5\wd\pgfnodeparttextbox%
    \advance\pgf@y by -.5\ht\pgfnodeparttextbox%
    \advance\pgf@y by +.5\dp\pgfnodeparttextbox%
  }

  % Define anchors for signal ports
  \anchor{A}{
    \pgf@process{\northeast}%
    \pgf@x=-1\pgf@x%
    \pgf@y=.66\pgf@y%
  }
  \anchor{B}{
    \pgf@process{\northeast}%
    \pgf@x=-1\pgf@x%
    \pgf@y=0\pgf@y%
  }
  \anchor{Cin}{
    \pgf@process{\northeast}%
    \pgf@x=-1\pgf@x%
    \pgf@y=-.66\pgf@y%
  }
  \anchor{S}{
    \pgf@process{\northeast}%
    \pgf@y=.66\pgf@y%
  }
  \anchor{C}{
    \pgf@process{\northeast}%
    \pgf@y=-.66\pgf@y%
  }

  % Draw the rectangle box and the port labels
  \backgroundpath{
    % Rectangle box
    \pgfpathrectanglecorners{\southwest}{\northeast}
    % Angle (>) for clock input

    % Draw port labels
    \begingroup
    \pgf@anchor@fulladder@A
    \pgftext[left,base,at={\pgfpoint{\pgf@x}{\pgf@y}},x=\pgfshapeinnerxsep]{\raisebox{-0.75ex}{A}}
    
    \pgf@anchor@fulladder@B
    \pgftext[left,base,at={\pgfpoint{\pgf@x}{\pgf@y}},x=\pgfshapeinnerxsep]{\raisebox{-0.75ex}{B}}

    \pgf@anchor@fulladder@Cin
    \pgftext[left,base,at={\pgfpoint{\pgf@x}{\pgf@y}},x=\pgfshapeinnerxsep]{\raisebox{-0.75ex}{$\text C_{\text{in}}$}}


    \pgf@anchor@fulladder@S
    \pgftext[right,base,at={\pgfpoint{\pgf@x}{\pgf@y}},x=-\pgfshapeinnerxsep]{\raisebox{-.75ex}{S}}

    \pgf@anchor@fulladder@C
    \pgftext[right,base,at={\pgfpoint{\pgf@x}{\pgf@y}},x=-\pgfshapeinnerxsep]{\raisebox{-.75ex}{$\text C_{\text{out}}$}}

    \endgroup
  }
}

% Define default style for this node
\tikzset{every fulladder node/.style={draw,minimum width=1.75cm,minimum 
  height=2.25cm,thick,inner sep=1mm,outer sep=0pt,cap=round}}

\makeatother

\newcommand{\dipcase}[5]{
  \def\dipcasewidth{#3}
  \def\dipcasename{#5}
  \pgfmathsetmacro{\dipcasepinshalf}{(#4)/2}
  \pgfmathsetmacro{\dipcasepinheight}{((\dipcasewidth)/3)}
  \pgfmathsetmacro{\dipcaseheight}{((\dipcasepinheight)*\dipcasepinshalf)}
  \def\dipcasepindistance{1cm}
  \draw (#1,#2) coordinate (dipcaseorigin);

  \draw (dipcaseorigin) rectangle ++(\dipcasewidth, -\dipcaseheight) 
    coordinate (dipcaserb);

  \draw (dipcaseorigin) ++(0.5*\dipcasewidth, 0)
    ++(-0.1*\dipcasewidth, 0)
    arc (-180:0:0.1*\dipcasewidth);

  \foreach \pin in {1, ..., \dipcasepinshalf}
  {
    \pgfmathsetmacro{\rpin}{\pin - 1}
    \foreach \xmirr in {1, -1}
    {
      \ifnum\xmirr=1
        \pgfmathtruncatemacro{\pinnumber}{\pin}
        \draw (dipcaseorigin) ++(0, -\rpin * \dipcasepinheight)
          ++(0, -0.5*\dipcasepinheight) coordinate(dipcasepincenter);
      \else
        \pgfmathtruncatemacro{\pinnumber}{\dipcasepinshalf + \pin}
        \draw (dipcaserb) ++(0, \rpin * \dipcasepinheight)
          ++(0, 0.5*\dipcasepinheight) coordinate(dipcasepincenter);
      \fi
      \draw (dipcasepincenter) coordinate (\dipcasename\pinnumber);

      \draw (dipcasepincenter) 
        ++(0, \dipcasepinheight/4) 
        -- ++(-\dipcasepinheight/4*\xmirr, 0)
        -- ++(0, -\dipcasepinheight/2)
        -- ++(\dipcasepinheight/4*\xmirr, 0);

    }
  }
}


\newcommand{\mymeter}[2] 
{  % #1 = name , #2 = rotation angle
\begin{scope}[transform shape,rotate=#2]
\draw[thick] (#1)node(){$\mathbf V$} circle (11pt);
\draw[rotate=45,-latex] (#1)  +(-17pt,0) --+(17pt,0);
\end{scope}
}
\usepackage[european resistors]{circuitikz}
\usepackage[ 
    top=2cm, 
    bottom=2cm, 
    outer=3cm, 
    inner=3cm,
    marginparwidth=2.5cm,
  ]{geometry}

\usepackage{parskip}
\setlength{\parindent}{0pt}

\newcommand{\experimentheader}[4]
{
  \iftutor{{\bf Schwierigkeitsgrad:} #1\\}
  \iftutor{{\bf Dauer:} #2\\}
  {\bf Ger\"ate:} #3\\
  {\bf Bauteile:} #4
}

\newcommand{\hintboxNone}{0}
\newcommand{\hintboxExclamation}{1}
\newenvironment{hintbox}[4][\hsize]
{
  \def\FrameCommand
  {%
    {\color{#3}\vrule width 3pt}%
    \hspace{0pt}%must no space.
    \fboxsep=\FrameSep\colorbox{#4}%
  }%
  \MakeFramed{\hsize#1\advance\hsize-\width\FrameRestore}%
  \mbox{\textbf{#2}:}%
}
{
  \endMakeFramed
}
\newcommand{\xhintbox}[1]
{
  \begin{hintbox}{Achtung}{red!50}{red!10}
    #1
  \end{hintbox}
}

\newenvironment{hint}
{
  \begin{hintbox}{Hinweis}{green!50}{green!10}
}
{
  \end{hintbox}
}

\newenvironment{definition}
{
  \begin{hintbox}{Definition\\}{white!50}{white!10}
}
{
  \end{hintbox}
}

\newenvironment{important}
{
  \begin{hintbox}{Hinweis}{gray!50}{gray!10}
}
{
  \end{hintbox}
}

\newenvironment{jason}
{
  \begin{hintbox}{Achtung}{red!50}{red!10}
}
{
  \end{hintbox}
}

\newcommand{\mandatoryenumi}
{
  \renewcommand{\labelenumi}{\arabic{enumi}.} 
}
\newcommand{\optionalenumi}
{
  \renewcommand{\labelenumi}{$\bigstar$\quad\arabic{enumi}.} 
}
\newcommand{\mandatoryenumii}
{
  \renewcommand{\labelenumii}{(\alph{enumii})} 
}
\newcommand{\optionalenumii}
{
  \renewcommand{\labelenumii}{$\bigstar$\quad(\alph{enumii})} 
}
\newcommand{\icname}[1]{\mbox{\tt #1}}

\IfFileExists{tmp/makefile-auto-tutor.tex}
{
  \input{tmp/makefile-auto-tutor.tex}
} %else
{
  \newcommand{\iftutor}[1]{}
\newcommand{\ifnotutor}[1]{#1}

  %\newcommand{\iftutor}[1]{#1}
\newcommand{\ifnotutor}[1]{}

}
\newenvironment{tutorhint}{\comment}{\endcomment}
\newenvironment{todo}{\comment}{\endcomment}
\newenvironment{solution}{\comment}{\endcomment}
\iftutor
{
  \renewenvironment{todo}
  {
    \hintbox{Todo}{red!50!yellow!90}{red!50!yellow!20}
  }
  {
    \endhintbox
  }
  \renewenvironment{tutorhint}
  {
    \hintbox{Tutorenhinweis der Stunde}{blue!50}{blue!10}
  }
  {
    \endhintbox
  }
  \renewenvironment{solution}
  {
    \hintbox{L\"osung}{black!80}{black!5}
  }
  {
    \endhintbox
  }
}
\newcommand{\etutorhint}[1]
{
  \iftutor{
    \tutorhint
      #1
    \endtutorhint
  }
}
\newcommand{\esolution}[1]
{
  \iftutor
  {
    \solution
    #1
    \endsolution
  }
}
\newcommand{\etodo}[1]
{
  \iftutor
  {
    \todo
    #1
    \endtodo
  }
}


%%%%%%%%%%%%%%%%%%%%%%%%%%%%%%%%%%%%%%%%%%%%%%%%%%%%%%%%%%%%%%%%%

 \setcounter{tocdepth}{0}
 \setcounter{totalnumber}{6}
 \setcounter{topnumber}{6}
 \setcounter{bottomnumber}{6}

%%%%%%%%%%%%%%%%%%%%%%%%%%%%%%%%%%%%%%%%%%%%%%%%%%%%%%%%%%%%%%%%%%%

 \newcommand{\Ohm}{$\mathrm\Omega$}
 \newcommand{\sym}[1]{$ #1 $}
 %\newcommand{\degree}{$^{\circ}$} 	%already defined somewhere else...
 \newcommand{\ret}{$[\hookleftarrow]$}
 \newcommand{\bs}{\tt\symbol{'134}}
 \newcommand{\us}{\tt\symbol{95}}

%%%%%%%%%%%%%%%%%%%%%%%%%%%%%%%%%%%%%%%%%%%%%%%%%%%%%%%%%

\hyphenation{ Tor-sions-mo-dul Tor-sions-schwing-ung  Boltz-mann Ab-len-kung Be-schleu-ni-gungs-span-nung}

%\makeindex

\let\person=\textsc
\newcommand{\komplex}[1]{\ensuremath{\mathfrak{#1}}}%

\begin{document}

\frontmatter


\title{\Huge Physikalisches Praktikum\\
f"ur Nebenfach Physik}
\author{Dr.~Jens Weingarten, Prof.~Dr.~Arnulf Quadt}
%\edition{1}
 
\maketitle
%das Folgende macht das inhaltverzeichnis schoen
\makeatletter
\renewcommand*{\l@chapter}{\@dottedtocline{0}{1.5em}{2.3em}}
\makeatother

\tableofcontents

%
%%%%%%%%%%%%%%%%%%%%%%%%%%%%%%%%%%%%%%%%%%%%%%%%%%%%%%%%%%%%%%%%%%%

\chapter*{Vorwort}

\addcontentsline{toc}{chapter}{Vorwort}


Herzlich Willkommen zum physikalischen Praktikum im Nebenfach an der Universit"at
G"ottingen. Diese Praktikumsanleitung enth"alt alle Informationen, die Sie ben"otigen, um das einsemestrige Praktikum erfolgreich zu absolvieren.
%wird Sie die n"achsten zwei bis drei Semester
%begleiten. Das Grundpraktikum f"ur das Fach Physik
%beinhaltet die Vorlesung \glqq{}Grundlagen des Experimentierens\grqq{} 
%(GdE) sowie 25~Versuche und erstreckt sich "uber zwei Semester. Es
%soll im Regelfall im 2.~und 3.~Semester absolviert werden. Das
%separate Projektpraktikum schlie"st sich im 4.~Semester an.

Das Praktikum (Modul B.Phy-NF.7004) wird ben"otigt f"ur den Bachelor in Chemie, Mathematik, Geologie, Biologie, Bio-Diversit"at, Mineralogie, Physiologie und Molekulare Medizin sowie f"ur den Zwei-Fach Bachelor in Biologie, Chemie, Geowissenschaften und Mathematik. Dieses Praktikum wird mit 4~SWS angerechnet und bringt 4~Credits.
%Physik und f"ur den 2-F"acher-Bachelor (Lehramt an Gymnasien) mit Physik
%als Fach. 
%Dieses Praktikum wird mit 4~SWS angerechnet und bringt 12~Credits -- 2~Credits f"ur
%die GdE als B.phy.410.1 und 10~Credits f"ur die 25~Versuche als
%B.phy.410.2. F"ur den Bachelor in Physik ist zus"atzlich auch
%das Projektpraktikum (Modul B.phy.604) erforderlich.\footnote{Das
%Projektpraktikum kann auch von Lehramtstudentinnen und -studenten
%durchgef"uhrt und als Studienleistung anerkannt werden. Bitte
%informieren Sie sich bei Ihrer Studienberaterin f"ur das
%Lehramt/Zweif"acher-Bachelor.} 


Das G"ottinger Nebenfach-Physikpraktikum f"uhrt anhand ausgew"ahlter, vorgefertigter Versuche
in einen weiten Bereich physikalischer Grundlagen, in den Umgang mit
Apparaturen und Messger"aten und in die Technik des physikalischen
Experimentierens ein und stellt damit einen wesentlichen Teil der
traditionellen Grundausbildung in Physik dar. Ziel ist hierbei auch 
eine Vertiefung des
in der Vorlesung ''Experimentalphysik I im Nebenfach'' erlernten Stoffes durch eigenes Umsetzen
und das Erfahren von Physik (\textit{learning by doing}). Sie erlernen
den Umgang mit verschiedensten Ger"aten und erfahren durch eigenes
Tun, wie eine physikalische Aufgabenstellung experimentell und
methodisch angegangen wird (\textit{hands on physics}). %Problem -
%Analyse - Bearbeitung - L"osung - Dokumentation, dies ist die
%Sequenz, die Sie in Ihrem ganzen "`Physik"=Leben"' begleiten wird.
Hierbei spielt auch Gruppen- oder Teamarbeit eine wichtige Rolle.
Nutzen Sie die Gelegenheit im Praktikum auch dies zu "uben, und
bringen Sie sich aktiv ein. Es wird sich auszahlen.

Dieses Handbuch beschreibt derzeit 20~Versuche, wovon nur 14
verpflichtend durchgef"uhrt werden m"ussen. %Zu Abschnitt 15 (Ferro-, Para-,
%Diamagnetismus) stehen zwei Versuche, zu Abschnitt 17 (Elektronik)
%stehen drei Versuche zur Auswahl, wovon jeweils einer durchgef"uhrt wird.
%Sprechen Sie die Auswahl bitte rechtzeitig mit Ihren Praktikumspartnern und
%Ihrem/r Betreuer/in ab.

%Mit dem seit 2002 eingef"uhrten Projektpraktikum, welches sich an die
%vorgefertigten Versuche anschlie"st, soll verst"arkt auch das
%eigenst"andige wissenschaftliche Arbeiten gef"ordert werden. Hierzu
%werden nur Themen (bzw. Aufgabenstellungen) vorgegeben oder auch von
%den Praktikantinnen und Praktikanten selbst ausgew"ahlt, die dann
%eigenst"andig in der Gruppe (max. 6~Personen) mit
%Hilfestellung eines Betreuers bearbeitet werden. Weitere Einzelheiten zum
%Projektpraktikum sind in diesem Handbuch im Teil \glqq
%Projektpraktikum\grqq{} angef"uhrt.

%Wir legen hiermit ein wiederum "uberarbeitetes Handbuch
%f"ur das G"ottinger Grundpraktikum Physik vor. 
%Wir m"ochten uns
%bei allen Praktikantinnen und Praktikanten, sowie allen Betreuerinnen
%und Betreuern bedanken, die durch ihre Hinweise und Vorschl"age geholfen
%haben, diese Anleitung zu verbessern. Ein besonderer Dank gilt den 
%Beteiligten des Lehrportals, die diese Anleitung mit einer online-Version
%und Videos unterst"utzen und mit verbesserten Abbildungen zu dieser
%Version des Handbuchs beigetragen haben. Trotz der Verbesserungen ist 
%es nur nat"urlich, dass sich auch wieder neue
%Fehler und Unzul"anglichkeiten in dieses Handbuch eingeschlichen haben.
%Wir w"aren dankbar, wenn Sie uns Fehler und auch Verbesserungsvorschl"age
%\emph{sofort} mitteilen w"urden (E-Mail: {\footnotesize
%\url{jgrosse1@uni-goettingen.de}}). Wir werden diese dann
%schnellstm"oglich beheben und auf den Web"=Seiten des Praktikums eine
%verbesserte Version der jeweiligen Anleitung zur Verf"ugung stellen.
%Noch sind nicht alle Versuche und ihre Anleitungen auf dem Stand, den
%Sie und wir gerne h"atten. Dies werden Sie sicherlich feststellen.
%Bedenken Sie, dass diese Anleitung und auch die "Uberarbeitung und
%Erneuerung der Versuche sehr viel Arbeit erfordert, und wir bei der
%derzeitigen Personal- und Betreuungssituation nicht alles umsetzen
%k"onnen, was Sie sich und wir uns w"unschen.

%Wir modernisieren das Grundpraktikum Physik weiterhin
%kontinuierlich durch Neuanschaffungen, Versuchsmodifikationen und
%Entwicklung neuer Versuche. Daneben gibt es Apparaturen, die etwas
%\glqq{}altmodischer\grqq{} aussehen, aber doch noch ganz ihrer
%(didaktischen) Aufgabe gerecht werden. Es kostet gro"se M"uhe, alle
%diese Apparaturen in einem einwandfreien Zustand zu erhalten.
%Sollten Sie dennoch Fehler feststellen, so geben Sie uns bitte
%sofort Bescheid. Nur dann k"onnen wir f"ur Abhilfe sorgen.

%Auch nach dem Druck dieser \glqq{}Praktikumsanleitung\grqq{} werden
%Versuche weiterentwickelt und verbessert, so dass es zu Abweichungen
%des aktuellen Versuches von dieser Anleitung kommen kann. Bedenken
%Sie bitte, dass zwischen Drucklegung und Ihrer Durchf"uhrung des
%Versuches schon eine lange Zeitspanne vergangen sein kann (im
%Extremfall "uber ein Jahr). Wir bem"uhen uns, Ihnen solche "Anderungen
%und Verbesserungen rechtzeitig mitzuteilen, hoffen aber auch, dass
%Sie diese Verbesserungen honorieren werden. Wir werden versuchen,
%auf den Webseiten immer aktuelle Versuchsanleitungen zur Verf"ugung
%zu stellen. Es lohnt sich also bestimmt, von Zeit zu Zeit auf den
%Web-Seiten des Praktikums {\footnotesize
%\url{http://www.praktikum.physik.uni-goettingen.de}} nachzusehen, da
%wir uns bem"uhen werden, dort immer die aktuellsten Informationen zu
%publizieren. Auf den Webseiten finden Sie auch wertvolle Hinweise,
%Kontaktadressen, Termine, Gruppeneinteilungen und eine E-Mail Liste.
%Die E-Mail-Liste ist zur Vermeidung von \acro{SPAM}-Mail 
%nur nach einem Login "uber das Benutzerkonto, welches Sie bei
%der online-Anmeldung angelegt haben, zu erreichen.

In diesem Handbuch finden Sie eine kleine Abhandlung "uber die
Grundlagen der Fehlerrechnung und Protokollerstellung. 
%W"ahrend der
%\glqq Grundlagen des Experimentierens\grqq{} werden Sie davon bereits
%profitiert haben und k"onnen das dort gelernte hier entsprechend anwenden.
%Sprechen Sie
%bitte Ihren Betreuer/Ihre Betreuerin an, damit diese/r an den ersten
%Versuchstagen die Fehlerrechnung und Protokollerstellung nochmal an
%einem konkreten Beispiel mit Ihnen und Ihrer Gruppe "ubt.

Bitte bedenken Sie auch immer, dass Ihre Betreuerinnen und Betreuer
f"ur \emph{Ihr} Praktikum, also f"ur \emph{Ihren} Lernerfolg, viel
Arbeit und Zeit investieren. Dies geschieht neben eigenem Studium
oder eigener Promotion und resultiert in einer Belastung, die weit
"uber das hinausgeht, was als Lehrverpflichtung von Betreuer(innen)
im Durchschnitt an der Fakult"at erbracht wird. Leider
stehen uns nicht so viele Betreuer(innen) zur Verf"ugung, wie wir dies aus
praktischen und didaktischen Erw"agungen f"ur sinnvoll erachten.
Erleichtern Sie deshalb bitte Ihren Betreuerinnen und Betreuern
diese Belastung durch Ihre engagierte, aktive, gut vorbereitete und
m"oglichst eigenst"andige Mitarbeit im Praktikum und
\glqq{}zahlen\grqq{} Sie deren Engagement mit Ihrem pers"onlichen
guten Lernerfolg zur"uck. Nur Ihr \emph{aktives und eigenst"andiges}
Arbeiten erzielt auch eine hohe Nachhaltigkeit des Erlernten und
schafft so das solide Wissensfundament, auf dem Sie Ihre Zukunft
aufbauen k"onnen.

Zusammenfassend w"unschen Ihnen alle Betreuerinnen und Betreuer des
Praktikums viel Spa"s im und einen guten Lernerfolg durch das
Praktikum. Wir alle, insbesondere Ihre Betreuerinnen und Betreuer,
bem"uhen uns, damit dies -- Ihre Mithilfe angenommen -- auch erreicht
werden kann.


%\signature{G"ottingen}{im Oktober 2015}{Dr. Jens Weingarten}



%%%%%%%%%%%%%%%%%%%%%%%%%%%%%%%%%%%%%%%%%%%%%%%%%%%%%%%%%%%%%%%%%%%

%%%%%%%%%%%%%%%%%%%%%%%%%%%%%%%%%%%%%%%%%%%%%%%%%%%%%%%%%%%%%%%%%%%
% JW: Zum kompilieren verwende Ausgabeprofil "LaTex => PDF"
%%%%%%%%%%%%%%%%%%%%%%%%%%%%%%%%%%%%%%%%%%%%%%%%%%%%%%%%%%%%%%%%%%%


\mainmatter

%%%%%% define header and footer style %%%%%%%%%%%%%%%%%%%%
\pagestyle{fancy}
\fancyhead{} % clear all header fields
\fancyhead[RO,LE]{\leftmark} %Kapitelnummer und -�berschrift Header au�en
\fancyfoot{} % clear all footer fields
\fancyfoot[LE,RO]{\thepage}	%Seitenzahl Footer au�en an der Seite
\renewcommand{\headrulewidth}{0.4pt}
\renewcommand{\footrulewidth}{0.4pt}
%%%%%%%%%%%%%%%%%%%%%%%%%%%%%%%%%%%%%%%%%%%%%%%%%%%%%%%%%%%%
%
\part{Vorbemerkungen}

\renewcommand{\thechapter}{\Alph{chapter}}

%%%%%%%%%%%%%%%%%%%%%%%%%%%%%%%%%%%%%%%%%%%%%%%%%%%%%%%%%%%%%%%%%%%

\chapter{Praktikumsordnung} \label{v:ordnung}

Das Physikalische Praktikum für Nebenfach Physik (B.Phy-NF.7004) besteht aus 14 Versuchen, die während der Vorlesungszeit eines Semesters durchgeführt werden. An einem Praktikumstag werden zwei Versuche durchgeführt, die Termine können dem Kursplan entnommen werden, der vor Beginn des Praktikums auf den Webseiten des Physikalischen Praktikums für Nebenfach Physik veröffentlicht wird. \\
\textbf{Versuche aus einem vorherigen Kurs können nicht angerechnet werden.}

Das Praktikum beginnt mit der obligatorischen Einführungsveranstaltung mit Sicherheitsbelehrung. Eine Teilnahme am Praktikum ohne Sicherheitsbelehrung ist nicht möglich. Sollte das Praktikum zum wiederholten Male belegt werden, ist die Einführungsveranstaltung erneut zu besuchen.

%**********************************************************************************************************************
%**********************************************************************************************************************

%\section{Ablauf des Praktikums}
%
%\begin{enumerate}
	%%
	%\item Das Praktikum erstreckt sich über ein Semester. Wird das Praktikum nicht in diesem Semester abgeschlossen, so muss es erneut belegt werden. Versuche aus dem vorherigen Kurs können nicht angerechnet werden.
	%%
	%\item Das Praktikum umfasst 14 Versuche. Die Termine können dem Kursplan entnommen werden, der vor Beginn des Praktikums auf den Webseiten des Physikalischen Praktikums für Nebenfach Physik veröffentlicht wird.
	%%
	%\item Wenn ein Versuch nicht testiert wurde, 
	%%
	%\item Das Praktikum beginnt mit der obligatorischen Einführungsveranstaltung mit Sicherheitsbelehrung. Eine Teilnahme am Praktikum ohne Sicherheitsbelehrung ist nicht möglich. Sollte das Praktikum zum wiederholten Male belegt werden, ist die Einführungsveranstaltung erneut zu besuchen.
	%%
	%\item Mit der Anmeldung zum Praktikum verpflichtet sich der/die Praktikant/in zur regelmäßigen Teilnahme.
%\end{enumerate}

%**********************************************************************************************************************
%**********************************************************************************************************************


%Es hat Vorteile, wenn eine Zweiergruppe alle Versuch des Praktikums gemeinsam durchf"uhrt. Da dies aufgrund terminlicher Bedingungen nicht immer m"oglich ist, ist es jedoch nicht zwingend.\\
%An jedem Arbeitstag werden von jeder Arbeitsgruppe je 2 Versuche durchgef"uhrt. Nach der Vorbesprechung w"ahlen die Zweiergruppen sich ihre 14 Versuche (7 Arbeitstage) aus den m"oglichen Terminen aus. Es m"ussen aus den Bereichen Mechanik (Versuche 1-10) und Elektrik (Versuche 11-20) jeweils mindestens 6 Versuche ausgesucht werden.

%\noindent
%Es gelten folgende organisatorische Regeln\index{Regeln} f"ur das Praktikum, die einen reibungslosen und effektiven Ablauf des Praktikums erm"oglichen sollen. Diese sind immer zu beachten.

\section{Voraussetzungen zur Teilnahme}

Voraussetzungen zur Teilnahme am Nebenfachpraktikum Physik sind die erfolgreiche Teilnahme an der Vorlesung "`Experimentalphysik I im Nebenfach"' (B.Phy-NF.7001 oder B.Phy-NF.7002), als auch die persönliche Teilnahme an der Einführungsveranstaltung des Praktikums. 

\section{Anmeldung und Gruppeneinteilung}

Zur Teilnahme am Praktikum melden Sie sich bitte auf StudIP für die Veranstaltung an.\\
Das Praktikum wird in festen Gruppen von bis zu 10 Studierenden durchgeführt. Bitte tragen Sie sich bei der Anmeldung gleich in eine der vordefinierten Gruppen ein.\\
Die Versuche werden in Zweiergruppen durchgeführt, welche zusammen ein Protokoll abgeben. Sollte eine/r der Teilnehmer/innen zur Versuchsdurchführung nicht erscheinen, so kann in Absprache mit dem Betreuer eine Dreiergruppe gebildet werden.

\section{Termine}

Der Termin für die Einführungsveranstaltung wird rechtzeitig in UniVZ und auf StudIP angekündigt. Typischerweise findet diese in der ersten Vorlesungswoche statt.

Die Versuche werden jeweils Mittwoch oder Freitag an 14~Uhr~c.t. durchgeführt. Ab etwa 13:30~Uhr sind die Tutoren in den Praktikumsräumen anzutreffen, damit Sie Ihre Protokolle abgeben oder den Versuch testiert bekommen können. 

\section{Versuchsvorbereitung}

Jede Praktikantin und jeder Praktikant muss sich genügend auf den durchzuführenden Versuch\index{Versuchsvorbereitung} vorbereiten. Die Durcharbeitung der Anleitung zum Praktikum und das Literaturstudium sind obligatorisch. \\
Der schriftliche Kurztest vor Versuchsbeginn (''Quicky'') dient dem Nachweis genügender fachlicher Grundkenntnisse und ausreichender Vorbereitung der Teilnehmer, er bildet die Grundlage für die Teilnahme an den Praktikumsversuchen.

Wer unvorbereitet zu einem Versuch kommt, riskiert, dass er/sie den Versuch an diesem Tag nicht durchf"uhren darf und einen Nachholtermin in Anspruch nehmen muss. Zur Hilfe bei der Vorbereitung sind in der jeweiligen Versuchsanleitung einige Fragen gestellt. 
In der ersten Stunde jeden Praktikumtages prüft der Assistent, ob Sie sich auf die beiden Versuche vorbereitet haben. Dazu k"onnen Sie in den ersten 30 Minuten Fragen zu möglichen Unklarheiten im Versuch an den Assistenten stellen. Anschließend werden Zettel ausgeteilt, auf denen jeweils auf einer Viertelseite 2 der Vorbereitungsfragen schriftlich beantwortetet werden müssen (Quickie). Die Zettel werden sofort durchgesehen und bleiben bei den Assistenten.\\ 
\textbf{Wer beide Fragen falsch beantwortet, wird vom Versuch ausgeschlossen.} Nur eine falsche Antwort führt nicht zum Ausschluss, wohl aber, wenn es schon das zweite Mal ist. 

%Die Betreuerinnen und Betreuer sind gehalten, vor jedem Versuch
%nochmals die Sicherheitsaspekte zum Versuch zu erl"autern und deren
%Verst"andnis zu "uberpr"ufen.

\section{Durchführung}

\textbf{Die Versuchsdurchführung beginnt um 14~Uhr~c.t. Erheblich verspätetes Erscheinen führt zum Ausschluss an der Durchführung des Versuchs.}\\
Vor dem Beginn der Messungen mache man sich mit den Apparaturen vertraut, d.h. wie sind welche Messgeräte anzuschließen, wie funktionieren sie, wie werden sie abgelesen, welche Fehler haben sie, bei welchen Apparaturen ist besondere Vorsicht geboten, usw. Insbesondere bei elektrischen Stromkreisen ist darauf zu achten, dass Strom und
Spannungen sehr gefährlich sein können! Messgeräte sind vor dem Gebrauch - sofern möglich - auf Funktionsfähigkeit zu testen und auf den richtigen Messbereich einzustellen. Manchmal ist es hilfreich, Schalter und Messgeräte durch Zettel oder ähnliches (z.B. "`Post-it"') zu beschriften, um Irrtümer zu vermeiden. Bei elektrischen Schaltungen ist nach dem Aufbau zunächst der Assistent zu benachrichtigen, erst mit dessen Zustimmung wird die Stromversorgung eingeschaltet! Messkurven sind während der Versuchsdurchführung grafisch darzustellen (Millimeterpapier nicht vergessen!). Jeder ist selbst dafür verantwortlich, dass alle benötigten Daten richtig und vollständig gemessen werden. Bitte denken Sie nach, ob die gemessenen Werte sinnvoll sind!

\textbf{Während der Versuchsdurchführung ist ein Messprotokoll\index{Messprotokoll} \emph{dokumentenecht} anzufertigen.} \\
Es darf also nur Kugelschreiber oder Tusche verwendet werden (kein Bleistift). Es wird nichts radiert, sondern nur gestrichen. Datum und Mitarbeiter angeben, Seiten nummerieren. Die Versuchsdurchführung muss nachvollziehbar sein. \\
Darauf müssen folgende Informationen zu finden sein:
\begin{itemize}
	\item Name des Versuchs
	\item Datum der Durchführung 
	\item Namen aller beteiligten Praktikanten 
	\item die gemessenen Werte mit Fehlerangabe.
\end{itemize} 
Das Protokoll muss leserlich sein und sollte übersichtlich gestaltet sein, z.B. durch einleitende Sätze, was mit den dann folgenden Messwerten bestimmt werden soll. Jeder Messwert muss eindeutig mit der gemessenen Größe in Verbindung gebracht werden können, ggf. sollten Skizzen angefertigt werden. Es müssen die tatsächlich gemessenen (direkt abgelesenen) Werte aufgeschrieben werden, zusätzlich ausgerechnete Werte (z.B. Differenzen) dürfen nur zusätzlich aufgeschrieben werden. Dies soll (Kopf-)Rechen- und Denkfehlern
vorbeugen. Zu jedem Messwert ist die Einheit zu notieren! Zu jedem Messwert ist ein Fehler zu notieren (Ablesefehler, Gerätefehler, Schwankungen).

\textbf{Am Ende des Versuchs wird das Messprotokoll vom Assistenten testiert, ansonsten ist es ungültig!}

Sinnigerweise sollte der Versuch erst hiernach abgebaut werden, da u.U. bestimmte Dinge erneut gemessen werden m"ussen oder Daten fehlen.\\ 
{\bf Jede(r) Student/in sollte ein eigenes testiertes Messprotokoll haben.}\\
Erst nachdem der/die Betreuer/in die Werte kontrolliert, das Versuchsprotokoll testiert (Versuchs"=Testat) \index{Versuchs-Testat} und dies in die Karteikarte eingetragen hat,
ist der Versuch abzubauen und alles aufzuräumen. Weiterhin erteilt der Assistent nach Abschluss eines Versuches auch auf der Karteikarte durch Unterschrift ein Vortestat. Dieses dient gleichzeitig als Teilnahme-Beweis.

Nach Beendigung eines Versuchstages\index{Versuchsende}\index{Aufräumen} sind alle Versuche, Geräte und Räume wieder in den ursprünglichen Zustand zu versetzen. Messgeräte, Kabel und Stoppuhren sind wieder an die vorgesehenen Stellen zu bringen. \textbf{Defekte sind sofort einem/r Betreuer/in zu melden.} Flaschen und sonstige Abfälle sind bitte zu entsorgen.


\section{Protokolle}

%Beachten Sie zum Protokoll auch die Hinweise in
%Kapitel~\ref{c:protokoll}. 
Im Protokollkopf müssen der Name des Versuchs und das Datum der Durchführung stehen. Bei einem in Eigenarbeit geschriebenen Protokoll\index{Protokoll} steht als "`Praktikant"' der Name des Praktikanten und unter "`Mitarbeiter"' die Namen der übrigen an der Versuchsdurchführung beteiligten Personen. Bitte auch den Namen des/der Betreuers/in und die eigene E-Mail Adresse im Kopf angeben. 
%Jeder
%Praktikant braucht f"ur die Unterschrift auf der Karteikarte ein
%eigenes Protokoll, welches auch unterschrieben werden
%soll. 

Das Protokoll muss leserlich und übersichtlich gestaltet sein. Es ist für sich eigenständig, also keine Verweise auf die Praktikumsanleitung.\footnote{Sätze wie "`Versuchsdurchführung s. Praktikumsanleitung"' sind überflüssig.} Aus dem Protokoll alleine muss klar verständlich sein, was das Ziel des Versuchs war, wie Sie den Versuch durchgeführt haben und was das Ergebnis des Versuchs ist. Es muss aus dem Protokoll ersichtlich sein, wie die Auswertung aufgebaut ist, d.h. welche Werte gemessen wurden und was man aus diesen Messwerten bestimmen möchte.

Im eigentlichen Teil der Auswertung sind deutlich und nachvollziehbar die einzelnen Auswertungsschritte aufzuschreiben. Einleitende Sätze, was gemessen wurde, und was daraus berechnet wird, sind obligatorisch. Ergebnisse sind deutlich zu kennzeichnen (Rahmen, farbiges Markieren, größere Schrift, usw.). Was sind Zwischen- oder Hilfsergebnisse, was sind Endergebnisse? Alle benutzten Formeln müssen beschrieben sein bzw. deren Herleitung klar werden. Es ist auf eine durchgehend eindeutige Variablendefinition und Variablenbenutzung im Protokoll zu achten. Alle Variablen in den Funktionen sind zu benennen bzw. zu definieren. Die Fehlerrechnung muss nachvollziehbar beschrieben werden (einfacher Mittelwert oder gewichtetes Mittel, ggf. Formel der Fehlerfortpflanzung angeben). Fehler sind sinnvoll anzugeben! Alle Werte haben Einheiten, alle Grafen eine Beschriftung! Bei Vergleich mit Literaturwerten: Woher kommen die Werte (Quellenangabe)? Eine Diskussion der Ergebnisse und der Fehler ist obligatorisch. Dazu muss man sich natürlich vorher die Frage stellen, ob das, was man berechnet hat, ein sinnvolles Ergebnis ist.

Bei der Abgabe des Protokolls muss das dazugehörige original unterschriebene Messprotokoll mit abgegeben werden. Protokolle müssen in geeigneter Form zusammengeheftet sein (einfache Mappe oder Heftung reichen vollkommen), lose Blätter werden nicht akzeptiert.\\
\textbf{Wird ein/e Praktikant/in auf die Auswertung seines/ihres eigenen Protokolls angesprochen und kann keine Auskunft zu den gemachten Rechnungen geben, so gilt das Protokoll als nicht selbständig erstellt und wird nicht testiert.} \\
Protokolle mit einer Auswertung, die nicht auf den eigenen Messdaten basieren, bei denen die Messdaten nachträglich geändert wurden oder bei denen die Liste der am Versuch beteiligten Personen erweitert wurde, gelten als Täuschungsversuch/Urkundenfälschung und werden entsprechend geahndet.
%
%\begin{definition}[Abgabefrist der Protokolle]
% Die Abgabefrist f"ur ein Protokoll betr"agt 2~Wochen nach Durchf"uhrung
% des Versuchs.
%\end{definition}
%

{\bf Ein vollständiges Protokoll muss zum nächsten Praktikumstermin, also innerhalb einer Woche nach der Versuchsdurchführung, abgegeben werden.} 
%Wird das Protokoll {\bf nicht innerhalb von 2~Wochen} abgegeben, {\bf verf"allt das Vortestat und der Versuch muss wiederholt werden.} 

Ein Protokoll gilt nur dann als vollständig, wenn es oben genannte Bedingungen erfüllt. Insbesondere gilt es als nicht vollständig, wenn es außer dem Endergebnis keine Zwischenergebnisse enthält, die den Rechenweg und die Werte nachvollziehbar machen. Sollte das Protokoll für Korrekturen ohne Testat zurückgegeben werden, so gilt erneut die 1~Wochen-Frist ab dem Tag der Rückgabe. Die Korrekturen sind (z.B. als Anhang) zusammen mit dem vollständigen ursprünglichen Protokoll abzugeben. Sie haben die Möglichkeit, jedes Protokoll genau einmal zu überarbeiten. Wird das Protokoll nach der Korrektur immer noch nicht testiert wird, so gilt der Versuch als nicht bestanden und es muss ein Versuch (nicht notwendigerweise derselbe) nachgeholt werden.

Auch für die Rückgabe der Protokolle durch die Assistenten soll die 1-Wochen-Frist eingehalten werden. Das bedeutet, dass nach spätestens vier Wochen feststeht, ob der Versuch testiert wird, oder nicht.

Es ist zu beachten, dass es für jedes Semester einen Termin gibt, ab dem alle bis zu diesem Tag nicht testierten Protokolle nicht mehr angenommen und testiert werden. Dies ist in der Regel der 30.04. für das vergangene Wintersemester und der 31.10. für das vergangene Sommersemester. Dies ist erforderlich, da auch die Betreuer Fluktuationen
unterworfen sind und so der/die Betreuer/in später eventuell Göttingen schon verlassen hat.

\section{Nachholtermine}

\textbf{Es stehen genau zwei Nachholtermine\index{Nachholtermin} für versäumte oder nicht testierte Versuche zur Verfügung.} Sollten Sie aus triftigen Gründen (schwere Erkrankung, o.Ä.) mehr als zwei Versuche versäumen, melden Sie sich bitte umgehend bei der Praktikumsleitung. \textbf{Wenn aus anderen Gründen mehr als zwei Versuche nicht durchgeführt oder testiert werden, gilt das Praktikum als nicht bestanden.}\\
Bitte sorgen Sie dafür, dass ggf. ein/e Partner/in für die Durchführung des Nachholversuchs zur Verfügung steht. Die Nachholtermine finden in den beiden Wochen nach Ende der regulären Versuche statt.
%Um einen Nachholtermin zu bekommen, melden Sie sich bitte z"ugig beim Praktikumsleiter. Es kann vorkommen, dass ein Nachholtermin erst im n"achsten Semster m"oglich ist.


\section{Karteikarte}

Während der Vorbesprechung zum Praktikum werden die Karteikarten\index{Karteikarte} ausgeteilt, die die Praktikantin oder der Praktikant bis zum Abschluss des Praktikums behält. Auf dieser werden dann die Versuchs-Durchführung und Protokoll-Testate vom Assistenten eingetragen und mit Unterschrift bestätigt. Diese Karteikarte ist der Nachweis für die
Gesamtleistung im Praktikum und damit Zulassung zum Noteneintrag in FlexNow, also bitte nicht verlieren.

Nach Abschluss des Praktikums geben Sie bitte Ihre vollständig ausgefüllte Karteikarte bei der Praktikumsleitung ab.


\section{Leistungsnachweis}

Die Gesamtleistung besteht aus 14 durchgeführten und komplett testierten Versuchen. \\
Das Praktikum ist unbenotet und wird nur mit Bestanden oder Nicht Bestanden in FlexNow eingetragen. In besonderen Fällen kann auch ein Schein ausgestellt werden.

%
%%%%%%%%%%%%%%%%%%%%%%%%%%%%%%%%%%%%%%%%%%%%%%%%%%%%%%%%%%%%%%%%%%%

\chapter{Literatur f"ur das Praktikum}

%%%%%%%%%%%%%%%%%%%%%%%%%%%%%%%%%%%%%%%%%%%%%%%%%%%%%%%%%%%%%%%%%%%

Angesichts der sehr unterschiedlichen Vorkenntnisse in Physik und der gro{\ss}en Auswahl physikalischer Lehrb"ucher ist es leider nicht m"oglich, einen einzigen Text als verbindlich zu erkl"aren. Zu Beginn jeder Versuchsanleitung finden Sie eine kurze Einf"uhrung in die zugrundeliegende Physik, welche allerdings keine vollst"andige Darstellung der physikalischen Zusammenh"ange sein kann. Bitte eignen Sie sich daher selbst"andig die zugeh"orige Physik durch Nachlesen in mehreren B"uchern (zur Not auch in Vorg"angerprotokollen\index{Vorg"angerprotokolle}, aber auf Richtigkeit achten!) tiefgehender an. Die unterschiedlichen Darstellungsweisen f"ordern das Verst"andnis. Die nachfolgend  aufgef"uhrten Literaturhinweise sollen dazu  dienen, Ihnen die Suche nach weiterf"uhrendem Material zu erleichtern. Die Aufz"ahlung erhebt weder einen Anspruch auf Vollst"andigkeit, noch stellt sie eine Wertung dar. Die meisten B"ucher sind in der Bereichsbibliothek Physik \textsc{BBP} ausleih- oder einsehbar. Es ist nicht wichtig, von jedem Buch unbedingt die letzte Auflage zu bekommen, da die Physik, die im Praktikum behandelt wird, sich in den letzten 100 Jahren nicht mehr ver"andert hat!

\noindent
Wir empfehlen  Ihnen, die Textstellen nicht isoliert herauszugreifen (Physiklexika sind f"ur die meisten Anf"anger erfahrungsgem"ass nicht  brauchbar!),  sondern Zusammenh"ange herzustellen.   \\
Auch Wikipedia kann nur Anhaltspunkte liefern...


%Als begleitende Literatur\index{Literatur} f"ur das Physikalische
%Praktikum sind prinzipiell alle Physikb"ucher geeignet. Insbesondere
%sind die folgenden B"ucher\index{B"ucher} zu nennen. Die Aufz"ahlung
%erhebt weder einen Anspruch auf Vollst"andigkeit, noch stellt sie
%eine Wertung dar. Welches Buch f"ur Sie pers"onlich das Beste ist,
%k"onnen nur Sie selbst entscheiden. Schauen Sie sich die B"ucher an,
%vergleichen Sie dabei beispielsweise direkt die unterschiedlichen
%Darstellungen eines bestimmten engen Gebietes. W"ahlen Sie dann
%dasjenige Buch aus, welches Ihnen am besten liegt. Neben dieser
%Aufz"ahlung finden sich diese und weitere B"ucher auch im
%Literaturverzeichnis wieder. Die Abk"urzungen der B"ucher werden zum
%Teil auch bei den Versuchen zur Angabe vertiefender Literatur
%benutzt.

%W"ahlen Sie anhand der Sachverzeichnisse\index{Sachverzeichnis} und
%der Stichworte\index{Stichworte} in den Anleitungen die geeignete
%Literatur zum jeweiligen Versuch aus. Zu einigen Versuchen wird
%spezielle Literatur angegeben. Die meisten B"ucher sind in der
%Bereichsbibliothek Physik \acro{BBP} ausleih- oder einsehbar. Wir
%sind bem"uht, die wichtigsten physikalischen Grundlagen in diese
%Anleitung aufzunehmen. Dies kann aber keine vollst"andige Darstellung der den Versuchen zugrunde liegenden Physik sein. 
%Bitte eignen Sie sich daher selbst"andig die zugeh"orige Physik
%durch Nachlesen in mehreren B"uchern (zur Not auch in
%Vorg"angerprotokollen\index{Vorg"angerprotokolle}, aber auf
%Richtigkeit achten!) tiefgehender an. Die unterschiedlichen
%Darstellungsweisen f"ordern das Verst"andnis.


\section{Spezielle Praktikumsb"ucher}

Tabelle~\ref{t:litprakt} enth"alt eine Aufz"ahlung von B"uchern, die
speziell f"ur Physikalische Praktika gedacht sind
(Praktikumsb"ucher\index{B"ucher!Praktikumsb"ucher}) und somit auch
Methodisches und Handlungshinweise enthalten.
%
\begin{table}[h!]
  \centering
  \caption[Praktikumsb"ucher]{\label{t:litprakt}Dedizierte Praktikumsb"ucher}
  \begin{tabular}{lp{11cm}}
    \hline
    K"urzel & Autor, Titel, Verlag, Jahr, Referenz \\
    \hline
    NPP & \person{Eichler, Kronfeld, Sahm}, Das Neue Physikalische Grundpraktikum, Springer, 2001 \\
    Wal & \person{Walcher}, Praktikum der Physik, Teubner, 2004 \\
    Wes & \person{Westphal}, Praktikum der Physik, Springer, 1984 (vergriffen)  \\
    SK & \person{Stuart, Klages}, Kurzes Lehrbuch der Physik, 2009\\
%    Geschke & \person{Geschke}, Physikalisches Praktikum, Teubner, 2001  \cite{geschke} \\
%    BeJo & \person{Becker, Jodl}, Physikalisches Praktikum, VDI-Verlag, 1983 \cite{bejo} \\
%    CIP & \person{Diemer, Basel, Jodl}, Computer im Praktikum  Springer, 1999 \cite{cip} \\
%    Paus & \person{Paus}, Physik in Experimenten und Beispielen,Hanser \cite{paus} \\
    \hline
  \end{tabular}
\end{table}



\section{Allgemeine Physikb"ucher}

Folgende, in Tabelle~\ref{t:litallg} aufgef"uhrte, allgemeine
Physikb"ucher\index{B"ucher!Allgemeine Physik} sind f"ur das
Praktikum und das Studium insgesamt n"utzlich.
%
\begin{table}[h!]
  \centering
  \caption[Allgemeine Physikb"ucher]{\label{t:litallg}Allgemeine Physikb"ucher, die f"ur das Praktikum n"utzlich sind.}
  \begin{tabular}{lp{11cm}}
    \hline
    K"urzel & Autor, Titel, Verlag, Jahr, Referenz \\
    \hline
 BS 1-8 & \person{Bergmann-Schaefer}, Experimentalphysik 1-8, DeGruyter, 2000 \\
 Dem 1-4 & \person{W. Demtr"oder}, Experimentalphysik 1-4, Springer, 2002\\
 Gerthsen & \person{Meschede, Vogel, Gerthsen}, Gerthsen: Physik, Springer, 2003\\
% Halliday & \person{Halliday}, Physik, Springer, 2003 \cite{halli}\\
% Feyn & \person{Feynman} Physics Lectures \cite{feyn1,feyn2,feyn3}\\
% Kohlr 1-3 & \person{Kohlrausch}, Praktische Physik 1-3, Teubner, 2002 \cite{kohlr1,kohlr2,kohlr3} \\
% Wesp & \person{Westphal}, Physik, Springer  \cite{wesp} \\
% Pohl & \person{L"uders-Pohl}, Pohls Einf"uhrung in die Physik, Springer 2004 \cite{pohl}\\
 %Grim 1-4 & \person{Grimsehl}, Lehrbuch der Physik 1-4, Teubner \cite{grim1,grim2,grim3,grim4}\\
% Alonso & \person{Alonso, Finn}, Physik, Oldenbourg, 2000 \cite{alonso} \\
% L"usch 1-3 & \person{L"uscher}, Experimentalphysik 1-3, Piper \cite{luesch1,luesch2,luesch3} \\
% St"ocker & \person{St"ocker}, Taschenbuch der Physik, Harri Deutsch \cite{tabphys} \\
 Tipler & \person{Tipler}, Physik, Spektrum, 1994\\
 Kuhn & \person{Kuhn}, Physik 2, Westermann, 2000\\
 Dorn & \person{Dorn, Bader}, Physik Grundkursband 12/13, Hermann Schroedel Verlag, 1976\\
 Harten & \person{Harten}, Physik f"ur Mediziner: Eine Einf"uhrung, Springer, 2011\\
 Metzler & \person{Grehn, Krause}, Metzler Physik, Schroedel, 2007\\
 \hline
  \end{tabular}
\end{table}


\section{Handb"ucher und Nachschlagewerke}

N"utzliche Hinweise zur Auswertung und Fehlerrechnung, sowie eine Vielzahl von Werten und Materialdaten, findet man in den in Tabelle~\ref{t:litnach} aufgef"uhrten\index{B"ucher!Handb"ucher}
Nachschlagewerken\index{B"ucher!Nachschlagewerke}.
%
\begin{table}[h!]
  \centering
  \caption[Handb"ucher und Nachschlagewerke]{\label{t:litnach}Handb"ucher und Nachschlagewerke f"ur das Praktikum}
  \begin{tabular}{lp{11cm}}
    \hline
    K"urzel & Autor, Titel, Verlag, Jahr, Referenz \\
    \hline
 Bron & \person{Bronstein-Semendajev}, Taschenbuch der Mathematik, H.~Deutsch \\
 TBMathe & \person{St"ocker}, Taschenbuch mathematischer Formeln u.~moderner Verfahren, H.~Deutsch  \\
 TBPhys & \person{St"ocker}, Taschenbuch der Physik, H.~Deutsch \\
% TBRegel & \person{Lutz, Wendt}, Taschenbuch der Regelungstechnik, H.~Deutsch \cite{tbregel} \\
 TBStat & \person{Rinne}, Taschenbuch der Statistik, H.~Deutsch \\
% TBElektro & \person{Kories, Schmidt-Walter}, Taschenbuch der Elektrotechnik, H.~Deutsch \cite{tbelektro} \\
% TBChem & \person{Schr"oter, Lautenschl"ager, Bibrack}, Taschenbuch d.~Chemie, H.~Deutsch \cite{tbchem} \\
 Kneu & \person{Kneub"uhl}, Repetitorium der Physik, Teubner \\
 Lichten & \person{Lichten}, Scriptum Fehlerrechnung, Springer \\
% Beving & \person{Bevington, Robinson}, Data reduction and error analysis for the physical sciences, McGraw-Hill, 1992 \cite{beving} \\
% Tab & \person{Berber, Kacher, Langer}, Physik in Formeln und Tabellen, Teubner \cite{tab} \\
 Messunsicher & \person{Weise, W"oger}, Messunsicherheit und Messdatenauswertung, Wiley-VCH, Weinheim, 1999  \\
 UmgUnsich & \person{Drosg}, Der Umgang mit Unsicherheiten, facultas, 2006 \\
 Kunze & \person{Kunze}, Physikalische Messmethoden, Teubner, 1986 \\
% Webelm & Web-Elements \url{www.webelements.com} \\
% LandB"orn & \person{Landolt-B"ornstein} \url{www.springeronline.de} \cite{landbern} \\
 NIST & \person{NIST} \url{www.nist.gov} \\
 \hline
  \end{tabular}
\end{table}
%
%

%\section{Verwendung im Praktikum}
%
%W"ahlen Sie anhand der Sachverzeichnisse\index{Sachverzeichnis} und
%der Stichworte\index{Stichworte} in den Anleitungen die geeignete
%Literatur zum jeweiligen Versuch aus. Zu einigen Versuchen wird
%spezielle Literatur angegeben. Die meisten B"ucher sind in der
%Bereichsbibliothek Physik \acro{BBP} ausleih- oder einsehbar. Wir
%sind bem"uht, die wichtigsten physikalischen Grundlagen in diese
%Anleitung aufzunehmen. Dies ist aber erst f"ur einige Versuche
%gelungen. Bitte eignen Sie sich selbst"andig die zugeh"orige Physik
%durch Nachlesen in mehreren B"uchern (zur Not auch in
%Vorg"angerprotokollen\index{Vorg"angerprotokolle}, aber auf
%Richtigkeit achten!) tiefgehender an. Die unterschiedlichen
%Darstellungsweisen f"ordern das Verst"andnis.

\newpage
\section{Fundamentalkonstanten}

Viele physikalische Fundamentalkonstanten werden im Praktikum f"ur
Berechnungen ben"otigt oder werden dort gemessen.
Tabelle~\ref{t:fundamentconst} gibt eine Auswahl aus der von der
\textsc{IUPAP} (\textit{International Union of Pure and Applied
Physics}) festgelegten Zusammenstellung \textsc{CODATA} \cite{codata}
wieder.
%
\begin{table}[hb]
  \centering
  \caption[Fundamentalkonstanten]{\label{t:fundamentconst} Wichtige
  physikalische Fundamentalkonstanten \cite{codata}. $\Delta x/x$
  ist die relative Unsicherheit (\textit{ppm} - parts per million, $\times 10^{-6}$).}%
  \begin{tabular}{p{5cm}ccccc}\hline
%
 Konstante & Symbol & Wert & $\Delta x/x$ [ppm]  \\ \hline
%
 Vakuumlichtgeschwindigkeit & $c_0$ & $\mathrm{299792458\,m \, s^{-1}}$ & exakt \\
%
 Permeabilit"at des Vakuums  & $\mu_0$ & $4\pi \cdot 10^{-7}\mathrm{\,N \, A^{-2}}$ & exakt \\
%
 Permittivit"at des Vakuums  & $\epsilon_0 $ & $\mathrm{8.854187817\cdot 10^{-12}\,F \, m^{-1}}$ & exakt \\
%
 Gravitationskonstante  & $G, \, \gamma$ & $\mathrm{6.67259\cdot 10^{-11}\,m^3 \, kg^{-1} \, s^{-2}}$ & 128 \\
%
 Planck Konstante  & $h$ & $\mathrm{6.6260755\cdot 10^{-34}\,J\, s}$ & $\mathrm{0.60}$ \\
                   & $h$ & $\mathrm{4.1356692\cdot 10^{-15}\,eV \, s}$ & $\mathrm{0.30}$ \\
%
 Elementarladung   & $e$ &$\mathrm{1.60217733\cdot 10^{-19}\,C}$ & $\mathrm{0.30}$ \\
%
 Hall-Widerstand & $R_{\rm H}$ & $\mathrm{25812.8056\,\Omega}$ & $\mathrm{0.045}$ \\
%
 Bohr Magneton & $\mu_{\rm B}=\nicefrac{e\hslash}{2m_e}$ & $\mathrm{5.78838263\cdot 10^{-5}\,eV/T}$ & $\mathrm{0.089}$ \\
%
 Feinstrukturkonstante & $\alpha=\nicefrac{\mu_0 ce^2}{2h}$ & $\mathrm{0.00729735308}$ & $\mathrm{0.045}$ \\
%
      & $\alpha^{-1}$ & $\mathrm{137.0359895}$ & $\mathrm{0.045}$ \\
%
%
 Rydberg Konstante & $R_\infty$ & $\mathrm{10973731.534\,m^{-1}}$ & $\mathrm{0.0012}$ \\
%
                   & $cR_\infty$ & $\mathrm{3.2898419499e15\,Hz}$ & $\mathrm{0.0012}$ \\
%
                   & $h c R_\infty$ & $\mathrm{13.6056981\,eV}$ & $\mathrm{0.30}$ \\
%
 Bohr Radius & $a_0=\nicefrac{\alpha}{4 \pi R_\infty}$ & $\mathrm{0.529177249\cdot 10^{-10}\,m}$ & $\mathrm{0.045}$ \\
%
 Elektronenmasse & $m_e$ & $\mathrm{9.1093897\cdot 10^{-31}\,kg}$ & $\mathrm{0.59}$ \\
%
 Avogadro Konstante & $N_{\rm A}$ & $\mathrm{6.0221367\cdot 10^{23}\,mol^{-1}}$ & $\mathrm{0.59}$ \\
%
 Atomare Masseneinheit & $m_u$ & $\mathrm{1.6605402\cdot 10^{-27}\,kg}$ & $\mathrm{0.59}$ \\
%
          & $m_u$ & $\mathrm{931.49432\,MeV}$ & $\mathrm{0.30}$ \\
%
 Faraday Konstante & $F$ & $\mathrm{96485.309\,C\, mol^{-1}}$ & $\mathrm{0.30}$ \\
%
 Molare Gaskonstante & $R$ & $\mathrm{8.314510\,J \, mol^{-1} \, K^{-1}}$ & $\mathrm{8.4}$ \\
%
 Boltzmann Konstante & $k_{\rm B}$ & $\mathrm{1.380658\cdot 10^{-23}\,J \, K^{-1}}$ & $\mathrm{8.5}$ \\
%
           & $k_{\rm B}$ & $\mathrm{8.617385\cdot 10^{-5}\,eV \, K^{-1}}$ & $\mathrm{8.4}$ \\
%
 Molvolumen Ideales Gas\footnote{Normalbedingungen} & $V_{\rm m}$ & $\mathrm{22414.10\,cm^3 \, mol^{-1}}$ & $\mathrm{8.4}$ \\
%
 Loschmidt Konstante & $n_0=\nicefrac{N_A}{V_m}$ & $\mathrm{2.686763\cdot 10^{25}\,m^{-3}}$ & $\mathrm{8.5}$ \\
%
 Stefan-Boltzmann Konstante & $\sigma$ & $\mathrm{5.67051\cdot 10^{-8}\,W\, m^{-2} \, K^{-4}}$ & $\mathrm{34}$ \\
%
 Wien Konstante & $b=\lambda_{\rm max}T$ & $\mathrm{0.002897756\,m \, K}$ & $\mathrm{8.4}$ \\
%
 \hline
  \end{tabular}
\end{table}



%%%%%% noch Tabelle mit Umrechnungen ?

%%%%%%%%%%%%%%%%%%%%%%%%%%%%%%%%%%%%%%%%%%%%%%%%%%%%%%%%%%%%%%%%%%%

\chapter{Sicherheit im Praktikum} \label{v:sicherheit}

Nehmen Sie Ihre eigene Sicherheit\index{Sicherheit} und die Ihrer
Kommilitonen sehr wichtig. Auch im Praktikum gibt es viele
Gefahrenquellen (Spannung, Strom, Wasserdampf, Kochplatten, Radioaktivit"at, Druck, Vakuum, \emph{et
cetera}). Bitte machen Sie sich dies immer bewusst und handeln Sie
besonnen. \emph{Immer zuerst denken, dann handeln.} Sind Sie sich
"uber Gefahren, Prozeduren und Vorgehensweisen im Unklaren, wenden
Sie sich bitte zuerst an eine betreuende Person. Generell sind alle
Unfallverh"utungsvorschriften \textsc{UVV}\index{UVV} zu beachten.

Aus Sicherheitsgr"unden m"ussen w"ahrend des Aufenthaltes in den R"aumen
des Praktikums mindestens zwei Studierende anwesend und eine
betreuende Person in unmittelbarer N"ahe sein, damit bei einem Unfall
f"ur eine schnelle und wirksame Erste Hilfe\index{Erste Hilfe}
gesorgt werden kann. F"ur dringende Notf"alle\index{Notf"alle} sind bei
den Telefonen die Notrufnummern\index{Notrufnummern} 110 und 112
freigeschaltet.

Folgende Sicherheitsbestimmungen fassen die f"ur das Praktikum
wichtigsten Punkte zusammen und erheben keinen Anspruch auf
Vollst"andigkeit. Auf Wunsch k"onnen die einschl"agigen
Sicherheitsbestimmungen eingesehen werden.

In den Labors und Praktikumsr"aumen darf weder geraucht noch
gegessen oder getrunken werden.

Im Falle eines Feuers\index{Feuer} ist unverz"uglich eine
betreuende Person zu verst"andigen. Feuerl"oscher befinden sich in
den Fluren. Die Feuerwehr ist unter der Notrufnummer 112 zu
erreichen. Die Feuermelder sind im Notfall auf dem Weg aus dem
Geb"aude zu bet"atigen. Bei einem Feueralarm ist das Geb"aude auf den
gekennzeichneten Fluchtwegen z"ugig, aber ruhig zu verlassen. Man
muss sich am entsprechenden Sammelpunkt\index{Sammelpunkt} vor dem
Geb"aude einfinden (Mitte des Parkplatzes), damit festgestellt
werden kann, ob alle Personen das Praktikum verlassen haben. Es
gilt der generelle Grundsatz "`Personenschutz geht vor
Sachschutz"'.

Auch bei Unf"allen\index{Unfall} oder Verletzungen ist sofort eine
betreuende Person zu benachrichtigen. Ein Verbandskasten ist in
den Praktikumsr"aumen vorhanden. Die Notrufnummern 110 und 112 sind
freigeschaltet. Ein Notfallblatt mit entsprechenden Telefonnummern
ist an verschiedenen Stellen ausgeh"angt. Die betreuende Person
muss Unf"alle und Verletzungen sofort weitermelden.

Werden Sch"aden an einer Apparatur oder an einem Ger"at
festgestellt, d"urfen diese nicht weiter verwendet werden. Bitte
sofort eine betreuende Person benachrichtigen.

Bananenstecker geh"oren keinesfalls in Steckdosen! Bei Aufbau und
Arbeiten an elektrischen Schaltungen ist die Schaltung zuerst in
einen spannungsfreien Zustand zu bringen, d.h. Netzger"at nicht einschalten. Schaltungen sind vor deren
Einsatz durch eine betreuende Person zu kontrollieren.
Schwingkreise, Spulen und Kondensatoren k"onnen auch nach Abschalten
der Spannung noch eine l"angere Zeit Spannung f"uhren. Sollte ein
elektrischer Unfall passieren, ist sofort der NOT"=AUS
Schalter\footnote{Der NOT-AUS Schalter (Roter Knopf auf gelbem
Grund), welcher den ganzen Raum stromlos schaltet, befindet sich
immer direkt neben der Raumt"ur. Bei Gefahr einfach eindr"ucken.} zu
bet"atigen und dann Hilfe zu leisten. Danach sofort eine betreuende
Person verst"andigen oder weitere Hilfe veranlassen.

Beim Umgang mit Chemikalien und anderen
Gefahrstoffen\index{Gefahrstoffverordnung} sind die
Gefahrstoffverordnung und weitere Vorschriften zu beachten. Beim
Umgang mit radioaktiven Stoffen und ionisierender Strahlung ist die
Strahlenschutzverordnung\index{Strahlenschutzverordnung StrSchV}
(StrSchV) und R"ontgenverordnung\index{R"ontgenverordnung R"oV} (R"oV)
zu beachten. Beide liegen in den jeweiligen R"aumen aus. Da diese
Gesetze auch besondere Regeln f"ur Schwangere enthalten, m"ussen
Schwangerschaften dem Praktikumsleiter gemeldet werden. Am Versuch
"`Radioaktivit"at"' darf dann nicht teilgenommen werden.

%Der Umgang und das Hantieren mit tiefkalten Gasen (fl"ussiger
%Stickstoff) darf nur durch eine betreuende Person erfolgen.
%Fl"ussiger Stickstoff kann schwerwiegende Verbrennungen verursachen
%und durch die Verdr"angung des Sauerstoffs auch zu Sauerstoffmangel
%bis hin zum Ersticken f"uhren. Deshalb ist beim Umgang mit fl"ussigem
%Stickstoff immer f"ur ausreichende L"uftung zu sorgen.

Kochendes Wasser, Wasserdampf und hei"se Kochplatten stellen ein
Gefahrenpotenzial f"ur schwere Verbrennungen dar. Unter Druck
stehender Wasserdampf (Versuch "`Dampfdruck"') ist noch eine Stufe
gef"ahrlicher.

%Laserlicht\index{Laser} ist "au"serst intensiv und kann bei direkter
%Einstrahlung in das Auge zu Sch"adigungen, bis hin zur Erblindung,
%f"uhren. Im Praktikum werden Laser der Klasse~2 verwendet. Gehen
%Sie bitte entsprechend vorsichtig damit um.

%Druckgasflaschen stehen unter sehr hohem Druck, sie sind nur durch
%eine betreuende Person zu benutzen. Die Bet"atigung
%der zentralen Gasarmaturen in den Praktikumsr"aumen erfolgt
%ausschlie"slich durch den Praktikumstechniker.

Die Betreuerinnen und Betreuer sind gehalten, vor jedem Versuch
nochmals die Sicherheitsaspekte zum Versuch zu erl"autern und deren
Verst"andnis zu "uberpr"ufen.

\emph{Generell gilt: Alle Unf"alle und Verletzungen sind sofort einer
betreuenden Person zu melden, die dann das weitere veranlassen und
den Unfall weitermelden muss.}


%%%%%%%%%%%%%%%%%%%%%%%%%%%%%%%%%%%%%%%%%%%%%%%%%%%%%%%%%%%%%%%%%%%%

\chapter{Anfertigung eines Versuchsprotokolls}
\label{c:protokoll}

Das Protokoll ist nicht nur ein wichtiger Aspekt im Praktikum,
es ist auch eine "`Visitenkarte"' f"ur Ihre Arbeit im Praktikum. Das
Protokoll fasst Ihre Ergebnisse des Versuches zusammen und soll das
Nachvollziehen des Versuches erm"oglichen. Eine klare Gliederung und
eine pr"agnante Formulierung ist anzustreben. Die Fertigkeit, einen Versuch
gut darzustellen (mit Messprotokoll, Ergebnis, Diskussion) ist eine der wichtigen
Schl"usselkompetenzen Ihrer Ausbildung, die durch das Praktikum vermittelt werden.

Die "au"sere Form des Protokolles sollte zudem ein sauberes
Schriftbild und Seitenbild und einen Heft- und Korrekturrand
beinhalten. Das Protokoll ist zu heften.

%Die Muster f"ur ein Protokoll\index{Protokoll!Muster} (f"ur \LaTeX{}
%und \acro{WORD}) k"onnen von den Praktikumswebseiten herunter geladen
%werden. Wir schlagen folgenden schematischen Aufbau und Inhalt f"ur
%ein Versuchsprotokoll vor:
%
%\vspace{1cm}
%
%\noindent{\bf \Large Titel}
%
%\noindent Versuchstitel und Nummer:  \hfill  Datum der
%Durchf"uhrung:
%
%\noindent Praktikant/-in: Name, E-Mail: e-mail
%\hfill Mitarbeiter/in:
%
%\noindent Assistent: Name des Assistenten  \hfill Platz f"ur
%Stempel, Testat und Unterschrift
%

%\section{Einleitung}

%Was wird gemessen?  Was ist die Motivation? Hinweise auf Literatur.

%F"ur alle Teile des Protokolls (Messprotokoll, Auswertung, Diskussion) ist darauf zu achten, da{\ss} die benutzten/gemessenen Gr"o{\ss}en, benutzten Formeln, usw. klar ersichtlich sind. Bitte "uberlegen Sie, ob eine kleine Zeichnung zum Verst"andnis hilfreich w"are.

\section{Messprotokoll}

Das Messprotokoll wird schon w"ahrend der Durchf"uhrung des Versuchs angelegt, es bildet die Grundlage f"ur die folgende Auswertung des Versuchs und muss vom Assistenten abgezeichnet werden. Ab diesem Moment ist das Messprotokoll als offizielles Dokument anzusehen und darf nicht mehr ver"andert werden.

Zus"atzlich zu den Messwerten selbst, enth"alt das Messprotokoll eine Absch"atzung der systematischen Unsicherheit der Messfehler. Hierunter fallen Ungenauigkeiten bei der Ablesung eines Messwertes ebenso wie m"ogliche Schwierigkeiten mit dem Messaufbau selbst.

Achten Sie darauf, dass das Messprotokoll sauber geschrieben und der Auswertung beigeheftet wird. Werte im Messprotokoll werden mit Tinte (Kugelschreiber oder F"uller) geschrieben, nicht mit Bleistift. Sollten Sie einzelne Werte nachmessen oder haben Sie bei einer Messung einen Fehler gemacht, so wird der alte Wert im Messprotokoll durchgestrichen, auf keinen Fall ausradiert oder mit Tintenkiller oder TippEx unkenntlich gemacht.

\section{Auswertung}

Die Auswertung erfolgt nach dem Versuch zu Hause. Sie soll von der Zweiergruppe, die den Versuch durchgef"uhrt hat, gemeinsam erstellt werden.

Bei der Auswertung ist darauf zu achten, dass benutzte Formeln, Konstanten, etc. eindeutig erkennbar und leicht aufzufinden sind. Ausserdem m"ussen auch Rechnungen angegeben werden, anstatt nur eines Ergebnisses. Dies tr"agt zur allgemeinen Lesbarkeit des Protokolls bei und erleichtert dem Assistenten die Korrektur.

Bitte schreiben Sie in vollst"andigen S"atzen, anstatt nur Stichpunkte anzugeben. 

Wenn das Messprogramm abweichend von der Versuchsanleitung durchgef"uhrt wird, so ist dies unbedingt im Protokoll zu vermerken. Ebenso sind Skizzen des Versuchsaufbaus oder elektrischer Schaltungen Teil des Protokolls, sofern sie nicht in der Versuchsanleitung enthalten sind.

\begin{hint}
	Jede Messung ist mit einer Unsicherheit behaftet und daher auch jedes Messergebnis. Auch wenn die Fehlerrechnung nicht sehr beliebt ist, so ist sie dennoch ein wichtiger Teil jeder Versuchsauswertung. Erst die Angabe eines 	Fehlers auf Ihr Messergebnis macht den Vergleich mit einem Literaturwert m"oglich.
\end{hint}

Die Messergebnisse sind immer auf Plausibilit"at zu "uberpr"ufen. Wenn zum Beispiel bei der Messung des Durchmessers eines Haares etwas von der Gr"o{\ss}enordnung ein Meter rauskommt, ist Ihnen wohl ein Fehler unterlaufen. Ebenso gibt die Betrachtung der Einheiten eines Rechenergebnisses (\textit{Dimensionsanalyse}) Hinweise auf Fehler in der Rechnung. 

\section{Angabe von Messergebnissen}

Bei der Angabe von Zahlenwerten von Messergebnissen ist darauf zu achten, dass die Genauigkeit, i.e. die Anzahl der Nachkommastellen, sinnvoll gew"ahlt ist. Wenn Sie zum Beispiel mit einem Lineal mit Millimetereinteilung eine L"ange in der Gr"o{\ss}enordnung Zentimeter messen, macht es keinen Sinn, das Messergebnis mit mehr als zwei Nachkommastellen anzugeben. Dies w"urde n"amlich eine Messgenauigkeit von unter 0,1~mm andeuten, die mit dieser Methode einfach nicht zu erreichen ist.\\
In diesem Beispiel w"are also eine gute Angabe:
\begin{equation*}
	L = 24,34 \pm 0,05~\mathrm{cm}
\end{equation*}

Auch wenn Sie die Unsicherheit einer indirekt gemessene Gr"o{\ss}e berechnen, macht es wenig Sinn, alle Nachkommastellen anzugeben, die der Taschenrechner ausspuckt. In diesem Fall rundet man den Fehler auf, auf die gr"o{\ss}te signifikante Stelle, d.h. die erste Stelle (vor oder nach dem Komma), die nicht gleich Null ist. Am Beispiel der Fl"ache eines Rechtecks, gemessen mit eine Lineal mit Millimetereinteilung:
\begin{equation*}
	\sigma_F = 0,0236~\mathrm{cm^2} \approx 0,03~\mathrm{cm^2}
\end{equation*}
Wenn mit dem so berechneten Fehler noch weiter gerechnet wird, macht es Sinn, eine weitere Stelle anzugeben, in unserem Beispiel also $\sigma_F\approx 0,024~\mathrm{cm^2}$. So wird verhindert, dass die endg"ultige Unsicherheit durch zu grobes Runden k"unstlich vergr"o{\ss}ert wird.

In diesem Fall einer indirekt gemessenen Gr"o{\ss}e ergibt sich die Anzahl an Stellen (vor oder nach dem Komma), die beim Ergebnis angegeben werden, wieder aus der erreichten Genauigkeit, i.e. aus der Gr"o{\ss}enordnung des Fehlers. Beim Ergebnis werden n"amlich genausoviele Stellen angegeben, wie bei der Unsicherheit. Am Beispiel der Fl"ache eines Rechtecks:
\begin{equation*}
	F = 12,26746~\mathrm{cm^2} \approx 12,27~\mathrm{cm^2}
\end{equation*}

Das Ergebniss der Messung der Fl"ache des Rechtecks w"urde also lauten:
\begin{equation*}
	F = 12,27 \pm 0,03~\mathrm{cm^2}.
\end{equation*}

%\section{Anfertigung graphischer Darstellungen}

%Graphische Darstellungen von Messergebnissen sind grunds"atzlich von Hand auf Millimeterpapier auszuf"uhren, da diesen ein immenser p"adagogischer Wert zukommt. Eingescanntes Millimeterpapier steht auf StudIP zum Download bereit.

%Die Form von Diagrammen ist wohldefiniert und soll hier trainiert werden. Daher muss jedes Diagramm ordentlich beschriftet sein. Zur Beschriftung geh"oren sowohl die Angabe der dargestellten Gr"o{\ss}e mit Einheit, als auch die Einteilung der entsprechenden Achse.

\section{Checkliste}

Benutzen Sie die nachfolgende Checkliste, um zu überprüfen ob das Protokoll, welches Sie angefertigt haben vollständig ist. Die Betreuerinnen und Betreuer benutzen dieselbe Checkliste bei ihrer Korrektur des Protokolls.
\begin{itemize}[label={$\square$}]
	\item Deckblatt vollständig (Namen und Unterschrift der Teilnehmer, Name des Betreuers, Datum). \\
		\textbf{Benutzen Sie bitte das Deckblatt auf der nächsten Seite.}
	%
	\item Ordentliche Form
	\begin{itemize}[label={$\square$}]
		\item Wenn handschriftlich, dann lesbare Handschrift
		%
		\item Sinnvolle Gliederung
		%
		\item Ganze Sätze
	\end{itemize}
	%
	\item Wichtigste Punkte der Versuchsdurchführung beschrieben
	\begin{itemize}[label={$\square$}]
		\item Was wurde gemacht? Wieso?
		%
		\item Was hat nicht geklappt?
	\end{itemize}
	%
	\item Auswertung komplett und richtig?
		\begin{itemize}[label={$\square$}]
			\item Formeln zur Ergebnis- und Fehlerberechnung angegeben und richtig
			\begin{itemize}[label={$\square$}]
				\item Rechenschritte nachvollziehbar
				%
				\item Benutzte Variablen definiert, einheitlich
			\end{itemize}
			%
			\item Plots korrekt
			\begin{itemize}[label={$\square$}]
				\item Achsenbeschriftung, -einteilung, evtl. Legende
				%
				\item Fehlerbalken
				%
				\item Ausgleichsgerade, inkl. Steigungsdreiecken und Fehlergeraden
				%
				\item sauber, ordentlich auf Millimeterpapier gezeichnet
			\end{itemize}
			%
			\item Zwischenschritte nachvollziehbar
			%
			\item Ergebnis inkl. Fehler angegeben und kenntlich gemacht
			%
			\item Ergebnisdiskussion: Vergleich mit Literaturwert oder Erwartung, Plausibilität des Ergebnisses, Fehlerbetrachtung
		\end{itemize}
	%
	\item Protokoll links oben getackert
\end{itemize}

\includepdf[pages=-]{00_einl/Deckblatt.pdf}
\chapter{Auswertung von Messungen: Messunsicherheit} \label{v:fehlerJW}

%**********************************************************************************************************************
%**********************************************************************************************************************
\section{Allgemeines}

Der folgende Abschnitt zu Messungen und Messunsicherheiten ist sehr wichtig nicht nur in der Physik, sondern in allen Fächern, in denen quantitative Messungen ausgewertet und sinnvoll interpretiert werden sollen. Zunächst wollen wir verdeutlichen, was unter einer Messung überhaupt zu verstehen ist. So trivial das klingt, so hilft dieses formalisierte Verständnis doch später dabei, Begriffe wie \textit{wahrer Wert, Bestwert, usw.} zu verstehen. Im Anschluss daran werden wir uns mit der Messunsicherheit und deren Behandlung beschäftigen.

%Im Folgenden werden wir versuchen, den irreführenden Begriff des Messfehlers zu umgehen, da die Unsicherheit eines Messwertes nichts mit Fehlern während der Messung zu tun hat. Die Begriffe werden später noch genauer erklärt.

%**********************************************************************************************************************
%**********************************************************************************************************************
\section{Messung}

Die Messung einer physikalischen Größe bedeutet, diese Größe mit einer \textit{Einheit dieser Größe} zu vergleichen. Bei einer Längenmessung mit Lineal etwa, ist dessen kleinste Untereinteilung, z. Bsp. ein Millimeter, die Einheit der gemessenen Länge. Das Messergebnis ist also eine Länge in Millimetern.

Wiederholt man die Messung unter den gleichen Bedingungen, so weichen die Messwerte \textbf{immer} voneinander, und damit auch vom \textit{wahren Wert $\mu$} der Messgröße, ab. Das liegt daran, dass eine Messung niemals beliebig genau sein kann. Es hat also nichts mit einem Fehlverhalten oder einer falschen Messung zu tun, dass ein Ergebnis vom wahren Wert abweicht. Deshalb werden wir im Folgenden versuchen, den historisch benutzten Begriff des \textit{Messfehlers} zu vermeiden und stattdessen den Begriff der \textit{Unsicherheit} zu benutzen. Beide Begriffe sind in DIN 1319 wohldefiniert, und gesetzlich nicht synonym.

Die Aufgabe ist es also, aus den Messwerten den bestmöglichen Schätzwert für den wahren Wert der Größe, den \textit{Bestwert}, sowie ein Maß für dessen \textit{Unsicherheit}, zu bestimmen. 

Da es unmöglich ist, die Unsicherheit auf Null zu reduzieren, ist die Angabe eines Schätzwertes ohne seine Unsicherheit wissenschaftlich unsinnig. Die Unsicherheit einer Größe $x$ wird durch ein vorangestelltes $\Delta$ gekennzeichnet, hier also $\Delta x$. Die Angabe des Ergebnisses geschieht also in der Form
\begin{equation}
	\left( x\pm\Delta x\right)\;\mathrm{Einheit}
\end{equation}

Diese Angabe ist gleichbedeutend mit der Aussage, dass der wahre Wert der Größe der gemessenen Größe mit einer genau bekannten Wahrscheinlichkeit im Intervall $\left[x-\Delta x, x + \Delta x\right]$ liegt. Dabei ist der \textit{Fehlerintervall} als homogen zu betrachten, d.h. keiner der Werte innerhalb des Intervalls ist gegenüber den anderen Werten ausgezeichnet (keiner ist ''richtiger'' als andere) und der Zentralwert $x$ als ''Ergebniswert'' im engeren Sinne ist nicht besser oder gewichtiger als die Grenzwerte $x+\Delta x$ oder $x-\Delta x$.

%**********************************************************************************************************************
%**********************************************************************************************************************
\section{Messunsicherheiten}

Die meisten Unsicherheiten mit denen wir es im Praktikum, oder mit denen es Naturwissenschaftler in Forschung und Anwendung, zu tun bekommen, beruhen auf Unvollkommenheiten unserer Messgeräte, sowie unserem Umgang mit ihnen. Jedes Messgerät hat eine (bauartbedingte) Genauigkeit, welche die Messung begrenzt. So kann man mit einem Lineal mit Millimetereinteilung nicht sinnvoll Längen messen, die wesentlich kleiner als ein Millimeter sind.

Neben der Genauigkeit der Messgeräte spielt aber auch deren Einfluss auf die Messgröße selbst eine Rolle. Im Versuch ''Nicht stationäre Diffusion'' messen Sie zum Beispiel wie Tinte in Wasser diffundiert anhand der Lichtdurchlässigkeit der Mischung der beiden Flüssigkeiten. Da, wo Tinte hin diffundiert ist, lässt die Mischung weniger Licht durch. Die Messung geschieht, indem Sie mit einer Lampe auf die Flüssigkeit scheinen und die Helligkeit auf der anderen Seite messen. Wenn Sie die Flüssigkeit mit Licht bescheinen, erwärmt sich diese allerdings, was direkt die Diffusion beeinflusst und damit zu einem (wenn auch nur leicht) verfälschten Ergebnis führt.\\


Bei der Behandlung der Messunsicherheit ist es zweckmäßig, zwischen zwei Typen von Unsicherheiten\footnote{Auch wenn die hier gemachte Zuordnung statistische = unkorrelierte Unsicherheit, und systematische = korrelierte Unsicherheit in dieser Einfachheit sicher nicht ganz korrekt ist, ist sie für die Anwendung im Praktikum ausreichend. Vom mathematischen Gesichtspunkt ist die Korrelation zwischen den Größen und Unsicherheiten ausschlaggebend.} zu unterscheiden:
%**********************************************************************************************************************
\subsection{Systematische Unsicherheiten}

Systematische Unsicherheiten entstehen aus Effekten des Messvorgangs selbst und sind daher korreliert. Für unsere Zwecke bedeutet das, dass der Effekt für alle Einzelmessungen derselbe ist. Daraus wird klar, dass, wenn man den Effekt sehr genau kennt, es möglich ist, ihn zu korrigieren.\\
Da der Effekt über alle Messwerte korreliert ist, verringert er sich nicht, wenn man die Messung wiederholt.

Ursachen systematischer Unsicherheiten sind zum Beispiel der Einfluss des Messgerätes auf die Messgröße und falsche Eichung oder Kalibrierung des Messgeräts.\\

Beispiel: \textit{Das zur Längenmessung benutzte ''Metermaß'' ist tatsächlich nur 999~mm lang.}\\

Die Beurteilung systematischer Abweichungen, die das Messergebnis meist einseitig ver-fälschen, erfordert eine kritische Analyse aller relevanten Umstände. Es ist eine wichtige Aufgabe des Experimentators, die systematischen Abweichungen zu erkennen und zu minimieren. Da man diese nie ganz ausschalten kann, gehört eine möglichst genaue Abschätzung der verbleibenden systematischen Unsicherheit zum jedem Experiment dazu.

Allgemein gültige Regeln können dabei leider nicht gemacht werden, da die systematischen Effekte für jeden Aufbau unterschiedlich sind. Deren Abschätzung erfordert also eine kritische Auseinandersetzung mit dem jeweiligen Aufbau.
%**********************************************************************************************************************
\subsection{Statistische Unsicherheiten}

\label{ssect:StatistUnsicher}
Statistische Unsicherheiten entstehen dadurch, dass die einzelnen Messwerte bei wiederholter Messung niemals genau miteinander übereinstimmen. Gründe hierfür sind zum einen die nur endlich genaue Auflösung des Messgerätes (z. Bsp. Millimetereinteilung eines Maßbandes) und zum anderen unkontrollierbare zufällige Änderungen der Messgröße selbst (''Rauschen''). Die Abweichungen sind zwischen verschiedenen Messungen nicht korreliert, das bedeutet das Abweichungen verschiedene Beträge und Richtungen haben. Im Mittel finden ebenso viele Abweichungen um $+\Delta x$ statt, wie um $-\Delta x$. Daher können statistische Unsicherheiten durch wiederholte Messung unter gleichen Bedingungen unterdrückt werden, wenn man den Mittelwert all dieser Messungen betrachtet.

%**********************************************************************************************************************
%**********************************************************************************************************************
\section{Direkte Messung} \label{chap:Direkt}

Eine direkte Messung liegt vor, wenn die gesuchte Größe nicht aus mehreren Messgrößen zusammengesetzt ist. Ein Beispiel wäre die Messung der Länge eines Stocks.

Direkte Messungen können prinzipiell unendlich oft wiederholt werden. Die Gesamtheit aller (unendlich vieler) Messwerte nennt man in der Statistik dann die \textit{Grundgesamtheit}. Praktisch wiederholt man eine Messung natürlich nur endlich oft, z. Bsp. $n$ mal. Die Messwerte $l_i (i=1...n)$ ergeben dann eine Untermenge der Grundgesamtheit, man nennt das eine \textit{Stichprobe}. 

Eine gängige Darstellung solcher Messwerte ist das \textit{Histogramm}, bei dem auf der x-Achse die Messwerte aufgetragen sind, auf der y-Achse die Häufigkeit, mit der der Messwert in der Stichprobe vorkommt.


%**********************************************************************************************************************
\subsection{Der Mittelwert}

Der Mittelwert, um den die Messwerte verteilt sind, nähert sich mit wachsender Anzahl $n$ von Messungen dem wahren Wert $\mu$, ist also ein guter Schätzwert für den wahren Wert basierend auf der Stichprobe der Messungen. Wenn keine besonderen Umstände vorliegen, die eine spezielle Gewichtung der Abweichung notwendig machen, verwendet man als Schätzwert das \textit{arithmetische Mittel}\footnote{Das Symbol $\sum_{i=1}^n$ bezeichnet die Summe über die Größen $x_i$, wobei die Werte aufsummiert werden, deren Zählindex $i$ zwischen 1 und $n$ liegt.}:
\begin{hint}
	\begin{equation}
		\bar{x} = \frac{1}{n}\sum^n_{i=1} x_i = \frac{\left(x_1 + x_2 + ... + x_n\right)}{n} 
	\end{equation}
\end{hint}

Als Messergebnis gibt man also den so definierten Mittelwert $\bar{x}$ an.

%**********************************************************************************************************************
\subsection{Die Standardabweichung} \label{chap:Standardabweichung}

Das arithmetische Mittel $\bar{x}$ hat die Eigenschaft, dass die Summe der Abweichungen $\Delta x_i = \left( x_i - \bar{x}\right)$ der Einzelmessungen gerade verschwindet
\begin{equation}
	\sum^n_{i=1} \Delta x_i = \sum^n_{i=1} \left(x_i - \bar{x}\right) = 0
\end{equation}
Dies folgt daraus, dass, wie in Abschnitt \ref{ssect:StatistUnsicher} erwähnt, für ein große Anzahl $n$ an Messungen, die statistischen Schwankungen symmetrisch um den Mittelwert erfolgen. Die Summe der Quadrate der Abweichungen verschwindet hingegen nicht.

Als Maß für die Breite der Streuung der Messwerte um den Mittelwert definiert man (für die Stichprobe) die \textit{Standardabweichung} $\sigma$:
\begin{hint}
	\begin{equation}
		\sigma_x = \sqrt{\frac{1}{n-1}\sum^n_{i=1}\left(x_i - \bar{x}\right)^2}
	\end{equation}
\end{hint}

$\sigma_x$ ist eine positive Größe, die genau dann verschwände, wenn alle Messwerte übereinstimmten, was allerdings nie passiert. 

$\sigma_x$ liefert eine Schätzung der Abweichung der Messwerte vom wahren Wert $\mu$, der allerdings unbekannt ist. Daher wird zur Berechnung von $\sigma_x$ der Mittelwert $\bar{x}$ benutzt. Für eine Einzelmessung ist diese Schätzung nicht durchführbar, was durch den Faktor $1/n-1$ berücksichtigt wird. Für $n=1$ ist $\sigma_x$ also mathematisch nicht definiert.

\paragraph{Bemerkung:}

Die Formel für die Standardabweichung kann so umgestellt werden, dass der Mittelwert zur Berechnung nicht benötigt wird. Dies ist insbesondere dann von Vorteil, wenn die Unsicherheit schon während der Messwertaufnahme berechnet werden soll, der Mittelwert also noch gar nicht bekannt ist. Zu diesem Zweck kann folgende Formel verwendet werden
\begin{equation*}
\sigma_x = \sqrt{\frac{1}{n(n-1)}\left[n\sum_{i=1}^n{x_i^2} - \left(\sum_{i=1}^n{x_i} \right)^2\right]}
\end{equation*}

%**********************************************************************************************************************
\subsection{Die Normalverteilung}

Die Standardabweichung gibt ein Maß für die mittlere Abweichung der Messwerte vom wahren Wert an. Natürlich gibt es Messwerte, die weniger abweichen und solche die mehr abweichen. Für eine sehr große Anzahl an Messwerten, $n\rightarrow \infty$, kommt jeder einzelne Messwert mit der Häufigkeit $h$ vor. Diese nennt man die \textit{Wahrscheinlichkeitsdichte}. \\
Aufgrund des \textit{zentralen Grenzwertsatzes} wird in den meisten Fällen die Häufigkeit durch die \textit{Gaußsche Normalverteilung} beschrieben:
\begin{hint}
	\begin{equation}
		h(x) = \frac{1}{\sqrt{2\pi}\sigma}\cdot e^{-\frac{\left(x-\bar{x}\right)^2}{2\sigma^2}}
	\end{equation}
\end{hint}

Die Gauß-Funktion (''Glockenkurve'')  ist symmetrisch um den Mittelwert $\bar{x}$ und so normiert, dass das Integral von $-\infty$ bis $\infty$ gerade 1 ergibt. 

Tabelle \ref{tab:MessungDINA4} enthält die Messwerte von $n=30$ Messungen der Breite $x_i$ eines DIN A4 Blattes, in Abbildung \ref{fig:messung_histo} sind die Werte der Breite $x$ als Histogramm dargestellt, inklusive der entsprechend angepassten Normalverteilung.

\begin{minipage}{.45\textwidth}
%\begin{table}[h]	
	\centering
		\begin{tabular}[t]{|c|c||c|c||c|c|} 
%			\hline
			i & x [cm] & i & x [cm] & i & x [cm]\\
			\hline
			1 & 21,10 & 11 & 21,00 & 21 & 21,05\\
			2 & 20,95 & 12 & 21,00 & 22 & 20,95\\
			3 & 21,00 & 13 & 20,95 & 23 & 21,00\\
			4 & 21,00 & 14 & 21,05 & 24 & 21,00\\
			5 & 20,90 & 15 & 20,85 & 25 & 21,00\\
			6 & 21,00 & 16 & 21,00 & 26 & 21,05\\
			7 & 21,00 & 17 & 20,95 & 27 & 21,00\\
			8 & 21,00 & 18 & 21,00 & 28 & 21,00\\
			9 & 21,05 & 19 & 20,95 & 29 & 21,05\\
			10& 20,90 & 20 & 21,00 & 30 & 21,00\\
%			\hline
		\end{tabular}
	\captionof{table}{Messung der Höhe und Breite eines DIN A4 Blattes}
	\label{tab:MessungDINA4}
%
\end{minipage}\hfill
%
\begin{minipage}{.45\textwidth}
	\includegraphics[width=1.00\textwidth]{00_einl/messung_histo.pdf}
	\captionof{figure}{Häufigkeitsverteilung der Messwerte aus Tabelle \ref{tab:MessungDINA4}}
	\label{fig:messung_histo}
\end{minipage}

Die Breite der Häufigkeitsverteilung wird durch die Standardabweichung $\sigma$ beschrieben, je größer diese ist, desto breiter ist die Kurve. Der Wendepunkt der Kurve befindet sich im Abstand $\sigma$ vom Mittelwert. Berechnet man die Fläche unter der Kurve zwischen den beiden Wendepunkten, so ergibt sich:
\begin{equation}
	\frac{1}{\sqrt{2\pi}\sigma}\int_{\bar{x}-\sigma}^{\bar{x}+\sigma}{e^{-\frac{\left(x-\bar{x}\right)^2}{2\sigma^2}}dx} = 0,683
\end{equation}

Das bedeutet, dass die Wahrscheinlichkeit, einen Messwert im Intervall $\left[\bar{x}-\sigma; \bar{x}+\sigma\right]$ zu finden (für $n\rightarrow\infty$), 68,3\% beträgt.

\noindent
Andere oft genutzte Intervalle sind:
\begin{align*}
	\mathrm{1-fache Standardabweichung} &\quad \phantom{0}\bar{x}-\sigma \; \mathrm{bis}\; \bar{x}+\sigma & (68,3\%)\\
	\mathrm{2-fache Standardabweichung} &\quad \bar{x}-2\sigma\; \mathrm{bis}\; \bar{x}+2\sigma & (95,5\%)\\
	\mathrm{3-fache Standardabweichung} &\quad \bar{x}-3\sigma\; \mathrm{bis}\; \bar{x}+3\sigma & (99,7\%)
\end{align*}
%**********************************************************************************************************************
\subsection{Die Unsicherheit eines Messergebnisses}

Da wir als Messergebnis den Mittelwert angeben, ist die Messunsicherheit definiert durch die Frage: \textit{Wie weit weicht der Mittelwert $\bar{x}$ aller Messungen vom wahren Wert $\mu$ ab?}

Diese Abweichung nimmt mit zunehmender Anzahl an Messungen immer weiter ab, der Mittelwert nähert sich dem wahren Wert immer mehr an. Man könnte auch sagen, dass der Mittelwert den wahren Wert immer besser beschreibt.\\
Bei einer großen Anzahl an Messungen gilt für die \textit{statistische Unsicherheit des Mittelwertes}:
\begin{hint}
	\begin{equation}
		\Delta{\bar{x}} = \frac{\sigma_x}{\sqrt{n}}
	\end{equation}
\end{hint}

Den Intervall $\left[\bar{x}-\Delta\bar{x}; \bar{x}+\Delta\bar{x}\right]$ nennt man den \textit{Konfidenzbereich auf dem Vertrauensniveau 68,3\%}. Die Aussage ''Der Messwert beträgt $\bar{x}\pm\Delta\bar{x}$'' bedeutet, dass wenn man den Messvorgang wiederholen würde (mit der gleichen Anzahl an Einzelmessungen $n$), für mindestens $68,3\%$ der auf Grundlage der Messdaten berechneten Konfidenzintervalle der wahre Wert im jeweiligen Konfidenzintervall liegt.

\paragraph{Bemerkung:} Korrektur für kleine Anzahl von Messungen

\begin{table}[t]	
	\centering
		\begin{tabular}[t]{|c|c|c|c|} 
			 & 68,3\% & 95\% & 99,7\% \\ 
			n & t & t & t \\\hline
			2 & 1,84 & 12,71 & 235,8 \\
			3 & 1,32 & 4,30 & 19,21 \\
			4 & 1,20 & 3,18 & 9,22 \\
			5 & 1,15 & 2,78 & 6,62 \\
			6 & 1,11 & 2,57 & 5,51 \\
			8 & 1,08 & 2,37 & 4,53 \\
			10 & 1,06 & 2,26 & 4,09 \\
			20 & 1,03 & 2,09 & 3,45 \\
			30 & 1,02 & 2,05 & 3,28 \\
			50 & 1,01 & 2,01 & 3,16 \\
			100 & 1,00 & 1,98 & 3,08 \\
			200 & 1,00 & 1,97 & 3,04 
		\end{tabular}
	\captionof{table}{Werte der Student-t-Verteilung für verschiedene Konfidenzintervalle}
	\label{tab:Student-t}
\end{table}

\noindent
Liegen nur wenige Messungen vor, so ist die Annahme nicht mehr wahr, dass die Verteilung der Messwerte durch die Normalverteilung beschrieben wird. Stattdessen muss die Student-t-Verteilung verwendet werden. Den Unterschied der beiden Verteilungen berücksichtigen wir durch einen Korrekturfaktor $t$, der Tabelle \ref{tab:Student-t} entnommen werden kann.

\noindent
Für die Unsicherheit des Messwertes gilt dann:
\begin{equation}
 \Delta \bar{x} = \frac{t}{\sqrt{n}}\sigma_x = t\cdot \sigma_{\bar{x}}
\end{equation}

Wie man sieht, geht der Korrekturfaktor für $n\rightarrow\infty$ gegen $t=1$, denn für große Anzahlen an Einzelmessungen nähert sich die Student-t-Verteilung der Normalverteilung an.

%**********************************************************************************************************************
\subsection{Der gewichtete Mittelwert und seine Unsicherheit}

Bei der Bildung des arithmetischen Mittelwertes geht man implizit davon aus, dass alle Messwerte, die in die Mittelung aufgenommen werden, dieselbe Unsicherheit haben. Dies muss nicht immer der Fall sein.\\

Beispiel: \textit{Aus irgendeinem Grund verwenden Sie zur Messung derselben Länge unterschiedliche Maßstäbe. Z. Bsp. benutzen Sie für einige Messungen ein Lineal mit einer Genauigkeit von 1~mm, für andere Messungen eine Schieblehre, die eine Auflösung von 0,1~mm hat.}\\

Um diese unterschiedlich genauen Messungen sinnvoll in einem Mittelwert zu kombinieren, benutzt man den \textit{gewichteten Mittelwert}:
\begin{hint}
	\begin{equation}
		\bar{x} = \frac{\sum_{i=1}^n{w_i\cdot x_i}}{\sum_{i=1}^n{w_i}}
	\end{equation}
\end{hint}
wobei die Wichtungsfaktoren $w_i$ die Kehrwerte der Abweichungsquadrate sind:
\begin{equation*}
	w_i = \frac{1}{\Delta x_i^2}
\end{equation*}
Auf diese Weise tragen Messwerte mit größerer Unsicherheit weniger stark zum Mittelwert bei. Die Unsicherheit des gewichteten Mittelwertes beträgt:
\begin{equation}
	\Delta\bar{x} = \sqrt{\frac{1}{\sum_{i=1}^n{w_i}}}
\end{equation}
Sind alle Wichtungsfaktoren $w_i$ gleich, geht der gewichtete Mittelwert in den 'normalen` über, und dessen Abweichung sinkt wieder mit der Wurzel der Zahl der Werte.\\

Der gewichtete Mittelwert gilt strenggenommen nur für normalverteilte Größen. Er wird aber mangels Alternativen auch auf andere statistische oder systematische Abweichungen angewandt. Vor einer Anwendung ist aber zu prüfen, ob die unterschiedlichen Messungen mit dem zugehörigen Konfidenzbereich überlappen. Falls dies nicht der Fall ist, liegt eine zusätzliche, nicht erkannte Abweichung vor (z.B. grobe Fehler).\\

Beispiel:
\textit{Drei Praktikumsgruppen erhalten die Aufgabe, den Durchmesser einer Dose zu bestimmen. Die erste Gruppe erhält eine Schieblehre als Hilfsmittel, die zweite ein 30~cm langes Lineal mit Millimeterskala und die Studierenden der dritten Gruppe müssen mit bloßem Auge den Durchmesser abschätzen:}

\begin{tabular}{|l|l|l|l|} \hline
		& Gruppe 1: & Gruppe 2: & Gruppe 3:\\
		& Durchmesser [mm] & Durchmesser [mm] & Durchmesser [cm]\\ \hline
	Messung 1 & 100,3 & 102,5 & 8 \\
	Messung 2 & 100,2 & 105,0 & 9\\
	Messung 3 & 100,4 & 102,0 & 10 \\
	Messung 4 & 100,1 & 107,0 & 13 \\
	Messung 5 & 100,2 & 100,5 & 6 \\
	Messung 6 & 100,3 & 101,0 & 12 \\
	Messung 7 & 100,5 & 102,5 & 9\\
	Messung 8 & 100,0 & 103,0 & 11\\
	Messung 9 & 100,4 & 102,5 & 12\\
	Messung 10 & 100,3 & 103,5 & 8 \\ \hline
	Mittelwert & 100,270 & 102,95 & 9,8\\ \hline
	Std. abw. & 0,047 & 0,60 & 0,70\\ \hline
\end{tabular}

\textit{Der Mittelwert aus allen diesen Messungen für den Dosendurchmesser $d$ ergibt:}
\begin{align}
	\bar{d} & = \frac{\frac{100,27}{0,047^2}+\frac{102,95}{0,60^2}+\frac{98}{7^2}}{\frac{1}{0,047^2}+\frac{1}{0,60^2}+\frac{1}{7^2}}\;\mathrm{mm} = 100,286\;\mathrm{mm}\\
	\Delta\bar{d} & = \frac{1}{\sqrt{\frac{1}{0,047^2}+\frac{1}{0,60^2}+\frac{1}{7^2}}}\;\mathrm{mm} = 0,047\;\mathrm{mm}
\end{align}
\textit{Als Messwert für den Durchmesser ergibt sich also: $d = (100,286\pm 0,047)\;\mathrm{mm}$.}
%**************************************************************
% Hier noch was zu Geräten und Aufspüren systematischer Fehler?
%**************************************************************

%**********************************************************************************************************************
%**********************************************************************************************************************
\section{Indirekte Messung: Fehlerfortpflanzung}

Wenn sich die gesuchte Größe aus zwei oder mehr einzeln gemessenen Größen zusammensetzt, so nennt man die Bestimmung der zusammengesetzten Größe eine \textit{indirekte Messung}. Ein Beispiel dafür ist schon die einfache Bestimmung der Fläche eines DIN A4 Blattes, die wir uns früher schon angeschaut haben.

Wie wir wissen, sind die einzelnen, direkten Messungen mit Unsicherheiten behaftet. Wie können wir aus diesen die Ungenauigkeit der indirekten Messung der zusammengesetzten Größe berechnen?

%**********************************************************************************************************************
\subsection{Größtfehlerbetrachtung}

Die einfachste Möglichkeit, den Fehler abzuschätzen, ist die Annahme, dass die Unsicherheiten aller Einzelgrößen voll zur Gesamtunsicherheit beitragen und sich nicht untereinander beeinflussen (man sagt \textit{sie sind nicht korreliert}).

Rechnerisch wird das umgesetzt, indem man in die Formel zur Berechnung der gesuchten zusammengesetzten Größe die Einzelmessungen $x_i \pm \Delta x_i$ so einsetzt, dass das Ergebnis maximal wird. Anschließend wird der Vorgang so wiederholt, dass das Ergebnis minimal wird. Die Differenz der beiden Ergebnisse bezeichnet man als den \textit{Größtfehler}.\\

%\noindent
Dieses Verfahren berücksichtigt den statistischen Zusammenhang der einzelnen Beiträge nicht, daher liefert es im Allgemeinen zu große Werte für die Unsicherheit. Darüber hinaus können sich rechnerische Schwierigkeiten ergeben, wenn z. Bsp. die entsprechende Größe in einer Summe im Nenner und im Zähler der Formel vorkommt. In diesem Fall ist es schwierig herauszufinden, ob die positive oder die negative Abweichung den größten bzw. den kleinsten Wert ergibt. Dieses Verfahren ist daher nur für den Notfall geeignet, liefert jedoch oft eine brauchbare Abschätzung.

%**********************************************************************************************************************
\subsection{Gauß'sche Fehlerfortpflanzung}\label{chap:Fehlerfortpflanzung}

Betrachten wir eine Größe $g$, die nicht direkt gemessen werden kann, sich aber aus mehreren direkt gemessenen Größen $x$, $y$, $z$,... berechnen läßt:
\begin{equation}
	g = g(x,y,z,...)
\end{equation}
Der Einfachheit halber wollen wir uns im Folgenden auf die drei Messgrößen $x$, $y$ und $z$ beschränken, die Rechnungen können jedoch prinzipiell mit beliebig vielen Messgrößen durchgeführt werden.

Wir gehen davon aus, dass die Messserien für die Größen $x$, $y$ und $z$ normalverteilt sind. Dann kann der Mittelwert $\bar{g}$ der gesuchten Größe bestimmt werden, indem man die Mittelwerte der Messgrößen in die Formel einsetzt:
\begin{hint}
	\begin{equation}
		\bar{g} = g(\bar{x},\bar{y},\bar{z})
	\end{equation}
\end{hint}

Wenn die Messgrößen $x$, $y$ und $z$ statistisch unabhängig voneinander sind, man sagt \textit{unkorreliert}, dann kann die gesamte Unsicherheit wie folgt berechnet werden:
\begin{hint}
	\begin{equation}
		\Delta\bar{g} = \sqrt{\Delta\bar{x}^2 \cdot \left(\frac{\partial g}{\partial x} \right)^2_{\bar{x},\bar{y},\bar{z}} + \Delta\bar{y}^2 \cdot \left(\frac{\partial g}{\partial y} \right)^2_{\bar{x},\bar{y},\bar{z}} + \Delta\bar{z}^2 \cdot \left(\frac{\partial g}{\partial z} \right)^2_{\bar{x},\bar{y},\bar{z}}}
	\end{equation}
\end{hint}

Diese Formel ist bekannt als die \textit{Gauß'sche Fehlerfortpflanzung}, wenn für die Unsicherheiten $\Delta\bar{x}$, $\Delta\bar{y}$, $\Delta\bar{z}$ die jeweilige Standardabweichung eingesetzt wird.

Die Ausdrücke $\partial g/\partial x$, $\partial g/\partial y$ und $\partial g/\partial z$ stehen dabei für die partiellen Ableitungen der Formel zur Berechnung von $g$ nach den Variablen $x$, $y$ und $z$.

\paragraph{Beispiel:} Messung der Dichte eines Quaders\\
\textit{
Die Dichte eines Quaders ergibt sich aus seiner Masse $m$ und seinem Volumen $V = x\cdot y\cdot z$, wenn $x$, $y$ und $z$ die Kantenlängen sind, nach der Formel
\begin{equation}
	\rho = \frac{m}{V} = \frac{m}{x\cdot y\cdot z}
	\label{eq:Dichte}
\end{equation}
Bei der Messung der Masse und der Kantenlängen findet man die Werte $\bar{m}\pm \sigma_m$, $\bar{x}\pm \sigma_x$, $\bar{y}\pm \sigma_y$ und $\bar{z}\pm \sigma_z$. Die partielle Ableitung von Gleichung \ref{eq:Dichte} nach der Masse $m$ ist:
\begin{equation*}
	\frac{\partial \rho}{\partial m} = \frac{1}{x\cdot y\cdot z}
\end{equation*}
Die partielle Ableitung nach $x$ lautet (für die anderen beiden Richtungen genauso vorgehen):
\begin{equation*}
	\frac{\partial \rho}{\partial x} = -\frac{1}{x}\cdot\frac{m}{x\cdot y\cdot z}
\end{equation*}
Damit ergibt sich für die Unsicherheit der Dichte:
\begin{equation}
	\Delta\bar{\rho} = \sqrt{\Delta\bar{x}^2\left(-\frac{1}{\bar{x}}\frac{m}{\bar{x}\bar{y}\bar{z}} \right)^2 + \Delta\bar{y}^2\left(-\frac{1}{\bar{y}}\frac{m}{\bar{x}\bar{y}\bar{z}} \right)^2 + \Delta\bar{z}^2\left(-\frac{1}{\bar{z}}\frac{m}{\bar{x}\bar{y}\bar{z}} \right)^2 + \Delta\bar{m}^2\left(\frac{1}{\bar{x}\bar{y}\bar{z}}\right)^2}
\end{equation}
In diese Gleichung sind jetzt nur noch die jeweiligen Mittelwerte und Unsicherheiten einzusetzen, um die Unsicherheit der Dichte zu berechnen.
}\\

Falls die einzelnen Messgrößen nicht normalverteilt sind, oder wenn sie nicht statistisch von einander unabhängig sind, muss man bei der Fortpflanzung der Unsicherheiten noch einen Korrekturfaktor berücksichtigen, der diese Korrelation der Messgrößen beschreibt. Diesen nennt man den \textit{Korrelationskoeffizienten}.

%**********************************************************************************************************************
\subsection{Fortpflanzung systematischer Abweichungen}

Systematische Abweichungen entziehen sich naturgemäß einer statistischen Behandlung. Man kann insbesondere nicht davon ausgehen, dass es unwahrscheinlich ist,
dass alle Abweichungen der Beteiligten Einzelgrößen gleichzeitig extremal sind.\\

Beispiel:
\textit{Eine Veränderung der Umgebungstemperatur bei der Messung zieht eine Abweichung aller Längenmessungen gemeinsam nach sich, da der Maßstab natürlich bei jeder Messung
'falsch' ist. Man muss also in diesem Fall davon ausgehen, dass die Abweichungen nicht quadratisch addiert werden dürfen, sondern linear addiert werden müssen. Der Einfluss jeder
Einzelabweichung muss voll auf die Gesamtabweichung durchschlagen.}\\

\textit{Im Fall einer systematischen Abweichung der Längenmessung durch einen nicht gleichmäßig geteilten Maßstab erscheint es jedoch sinnvoll anzunehmen, dass je nach Position die Abweichung unterschiedlich ist. Entsprechend kann man auch davon ausgehen, dass nicht gerade alle Abweichungen zum gleichen Zeitpunkt in die gleiche Richtung gehen. Hier ist also eine quadratische Addition sinnvoll.}\\

Es hängt also vom Einzelfall ab welche Methode der Fortpflanzung bei einer systematischen Abweichung angewandt werden muss. Die lineare Addition bei potentiell statistisch gekoppelten Vorgängen oder die quadratische Addition bei definitiv statistisch unabhängigen Vorgängen.\\
Sei beispielsweise die Größe $g$ aus drei Variablen $x, y, z$ zusammengesetzt, die jeweils die systematischen Unsicherheiten $\Delta x, \Delta y, \Delta z$ haben. Dann erfolgen lineare und quadratische Addition wie folgt:

\textbf{Lineare Addition}
\begin{equation} \label{eq:error_lin_add}
	\Delta\bar{g} = \left|\Delta\bar{x}\cdot \left[\frac{\partial g}{\partial x}\right]_{\bar{x}, \bar{y}, \bar{z}}\right|
								+ \left|\Delta\bar{y}\cdot \left[\frac{\partial g}{\partial y}\right]_{\bar{x}, \bar{y}, \bar{z}}\right|
								+ \left|\Delta\bar{z}\cdot \left[\frac{\partial g}{\partial z}\right]_{\bar{x}, \bar{y}, \bar{z}}\right|
\end{equation}

\textbf{Quadratische Addition}
\begin{equation} \label{eq:error_quad_add}
	\Delta\bar{g} = \sqrt{\Delta\bar{x}^2\cdot \left[\frac{\partial g}{\partial x}\right]^2_{\bar{x}, \bar{y}, \bar{z}}
											+ \Delta\bar{y}^2\cdot \left[\frac{\partial g}{\partial y}\right]^2_{\bar{x}, \bar{y}, \bar{z}}
											+ \Delta\bar{z}^2\cdot \left[\frac{\partial g}{\partial z}\right]^2_{\bar{x}, \bar{y}, \bar{z}}}
\end{equation}
%**********************************************************************************************************************
\subsection{Verknüpfung statistischer und systematischer Unsicherheiten}

Auch bei der Verknüpfung systematischer und statistischer Unsicherheiten ist die Frage nach de Kopplung der Gr"o{\ss}en zu beachten. Die Frage, ob die statistische Unsicherheit von der systematischen Abweichung abhängig ist oder nicht, ist jedoch letztlich nicht beantwortbar. Entsprechend werden in der Literatur auch beide Meinungen vertreten.\\

\noindent
\textbf{Im Praktikum} wollen wir vom konservativen Fall ausgehen und annehmen, dass die beiden Beitr"age nicht voneinander unabh"angig sind und befolgen die folgenden Schritte:
\begin{enumerate}
	\item Bestimmung der statistischen Unsicherheit aus allen Einzelgrößen mittels \textit{quadratischer Addition} nach Gleichung \ref{eq:error_quad_add}.
	\item Bestimmung der systematischen Unsicherheit aus \textit{linearer Addition} aller Einzelgrößen nach Gleichung \ref{eq:error_lin_add}.
	\item Verknüpfung der statistischen und systematischen Unsicherheit mittels normaler linearer Addition, nicht Gleichung \ref{eq:error_lin_add}, zur Gesamtunsicherheit, welche dann beim Ergebnis angegeben wird.
\end{enumerate}
%**********************************************************************************************************************
%\begin{todo}
	%Einen Abschnitt über systematische Unsicherheiten und wie man sie abschätzt.
%\end{todo}

%**********************************************************************************************************************
%**********************************************************************************************************************
%\section{Graphische Auswertung bei korrelierten Messwerten}
%
%Dein einfachste Beziehung zwischen zwei Messgrößen ist die \textit{lineare Abhängigkeit}
%\begin{equation}
	%y(x) = a\cdot x + b
%\end{equation}
%mit den beiden freien Parametern $a$ und $b$.
\chapter{Fehlerrechnung und Auswertungen im Praktikum} \label{v:fehler}

\section{Allgemeines}

Dieser Abschnitt m"usste eigentlich richtiger hei"sen: "`Rechnen mit
Ungenauigkeiten"'\index{Ungenauigkeiten}. Im Folgenden sollen kurz
einige Grundlagen zur Fehlerrechnung\index{Fehlerrechnung},
Statistik\index{Statistik} und Auswertung von
Messdaten\index{Auswertung} dargelegt werden, soweit sie f"ur
dieses Praktikum wichtig sind. F"ur eine genauere Betrachtung sei
auf die Spezialliteratur
%\cite{beving,bron,npp,kamke,physmess,lichten,tbstat,tbnum,taylor,messunsicher}
verwiesen.\\

\noindent
Dieser Abschnitt ist der Praktikumsanleitung für Physiker entnommen. Inhaltlich sollte er größtenteils mit dem vorherigen Abschnitt übereinstimmen, führt aber einige der Konzepte in größerer Tiefe ein.

\section{Vorbemerkung}

Eine physikalische Gr"o"se\index{Gr"o"se!Physikalische} kennzeichnet
Eigenschaften und beschreibt Zust"ande sowie Zustands"anderungen von
Objekten der Umwelt. Sie muss nach einer Forderung von
\person{Einstein} messbar sein. Die Vereinbarung, nach der die
beobachtete physikalische Einheit quantifiziert wird, ist die
Einheit der physikalischen Gr"o"se. Somit besteht eine physikalische
Gr"o"se {\bf G} immer aus einer quantitativen Aussage G (Zahlenwert)
und einer qualitativen Aussage [G] (Einheit): {\bf G} = G $\cdot$
[G].

F"ur physikalische Gr"o"sen gilt: "`Physikalische Gr"o"se = Zahlenwert
$\cdot$ Einheit"', also bitte immer Einheiten angeben.\footnote{Die
Betreuerinnen sollen Protokolle mit fehlenden Einheiten
zur"uckgeben.} Gesetzlich vorgeschrieben ist die Verwendung des
\emph{Internationalen Einheitensystems
(SI-Einheiten)}\index{SI-Einheiten}. Im amtlichen und gesch"aftlichen
Verkehr d"urfen nur noch SI-Einheiten\index{Einheit} benutzt werden. Teilweise muss man als Naturwissenschaftler aber
auch mit anderen, "alteren Einheiten umgehen k"onnen (Bsp.: Torr,
Gau"s). %Einen sehr sch"onen "Ubersichtsartikel "uber physikalische
%Gr"o"sen, deren Nomenklatur und Einheiten, finden Sie in Ref.~\cite{codata}.

Die Messung einer physikalischen Gr"o"se\index{Gr"o"se!physikalische}
erfolgt %durch Vergleich der Einheit dieser Gr"o"se nach der
Messmethode der (SI-)Vereinbarung oder einem darauf aufbauenden
Messverfahren. Je nach Genauigkeit des Messverfahrens tritt ein
unterschiedlich grosser Messfehler (Ungenauigkeit, Abweichung) auf.
Dabei ist zwischen den \emph{systematischen}, f"ur das Messverfahren
charakteristischen Messfehlern\index{Messfehler} und den
\emph{zuf"alligen} oder \emph{statistischen}, vom einzelnen
Experiment abh"angigen Fehlern zu unterscheiden. Zu systematischen
Messfehlern geh"oren z.B. eine falsche Kalibrierung eines
Messger"ates, Ablesefehler (Parallaxe), falsche Justierung,
Messwertdrift, etc. Zu statistischen Fehlern geh"oren (zuf"allige)
Schwankungen wie elektronische Triggerschwankungen,
Temperaturschwankunken, Rauschen, ungenaues Anlegen von Ma"sst"aben
etc. Zur grafischen Analyse der Messwertschwankungen dient das
Histogramm. Bei zuf"alligen Messfehlern ist die H"aufigkeitsverteilung
der Messwerte $N_j(x_j)$ symmetrisch zu einem Mittelwert, dem
Erwartungswert $\mu$. Wird die Anzahl $n$ der Wiederholungsmessungen
stark erh"oht, so geht die (diskrete) relative H"aufigkeitsverteilung
$N_j(x_j) \rightarrow h(x)$ in eine glockenf"ormige Normalverteilung
(\person{Gau"s}sche Verteilungsfunktion) mit der
Halbwertsbreite\index{Halbwertsbreite} $\Gamma$ (= halbe Breite der Kurve
in halber H"ohe des Maximums, engl.~\textsc{HWHM=Half Width at Half
Maximum}) der Messwerte "uber:
%
\begin{important}
\begin{equation}\label{e:gaussnormal}
    h(x) = \frac{1}{\sqrt{2 \pi}\sigma} e^{-\frac{(x-\mu)^2}{2 \sigma^2}}
    \qquad \mbox{mit} \qquad
    \sigma = \frac{\Gamma}{\sqrt{2 \, \ln 2}} \, .
\end{equation}
\end{important}

Der Parameter $\sigma$ ist ein Ma"s f"ur die Breite der
Verteilungsfunktion $h(x)$: 68,3~\% der Messwerte liegen im Bereich
$\mu - \sigma < x < \mu + \sigma $. Aus der H"aufigkeitsverteilung $h(x_j)$
einer endlichen Anzahl $N$ von Messungen der $m$ diskreten
Messwerte $x_1, \ldots x_m$ lassen sich f"ur 
$\mu$ und $\sigma$ nach der Theorie der
Beobachtungsfehler von \person{Gau"s} Sch"atzwerte berechnen.
Demnach ist die beste N"aherung f"ur $\mu$ der arithmetische
Mittelwert $\bar{x}$, f"ur $\sigma$ die
Standardabweichung $s$, die sich aus der
Fehlersumme berechnet (s.u.). 
%Letztere ist minimal, wenn der
%arithmetische Mittelwert als Erwartungswert gesetzt wird (siehe
%unten).

\section{Ungenauigkeiten und Fehler}

Alle Messvorg"ange liefern Messergebnisse mit einem Fehler, der
nach einer verbindlichen "Ubereinkunft ein Ma"s f"ur die Genauigkeit
des Messergebnisses darstellt. Ein Beispiel: Die L"ange eines
Stabes wird durch Anlegen eines Ma"sstabes bestimmt. Dann sind zwei
Arten von Fehlern m"oglich:
%
\begin{enumerate}
  \item der systematische Fehler des Ma"sstabs, der sich durch genauen Vergleich
   mit dem Urmeter\index{Urmeter} ermitteln l"asst,
  \item der zuf"allige Fehler, der sich durch Unsicherheiten beim Anlegen des
 Ma"sstabs ergibt.
\end{enumerate}

Alle Messergebnisse m"ussen deshalb mit Fehlerangabe $\bar{x} \pm
\Delta x$ angeben werden. In den meisten
F"allen darf man annehmen, dass die Messwerte um den wahren Wert
statistisch streuen, d.h. dass die Abweichungen im Betrag
schwanken und im Mittel gleich oft positiv wie negativ
ausschlagen. Dann ist der beste Wert (Bestwert)\index{Bestwert},
den man aufgrund von $n$ wiederholten Messungen mit Messergebnis
$x_i$ angeben kann, der Mittelwert\index{Mittelwert} $\bar{x}$:
%
\begin{important}
\begin{equation}\label{e:mw}
\mbox{Mittelwert: } \,  \bar{x} = \frac{1}{n} \sum_{i=1}^n x_i \,
.
\end{equation}
\end{important}
%
Au"ser den Streufehlern, die man auch zuf"allige - oder statistische
- Fehler nennt, treten gew"ohnlich auch so genannte systematische
Fehler auf. Ist ein Messger"at falsch kalibriert, wird es zum
Beispiel immer zu gro"se Werte liefern. Um eine Aussage "uber die
Zuverl"assigkeit des Messergebnisses machen zu k"onnen, muss die
Gr"o"se dieser beiden Fehlereinfl"usse abgesch"atzt werden. Die
Aufgabe der Fehlerrechnung ist also die Bestimmung des Fehlers
$\Delta x = \Delta x_{\rm syst.} + \Delta x_{\rm stat.}$. Das Ergebnis der
Messung mit Fehlerangabe\index{Fehlerangabe} lautet dann:
%
\begin{equation}\label{e:ergfehler}
  \mbox{Ergebnis mit Fehlerangabe: } \,  \bar{x} \pm \Delta x \, .
\end{equation}
%
Diese Angabe bedeutet: Man erwartet, dass der wahre Wert $x_w$ im
Bereich $\bar{x} - \Delta x \leq x_w \leq \bar{x} + \Delta x $
liegt. $\Delta x$ nennt man den absoluten Fehler. Es kann auch der
relative Fehler angegeben werden: $\Delta x / \bar{x}$.

Da jede gemessene physikalische Gr"o"se mit einem Fehler behaftet ist,
macht es keinen Sinn als Ergebnis eine Zahl mit vielen Ziffern
anzugeben. Die Zahl der angegebenen Ziffern sollte an die Gr"o"se des
Fehlers angeglichen werden, d.h. das Ergebnis ist entsprechend
sinnvoll zu runden. Das Ergebnis und der Fehler werden an der
gleichen Stelle gerundet. Der Fehler wird normal gerundet.


\section{Systematische Fehler}

Systematische Fehler\index{Fehler!systematische} bei Messungen im
Praktikum r"uhren haupts"achlich von Ungenauigkeiten der Messger"ate
oder der Messverfahren her. Abweichungen der Messbedingungen, wie
z.B. der Temperatur, spielen in der Regel eine untergeordnete
Rolle. Beispiele f"ur systematische Fehler sind:
%
\begin{itemize}
  \item Eine Stoppuhr geht stets vor oder nach
  \item Ein Voltmeter zeigt wegen eines Kalibrierfehlers einen stets zu gro"sen
   (oder zu kleinen) Wert an
  \item Der Ohmsche Widerstand in einer Schaltung weicht vom
  angegebenen Nominalwert ab.
  \item Eine Beeinflussung der Messung durch Messger"ate (z.B.
  Innenwiderst"ande) wird nicht ber"ucksichtigt.
\end{itemize}
%
Systematische Fehler haben stets einen festen Betrag und ein
eindeutiges Vorzeichen. Sie "andern sich auch nicht, wenn die Messung
mit der gleichen Anordnung und den gleichen Ger"aten wiederholt wird.
Da das Vorzeichen nicht bekannt ist, muss man sie auch mit dem
unbestimmten Vorzeichen $\pm$ angeben. F"ur die Absch"atzung des
Betrages gelten die folgenden Hinweise.

F"ur Messger"ate sind die maximal erlaubten Abweichungen $\Delta
x_{\rm syst.}$ einer Anzeige $x$ vom wahren Wert in der Regel durch
Herstellungsnormen festgelegt (G"ute des Ger"ates, siehe
Beschreibung bei "`Messger"aten"'). F"ur elektrische Messger"ate ist
der Begriff "`G"uteklasse"' eingef"uhrt worden. Diese gibt den
erlaubten systematischen Fehler als Prozentwert vom Vollausschlag
an. Diesen Fehler setzt man dann f"ur alle Messungen in diesem
Messbereich an.

Bei L"angenmessger"aten betr"agt der m"ogliche systematische Fehler
selten mehr als wenige Promille vom Messwert und kann daher
gegen"uber den Streufehlern in den meisten F"allen vernachl"assigt
werden. F"ur eine quantitative Absch"atzung kann die folgende Formel
verwendet werden:
%
\begin{equation}\label{e:skalafehler}
  \frac{\Delta x_{\rm syst}}{x} =
  \frac{\mbox{1 Skalenteil der Skala}}{\mbox{Skalenteile bei Vollausschlag.}}
\end{equation}

Stoppuhren sind noch genauer und Ihr Fehler kann zu $\Delta
x_{\rm syst} = \mbox{kleinster Skalenwert} + {\rm 0.005} \cdot
\mbox{Messwert} $ abgesch"atzt werden.

Bei der Messung von Temperaturen mit einem Fl"ussigkeitsthermometer
betr"agt der Ger"atefehler etwa 1~Strichabstand.


\section{Statistische, zuf"allige Fehler}

Ursachen f"ur zuf"allige Fehler\index{Fehler!statistische} sind z.B.
Schwankungen der Messbedingungen w"ahrend der Messung oder auch
Ungenauigkeiten bei der Ablesung von Messinstrumenten (z.B.
Parallaxe). Um den Betrag des Streufehlers absch"atzen zu k"onnen,
wiederholt man die Messung mehrfach. Ein Ma"s f"ur die Streuung kann
dann aus den Abweichungen $x_i - \bar{x}$ der einzelnen Messwerte
vom Mittelwert gewonnen werden, die von Gau"s als
Standardabweichung $s$ f"ur $n$ Messungen $x_i$ definiert wurde:
%
\begin{equation}\label{e:standabw}
  s = \sqrt{ \frac{1}{n-1} \sum_{i=1}^n \left( x_i - \bar{x} \right)^2 }
\end{equation}

Die Standardabweichung $s$ repr"asentiert die Genauigkeit der
einzelnen Messung und damit auch des Messverfahrens. Deshalb wird
$s$ auch als mittlerer quadratischer Fehler der Einzelmessung
bezeichnet. Je mehr Einzelmessungen vorliegen, umso genauer wird
der Mittelwert sein. Der mittlere quadratische Fehler des
Mittelwertes $\Delta x_{\rm stat.}$ ist nach der Fehlertheorie um den
Faktor $\nicefrac{1}{\sqrt{n}}$ kleiner.
%
\begin{important}
\begin{equation}\label{e:fmw}
 \Delta x_{\rm stat} = \frac{s}{\sqrt{n}}
     = \sqrt{ \frac{1}{n(n-1)} \sum_{i=1}^n \left( x_i - \bar{x} \right)^2 }
\end{equation}
\end{important}
%
Dieser Wert $\Delta x_{\rm stat}$ wird manchmal in Anlehnung an
die Normalverteilung als $\sigma_{\bar{x}}$ bezeichnet. 
Dabei sollte der Wert $\sigma_{\bar{x}}$, der den Fehler auf
den arithmetischen Mittelwert $\bar{x}$ angibt, {\it nicht} mit dem
Fehler $\sigma=s$ der Einzelmessung $x_i$ verwechselt werden!
Die Fehlerrechnung
erlaubt dann die Aussage (wenn systematische Fehler wesentlich
kleiner sind), dass der wahre Wert mit einer Wahrscheinlichkeit
von 68\% im Intervall mit der Breite $\sigma_{\bar{x}}$ um den Mittelwert
liegt: $\bar{x} - \sigma_{\bar{x}} < x_w < \bar{x} + \sigma_{\bar{x}} $.

%Nach der Theorie der Beobachtungsfehler (t-Verteilung nach
%Student, alias \person{W. S. Gosset}) sind bei normalverteilten
%Messgr"ossen die Vertrauensgrenzen abh"angig von der Anzahl $n$ der
%Messungen und der Standardabweichung $s$ des Messverfahrens:
%
%\begin{equation}\label{e:t-vert}
%    x = \bar{x} \pm t_P \cdot \frac{s}{\sqrt{n}}
%\end{equation}
%
%Der Faktor $t_P$ folgt aus der Student-t-Verteilung und ist
%abh"angig von der Anzahl der Wiederholungsmessungen und der
%geforderten statistischen Sicherheit $P$
%\cite{bron,tbstat}. F"ur gro"se $n$ entspricht
%$t_{\numprint[\%]{68.3}}$=1. %, d.h. $\Delta x = \Delta \bar{x}$.
%Einige Werte f"ur $t_P$ sind in Tabelle~\ref{t:student} aufgef"uhrt.

Im Praktikum, wie meistens in der
Physik, k"onnen wir uns mit 1$\sigma$, also 68,3~\% Sicherheit zufrieden geben.
Bitte geben Sie Ihre Ergebnisse in den Protokollen auch so an,
d.h. benutzen Sie die $1\sigma$"=Regel f"ur Ihre Fehlerangaben.
Liegt neben der statistischen Unsicherheit auch noch ein
systematischer Fehler vor, so ist als Gesamt-Messfehler die quadratische Summe
der beiden Fehler anzugeben.
%
%\begin{table}[htb]%
%  \centering%
%  \caption[Student-Verteilung: Werte von $t_P$]{\label{t:student}Einige Werte
%  von $t_P$ bei der Student"=t"=Verteilung
%   f"ur die angegebene statistische Sicherheit.}%
%  \begin{tabular}{rrrr}%
%    \toprule
%    $n$ & 68.3\% & 95\% & 99.7\% \\ \midrule
%    3   & 1.32 & 4.3 & 19.2 \\
%    5   & 1.15 & 2.8 & 6.6\\
%    10  & 1.06 & 2.3 & 4.1\\
%    100 & 1.00 & 2.0 & 3.1\\ \bottomrule
%  \end{tabular}
%\end{table}


\section{Gewichteter Mittelwert}

Bei Vorliegen mehrerer \emph{unabh"angiger} Ergebnisse ist es "ublich,
den gewichteten Mittelwert\index{Mittelwert!gewichtet} anzugeben:
%
\begin{important}
\begin{equation}\label{e:gewmittel1}
 \bar x = \frac{
   \sum\limits_i \frac{x_i}{\sigma_i^2 }
    }{
    \sum\limits_i \frac{1}{\sigma_i^2}
    }
    \quad \mbox{mit Fehler: } \quad
    \sigma  = \sqrt{\frac{1}{\sum\limits_i \frac{1}{\sigma_i^2}} } \, .
\end{equation}
\end{important}
%
Bei stark unterschiedlich genauen Werten greift man besser auf
folgende Berechnung des Fehlers zur"uck:
%
\begin{equation}\label{e:gewsigma2}
 \sigma  = \sqrt {\frac{{\sum {\frac{{\left( {x_i  - \bar x}
 \right)^2 }}{{\sigma _i^2 }}} }}{{\left( {n - 1} \right)\sum
 {\frac{1}{{\sigma _i^2 }}} }}} \, ,
\end{equation}
%
oder nimmt das Maximum des mit den beiden obigen Formeln
berechneten Fehlers.

\section{Lineare Regression}

Hat man die Messwerte $y_i(x_i)$ vorliegen und vermutet einen
linearen Zusammenhang $y=m\cdot x + b$, so kann man dies einfach
mit der linearen Regression\index{Lineare Regression} testen. 

\subsection{Einfache Regression}

Ohne Ber"ucksichtigung bzw.~Kenntnis der Fehler auf die Messwerte 
$y_i$ ergibt sich f"ur die Steigung aus der linearen Regression:
%
\begin{equation}
m = \frac{{n\sum {x_i y_i  - \sum {x_i \sum {y_i } } } }}{{n\sum
{x_i^2  - \left( {\sum {x_i } } \right)^2 } }}
\end{equation}
%
und der Achsenabschnitt ist
%
\begin{equation}
b = \frac{{\sum {x_i^2 \sum {y_i }  - \sum {x_i \sum {x_i y_i } }
} }}{{n\sum {x_i^2  - \left( {\sum {x_i } } \right)^2 } }}
\end{equation}
%
Die jeweiligen Fehler berechnen sich zu:
%
\begin{equation}
\sigma_m^2  = \frac{{n\sum {\left( {y_i  - b - mx_i } \right)^2 }
}}{{\left( {n - 2} \right)\left( {n\sum {x_i^2  - \left( {\sum
{x_i } } \right)^2 } } \right)}}
\end{equation}
%
\begin{equation}
\sigma_b^2  = \frac{{\sum {x_i^2  \cdot } \sum {\left( {y_i  - b -
mx_i } \right)^2 } }}{{\left( {n - 2} \right)\left( {n\sum {x_i^2
- \left( {\sum {x_i } } \right)^2 } } \right)}}
\end{equation}
%
und der Korrelationskoeffizient\index{Korrelationskoeffizient}
berechnet sich folgenderma"sen:
%
\begin{equation}
 r = \frac{
 {n\sum {x_i y_i  - \sum {x_i \sum {y_i } } } }
 }
 {
 {\sqrt{n\sum {x_i^2  - \left( {\sum {x_i } } \right)^2 }} \cdot
  \sqrt{n\sum {y_i^2  - \left( {\sum {y_i } } \right)^2 } } }
 }
\end{equation}
%
Mathematisch liefert der Korrelationskoeffizient ein Ma{\ss} daf"ur, ob die Annahme eines linearen Zusammenhangs zwischen den $x_i$ und $y_i$ sinnvoll ist. Je dichter der Betrag des Korrelationskoeffizienten bei Eins liegt, desto besser ist die Linearit"at.\\
Man sei aber vorsichtig, aus einem guten $r$ sofort auf einen
wirklich physikalischen linearen Zusammenhang zu schlie"sen.

\subsection{Regression mit Messfehlern} \label{v:LinRegErr}

Unter der Voraussetzung, dass nur die
$y_i$-Werte mit dem Fehler $\sigma_i$ fehlerbehaftet, w"ahrend die
$x_i$-Werte fehlerfrei sind (exakte Definition siehe
\cite{beving}), ergibt sich f"ur die lineare Regression:
\begin{align*}
\Delta&=\sum{\frac{1}{\sigma_i^2}}\sum{\frac{x_i^2}{\sigma_i^2}}-
\left(\sum{\frac{x_i}{\sigma_i^2}}\right)^2 & &
\\
%
m&=\frac{1}{\Delta}\left(\sum{\frac{1}{\sigma_i^2}}
\sum{\frac{x_iy_i}{\sigma_i^2}}-\sum{\frac{x_i}{\sigma_i^2}}\sum{\frac{y_i}{\sigma_i^2}}
\right)
&\sigma_m&=\sqrt{\frac{1}{\Delta}\sum{\frac{1}{\sigma_i^2}}} \\
%
b&=\frac{1}{\Delta}\left(\sum{\frac{x_i^2}{\sigma_i^2}}
\sum{\frac{y_i}{\sigma_i^2}}-\sum{\frac{x_i}{\sigma_i^2}}\sum{\frac{x_iy_i}{\sigma_i^2}}
\right) \hspace{11mm}
&\sigma_b&=\sqrt{\frac{1}{\Delta}\sum{\frac{x_i^2}{\sigma_i^2}}} \\
%
\chi^2&=\sum{\left[\frac{1}{\sigma_i}\left(y_i-mx_i-b\right)\right]^2} & &
\end{align*}


\section{Mathematische Behandlung}

\subsection{Grundlagen der Fehlerrechnung: Bestwert und Fehler}

Wir betrachten im Folgenden einen vorher berechneten Mittelwert
aus Messergebnissen f"ur eine physikalische Gr"o"se und bezeichnen
diesen mit $M$. F"ur diese Gr"o"se $M$ kennen wir den wahren Wert
$M_W$, der in einem wirklichen Experiment nat"urlich unbekannt ist,
aber als existent angenommen werden kann. Jeder Messwert $M_i$ der
Gr"o"se $M$ weicht vom wahren Wert um den absoluten Fehler $\Delta M_i$
ab:
%
\begin{equation} \label{b}
 \Delta M_i = M_i - M_W \, .
\end{equation}
%
Das Endergebnis einer $n$-mal wiederholten Bestimmung von $M$ soll
durch einen Bestwert $M_B$ beschrieben werden, der der Vorschrift
%
\begin{equation} \label{c}
\sum_{i=1}^{n}\, (M_{i} - M_{B})^{2} = f(M_{B}) = \mbox{Minimum}
\end{equation}
%
gen"ugt, die als \person{Gau"s}sche Methode der kleinsten
(Fehler-)Quadrate zur Bestimmung des Bestwertes\index{Bestwert}
bezeichnet wird. F"uhrt man die Bestimmung des Minimums nach der
Vorschrift
%
\begin{equation} \label{d}
 \frac{d}{d M_B} \sum_{i=1}^{n}(M_{i} - M_{B})^{2} = 0
\end{equation}
%
aus, so ergibt sich
%
\begin{equation} \label{e}
 M_{B} = \frac{1}{n} \sum_{i=1}^{n} M_{i} = \bar M
\end{equation}
%
und damit die Definition:
%
\begin{definition} \label{def:bestwert}
  Der Bestwert ist gleich dem arithmetischen Mittel.
\end{definition}

Der in (\ref{b}) definierte absolute Fehler der
Einzelmessung\index{Fehler!der Einzelmessung} l"asst sich in der
Praxis nicht ermitteln. Deshalb f"uhren wir nach der Vorschrift
%
\begin{equation} \label{f}
 \Delta M = \sqrt{
  \frac{1}{n-1} \sum_{i=1}^{n} \left( M_{i} - M_{B} \right)^{2}
 }
\end{equation}
%
den mittleren quadratischen Fehler der Einzelmessung ein. Der in
(\ref{f}) eigentlich erwartete Gewichtsfaktor $1/n$ wurde durch
$1/(n-1)$ ersetzt, weil man f"ur 1~Messwert nat"urlich keinen Fehler
berechnen kann \cite{beving,tbstat}.


\subsection{Die Normalverteilung}

Normalerweise sind die Messdaten $M_i$ gen"ahert in Form einer
Glockenkurve um den wahren Wert $M_W$, angen"ahert durch den
Bestwert $M_B$, verteilt. Die mathematische Form der
Glockenkurve\index{Glockenkurve} ist gegeben durch die
\person{Gau"s}sche
Normalverteilung\index{Normalverteilung}\index{Gau"s}:
%
\begin{important}
\begin{equation} \label{g}
  P(x) =
  \frac{1}{\sqrt{2\pi} \sigma} \cdot
  \exp\left( -\frac{(x-\bar{x})^{2}}{2\sigma^{2}} \right)
\end{equation}
\end{important}
%
wobei $x = M_i - M_B$ gesetzt wurde und somit hier $\bar{x}=0$
gilt. Dar"uber hinaus wird die Normierung erf"ullt:
%
\begin{equation} \label{h}
  \int_{-\infty}^{+\infty}\,P(x) \d x = 1 \, .
\end{equation}
%
Eine solche \emph{Glockenkurve} ist in Bild~\ref{a:glockenkurve}
schematisch dargestellt.
%
\begin{figure}[htb]
  \centering
% \setcapindent{1em}
% \begin{captionbeside}
  \includegraphics[width=8cm]{00_einl/gaussfkt}
% \end{captionbeside}
 \caption[Gau"ssche Glockenkurve]{\label{a:glockenkurve}Schematische Darstellung
   der Glockenkurve. Die Abzisse ist in Vielfachen von $\sigma$ angegeben.
   Auf die Normierung
   der Ordinate wurde der "Ubersicht wegen verzichtet. Die schraffierten
   Intervalle geben die jeweiligen Sicherheitsintervalle von 68.3~\%
   ($1\sigma$) und 95~\% ($2\sigma$) wieder (siehe Text).}
\end{figure}
%
Weiter kann folgende Formel hergeleitet werden:
%
\begin{equation} \label{i}
 s^2 = \int_{-\infty}^{+\infty}\,P(x) \cdot (x-\bar{x})^{2} \d x = \sigma^{2}
\end{equation}
%
Das bedeutet, dass der Parameter $\sigma$ der Glockenkurve mit deren 
Standardabweichung $s$ "ubereinstimmt. Man kann
ferner zeigen, dass die Wahrscheinlichkeit, den wahren Wert
innerhalb des $1\sigma$"=Intervalls um $\bar{x}$ zu finden

\begin{equation} \label{j}
  W(\sigma) = \int_{\bar{x}-\sigma}^{\bar{x}+\sigma} \, P(x) \d x =
  0.68
\end{equation}
%
betr"agt. Beziehung (\ref{j}) beinhaltet, dass die Angabe des
mittleren quadratischen Fehlers nicht bedeutet, dass f"ur alle
Messwerte $M_i$ die Abweichung vom Bestwert $M_B$ kleiner als
$\sigma$ ist. Vielmehr betr"agt die relative H"aufigkeit
(Wahrscheinlichkeit oder Sicherheit) hierf"ur nur 68\%.


\subsection{Der Bestwert einer Funktion und Fehlerfortpflanzung}

Der Bestwert einer Funktion $f(x,y,...)$ von verschiedenen
unabh"angigen Messgr"o"sen $x,y,...$ erschwert die Fehlerrechnung
etwas, und es muss das
Fehlerfortpflanzungsgesetz\index{Fehlerfortpflanzungsgesetz}
angewandt werden. Gegeben seien die Messwerte
%
\begin{equation} \label{l}
  x_{i},\quad i=1\ldots r; \hspace{2cm} y_{k},\quad k=1\ldots s \,
  ,
\end{equation}
%
aus denen ein Endergebnis $f_{i,k}=f(x_{i},y_{k})$ berechnet wird.
Beispiel: Berechnung der Fl"ache $A$ eines Rechtecks aus den
Kantenl"angen $x$ und $y$. Es l"asst sich zeigen, dass der Bestwert
$\bar A$ von $A$ gegeben ist durch
%
\begin{equation} \label{m}
 \bar f \equiv  \frac{1}{r} \cdot \frac{1}{s} \cdot
  \sum_{i=1}^{r} \sum_{k=1}^{s} f(x_{i},y_{k}) = f(\bar x, \bar y)
  \, .
\end{equation}
%
Dieses Ergebnis gilt f"ur beliebige Funktionen und beliebig viele
Variablen. Wir bezeichnen jetzt die mittleren quadratischen Fehler
von $f$, $x$ und $y$ mit $\sigma_f$, $\sigma_x$ bzw. $\sigma_y$.
Dann l"asst sich unter Benutzung der Definitionen der mittleren
quadratischen Fehler dieser drei Gr"o"sen zeigen, dass ein
Fehlerfortpflanzungsgesetz\index{Fehlerfortpflanzung} in der Form
%
\begin{equation} \label{n}
 \sigma_{f} =
   \sqrt{\sigma_{x}^{2} \left( \frac{\partial f}{\partial x} \right)^{2}
    +
         \sigma_{y}^{2} \left( \frac{\partial f}{\partial y} \right)^{2}
   }
\end{equation}
%
gilt\footnote{Die gilt, wie eingangs angenommen, {\it nur} f"ur unab"angige
Messgr"o"sen. Andernfalls muss ein zus"atzlicher Term, der die Korrelation
zwischen $x$ und $y$ ber"ucksichtig, hinzugef"ugt werden.}. 
Auch in (\ref{n}) sind beliebig viele Variablen zugelassen.
Spezialf"alle von (\ref{n}) sind:
%
\begin{equation} \label{o}
  \bar f = \bar x + \bar y \qquad \mbox{mit: } \qquad
  \sigma_{f} = \sqrt{\sigma_{x}^{2} + \sigma_{y}^{2}}
\end{equation}
%
\begin{equation} \label{q}
  \bar f = \bar x \cdot \bar y \qquad \mbox{mit: } \qquad
  \frac{\sigma_{f}}{\bar f}  = \sqrt{\left(\frac{\sigma_{x}}{\bar
 x}\right)^{2} + \left(\frac{\sigma_{y}}{\bar y}\right)^{2}} \, .
\end{equation}
%
%


\subsection{Der mittlere quadratische Fehler des Bestwertes}

Die Beziehung (\ref{f}) gibt den mittleren quadratischen Fehler
$\Delta M_{i}$ der Einzelmessung $M_{i}$ an.\footnote{Hier werden
h"aufig auch die Begriffe "`Standardabweichung"' $s$ oder
"`Varianz"' $s^2$ verwendet. Die Verwendung der Begriffe erfolgt
nicht immer einheitlich, man sollte daher auf die jeweilige
Definition achten.} Da aber nicht die Einzelmessung sondern der
Bestwert $M_B$ das Endergebnis darstellt, muss der Fehler des
Bestwertes\index{Fehler!des Mittelwertes} $\Delta M_{B}$ bestimmt
werden. Dazu fassen wir $M_B$ als Funktion der Gr"o"sen $M_{i}$ auf,
d.h.:
%
\begin{equation} \label{s}
M_{B} = \frac{1}{n} \sum_{i=1}^{n}\,M_{i} = f(M_{1}, M_{2},...)
\end{equation}
%
und wenden hierauf das Fehlerfortpflanzungsgesetz
%
\begin{equation} \label{t}
 \Delta M_{B} = \sqrt{\sum_{i=1}^{n}\,\left(\sigma_{i}\,\frac{\partial
 f}{\partial M_{i}}\right)^{2}}
\end{equation}
%
an. Da alle $\sigma_i$ als gleich angenommen werden k"onnen , d.h.
$\sigma_i$ =$\sigma$, und
%
\begin{equation} \label{u}
 \frac{\partial f}{\partial M_{i}} = \frac{1}{n}
\end{equation}
%
ist, ergibt sich schlie"slich:
%
\begin{important}
\begin{equation} \label{v}
 \Delta M_{B} = \frac{\sigma}{\sqrt{n}}  =
  \sqrt{\frac{1}{n(n-1)} \sum_{i=1}^{n} (\Delta M_{i})^{2}} \, .
\end{equation}
\end{important}
%
Genau dies sollte auch in den Protokollen zur Fehlerangabe
verwendet werden.




\subsection{Methode der kleinsten Fehlerquadrate (Minimales $\chi^2$)}

Ein h"aufig vorkommendes Problem ist die Anpassung einer glatten
Kurve an eine Folge von Messpunkten, die zur Bestimmung der
Kurvenparameter dienen sollen. Die Anpassung von Funktionen an
Messwerte erfolgt meist nach der Methode der kleinsten Quadrate
(minimales $\chi^2$). Als Beispiel benutzen wir das radioaktive
Zerfallsgesetz:
%
\begin{equation} \label{w}
  N(t)\,=\,N_{0} \exp(-\lambda t) \, ,
\end{equation}
%
welches die Anzahl $N(t)$ der nicht zerfallenen
radioaktiven Kerne als Funktion der Zeit $t$ beschreibt. In
(\ref{w}) ist $N_0$ die Zahl der Kerne zum Zeitpunkt $t=0$ und
$\lambda$ die Zerfallskonstante, die "uber die Beziehung
%
\begin{equation} \label{x}
  \lambda = \frac{\ln 2}{T_{1/2}}
\end{equation}
%
mit der Halbwertszeit $T_{1/2}$ des radioaktiven Materials
(Nuklids) zusammenh"angt. Zur Vereinfachung setzen wir voraus, dass
$N_0$ aus anderen Messungen bekannt ist, so dass nur noch
$T_{1/2}$ zu bestimmen ist. Wir messen $n$ mal die pro Zeiteinheit
stattfindenen Zerf"alle (Aktivit"at) und stellen uns die Aufgabe,
$T_{1/2}$ durch geeignete Anpassung der Funktion~(\ref{w}) an die
Messwerte zu bestimmen. Die Messung liefert
%
\begin{equation} \label{y}
 \mbox{Wertepaare} (N_i, t_i) \, ,
\end{equation}
%
d.h. die zum Zeitpunkt $t_i$ pro Zeiteinheit gemessene Anzahl
$N_i$ radioaktiver Kerne. Die Vorschrift f"ur die Anpassung der
Funktion (\ref{w}) lautet:
%
\begin{equation} \label{z}
  \chi^{2}(T_{1/2}) =
    \sum_{i=1}^{n} \frac{(N_{i}-N(t_{i}))^{2}}{\sigma_{i}^{2}} = \mbox{Minimum} \, .
\end{equation}
%
In (\ref{z}) bedeutet $\sigma_i$ den mittleren quadratischen
Fehler von $N_i$, der nach den Gesetzen der Statistik f"ur
diskrete, z"ahlbare Ereignisse
(Poisson-Statistik\index{Poisson-Statistik}) durch die Beziehung
\cite{tbstat}
%
\begin{equation} \label{aa}
  \sigma_i^2 = N_i
\end{equation}
%
gegeben ist. Wir vergleichen in (\ref{z}) also die Abweichung jedes
Messwertes $N_i$ von der gew"ahlten Kurve mit dem Fehler des
Messwertes und minimieren die Summe der mit dem reziproken Fehler
gewichteten Abweichungsquadrate\index{Abweichungsquadrate}.
Ausdr"ucke der Form (\ref{z}) werden allgemein mit $\chi^2$
bezeichnet und beinhalten die Gau"ssche Methode der kleinsten
Quadrate. Die praktische Auswertung der Vorschrift (\ref{z})
erfolgt, indem man $\chi^2$ f"ur eine Anzahl geeignet erscheinender
Werte $T_{1/2}$ berechnet und das Minimum mit Hilfe einer grafischen
Darstellung der Funktion $\chi^2$($T_{1/2}$) bestimmt.
%%%%%%%%%%%%%%%%%%%%%%%%%%%%%%%%%%%%%%%%%%%%%%%%%%%%%%%%%%%%%%%%%%%

\chapter {Erstellung von Diagrammen} \label{v:diagramme}

Hier folgen einige kurze Hinweise zur Erstellung von Diagrammen in
den Protokollen des Praktikums.

\section{Allgemeines}

Die meisten Auswertungen in der Physik werden heute mit sehr
umfangreichen Programmpaketen durchgef"uhrt, die auch gleichzeitig
eine komfortable Diagrammerstellung erlauben. Dennoch kann es
vorkommen, dass man eine einfache Auswertung sehr viel schneller
und mit guter Genauigkeit auch auf konventionellem Wege auf
Millimeterpapier oder Logarithmenpapier ausf"uhren kann. Zudem sind
die hier folgenden Hinweise auch sehr n"utzlich, wenn man
Computerprogramme zur Diagrammerstellung verwendet.

Manuelle Diagramme sind grunds"atzlich immer auf Millimeter- oder
Logarithmenpapier anzufertigen. Bei Computerprogrammen kann auf dies
verzichtet werden, dennoch sollte man auch hier auf eine leichte
Ablesbarkeit der Daten achten.

\section{Achsen}


\begin{enumerate}
    \item Wahl der Achsen: Die unabh"angige (die eingestellte)
    Variable sollte auf der waagerechten Achse, der "`x-Achse"'
    (Abszisse\index{Abszisse}), aufgetragen werden. Die abh"angige (die
    gemessene) Variable sollte auf der vertikalen Achse, der "`y-Achse"'
    (Ordinate\index{Ordinate}), aufgetragen werden.
    \item Achseneinteilung: Die Achseneinteilung sollte so gew"ahlt
    werden, das die Werte eines Datenpunktes einfach und schnell
    ermittelt werden k"onnen. Drei Einheiten einer Gr"o{\ss}e auf einem Zentimeter einzuzeichnen macht also nicht viel Sinn.
    \item Nullpunktsunterdr"uckung: Der Wertebereich der Achsen
    sollte so gew"ahlt werden, dass ein m"oglichst gro"ser Bereich
    ausgef"ullt wird (mindestens 75\% des Diagrammbereiches). Hierbei kann der Nullpunkt
    unterdr"uckt werden, wenn kein triftiger Grund dagegen
    spricht. Das bedeutet, dass die Einheiten auf Abszisse und Ordinate nicht unbedingt bei x=y=0 anfangen m"ussen. Dies gilt auch f"ur logarithmische Skalen.
    \item Achsen sind zu beschriften! Was ist aufgetragen?
    Zahlenwerte sind anzugeben. Einheiten sind unverzichtbar.
    \item Bei logarithmierten Werten ist die Angabe der Einheit problematisch,
     da der Logarithmus im Argument keine Einheit haben darf. Hier teilt man die
     Messwerte einfach durch die Einheit und erh"alt so reine Zahlenwert. Dies ist
     dann auch so anzugeben.
\end{enumerate}

\section{Datenpunkte und Fehlerbalken}

\begin{itemize}
    \item Symbole: Die Messpunkte sollten durch deutliche Symbole
    gekennzeichnet werden. "Ublich sind zum Beispiel: $ \square \blacksquare
    \blacktriangle \blacktriangledown \lozenge \blacklozenge \vartriangle
    \triangledown \bullet \bigcirc \circ \ast \star$.
    \item Unterschiedliche Messreihen sollten auch durch
    unterschiedliche Symbole gekennzeichnet werden. Auf eine
    eindeutige Legende ist zu achten. Farbe kann hier sehr
    n"utzlich sein, doch diese geht leider beim Kopieren verloren.
    \item Fehlerbalken: Normalerweise sind alle Messpunkte mit
    Fehlerbalken zu versehen, deren L"ange der Gr"o"se des (1$\sigma$-)Fehlers
    auf den jeweiligen Messwert entspricht. Dabei kann es notwendig sein, in
    den verschiedenen Achsrichtungen unterschiedlich gro"se
    Fehlerbalken zu verwenden.
\end{itemize}

\section{Kurven und Verbindungslinien}


\begin{itemize}
    \item Eine durchgezogene Kurve kann die Lesbarkeit einer
    Darstellung deutlich erh"ohen. Dennoch sollte dies mit Bedacht
    angewendet werden.
    \item In allen F"allen des Praktikums sind "`glatte"' Kurven zu
    erwarten. Zumeist ist der funktionale Zusammenhang der
    Messdaten auch durch die Theorie schon bekannt. Im Allgemeinen
    sollte eine Kurve m"oglichst wenig Wendepunkte haben.
    \item Es ist nicht zwingend erforderlich, dass die
    durchgezogene Kurve alle, oder "uberhaupt, Messpunkte trifft.
    Endpunkte sind meist weniger genau und m"ussen nicht unbedingt
    getroffen werden. Das einfache lineare Verbinden der
    Messpunkte durch eine
    "`Zick-Zack-Kurve"' ist physikalisch kompletter Unsinn und
    sollte unterlassen werden.
    \item Die Kurve sollte m"oglichst dicht an den Messpunkten
    liegen (Minimierung der Fehlerquadrate). Die eingezeichneten
    Fehlerbalken k"onnen hier eine gute Hilfe sein. Zudem ist das
    Auge ein sehr guter "`Computer"' f"ur die Ausgleichskurve.
    \item Im Wesentlichen sollte die Kurve die Messpunkte
    halbieren, d.h. eine H"alfte der Punkte "uber der Kurve, die
    andere darunter. Das gilt sinngem"a"s auch f"ur Teilst"ucke.
    \item An Regressionsgeraden sind auch die Ergebnisse der
    Regression mit den richtigen Einheiten anzugeben.
    \item Bei Ermittlung des Fehlers "uber Grenzgeraden, sind auch
    diese in der Zeichnung anzudeuten.
\end{itemize}

Als Beispiel f"ur die Diagrammerstellung sind in
Bild~\ref{a:graph_beisp} zwei typische Diagramme aufgef"uhrt.
%
\begin{figure}[htb]
  \centering
  \includegraphics[width=6.5cm]{00_einl/graph_expzerf}
  \includegraphics[width=6.5cm]{00_einl/graph_dampf}
  \caption[Diagramm-Beispiele]{\label{a:graph_beisp}Beispiele f"ur grafische Auftragungen
   zu Messwerten und Auswertungen:
  a) (links) Halblogarithmische Darstellung des exponentiellen Zerfalls;
  b) (rechts) Arrheniusplot der Dampfdruckkurve von Wasser. }
\end{figure}
%
Die Aktivit"atskurve wird
halblogarithmisch\index{Logarithmenpapier!halblogarithmisch}
aufgetragen, d.h. die Zeit normal (linear) als x-Achse, w"ahrend
die Aktivit"at logarithmisch
(Logarithmenpapier\index{Logarithmenpapier}) aufgetragen
wird.\footnote{Es erfolgt keine Umrechnung der Messwerte, die
Skalenteilung ist logarithmisch, genauer gesagt
halblogarithmisch.} Dies hat zur Folge, dass aus der
Exponentialfunktion eine lineare Funktion wird:
%
\begin{equation}\label{e:bsp_exp}
    A(t)=A_0 \cdot \exp(-\lambda t) \Longrightarrow \ln(A(t)) =
    \ln(A_0) - \lambda t
\end{equation}
%
Die Zerfallskonstante $\lambda$ kann dann aus der Steigung $m$ der
sich ergebenden Geraden ermittelt werden, woraus dann die
Halbwertszeit folgt. Man beachte die unterschiedliche L"ange der
Fehlerbalken. F"ur viele Skalengesetze und stark streuende Messwerte,
oder wenn man den Zusammenhang nicht genau kennt, verwendet man die
doppeltlogarithmische
Auftragung\index{Logarithmenpapier!doppeltlogarithmisch}, die f"ur
fast alle beliebigen Messungen eine Gerade entstehen l"asst.

In Bild~\ref{a:graph_beisp}b) ist ein Arrheniusplot des Dampfdruckes
von Wasser aufgetragen. Hier werden die Messwerte (in Pa) vor dem
Auftragen durch ihren Logarithmus ersetzt. Da das Argument des
Logarithmus keine Einheit enthalten kann, behilft man sich, indem
man durch die Einheit dividiert (d.h. diese "`Basiseinheit"' muss
auch im Diagramm mit angegeben werden). Durch die Auftragung der
logarithmierten Messwerte gegen die reziproke Temperatur erh"alt man
eine Gerade
%
\begin{equation}\label{e:bsp_dampf}
    p(T) = p_0 \cdot \exp\left( - \frac{\Lambda}{RT} \right) \Rightarrow
    \ln\left( p(T) \right) = \ln(p_0) - \frac{\Lambda}{R} \cdot
    \frac{1}{T} \, ,
\end{equation}
%
aus deren Steigung die Verdampfungsenthalpie $\Lambda$ berechnet
werden kann. Das Steigungsdreieck ist eingezeichnet und die Steigung
ergibt sich zu
$m=\frac{\mathrm{4.883}}{\mathrm{1\cdot 10^{-3}\, K^{-1}}}=\mathrm{4883~K}$.\footnote{Bitte
beachten Sie, dass die Steigung nat"urlich eine Einheit hat, die
angegeben werden muss.} Mit der allgemeinen Gaskonstanten
$R=\mathrm{8.3145\,J \, mol^{-1} \, K^{-1}}$ ergibt sich die
Verdampfungsw"arme zu $\Lambda=m \cdot R = \mathrm{40597\,J \,mol^{-1}}$ mit einem Fehler von $\Delta \Lambda = \mathrm{20\,J \,mol^{-1}}$.

%%%%%%%%%%%%%%%%%%%%%%%%%%%%%%%%%%%%%%%%%%%%%%%%%%%%%%%%%%%%%%%%%%%

\chapter{Verzeichnis der Versuche}

\begin{table}[h!]
 \centering
 \begin{tabular}{rll}
 \hline
 Versuch & Thema & Raum\\
 \hline
 1 & Solarzelle und Halbleiterdiode & A.02.106\\ 
 2 & Stoß & A.02.105\\
 3 & Wechselstrom und R-C-Kreis & A.01.108 \\ 
 4 & Spule und Transformator & A.01.108 \\ 
 5 & Pohl'scher Resonator & A.02.104 \\
 6 & Trägheitsmoment & A.02.105\\
 7 & Kapillarität und Auftrieb & A.02.102 \\ 
 8 & Innere Reibung von Flüssigkeiten & A.02.102 \\ 
 9 & Linsengesetze & A.01.104 \\ 
 10 & Mikroskop & A.02.107 \\ 
 11 & Brechungsindex von Glas & A.02.108 \\ 
 12 & Beugung am Gitter & A.02.108\\ 
 13 & Thermoelement & A.01.102 \\ 
 14 & Ultraschall & A.01.116\\
 15 & Signalausbreitung auf Leitern & A.01.117 \\ 
 16 & Diffusion & A.01.102\\ 
 17 & Künstliche Radioaktivität & A.01.116 \\ 
 18 & Spezifische Elektronenladung & A.01.105 \\ 
 19 & Spezifische Wärmekapazität & A.02.102 \\ 
 20 & Spezifische Wärmekapazität von Luft & A.01.102 \\ 
 \hline
 \end{tabular}
\end{table}
  
%
%%%%%%%%%%%%%%%%%%%%%%%%%%%%%%%%%%%%%%%%%%%%%%%%%%%%%%%%%%%%%%%%%%%%
\cleardoublepage
\part{Versuche}


\renewcommand{\thechapter}{\arabic{chapter}}
\setcounter{chapter}{0}
\def\chaptername{Versuch}
\setcounter{tocdepth}{0}
%
%%%%%%% Vorf�hrversuch %%%%%%%
\include{Versuch_0/Versuch-0}
%%%%%%%%%%%%%%%%%%%%%%%%%%%%%%
%
\chapter{Solarzelle und Halbleiterdiode}
\label{v:17}

Anhand der Charakteristiken von Halbleiterdioden und Solarzellen soll diese weit verbreitete Technologie näher gebracht werden.

%------------------------------------------------
\section{Stichworte}
%------------------------------------------------

Halbleiterdiode; Kennlinie einer Diode; Kennlinie einer Solarzelle (SZ), Arbeitspunkt (MPP) der SZ, Spektrale Empfindlichkeit der SZ; Farbe von Licht.
%
%------------------------------------------------
\section{Literatur}
%------------------------------------------------

Gehrtsen, Kapitel 11.1.2, 11.2.10, 11.3 und 14.4.3
%
%------------------------------------------------
\section{Anwendungsbeispiele}
%------------------------------------------------

Die Sonne ist die Hauptenergiequelle für unser tägliches Leben. Dabei kann diese Energie vor sehr langer Zeit geliefert worden sein, wie bei den fossilen Brennstoffen, indirekt, wie bei Windkraftwerken oder direkt, wie bei Sonnenkollektoren und Solarzellen. Solarzellen sind in Reihe geschaltete photoempfindlichen Halbleiterdioden, die zur Stromerzeugung genutzt werden. Die Spannung entsteht dabei durch optische Anregung von Valenzelektronen in das Leitungsband. Die mögliche Leistung, die eine Solarzelle zu erbringen vermag hängt dabei von der Beleuchtungsstärke und Farbe des beleuchtenden Lichtes ab. Im Versuch werden die physikalischen und elektrischen Eigenschaften von Halbleiterdioden studiert und die Energieerzeugung mit Hilfe einer Solarzelle quantitativ untersucht.

%------------------------------------------------
\section{Theoretischer Hintergrund}
%------------------------------------------------

\subsection{Halbleiter}

Durch die Anordnung im Kristallgitter verschieben sich die Energieniveaus der Siliziumatome ein klein wenig gegeneinander. Das führt dazu, daß die Niveaus in sogenannten Bändern nahe beieinanderliegen. Dasjenige Band, in dem sich die am schwächsten gebundenen Elektronen (Valenzelektronen) befinden, nennt man das \textit{Valenzband}. \\
Dieses ist durch eine \textit{Bandlücke}, in der es keine erlaubten Energieniveaus gibt, vom nächsthöheren Band getrennt. Da unter normalen Bedingungen keine Elektronen Zustände in diesem Band besetzen und die Energieniveaus benachbarter Atome energetisch sehr dicht aneinander liegen, kann ein Elektron, welches durch äußere Anregung in dieses Band gelangt, sich quasi-frei durch den Kristall bewegen und damit zur elektrischen Leitung beitragen. Daher bezeichnet man dieses Band als \textit{Leitungsband}.

\subsubsection{Dotierung}

Man kann in einen reinen Kristall aus vierwertigem Silizium bei der Herstellung Fremdatome einbringen, um die elektrische Charakteristik des Materials zu verändern. Dieser Prozess heißt \textit{Dotierung}.\\
Bringt man dreiwertige Atome in den Kristall (\textit{Akzeptor}, typischerweise Bor), so bleibt eine der Kristallbindungen zu den vier Nachbaratomen unvollständig, da ein Elektron für die kovalente Bindung fehlt. Diese unvollständige Bindung absorbiert leicht Elektronen aus dem Leitungsband, was ein fehlendes Elektron an einer anderen Stelle hinterläßt, da der Kristall insgesamt elektrisch neutral ist. Da auch diese neue, unvollständige Bindung leicht ein Elektron absorbiert, kann man dieses ''fehlende'' Elektron im Kristall als beweglich ansehen, es verhält sich also wie eine freie, positive Ladung. Man bezeichnet diese als \textit{Loch} im Valenzband. Auf diese Weise reichert eine Bor-Dotierung den Halbleiter mit positiven Ladungsträgern an, man nennt den Kristall auch \textit{p-dotiert}.\\
Auf dieselbe Weise führt Dotierung mit einem fünfwertigen Element (\textit{Donator}, typischerweise Phosphor) zu einem Überschuß an negativen Ladungsträgern, der Kristall ist \textit{n-dotiert}.

\subsection{$pn$-Diode}

Beim idealen $pn$-\"Ubergang grenzen zwei gleichm\"assig dotierte $p$-
und $n$-dotierte Schichten aneinander. Durch die unterschiedliche
Konzentration freier Ladungstr\"ager auf beiden Seiten kommt es zur
Diffusion: Elektronen des $n$-dotierten Gebiets wandern zur
$p$-dortierten Seite und k\"onnen dort mit L\"ochern rekombinieren. Die 
ortsfesten Atome, die ja nun ionisiert sind, stellen eine
positive {\it Raumladung} dar. Analog wandern L\"ocher aus der
$p$- in die $n$-Zone und rekombinieren. Eine negative Raumladung
entsteht.\\
Diese Raumladungen erzeugen ein elektrisches Feld,
welches der Diffusion entgegenwirkt und zu einem Gleichgewicht
f\"uhrt. Die Spannung, welche durch das elektrische Feld erzeugt
wird, nennt man {\it Diffusionsspannung}, $U_{\mathrm{D}}$. Das Gebiet
um den $pn$-\"Ubergang, welches frei von beweglichen Ladungstr{\"a}gern
 ist, wird {\it Verarmungszone} genannt. 
Die Ausdehnung der Verarmungszone auf die
jeweilige Seite ist proportional zum inversen der
Ladungstr\"agerkonzentration.

\subsubsection{$pn$-\"Ubergang mit \"ausserer Spannung in Durchlaßrichtung}

Legt man an den $pn$-\"Ubergang eine externe Spannung an, so dass die
Anode an die $p$-Schicht und die Kathode an die $n$-Schicht
angeschlossen ist, werden die Elektronen aus dem $n$-Gebiet zum
$p$-Gebiet hin verschoben und andersherum. Die angelegte Spannung,
$U_{\mathrm{F}}$, wirkt der Diffusionsspannung entgegen. Es gilt f\"ur
die Potentialdifferenz am $pn$-\"Ubergang:
%
\begin{equation}
U_{pn} = U_{\mathrm{D}} - U_{\mathrm{F}} \, .
\end{equation}

\noindent
Mit zunehmender externer Spannung w\"achst der Strom durch den
$pn$-\"Ubergang. Man spricht von einem {\it Durchlassstrom}.

\subsubsection{$pn$-\"Ubergang mit \"ausserer Spannung in Sperrrichtung}

Legt man eine externe Spannung mit umgekehrter Polung,
$U_{\mathrm{R}}$, an den $pn$-\"Ubergang, so verst\"arkt diese die
Diffusionsspannung und man erh\"alt
%
\begin{equation}
U_{pn} = U_{\mathrm{D}} + U_{\mathrm{R}} \, .
\end{equation}

Die Elektronen der $n$-Schicht und die L\"ocher der $p$-Schicht werden
von der Verarmungszone weggezogen. Die Verarmungszone w\"achst. Da
kein Strom fließen kann wird diese Art der Beschaltung auch {\it
Sperrrichtung} genannt.

\begin{figure}[ht!]
\begin{center}
\includegraphics[width=0.45\textwidth]{Versuch_17-18/Abbildungen/pn_Uebergang.jpg}
\end{center}
\caption{$pn$-\"Ubergang ohne angelegte Spannung (oben), mit \"ausserer Spannung in Durchlaß- (Mitte) und Sperrrichtung (unten).}
\label{fig:Verarmung}
\end{figure}

\subsubsection{Durchbruch}

Im Sperrbetrieb k\"onnen hohe elektrische Felder in der Sperrschicht
auftreten. Durch den {\it Lawineneneffekt} kommt
es zu einem starken Anstieg des {\it Sperrstroms}. Diesen Effekt bezeichnet 
man als {\it Durchbruch} des $pn$-\"Ubergangs.\\

\noindent
\textbf{Lawineneffekt.} 
%
Elektronen, welche durch thermische Anregung in der Sperrschicht entstehen,
werden durch das elektrische Feld beschleunigt. Bei ausreichender
Energie k\"onnen Elektronen aus Atomverb\"anden herausgeschlagen
werden. Diese sekund\"aren Elektronen k\"onnen ihrerseits durch
St\"osse Atome ionisieren und sorgen damit f\"ur einen ansteigenden
Strom.

\subsection{Ladungsdeposition in der verarmten Diode}

Geladene Teilchen (Elektronen, Myon, etc.) und Licht (Photonen) geben über verschiedene Prozesse Energie ab, wenn sie durch ein Material fliegen. Beim Halbleiter erzeugt diese Energie durch Ionisation Paare von Elektronen und Löchern, welche sich im Kristall frei bewegen können. Im elektrischen Feld, welches in der Verarmungszone herrscht, werden diese Ladungen zu den Elektroden hin beschleunigt und erzeugen so einen Strom durch die Diode und den außen angeschlossenen Stromkreis.\\
Je nach den benutzten Materialien der Diode, ihrer Dicke und der Anzahl an Teilchen oder Photonen, die die Diode pro Sekunde treffen, kann dieser Strom groß genug werden, um technisch genutzt werden zu können. Das ist das Prinzip der Solarzelle.

\subsection{Solarzellen}

Man kann sich eine Solarzelle als aus einer Stromquelle und einer Diode zusammengesetzt vorstellen. Dabei hängt die Stromstärke von der Beleuchtung ab. Wenn man die Solarzelle verdunkelt, dann misst man die Kennlinie einer Diode. Die Kennlinie einer beleuchteten Solarzelle ist daher lediglich um den Kurzschlussstrom $I_k$ nach unten verschoben (siehe Abbildung unten).\\ %\ref{fig:Kennlinienfeld}).\\
\begin{figure}[h]
	\centering
		\includegraphics[width=0.5\textwidth]{Versuch_17-18/Abbildungen/Kennlinienfeld.jpg}
	\label{fig:Kennlinienfeld}
\end{figure}

Da bei Beleuchtung der Solarzelle in der Regel nur der vierte Quadrant der Darstellung interessiert, wird auch nur dieser gezeichnet. Trägt man die abgegebene Leistung $P$ über der Spannung auf, kann man ein Maximum ablesen, das als Maximum Power Point bezeichnet wird (siehe Abbildung nächste Seite).\\ %\ref{fig:Leistung}).\\
Schaltet man Solarzellen in Serie, addieren sich die Spannungen, während sich der Strom nicht ändert. Bei Parallelschaltung addieren sich dagegen die Ströme, wobei die Spannung erhalten bleibt. Solarzellen werden daher zu leistungsfähigen Modulen zusammen geschaltet.\\
Im Versuch wird ein in aus 19 Einzelelementen geschaltetes Solarmodul verwendet.

\begin{figure}[h]
	\centering
		\includegraphics[width=0.5\textwidth]{Versuch_17-18/Abbildungen/Leistung.jpg}
	\label{fig:Leistung}
\end{figure}

%------------------------------------------------
\section{Fragen zur Vorbereitung}
%------------------------------------------------

\begin{enumerate}
	%
	%\item Was soll heute im Praktikum gemessen werden? Warum?
	%
	\item Was sind Valenz- und Leitungsband eines elektrischen Leiters?
	%
	\item Wie schaltet man ein Volt- oder Amperemeter in einen elektrischen Schaltkreis?
	%
	\item Was ist die Energielücke zwischen Valenz- und Leitungsband?
	%
	\item Was versteht man unter Löchern im Valenzband? Wie tragen sie zum Ladungstransport bei?
	%
	\item Was versteht man unter der Dotierung eines Halbleiters? Nennen Sie ein Beispiel!
	%
	\item Was sind Akzeptor- / Donatorniveaus im Bändermodell eines Halbleiters? Wie wirken sie auf die Energielücke im Halbleiter?
	%
	\item Was passiert beim Zusammenbringen eines p- und eines n-dotierten Halbleiters mit den freien Ladungsträgern?
	%
	\item Was ist die Diffusionsspannung (auch Diodenflussspannung)?
	%
	\item Was bedeutet Schaltung in Sperr- / Flussrichtung?
	%
	\item Wie sieht die I(U)-Kennlinie der Diode aus? %Was ist die Antidiffusionsspannung?
	%
	\item Was ist eine Solarzelle? Wie erzeugt sie eine elektrische Spannung (Stichwort: Elektronen-Loch-Paar Erzeugung)?
	%
	\item Wie sieht die I(U)-Kennlinie einer Solarzelle im Vergleich zu einer gewöhnlichen Diodenkennlinie aus? Wie unterscheiden sie sich?
	%
	\item Was ist der Maximal Power Point im P(U)-Diagramm einer Solarzelle?
	%
\end{enumerate}

%------------------------------------------------
\section{Durchführung} 
%------------------------------------------------

\begin{figure}[h]
	\centering
		\includegraphics[width=0.5\textwidth]{Versuch_17-18/Abbildungen/Schaltung_Kennlinie.jpg}
	\label{fig:Schaltung_Kennlinie}
	\caption{Schaltung zur Messung der Kennlinie der Halbleiterdiode.}
\end{figure}

\begin{enumerate}
	%
	\item \textbf{Die Kennlinie einer unbeleuchteten Halbleiterdiode:}\\
		Messen Sie die I(U)-Kennlinie der Diode im Bereich 0\,V - 0.4\,V in Schritten von 0.1\,V und im Bereich von 0.4 V\,- 0.8\,V in Schritten von 0.02\,V und bestimmen Sie daraus $U_F$.\\
		\begin{minipage}{0.6\textwidth}
		\textbf{Achtung: }\\
		Keine Spannung über 2\,V anlegen (Schalter auf der Versuchsbox)!\\
		Wenn die Überlastwarnleuchte aufleuchtet wird dies im Protokoll vermerkt und die Messung abgebrochen.\\
		Vor dem Einschalten den Assistenten fragen!
		\end{minipage}
		\begin{minipage}{0.4\textwidth}
			\includegraphics[width=0.8\textwidth]{Versuch_17-18/Abbildungen/Bild15.jpg}
			\label{fig:Bild15}
		\end{minipage}
	%
	\item \textbf{Die Kennlinie einer unbeleuchteten Solarzelle:}\\
		Ersetzen Sie die Diode durch das Solarmodul.\\
		Verdunkeln Sie die auf dem Abschirmgehäuse aufgesteckte Solarzelle mit einem der optischen Filter und messen Sie die I(U)-Kennlinie im Bereich von 0\,V bis etwa 10\,V in Schritten von 1\,V.
	%
 
	\begin{minipage}{0.6\textwidth}
			\item \textbf{Die Kennlinie einer beleuchteten Solarzelle:}\\
		Ersetzen Sie das Netzgerät durch den regelbaren Lastwiderstand und entfernen Sie den Filter.
	\end{minipage}
	\begin{minipage}{0.4\textwidth}
				\includegraphics[width=1.00\textwidth]{Versuch_17-18/Abbildungen/solarzelle.jpg}
				\label{fig:solarzelle }
		\end{minipage}
	%
	\item \textbf{Die Kennlinien einer spektral gefiltert beleuchteten Solarzelle:}\\
		Setzen sie nacheinander die verschiedenen Spektralfilter in das Abschirmgehäuse ein. Nehmen Sie die Kennlinien für blaues, grünes und gelbes Licht von 0\,V bis 5\,V in 1\,V-Schritten, darüber hinaus in 0.2\,V-Schritten auf.
\end{enumerate}

%------------------------------------------------
\section{Auswertung} 
%------------------------------------------------

Die Diodenkennlinie lässt sich für kleinere Ströme annäherungsweise durch die Schottky-Gleichung beschreiben:
\begin{equation}
I = I_{Sp}\cdot\left(\exp\left(\frac{eU}{kT}\right)-1\right)
\end{equation}
Dabei ist $I_{Sp}$ der Sperrstrom, $e$ die Elementarladung, $k$ die Boltzmannkonstante, $T$ die absolute Temperatur (in Kelvin gemessen) und $U$ die Spannung, die über Anode und Kathode der Diode anliegt. Die Diodenflussspannung $U_F$ entspricht dem Schnittpunkt der U-Achse mit der Geraden, durch die man den steil ansteigenden Teil der Diodenkennlinie im Durchlassbereich approximiert.

\begin{enumerate}
	%
	\item Zeichnen Sie die $I(U)$-Kennlinien der Halbleiterdiode und bestimmen Sie aus der Auftragung graphisch die Diodenflussspannung $U_F$ mit Fehler!
	%
	\item Zeichnen Sie die $I(U)$-Kennlinien der unbeleuchteten Solarzelle und bestimmen Sie aus der Auftragung graphisch die Diodenflussspannung $U_F$ mit Fehler!
	%
	\item Tragen Sie die $I(U)$-Kennlinien aus Versuchsteil 3 und 4 auf und bestimmen Sie daraus den Kurzschlussstrom $I_k$ sowie die Leerlaufspannung $U_L = U (I = 0)$. Dazu müssen Sie gegebenenfalls den Graphen extrapolieren. 	
		Auch diese Größen sind mit Fehler behaftet, schätzen Sie diesen daher grob ab!
	%
	\item Bestimmen Sie die von der Solarzelle abgegebenen Leistungen $P = U \cdot I$ für die vier mit Beleuchtung gemessenen Kennlinien und tragen Sie diese in einen Graphen über der Spannung $U$ auf. Aus den Graphen wird der Maximal Power Point abgelesen $(P_{max}$, $I(P_{max})$ und $U(P_{max}))$. Berechnen Sie den Fehler mithilfe der Fehlerfortpflanzung. Sie können davon ausgehen, daß die Messgerätegenauigkeit bei 1\% des Messwertes liegt.
	%
	\item Die spektrale Empfindlichkeit $S$ der Solarzelle soll überprüft werden. Dazu wird das Verhältnis der spektralen Empfindlichkeiten bei gelbem (Wellenlänge $\lambda$ = 670\,nm) und grünem (Wellenlänge $\lambda$ = 530\,nm)
		Licht gebildet und mit dem Literaturwert (siehe Diagramm: \textit{Empfindlichkeit gegen Wellenlänge einer Photodiode}) für dieses Verhältnis verglichen.
		\begin{equation}
			\frac{S_{Gelb}}{S_{Gruen}}=\frac{\frac{I_K(Gelb)}{P_{Lampe}(Gelb)}}{\frac{I_k(Gruen)}{P_{Lampe}(Gruen)}}
		\end{equation}
		Hier sind $I_k$ die Kurzschlussströme der Solarzelle für die verschiedenen Filter, $P_{Lampe}$ ist die Strahlungsleistung der Lampe.
\end{enumerate}
Beim verwendeten Aufbau ist die Strahlungsleistung der Lampe mit dem gelben Filter $P_{Lampe}(Gelb)$ um 10\% größer als bei Verwendung des grünen Filters ($P_{Lampe}(Gruen)$). Daher benötigen Sie nicht die absolute Leistung der Glühlampe.

\noindent
Da die Intensität des Sonnenlichts bei etwa 880\,nm maximal ist, baut man Solarzellen so, dass sie in diesem Bereich ihre maximale Empfindlichkeit haben. Je weiter man sich von diesem Maximum entfernt, umso geringer wird auch die Empfindlichkeit (Messung 3).

\begin{minipage}{0.5\textwidth}
	\includegraphics[width=1.00\textwidth]{Versuch_17-18/Abbildungen/QE_Solarzelle.JPG}
	\label{fig:BILD17}
\end{minipage}
\begin{minipage}{0.5\textwidth}
Das nebenstehende Bild zeigt die Empfindlichkeit einer Si-Photo-Diode in Abhängigkeit von der Wellenlänge. Gestrichelt eingezeichnet ist der Verlauf der Empfindlichkeit einer idealen Diode mit einer Ausbeute von 100\%.
Die gemessene Kurve liegt generell tiefer, verläuft aber bei kurzen Wellenlängen etwa parallel zur theoretischen Kurve.
Der steile Abfall auf der langwelligen Seite kommt daher, dass die Photonen nicht mehr genügend Energie haben, um Elektronen über die Bandlücke anzuregen.
\end{minipage}	% Solarzelle und Hlableiterdiode
%\chapter{Transistor}
\label{v:18}

In diesem Versuch soll die grundlegende Funktion als Leistungsverstärker eines Transistors studiert werden.

%------------------------------------------------
\section{Stichworte}
%------------------------------------------------

Halbleiter; $pn$-Übergang; Wirkungsweise eines Transistors; Emitterschaltung; Transistorkennlinie; Transistor als Verstärker.
%
%------------------------------------------------
\section{Literatur}
%------------------------------------------------

Gehrtsen, Kapitel 14.4.1-3
%
%------------------------------------------------
\section{Anwendungsbeispiele}
%------------------------------------------------

Transistoren gehören zu den wichtigsten elektronischen Bauelementen. Sie werden zum Verstärken und Schalten elektrischer Signale verwendet. Meist bestehen sie aus drei Zonen von halbleitendem Material, das verschieden dotiert ist (p- oder n-leitend). Diese Dreiteilung erinnert noch an ihre Vorläufer, die Trioden. Daher existieren in der Nomenklatur auch viele Ähnlichkeiten. Diese Vakuum-Röhren, deren Funktion man meist schneller begreift, erhalten ihre beweglichen Ladungsträger (Elektronen) aus einem Heizdraht, sie werden auf eine positiven Elektrode hin beschleunigt und können durch ein dazwischen liegendes Gitter abgebremst oder durchgelassen werden. Im Transistor bewegen sich die (negativen oder positiven) Ladungsträger durch Festkörpergrenzflächen, wieder geregelt durch Ladungen, die von außen beeinflusst werden.\\
Eine Vielzahl von Transistoren oder anderen Bauelementen (z.B. Widerstände) können in einem einzigen Fertigungsprozess auf einem einkristallinen Siliziumplättchen (Chip) hergestellt werden. Vor einigen Jahren überschlugen sich die Computer-Chip Hersteller noch mit Angaben, wie viel Tausend Transistoren auf einem Quadratzentimeter Platz hätten. Heute ist es um diese Zahlen still geworden, weil sie so astronomisch groß geworden sind, dass man sie nur noch in Zehnerpotenzen angeben kann, die dann aber doch keiner mehr begreift.\\
Es lohnt sich, sich mit diesem, zugegebenermaßen komplizierten, Bauelement näher zu beschäftigen, das in allen Regelungen und Computern steckt und schon unsichtbar klein geworden ist und unser Leben im letzten Jahrhundert so grundlegend geändert hat.

%------------------------------------------------
\section{Theoretischer Hintergrund}
%------------------------------------------------

\subsection{Der (unbelastete) Spannungsteiler}

	\begin{figure}[h!]
		\centering
		\begin{circuitikz}
			\begin{scope}[xshift=0cm]
				\draw (2.5,2)
					to [R, i<_=$I$, l=$R_1$](0,2) 
					to [V_=$U_o$] (0,0) 
					to [short] (2.5,0) coordinate (B)
					to [R, l=$R_2$] (2.5,2) coordinate (A);
				\draw (A) to [short, *-o] ++(0.5,0) coordinate (A1);
				\draw (A1) node [anchor=south] {$A$};
				\draw (B) to [short, *-o] ++(0.5,0) coordinate (B1);
				\draw (B1) node [anchor=north] {$B$};
				\draw (B1) to [open, v=$U_{AB}$] (A1);
			\end{scope}
		\end{circuitikz}
		\caption{Spannungsteiler}
		\label{fig:spannungsteiler}
	\end{figure}
Schaltet man, wie in Abbildung \ref{fig:spannungsteiler}, zwei Widerstände $R_1$ und $R_2$ hintereinander (in Serie), so fließt durch beide Widerstände derselbe Strom $I$. Die Spannung $U_0$, die die Spannungsquelle zur Verfügung stellt, fällt über beide Widerstände ab nach dem Ohm'schen Gesetz $U_0 = R\cdot I = \left( R_1 + R_2\right)\cdot I$. Über jeden einzelnen Widerstand fällt also nur ein Teil der gesamten Spannung ab: $U_0 = U_1 + U_2 = R_1\cdot I + R_2\cdot I$.\\
Diese \textit{Spannungsteiler} genannte Schaltung wird häufig benutzt, wenn man an einen Verbraucher nicht die ganze Spannung $U_0$ anlegen will oder darf. Stattdessen schließt man den Verbraucher zum Beispiel an die Klemmen $A$ und $B$ an, so dass an ihm die Spannung $U_{AB}$ liegt. Wie groß diese Spannung ist, kann man nun aus den obigen Gleichungen berechnen. Diese folgen aus den sogenannten Kirchhoff'schen Gesetzen, die Sie in Versuch \ref{v:13} genauer kennenlernen.\\
Für den Fall, dass der Verbraucher nicht viel Strom zieht, die Spannung $U_{AB}$ wie folgt berechnet werden kann:
\begin{equation}
	U_{AB} = U_0\cdot\frac{R_2}{R_1 + R_2}
\end{equation}
In diesem Versuch werden Sie die Spannungsteilerschaltung benutzen, um mehrere verschiedene, variable Spannungen an die Anschlüsse des Transistors anzulegen, obwohl nur ein einziges Netzgerät zur Verfügung steht.

\subsection{Der Bipolartransistor}

Werden zwei Elektronen-Halbleiter (n-Leiter) durch eine Schicht eines Löcher-Halbleiters (p-Leiter) getrennt, deren Schichtdicke in der Größenordnung der Ausdehnung des an den Grenzflächen aufgebauten elektrischen Feldes liegt 
($10^{-5}$ m), so kann ein Teil der Elektronen durch die Schicht hindurchtreten. Das Ausmaß des durchtretenden Teiles hängt von der Stärke des elektrischen Feldes ab. Dieses Feld kann von aussen durch Anlegen einer elektrischen Spannung an die dünne Schicht beeinflusst werden. Damit wird der Durchtritt der Elektronen steuerbar!\\
Der Strom $I_C$, der durch den Transistor fließt, lässt sich durch den Basisstrom $I_B$ steuern. Die Verstärkerwirkung des Transistors beruht darauf, dass im Sättigungsbereich der Transistorkennlinie kleine Änderungen des Basisstroms große Änderungen des Kollektorstroms bewirken.\\

\noindent
Die dünne Schicht, Basis $B$ genannt, übernimmt damit ähnliche Funktion wie das Gitter der Triodenröhre. Der an die Basis Elektronen abgebende n-Leiter heißt Emitter $E$. Er hat eine ähnliche Funktion wie die Kathode in der Triodenröhre. Der Elektronen aufnehmende Teil heißt Kollektor $C$ und entspricht der Anode in der Triodenröhre.\\
Das beschriebene System ist ein npn-Transistor. Mit p werden die positiven Ladungsträger (Löcher) der Basis $B$, symbolisiert, während n für die negativen Ladungsträger (Elektronen) im Emitter $E$ und im Kollektor $C$ steht. Ein pnp-Transistor funktioniert analog, aber mit umgekehrten Vorzeichen der Ladungen, Ströme und Spannungen.\\

\noindent
Dieser Transistor hat zwei pn-Übergänge und man kann ihn sich stark vereinfacht als zwei entgegengesetzt geschaltete Dioden vorstellen (siehe Abbildung).\\
In der sog. Emitterschaltung (s. Versuchsaufbau) ist die E-B-Diode in Durchlassrichtung geschaltet, d.h. Elektronen können von E nach B gelangen und es fließt ein (kleiner) Basisstrom. Der Kollektor liegt gegenüber der Basis auf positivem Potential, also ist die C-B-Diode in Sperrrichtung geschaltet. Der Strom $I_C$ ist der Sperrstrom der C-B-Diode und ist abhängig von der Anzahl der Elektronen in der p-leitenden Basis (s. Versuch \ref{v:17}). Da die B-E-Diode in Durchlassrichtung geschaltet ist, hängt die Elektronenkonzentration in der Basis von der äußeren Spannung $U_{BE}$ bzw. vom Basisstrom $I_B$ ab. Dadurch ist der Kollektorstrom $I_C$ eine Funktion sowohl von der äußeren Spannung $U_{EC}$ zwischen Emitter und Kollektor als auch vom Basisstrom $I_B$: $I_C = I_C (U_{EC} , I_B)$\\

\noindent
Sehr stark vereinfachend kann man sich einen Transistor in dieser Beschaltung, bei konstanter Kollektor-Emitter-Spannung $U_CE$, also ähnlich wie eine Stromquelle vorstellen, deren Ausgangsstrom durch die zwischen Basis und Emitter angelegte Spannung $U_{BE}$ eingestellt werden kann.\\
Eine der angenehmen Eigenschaften der Bipolartransistoren ist, dass der Kollektorstrom $I_C$ linear vom Basisstrom abhängt:
\begin{equation}
	I_C = \beta\cdot I_B\; .
\end{equation}
Den Proportionalitätsfaktor $\beta$ bezeichnet man als Stromverstärkungsfaktor des Transistors. Dieser hängt vom genauen Aufbau des Transistors ab, also der Breite der Basiszone und den Ladungsträgerkonzentrationen in den verschieden dotierten Bereichen, und kann somit von Transistor zu Transistor signifikant unterschiedlich sein.

%------------------------------------------------
\section{Fragen zur Vorbereitung}
%------------------------------------------------

\begin{enumerate}
	%
	%\item Was soll heute im Praktikum gemessen werden? Warum?
	%
	\item Was ist eine Spannungsteilerschaltung?
	%
	\item Wozu kann ein Transistor benutzt werden? Nennen Sie Beispiele!
	%
	\item Aus welchen Halbleitern wird ein Transistor aufgebaut?
	%
	\item Wie sind die Spannungen an den einzelnen Schichten bei der im Versuch verwendeten Emitterschaltung geschaltet?
	%
	\item Was versteht man unter Basis, Emitter und Kollektor?
	%
	\item Wie funktioniert ein Transistor in Emitterschaltung? Welche entscheidende Rolle spielt dabei die Basis?
	%
	\item Welcher Strom kann mit dem Basisstrom geregelt werden?
	%
	\item Wie sehen die Ausgangskennlinien eines Transistors in Emitterschaltung aus? Von welchem Parameter hängen die Kennlinien ab?
	%
	\item Was vergleicht man mit dem Verstärkungsfaktor eines Transistors? Wie berechnet sich der Verstärkungsfaktor des Transistors?
	%
\end{enumerate}

%------------------------------------------------
\section{Durchführung} 
%------------------------------------------------

\begin{figure}[h]
	\centering
		\includegraphics[width=0.75\textwidth]{Versuch_17-18/Abbildungen/BILD23.JPG}
	\label{fig:BILD23}
\end{figure}

\begin{enumerate}
	%
	\item Bauen Sie die Schaltung gemäß des Schaltplanes auf und lassen sie diese vom Assistenten kontrollieren.
	%
	\item Stellen sie einen Basisstrom $I_B$ von 0.2\,mA ein. Messen Sie den Kollektorstrom $I_C$ in Abhängigkeit von der Spannung $U_{EC}$ ($U_{EC}$ = 0, 0.1, 0.2, 0.3, 0.4, 0.5\,V anschließend 1\,V, 2\,V, 3\,V, ..., 10\,V).
		Regeln Sie dabei den Basisstrom $I_B$ immer nach.
	%
	\item Wiederholen Sie die Messung für verschiedene Basisströme $I_B$ = 0.3, 0.4, 0.5\,mA.
	%
\end{enumerate}

%------------------------------------------------
\section{Auswertung} 
%------------------------------------------------

\begin{enumerate}
	%
	\item Tragen Sie das Ausgangskennlinienfeld $I_C (U_E)$ für die vier Basisströme auf.
	%
	\item Entnehmen Sie aus Ihren Messreihen zu jedem Basisstrom $I_B$ die Werte des Kollektorstoms $I_C$ bei einer Spannung $U_{EC}$ = 8\,V. Tragen Sie diese auf und bestimmen sie graphisch den Verstärkungsfaktor $\beta$ mit Fehler.
\end{enumerate}	% Transistor
\include{Versuch_1-2/Versuch-1}			% Sto�

\chapter{Wechselstrom und R-C-Kreis}
\label{v:15}

In diesem Versuch lernen Sie die Grundlagen der Funktion und Bedienung eines Digital-Speicher-Oszilloskops (DSO), sowie die Auf- und Entladekurve eines Kondesnators kennen.

%------------------------------------------------
\section{Stichworte}
%------------------------------------------------

Kathodenstrahl; Braun'sche Röhre; Oszillograph; Ablenkung im elektrischen Feld; Wechselspannung; Kondensator; Kapazität.
%
%------------------------------------------------
\section{Literatur}
%------------------------------------------------

Gehrtsen, Kapitel 7.5.2/3, 8.2.1, 8.2.3
%Teile übernommen aus der Anleitung \textit{''Oszilloskop und Funktionsgenerator''} der Universität Oldenburg
%
%------------------------------------------------
\section{Anwendungsbeispiele}
%------------------------------------------------

R ist die Abkürzung (Symbol) für den elektrischen Widerstand, C für die Kapazität eines Kondensators und L für die Induktivität einer Spule; R-C-Kreise sind Kombinationen aus Widerständen und Kondensatoren, die ein für viele physikalische Vorgänge charakteristisches exponentielles Abklingverhalten zeigen. R-L-Kreise, aus Widerstand und Spule aufgebaut, zeigen auf vereinfachte Weise, was passiert wenn bei den meisten Haushaltsgeräten die Spannung eingeschaltet wird (z. Bsp. Glühbirnen).\\

Eine Zellmembran mit geringer Leitfähigkeit als Grenzschicht zwischen gut leitenden elektrolytischen Flüssigkeiten stellt eine elektrische Parallelschaltung eines Widerstandes (Membranwiderstand) und eines Kondensators (Membrankapazität) dar. Aus der Anwendung entsprechender physikalischer Modellvorstellungen kann ein Verständnis der physiologischen Vorgänge und Messtechniken zur Untersuchung von Zelleigenschaften und interzellulären Vorgängen gewonnen werden.

%------------------------------------------------
\section{Theoretischer Hintergrund}
%------------------------------------------------

\subsection{Oszilloskop}

\begin{figure}[hb]
	\centering
		\includegraphics[width=0.5\textwidth]{Versuch_15-16/Abbildungen/TDS2000.JPG}
	\caption{Front des TDS 2001C.}
	\label{fig:TDS2000}
\end{figure}

%Das Oszilloskop ist das vielseitigste Messgerät im Bereich der Elektrik und Elektronik. Es dient der Beobachtung und Messung zeitabhängiger, schneller und wiederkehrender elektrischer Signale. Bis vor einigen Jahren waren noch hauptsächlich analoge Elektronenstrahl-Oszilloskope im Einsatz, die inzwischen jedoch weitestgehend von Digital-Speicher-Oszilloskopen verdrängt wurden. Dennoch wollen wir mit einer kurzen Beschreibung der Funktionsweise des Elektronenstrahl-Oszilloskops beginnen, diese viele grundlegende Funktionen des Oszilloskops anschaulich erklärt.
%
%\begin{figure}[h!]
	%\centering
		%\includegraphics[width=0.5\textwidth]{Versuch_15-16/Abbildungen/Elektronenstrahlroehre.jpg}
	%\caption{Aufbau der Elektronenstrahlröhre des Oszilloskops.}
	%\label{fig:Elektronenstrahlroehre}
%\end{figure}
%
%Kernstück eines Oszilloskops ist eine Elektronenstrahlröhre, die einen Strahl beschleunigter Elektronen als quasi trägheitslosen Zeiger auf einen szintillierenden Schirm schießt. Zur Auslenkung dieses Strahls in der vertikalen Y Richtung wird dabei die zu untersuchende Eingangsspannung an die Platten eines Ablenkkondensators gelegt, so dass die Elektronen im elektrischen Feld des Kondensators abgelenkt werden.\\
%Um die zeitliche Abhängigkeit der Eingangsspannung darzustellen, wird eine linear steigende (Sägezahn-)Spannung (Kippspannung) an die horizontalen Ablenkplatten gelegt, sodass der Elektronenstrahl mit gleichbleibender Geschwindigkeit die X Achse überstreicht.\\
%Verschieden schnelle Signale können über Änderungen der Frequenz der Sägezahnspannung dargestellt werden (Einstellknopf 'SEC/DIV', 2 in Abbildung \ref{fig:TDS2000}). Signale mit verschiedener Amplitude können durch Anpassung der Verstärkung der Eingangsspannung sichtbar gemacht werden (Einstellknopf 'VOLTS/DIV', 1 in Abbildung \ref{fig:TDS2000}).\\
%Ein Trigger-Netzwerk sorgt dafür, dass die Zeitbasis immer in dem Moment ''angestoßen'' wird, in dem das Signal am Y-Eingang erscheint und damit Signal und Zeitbasis synchronisiert werden. Für ein wiederkehrendes Signal erhält man auf diese Weise ein stehendes Bild auf dem Bildschirm des Oszilloskops.\\
%

%
%\noindent
%Auf der Frontseite des hier benutzten TDS2001C Oszilloskops erkennt man 3 Bedienfelder:\\
%Im linken Teil findet man die Einstellungen für die Verstärkung der Eingangssignale der beiden Kanäle (2), sowie die Knöpfe für die Kanalmenüs (1). Hier lässt sich die Einkopplung des Eingangssignals (DC), die Bandbreite des Kanals (Voll), sowie die Grob- bzw. Feineinstellung der Verstärkung einstellen. Des Weiteren findet man hier einen einstellbaren Verstärkungsfaktor für den Eingang (sollte 1X sein), sowie die Möglichkeit, das Eingangssignal zu invertieren (Aus).\\
%Im mittleren Teil finden sich die Einstellungen für die Zeitachse, welche für alle Kanäle gleich ist. Hier können Sie die Vergrößerung der Zeitachse ('SEC/DIV', Drehknopf 2) einstellen, sowie die Zeitachse hin- und herbewegen. Dabei gibt der vertikale weiße Pfeil den Zeitpunkt an, an dem die Triggerbedingung erfüllt wurde.\\
%Im rechten Teil können Sie den Spannungspegel einstellen, 
%Rechts oben im Feld befinden sich der Einschaltknopf (1), die anderen Knöpfe bestimmen die Zeitablenkung des Oszilloskops, d.h. die x-Achse des Bildes und die sog. Triggerung, d.h. den Beginn der Ablenkung des Elektronenstrahls von links nach rechts.\\
%Im unteren rechten Feld werden die zu untersuchenden Spannungen eingegeben. Dieses Gerät kann zwei Signale gleichzeitig anzeigen, daher sind viele Funktionen doppelt. Es regelt also die Y-Achse.\\
%Im schmalen linken Feld unter dem Bildschirm gibt es die Helligkeits- (18) und Schärfe-regelung (20) des Bildes, sowie weitere Testfunktionen des Gerätes.

Will man schnell laufende Vorgänge, z.B. Wechselspannungen oder Impulse, stetig messen und in Abhängigkeit von der Zeit registrieren, so verwendet man ein Oszilloskop. Dieses stellt Spannungen direkt als Funktion der Zeit dar und ermöglicht so die Beobachtung von Vorgängen mit hoher Frequenz.

  Meist benutzt man das Oszilloskop im sogenannten \textit{YT-Modus}, bei dem auf der horizontalen X-Achse des Displays die Zeit $t$ dargestellt wird. Die \textit{Zeitbasis}, d.h. die Skalierung der Zeitachse, kann durch den horizontalen Wahlschalter zwischen \unit{5}{\nano\second} pro Kästchen und \unit{50}{\second} pro Kästchen einstellen. 
  Die K\"astchen werden am Oszilloskop als DIV (f\"ur Division) bezeichnet. 
  Weitere Eigenschaften der Zeitachse können in dem Menü eingestellt werden, welches nach Drücken der 
  Taste HORIZ MENU angezeigt wird. Für die richtige Wahl der Zeitbasis sollten Sie sich klar machen, mit welcher Frequenz die Spannung am Eingang des Oszilloskops sich verändert.

  Auf der vertikalen Y-Achse wird die am Eingang gemessene Spannung dargestellt. Die Skala der Y-Achse kann über den vertikalen Wahlschalter für jeden der vier Kanäle des Oszilloskops getrennt zwischen \unit{20}{\milli\volt} pro Kästchen und \unit{50}{\volt} pro Kästchen eingestellt werden. Zur richtigen Wahl des Y-Verstärkungsfaktors sollten Sie sich klarmachen, welche Spannung Sie an der zu messenden Stelle Ihrer Schaltung erwarten. 

  \paragraph{Kanaleigenschaften}
  Weitere Eigenschaften der Y-Achse können im sogenannten \textit{Kanalmenü} eingestellt werden, welches nach Drücken der Taste CH MENU für den entsprechenden Kanal angezeigt wird. Diese Eigenschaften können für die beiden Kanäle unabhängig von einander eingestellt werden. 
  \begin{itemize}
    \item KOPPLUNG: Es ist möglich, über eine zum Eingang in Reihe geschaltete Kapazität, den Gleichspannungsanteil der Eingangsspannung zu unterdrücken. Dies geschieht, wenn die Kopplung des Kanals auf AC gestellt wird. Dieser Modus ist nützlich, um Spannungsänderungen zu untersuchen und beschleunigt die Arbeit mit dem Oszilloskop erheblich,
      falls der Gleichspannungsanteil tats\"achlich irrelevant ist.
      \begin{important}
        Während des Versuchs stellen Sie die Kopplung bitte auf DC.
      \end{important}

    \item BANDBREITE: Die Bandbreiteneinstellung bestimmt die frequenzabhängig Unterdrückung von Eingangssignalen. Wenn Sie Signale in der Gößenordnung MHz darstellen wollen, sollten Sie darauf achten, dass die Bandbreite auf den maximalen Wert für den Oszilloskoptyp eingestellt ist, da ansonsten die Signalform stark verzerrt dargestellt wird.
    \item TASTKOPF: Die Tastköpfe, mit denen Sie Ihre Schaltung untersuchen, können die Eingangsspannung über einen einstellbaren Spannungsteiler um den Faktor 10 unterdrücken. Dies dient dazu, größere Spannungen auf dem Oszilloskop darstellen zu können, als man sicher an den Eingang anschliessen könnte ohne Bauteile im Oszilloskop zu zerstören. 
    Stellen Sie sicher, dass die Unterdrückung des Tastkopfes und die im Kanalmenü eingestellte Unterdrückung übereinstimmen, da Sie ansonsten andere Spannungswerte am Oszilloskop ablesen, als wirklich in der Schaltung anliegen.
			\begin{important}
				Im Versuch stellen Sie bitte sicher, dass keine Unterdrückung (1X) eingestellt ist.
			\end{important}
    \item INVERTIERUNG: Hiermit können Sie anstelle der Spannung $U$ die invertierte Spannung $-U$ auf dem Oszilloskop darstellen. 
			\begin{important}
				Im Versuch stellen Sie bitte sicher, dass keine Invertierung eingestellt ist.
			\end{important}
  \end{itemize}

  \paragraph{Trigger} 
  Das Oszilloskop stellt die Eingangsspannung auf dem Display dar, wenn die sogenannte \textit{Triggerbedingung} erfüllt ist. Diese stellt man im Trigger Menü ein, welches nach Drücken der Taste TRIG MENU angezeigt wird. 
  Das Oszilloskop stellt mehrere verschiedene Arten von Triggerbedingungen zur Verfügung. So kann, unter anderem auf Pulse bestimmter Breite getriggert werden. Im Praktikum ist allerdings der meistbenutzte Triggertyp der sogenannte \textit{Flankentrigger}. Eine typische Triggerbedinung lautet: Triggere, wenn die Spannung am Kanal 1 einen Wert von \unit{0.1}{\volt} überschreitet.\\
  Diese Bedingung besteht aus drei separat einstellbaren Teilen:
  \begin{enumerate}
    \item Die Triggerquelle, d.h. die Spannung welches Kanals soll betrachtet werden? Wird im Triggermenü über den Punkt QUELLE eingestellt.
    \item Der Triggertyp, d.h. Eingangsspannung soll den Schwellenwert von unten überschreiten (positive oder steigende Flanke) oder von oben unterschreiten (negative oder fallende Flanke). Wird im Triggermenü über den Punkt FLANKE eingestellt.
    \item Der Schwellenwert, d.h. wie groß ist die Spannung, mit der die Engangsspannung verglichen werden soll? Wird über den Drehknopf LEVEL eingestellt.
  \end{enumerate}
  Eine weitere wichtige Eigenschaft des Triggers ist der \textit{Triggermodus} (Triggermenü, Punkt MODUS):
  \begin{itemize}
    \item Triggermodus AUTO: Das Display wird in regelmäßigen Abständen neu aufgebaut, unabhängig davon, ob die Triggerbedingung erfüllt ist. Dieser Modus eignet sich dafür, eine erste Vorstellung des Signals zu bekommen, w\"ahrend noch kein korrekter Trigger eingestellt ist.
    \item Triggermodus NORMAL: Das Display wird nur dann neu aufgebaut, wenn die Triggerbedingung erfüllt ist. Dieser Modus eignet sich, um schnelle Veränderungen der Eingangsspannung, wie zum Beispiel logische Signale, zu untersuchen.
      Ist die Triggerbedingung in kurzen Abst\"anden verl\"asslich erf\"ullt,
      verhalen sich NORMAL und AUTO identisch. Sie k\"onnen also h\"aufig im
      Modus AUTO verbleiben und nur zu NORMAL wechseln, wenn zwischen
      zwei Triggern zu viel Zeit vergeht.
    \item
      Triggermodus STOP: Das Oszilloskop beh\"alt die letzte Aufnahme.
    \item
      Triggermodus SINGLE: Falls Sie eine funktionierende Triggerbedingung haben
      und sich eine einzige Aufnahme des Eingangssignals anschauen m\"ochten,
      k\"onnen Sie auf SINGLE wechseln. Das Oszilloskop wartet dann auf eine
      Triggerbedingung, nimmt genau eine Aufnahme auf und wechselt zu STOP.
  \end{itemize}
	
\subsubsection{Betrieb}

Da wir nur auf Pos. I messen, wird der Tastkopf bei CH 1 eingesteckt, die zugehörige Massenverbindung erfolgt an der Bananenbuchse daneben.\\
Mit dem Drehknopf (2) wird die Empfindlichkeit der Y-Achse eingestellt: Stellung 2~V bedeutet, dass jedes Kästchen auf dem Bildschirm 2~V hoch ist.\\

\noindent
Alle Oszilloskope (dieser Welt ?) werden nach diesem Schema bedient. Die vielen Knöpfe verführen zum Spielen, und wir möchten alle ermutigen, zu probieren, was die einzelnen Schalter bewirken. Diese Anleitung führt - hoffentlich - wieder zu einer Schalterstellung zurück, die eine richtige Messung ermöglicht.

\subsection{Wechselspannung und Wechselstrom}

Als Wechselspannung bezeichnet man eine periodische Spannung mit sinus- oder kosinusförmigem Verlauf:
\begin{equation}
 U(t) = U_0 \sin\omega t
\end{equation}
%
Die Wechselspannung verursacht an einem Bauteil einen Wechselstrom, der eine zeitliche Versetzung $t'$ (Phasenverschiebung $\delta$) gegenüber der Spannung haben kann:
\begin{equation}
 I(t) = I_0 \sin\omega(t-t') = I_0 \sin(\omega t - \delta)
\end{equation}

\noindent
Wechselspannungen werden technisch meist durch Induktion erzeugt (Dynamogenerator). Wird eine Leiterschleife mit konstanter Winkelgeschwindigkeit in einem homogenen Magnetfeld gedreht, so wird in dieser auf Grund des Induktionsgesetzes eine Wechselspannung induziert.\\

\noindent
In Wechselstromkreisen gibt es unterschiedliche Angaben zur Charakterisierung der Größe von Spannung und Strom: die Amplituden oder Scheitelwerte ($U_0$, $I_0$), die Effektivwerte ($U_{eff}$, $I_{eff}$) und die Spitze-Spitze-Werte ($U_{SS}$):
\begin{itemize}
 %
 \item \underline{Amplitude (Scheitelwert):} Die Amplituden geben die Maxima von Spannung oder Strom an; sie entsprechen dem Amplitudenbegriff der trigonometrischen Funktionen.
 %
 \item \underline{Effektivwert:} Die Effektivwerte von Spannung und Strom sind charakteristische Werte, deren Produkt (wie im Fall des Gleichstroms) die (mittlere) joulesche Wärmeleistung ergibt:
  \begin{equation}
   \bar{P} = U_{eff}\, I_{eff} = \frac{U_0}{\sqrt{2}}\,\frac{I_0}{\sqrt{2}}
  \end{equation}
 %
 \item \underline{Spitze-Spitze-Wert:} In besonderen Fällen wird für Spannungen die Differenz zwischen größtem und kleinstem Wert als sogenannter Spitze-Spitze-Wert $U_{SS}$ angegeben.
 %
\end{itemize}

Die Amplituden und $U_{SS}$ können auf dem Oszilloskop direkt beobachtet werden. Die Effektivwerte werden mit Multimetern gemessen, die für Wechselspannungen und -ströme einen eingebauten Gleichrichter enthalten und für diese Messbereiche in Effektivwerten kalibriert sind.\\

Analog zum ohmschen Widerstand R wird als Wechselstromwiderstand Z (Scheinwiderstand oder Impedanz) das Verhältnis der Amplituden von Spannung und Strom definiert:
\begin{equation}
 Z = \frac{U_0}{I_0} = \frac{U_{eff}}{I_{eff}}
 \label{eq:Z}
\end{equation}
Bei ohmschen Widerständen stimmen Gleich- und Wechselstromwiderstand überein. Bei anderen Bauteilen, wie Kondensatoren und Spulen, ist dies nicht der Fall.

\subsection{Kondensator und R-C-Kreis}

Ein Kondensator ist ein Speicher für elektrische Ladung. Die in einem Kondensator befindliche Ladung $Q$ ($Q$ auf der einen Platte und $-Q$ auf der anderen) ist proportional zur ''Größe'' des Kondensators (Kapazität $C$) und zur Spannung $U$ (die Kapazität $C$ ist definiert als Verhältnis von Ladung zu Spannung):
\begin{equation}
 Q = C\cdot U
 \label{eq:Q}
\end{equation}

\noindent
Die Einheit der Kapazität ist:
\begin{equation}
 \mathrm{\left[C\right] = 1\frac{As}{V} = 1 F \quad (Farad)}
\end{equation}

\noindent
Der Strom $I_C$ durch den Kondensator ist die zeitliche Ableitung der Ladung, mit Gleichung (\ref{eq:Q}):
\begin{equation}
 I_C = \frac{dQ_C}{dt} = C\,\frac{dU_C}{dt}\; .
 \label{eq:I_C}
\end{equation}
Wie man sieht ist der Strom groß, wenn sich die Spannung schnell ändert.\\
Stellt man diese Gleichung nach der Spannung frei
\begin{equation}
 U_C = \frac{1}{C}\int{I_C\, dt}
\end{equation}
so sieht man, dass sich die Spannung beim Aufladen eines Kondensators verhält wie das Integral über den Strom.

Nun werde ein (geladener) Kondensator der Kapazität $C$ mit einem Widerstand $R$ zu einem geschlossenen Stromkreis verschaltet.
\begin{figure}[h]
	\centering
		\includegraphics[width=.1\textwidth]{Versuch_15-16/Abbildungen/RC-Kreis.jpg}
	\caption{R-C-Kreis}
	\label{fig:RC-Kreis}
\end{figure}
Die Spannung am Kondensator ($U_C$) ergibt sich aus Gleichung (\ref{eq:I_C}), die Spannung am Widerstand aus der Definition
\begin{equation}
 U_R = R\, I_R\; .
\end{equation}
Nach der Maschenregel muss die Summe aller Ströme in der Masche Null sein. Mit $I = I_C = I_R$ folgt:
\begin{equation}
 U_C + U_R = \frac{1}{C}\int{I\, dt} + R I = 0\; .
\end{equation}
Durch Ableitung nach der Zeit erhält man eine Differentialgleichung für den Strom als Funktion der Zeit:
\begin{equation}
 \frac{dI}{dt} + \frac{1}{RC}\,I = 0\; .
\end{equation}
Diese Differentialgleichung wird gelöst durch die Funktion
\begin{equation}
 I(t) = I_0\, e^{-t/RC}
 \label{eq:Entladekurve}
\end{equation}
Beim Entladen entwickelt sich ein exponentiell mit der Zeit abklingender Strom (ebenso beim Aufladen). Das Produkt $RC$ im Exponenten von Gleichung (\ref{eq:Entladekurve}) bestimmt quantitativ die Abnahme des Stromes und wird als Zeitkonstante bezeichnet.\\

Betrachten wir nun den Wechselstromwiderstand des Kondensators.\\
Mit einer Wechselspannung der Form $U(t) =  U_0\cos\omega t$ finden wir den Strom durch den Kondensator gemäß Gleichung (\ref{eq:I_C}). Mit der Definition des Wechselstromwiderstandes (Gleichung (\ref{eq:Z}) ) berechnet man die Impedanz des Kondensators zu:
\begin{equation}
 Z_C = \frac{U_0}{I_0} = \frac{1}{\omega\, C}\; .
\end{equation}
Ursache des Widerstandes ist die sich aufbauende Gegenspannung am Kondensator. Die dabei umgesetzte Energie bleibt jedoch als elektrische Feldenergie im Kondensator gespeichert und wird während der Entladephase an den Kreis zurückgegeben. Ein (idealer) Kondensator setzt dem Strom einen Widerstand entgegen, der ohne Energieabgabe nach außen bleibt und deshalb als Blindwiderstand bezeichnet wird.\\

Wegen der Frequenzabhängigkeit des Wechselstromwiderstandes von Kondensatoren (und von Spulen) können mit diesen Bauteilen sogenannte Filter gebaut werden, die aus einem Wechselspannungsspektrum bestimmte Frequenzbereiche heraussieben. Solche Filter werden oft benötigt, um in elektrischen Mess- und Steuerkreisen die interessierenden Signale auszuwählen und Störsignale mit anderen Frequenzen abzutrennen. \\
Ein einfaches Beispiel ist die Reihenschaltung eines Kondensators mit einem Widerstand als Spannungsteiler:
\begin{figure}[ht]
	\centering
		\includegraphics[width=0.5\textwidth]{Versuch_15-16/Abbildungen/Paesse_gross.jpg}
	\caption{Hoch- und Tiefpassschaltungen}
	\label{fig:Hoch-Tiefpass}
\end{figure}
Eine Teilspannung ist proportional dem Teilwiderstand, über dem sie abgegriffen wird. Für tiefe Frequenzen ist der Widerstand des Kondensators groß, so dass hier der Hauptanteil der Eingangsspannung abfällt. Der Abgriff über dem Kondensator stellt einen \textit{Tiefpass} dar. Für hohe Frequenzen ist umgekehrt der Widerstand des Kondensators klein und der des Widerstandes vergleichsweise groß, so dass der Abgriff über dem Widerstand als \textit{Hochpass} wirkt.
%------------------------------------------------
\section{Fragen zur Vorbereitung}
%------------------------------------------------

\begin{enumerate}
 %
 \item Was soll heute im Praktikum untersucht werden? 
 %
 \item Was ist eine Wechselspannung? Wie kann man sie mathematisch beschreiben (Beispiel)? 
 %
 \item Was versteht man unter der Schwingungsdauer/Periode einer Wechselspannung? Wie lautet der Zusammenhang zwischen Periode und Frequenz?
 %
 \item Was versteht man unter einer Effektivspannung/einem Effektivstrom? Wie lautet der entsprechende Zusammenhang für eine sinusförmige Wechselspannung?
 %
 \item Was ist ein Kondensator? Wie sieht ein Plattenkondensator aus?
 %
 \item Was ist die Kapazität eines Kondensators? Wie hängt sie mit Ladung und Spannung zusammen?
 %
 \item Beschreiben Sie anhand einer kleinen Skizze (Spannung in Abhängigkeit der Zeit) die Auf- und die Entladung eines Kondensators über einen ohmschen Widerstand! Wie hängt die Auf-/Entladung des Kondensators von der Größe des Widerstandes ab?
 %
% \item Was ist eine Braun'sche Röhre (Skizze und kurze Erklärung)? Stichworte: Wie wird der Elektronenstrahl erzeugt? Wie wird er abgelenkt?
 %
 \item Wozu dient ein Oszillograph?
 %
 \item Was ist eine Kippspannung? Wozu wird sie im Oszillograph gebraucht? Was wäre auf dem Bildschirm eines Oszillographen zu sehen, wenn man nur eine Wechselspannung anlegt, aber keine Kippspannung?
 %
 \item Wie stellt man eine Wechselspannung auf einem Oszillograph dar? (Stichwort: Überlagerung von zu messender Wechselspannung und Kippspannung)
 %
 \item Wozu dient der Trigger? Stichworte: Was sind Trigger-Level und Triggerflanke?
 %
\end{enumerate}

%------------------------------------------------
\section{Durchführung} 
%------------------------------------------------

\begin{enumerate}
 %
 \item In diesem ersten Versuchsteil sollen Sie sich mit dem Oszilloskop vertraut machen. Stellen Sie eine sinusförmige Wechselspannung auf dem Bildschirm des Oszilloskops dar und beobachten die Abbildung bei verschiedenen Einstellungen des Oszilloskops.\\
  (Anmerkung: In diesem Versuchsteil werden keine für die Auswertung relevanten Messungen durchgeführt.)\\
  
  \noindent
  %Einschalten des Oszilloskops (1), Regeln von Helligkeit (18) und Schärfe (20). Man vergewissere sich, dass folgende Schalter im oberen Bedienungsfeld die richtige Position haben (sie werden nicht 
  %gebraucht!): TV Set: off; Delay: off; Hold-off: rechter Anschlag; Trig: AT.\\
	Man wähle den größten Messbereich für die Eingangsspannung (2). Man verbinde den Eingang des Oszilloskops mit dem Ausgang des Netzgeräts mit Hilfe des Anschlusskabels und einer Massenleitung. Man
	verändere die Zeitauflösung (3) und den Trigger-Level (4) und beobachte das Bild auf dem Bildschirm. Man ändere die Polarität der Triggerflanke (Trig Menu). Man ändere die Ablenkempfindlichkeit (2). Überzeugen Sie sich davon, dass eine Änderung der Ablenkempfindlichkeit bzw. der Zeitauflösung zwar die Auflösung ändert, nicht aber die Spannungsamplitude (in Volt) bzw. die Periode (in Sekunden) der Wechselspannung.
 %
 \item Messen Sie mit Hilfe des Vielfachmessgerätes die Werte der Widerstände $R_C$ und $R_L$ in der Versuchsbox ''Wechselspannungsversuch''. Dazu verbinden Sie die Ausgänge des Vielfachmessgerätes mit 
 	den Anschlussbuchsen ober- und unterhalb des Widerstandes $R_C$ (bzw. $R_L$). Legen Sie keine äußere Spannung an! Diese verfälscht die Messung und kann das Vielfachmessgerät beschädigen.\\
  Bitte notieren Sie die Nummer der Versuchsbox. Anmerkung: Der Widerstand $R_L$ (und die Nummer der Versuchsbox dienen lediglich dem Assistenten als Referenz und werden in der Auswertung nicht 
  benötigt).
 %
	
	\begin{figure}[t]
		\centering
			\includegraphics[width=\textwidth]{Versuch_15-16/Abbildungen/Schaltung.jpg}
		\caption{Schaltskizze für den weiteren Versuchsverlauf.}
		\label{fig:Schaltung}
	\end{figure}
	
 \item Verbinden Sie das Netzteil mit dem Eingang des Frequenzgenerators. \\
	\begin{minipage}{0.45\textwidth}
		Verbinden Sie das Oszilloskop mit dem Ausgang des Frequenzgenerators. Die Ausgangsspannung des Frequenzgenerators wird auf dem Bildschirm dargestellt (siehe Abbildung \ref{fig:Schaltung}). Messen Sie die Maximalamplitude 
		$U_0$ sowie die Schwingungsdauer und die Frequenz des Ausgangssignals. Schätzen Sie jeweils Ihre Ablesefehler ab.
	\end{minipage} 
	\hfill
	%
	\begin{minipage}{0.45\textwidth}
		\raggedright
			\includegraphics[width=0.7\textwidth]{Versuch_15-16/Abbildungen/Oszi.jpg}
			\label{fig:Oszi}
	\end{minipage}
 %
 \item Auf- und Entladung eines Kondensators über einen Widerstand:\\
	Verbinden Sie den regelbaren Widerstand über die Leiterbrücke mit dem Kondensator. Messen Sie nun mit dem zweiten Kanal des Oszilloskops den Spannungsabfall am Kondensator. Dazu schalten Sie den
	zweiten Kanal am Oszilloskop zu (1). Wählen Sie für beide Eingänge des Oszilloskops die gleiche Empfindlichkeit und bringen Sie die Nulllinien der beiden Eingänge im unteren Bildbereich auf eine Linie. Beachten Sie, dass Sie dazu nur ein Massekabel benötigen (bei korrekter Schaltung)!\\
	Verändern Sie den regelbaren Widerstand und beobachten Sie die Veränderung der Auf- und Entladekurve des Kondensators. Notieren Sie kurz - in Stichworten - Ihre Beobachtung (in der Auswertung ausführlich formulieren)!
 %
 \item Messung der Kapazität eines Kondensators mit dem Oszilloskop:\\
	Verbinden Sie nun den Festwiderstand mit dem Kondensator. Bringen Sie die Aufladungskurve des Kondensators mit bestmöglicher Zeitauflösung auf den Bildschirm und messen Sie die Spannung am 	Kondensator als Funktion der Zeit. Legen Sie dazu eine Folie (liegt im Praktikum aus) über den Bildschirm und pausen sie den Kurvenverlauf ab. (Skalen und Einstellungen mit aufschreiben !!)\\
	Ebenso messe man die Entladung des Kondensators (dazu Triggerflanke (Trig Menu) ändern!). Kann man sich durch geschicktes Drehen/Spiegeln der Folie ggf. ein weiteres 
	Abpausen der Entladekurve sparen? Bitte begründen Sie Ihre Antwort.
 %
 \item Hochpass-Charakteristik des R-L-Kreises:\\
  Entfernen Sie die Leiterbrücke zwischen $R_C$ und dem Kondensator und verbinden stattdessen $R_L$ mit der Spule. Betrachten und skizzieren Sie den Verlauf der Spannung über der Spule. 
 %
\end{enumerate}

%------------------------------------------------
\section{Auswertung} 
%------------------------------------------------

\begin{enumerate}
 %
 \item Beschreiben Sie den Einfluss verschiedener Widerstände bei der Auf- und Entladung eines Kondensators über einen Widerstand (R-C-Kreis).Beachten Sie dabei den Zusammenhang $\tau = R\,C$ und ggf. die Ladung auf dem Kondensator.
 %
 \item Bestimmung der Kapazität $C$ des Kondensators:
  \begin{enumerate}
   %
   \item Übertragen Sie den Kurvenverlauf der Aufladung des Kondensators auf Millimeterpapier. Einheiten und Achsenbeschriftung nicht vergessen!
   %
   \item Für die Aufladung eines Kondensators gilt:
    \begin{equation}
     U_C(t) = U_0\, \left(1-e^{-t/\tau}\right), \quad \tau = R\, C
    \end{equation}
		Lösen Sie diese Gleichung nach $e^{-t/\tau}$ auf und berechnen Sie den natürlichen Logarithmus. Schreiben Sie die neue Gleichung hin. Berechnen Sie die Unsicherheit des Logarithmus.
	 %
   \item Tragen Sie $\ln\left(1-U_C(t)/U_0\right)$ gegen $t$ auf. 
   %
   \item Bestimmen Sie aus der Steigung der Geraden die Kapazität $C$ des Kondensators inklusive Fehler.
   %
  \end{enumerate}
 %
 \item Erläutern Sie kurz, warum man einen R-L-Kreis auch Hochpass nennt.
 %
\end{enumerate}	% Wechselstrom und RC Kreis
\include{Versuch_15-16/Versuch-16}	% Spule und Transformator

\include{Versuch_1-2/Versuch-2}			% Drehschwingungen --> Tr�gheitsmoment
\include{Versuch_neu_1-2/Versuch_Neu-1}	% Pohl'scher Resonator

\include{Versuch_3-4/Versuch-3}			% Kapillarit�t und Auftrieb
\chapter{Innere Reibung von Flüssigkeiten}
\label{v:4}

In diesem Versuch lernen Sie die Grundlagen der Reibung in Flüssigkeiten kennen.

%------------------------------------------------
\section{Stichworte}
%------------------------------------------------

Innere Reibung; dynamische Viskosität; laminare Strömung; Hagen-Poiseuillesches Gesetz.
%
%------------------------------------------------
\section{Literatur}
%------------------------------------------------

Gehrtsen, Kapitel 3.3.2 und 3.3.3; Walcher, Kapitel 2.6.0 und 2.6.1
%
%------------------------------------------------
\section{Anwendungsbeispiele}
%------------------------------------------------
%
Die Viskosität ist ein Maß für die Zähflüssigkeit eines Fluids, je höher die Viskosität, desto zähflüssiger. Die Viskosität spielt in allen Bereichen, wo fließende Flüssigkeiten auftreten eine Rolle:\\
Motorenöl ersetzt die Reibung von Metall auf Metall durch innere Reibung im Öl, mit kleinen Radiusänderungen der Adern erreicht der Körper eine Regelung der Durchflussmenge im Blutkreislauf in weiten Grenzen, das Durchmischungsverhalten von Flüssigkeiten ist sowohl bei Lösungen im Labor, als auch bei der Herstellung von Lacken etc. wichtig.
%
%------------------------------------------------
\section{Theoretischer Hintergrund}
%------------------------------------------------

\subsection{Innere Reibung}

Zwischen einer festen Wand und einer dazu parallelen, bewegten Platte befinde sich eine dünne Flüssigkeitsschicht der Dicke $z$. Um die Platte der Fläche $A$ mit konstanter Geschwindigkeit $\vec{v}$ parallel zur Wand zu verschieben, braucht man eine Kraft
\begin{equation} \label{eq:Kraft}
 \vec{F} = \eta\,A\frac{\vec{v}}{z}\; .
\end{equation}
Die dynamische Viskosität $\eta$ beschreibt die Eigenschaften der Flüssigkeit. Dass in Gleichung \ref{eq:Kraft} $A$ und $\vec{v}$ im Zähler stehen, ist leicht einzusehen. Warum aber steht die Schichtdicke $z$ im Nenner?\\
Hierzu muss man sich klar machen, dass es sich bei der Gegenkraft zu $\vec{F}$ nicht um die Reibung zwischen Festkörper und Flüssigkeit handelt. Das liegt daran, dass die direkt an die Wände angrenzenden Flüssigkeitsschichten an diesen haften. Vielmehr kommt die Gegenkraft durch die Reibung zwischen verschiedenen Schichten in der Flüssigkeit zustande. Zwischen Wand und Platte bildet sich in der Flüssigkeit ein Geschwindigkeitsprofil aus, bei dem die Geschwindigkeit der einzelnen Flüssigkeitsschichten mit wachsendem Abstand von der Wand linear ansteigen. Je kleiner $z$ bei einer gegebenen Geschwindigkeit der Platte ist, desto schneller müssen also die einzelnen Molekülschichten der Flüssigkeit übereinander weggleiten.\\
Die durch die Kraft $\vec{F}$ von aussen zugeführte Energie wird komplett in Wärme umgewandelt. Diesen Zustand der Flüssigkeit nennt man \textit{laminare Strömung}.\\

Steigert man $\vec{v}$, so tritt im Allgemeinen beim Überschreiten eines kritischen Wertes $v_{krit}$ ein weiterer Anteil zur Reibungskraft hinzu: Die Flüssigkeit zwischen Wand und Platte wird in wirbelnde Bewegung versetzt. Die von aussen zugeführte Energie wird nun zum Teil in kinetische Energie der Flüssigkeit umgesetzt. Diesen Zustand nennt man \textit{turbulente Strömung}.\\

Analoge Überlegungen gelten natürlich auch für strömende Flüssigkeiten zwischen ruhenden Wänden, z.B. Blut in Adern, Motoröl, etc.

Die Viskosität von Flüssigkeiten nimmt mit steigender Temperatur sehr stark ab (das Medium wird 'dünn-flüssiger'). Für viele Flüssigkeiten gilt in guter Näherung $\eta = \eta_{\inf}\, e^{b/T}$.

Ist die Viskosität unabhängig von der Geschwindigkeit $v$, so spricht man von eine \textit{Newton'schen Flüssigkeit}. Die meisten reinen Flüssigkeiten sind Newton'sch. Für bestimmte Mischungen hingegen lässt sich die innere Reibung nicht mehr einfach durch das \textit{Newton'sche Reibungsgesetz}, s. Gl. \ref{eq:Kraft}, beschreiben, die Fließeigenschaften ändern sich, wenn zum Beispiel eine äußere Kraft auf die Flüssigkeit einwirkt. Beispiele für solchen nichtnewtonsche oder anomalviskose Flüssigkeiten sind Blut, Zementleime, Treibsand oder Ketchup.

%------------------------------------------------
\section{Fragen zur Vorbereitung}
%------------------------------------------------

\begin{enumerate}
 %
% \item Was soll heute im Praktikum gemessen werden? Warum?
 %
 \item Wie ist der Druck definiert? (Einheiten)
 %
 %\item Was heißt hydrostatischer Druck?
 %
 \item Welche Strömungstypen gibt es?
 %
 \item Was ist Viskosität? Wodurch entsteht sie? Welche Einheit hat sie?
 %
 %\item Welche Einheit hat die Viskosität?
 %
 \item Wie verhält sich die Viskosität bei steigender Temperatur?
 %
 %\item Unter welchen Voraussetzungen gilt das Hagen-Poiseuillesche Gesetz?
 %
 \item Wie lautet das Hagen-Poiseuillesche Gesetz?
 %
 \item Worauf bezieht sich die Druckdifferenz $\Delta$p im Hagen-Poiseuilleschen Gesetz?
 %
\end{enumerate}

%------------------------------------------------
\section{Durchführung} 
%------------------------------------------------

\begin{enumerate}
 %
 \item Messen Sie die Auslaufzeit $t$ in Sekunden von $h\,=\,45\,$cm auf $h\,=\,35\,$cm für alle drei Kapillaren. Wiederholen Sie die Messung für jede Kapillare dreimal.
 %
 \item Messen Sie für die Kapillare mit mittlerem Durchmesser die Auslaufzeit in Abhängigkeit von der Flüssigkeitshöhe $h$. Notieren Sie dazu bei durchlaufender Stoppuhr jeweils die Zeit, bei der die Flüssigkeitssäule um weitere 5~cm gesunken ist. Messen Sie die Zeit im Intervall zwischen $h\,=\,50\,$cm und $h\,=\,10\,$cm.\\
 Bereiten Sie eine Tabelle wie folgt vor:
 \begin{table}[h]
	\centering
		\begin{tabular}{|c|c|c|c|}
			$h_i$ [m] & $t_i$ [s] & $h_i/h_0$ & ln($h_i/h_0$)\\ \hline \hline
			0,5 & 0 & & \\
		\end{tabular}
 \end{table}
 %
 \item Messen Sie die Länge $l$ der Kapillaren, den Radius $R_V$ des Vorratsbehälters, sowie die Temperatur des (destillierten ?) Wassers.
 %
\end{enumerate}

%------------------------------------------------
\section{Auswertung} 
%------------------------------------------------

\begin{enumerate}
 %
 \item Berechnen Sie den Druckunterschied zwischen der Ober- und Unterseite der Kapillare aus der mittleren Wasserhöhe. Benutzen Sie hierfür
  \begin{equation}
		\Delta p = \rho\cdot g\cdot h_{mittel}\; .  
  \end{equation}
 %
 \item \label{Aufg:Vis1}
 Berechnen Sie die Viskosität $\eta$ von Wasser nach dem Hagen-Poiseuilleschen Gesetz
  \begin{equation} \label{eq:HP}
   \frac{V}{t} = \frac{\Delta p\cdot\pi}{8\eta l}r_k^4
  \end{equation}
 für die drei Kapillaren. Berechnen Sie die Fehler mithilfe der Gaußschen Fehlerfortpfanzung.
 %
 \item Tragen Sie den Logarithmus der relativen Wasserhöhe $\ln(h_i/h_0)$ als Funktion der Auslaufzeit $t$ auf. Bestimmen Sie die Steigung $m$ der Gerade inklusive ihres Fehlers (Anlegen von Grenzgeraden).\\

	\underline{Herleitung:}\\
	Es gilt: $\Delta p(t) = \frac{Kraft}{Flaeche} = \frac{\rho g\cdot h(t) \pi R_V^2}{\pi R_V^2} = \rho g\cdot h(t)$\; .\\
	Damit wird das Hagen-Poiseuillesche Gesetz zu 
	\begin{equation} \label{eq:HP1}
		\frac{dV}{dt} = \frac{\pi r_K^4}{8\eta l}\Delta p(t) = \frac{\rho g\pi r_K^4}{8\eta l} h(t) := C_1\cdot h(t)\; .
	\end{equation}
	Das Flüssigkeitsvolumen im Vorratsbehälter beträgt $dV = \pi R_V^2 dh$, damit wird Gleichung \ref{eq:HP1} zu
	\begin{equation}
		\frac{dh}{dt} = \frac{C_1}{\pi R_V^2} h(t) := C\cdot h(t)\; .
	\end{equation}
	Separation der Variablen ergibt $\frac{dh}{h(t)} = m\cdot dt$.\\
	Integrieren wir diese Formel in den gemessenen Grenzen $\int^{h_1}_{h_0} \frac{dh}{h(t)} = m\int^{t_1}_0 dt$\\
	so ergibt sich
	\begin{equation}
		ln\left(\frac{h_1}{h_0}\right) = m\cdot t_1 = \frac{\rho g r_K^4}{8\eta l R_V^2}\cdot t_1
	\end{equation}
 %
 \item \label{Aufg:Vis2}
 Berechnen Sie aus der Steigung der Geraden die Viskosität $\eta$ nach
  \begin{equation}
   \eta = -\frac{\rho g\,r_K^4}{8\,l\,R^2_V}\frac{1}{m}\; .
  \end{equation}
 Bestimmen Sie den Fehler der Viskosität aus der Gaußschen Fehlerfortpflanzung.
 %
 \item Stimmen die in den Aufgaben \ref{Aufg:Vis1} und \ref{Aufg:Vis2} gemessenen Werte mit dem Literaturwert für die Viskosität mit Wasser überein? Wenn nicht, diskutieren Sie Quellen für Abweichungen vom Literaturwert (z.B.: Welche Annahmen sind gemacht worden?).
\end{enumerate}			% Innere Reibung von Fl�ssigkeiten

\include{Versuch_7-8/Versuch-7}			% Linsengesetze
\include{Versuch_7-8/Versuch-8}			% Mikroskop

\chapter{Brechungsindex von Glas}
\label{v:9}

In diesem Versuch untersuchen Sie die wellenlängenabhängige Beugung (Dispersion) von Licht in einem Glasprisma.

%------------------------------------------------
\section{Stichworte}
%------------------------------------------------

Reflexion; Brechungsgesetz nach Snellius; Huygenssches Prinzip; Dispersion; Brechungsindex; Minimalablenkwinkel.
%
%------------------------------------------------
\section{Literatur}
%------------------------------------------------

Gehrtsen, Kapitel 9.1.2 - 9.1.5
%
%------------------------------------------------
\section{Anwendungsbeispiele}
%------------------------------------------------
%
Optische Instrumente wie Linsen, Lichtleiter und Prismen werden seit jeher aus Glas oder Glas-ähnlichen Stoffen hergestellt. Ihre Funktion als optische Instrumente beruht dabei auf der Ausnutzung der unterschiedlichen Lichtgeschwindigkeit in Medien mit unterschiedlichem Brechungsindex $n$.\\
Luft wird dabei im Praktikum vereinfachend der Brechungsindex $n_{Luft} = 1$ zugeordnet. Das diese Annahme nicht immer stimmt sehen Sie am Flimmern der Luft über einer Flamme oder heißen Oberfläche, bzw. am Phänomen der Fata Morgana. 
%
%------------------------------------------------
\section{Theoretischer Hintergrund}
%------------------------------------------------

Fällt Licht auf die Grenzfläche zwischen zwei Medien (z.B. Luft und Glas), so kann es entweder reflektiert werden oder es dringt unter Änderung von Richtung, Geschwindigkeit und Wellenlänge ein. Ein Maß für diese Änderung ist der Brechungsindex $n$. Die Tatsache, dass $n$ für verschiedenfarbiges Licht verschieden ist, bezeichnet man als Dispersion; d.h. violettes Licht (kurze Wellenlänge) wird stärker abgelenkt als rotes Licht.

\subsection{Das Huygens-Fresnel'sche Prinzip und Brechung an Grenzflächen}

Die Wellenausbreitung im homogenen Medium und auch in komplizierteren Fällen lässt sich relativ intuitiv verstehen, wenn man sich nach \textit{Huygens} vorstellt, dass in jedem Punkt einer Wellenfront ein Streuzentrum sitzt, von welchem Kugelwellen (\textit{Elementarwellen}) ausgehen. Diese überlagern sich dann zu einer neuen Wellenfront.\\

\noindent
Diese Vorstellung kann man nachvollziehen, wenn man sich klar macht, wie das einfallende Licht die Atome der Materie beeinflusst. Licht besteht aus einer sich ausbreitenden elektromagnetischen Welle. Das bedeutet, dass mit dem Lichtstrahl ein sich schnell veränderndes elektrisches Feld einhergeht. Die Frequenz der Änderung des elektrischen Feldes ist dieselbe wie die, die wir dem Licht zuschreiben. Dieses elektrische Feld übt nun eine Kraft auf die Hüllenelektronen der Atome aus, indem es diese negativ geladenen Elektronen in Richtung des positiven elektrischen Feldes zu ziehen, es entsteht ein elektrisches Dipolmoment im Atom oder Molekül. Die schnellen Wechsel des elektrischen Feldes zwingen den Dipol dazu, mit derselben Frequenz zu schwingen. er ruft dadurch ein selbst ein oszillierendes elektrisches Feld hervor, welches dem Feld des einfallenden Lichtstrahls entgegengesetzt ist und dieses deshalb auslöscht. Gleichzeitig emittiert der Dipol aufgrund seiner Schwingung selbst wieder eine elektromagnetische Welle, welche dieselbe Frequenz hat, wie der ursprüngliche Lichtstrahl. Das sind die oben erwähnten Elementarwellen. Die von vielen Dipolen an der Oberfläche ausgesandten Elementarwellen überlagern sich nun konstruktiv und bilden eine ebene Wellenfront, wenn der Abstand zur Oberfläche groß gegenüber der Wellenlänge ist. Damit senden die vielen Dipole einen neuen Lichtstrahl aus, der dieselbe Frequenz hat wie der einfallende, sich jedoch in eine andere Richtung ausbreitet.\\

\noindent
Fällt eine ebene Welle schräg auf die ebene Grenze zwischen zwei Medien, in denen die Wellen unterschiedliche Ausbreitungsgeschwindigkeiten $c_1$ und $c_2$ haben, so haben die Elementarwellen in den beiden Medien nach einer bestimmten Zeit unterschiedliche Radien, welche sich wie $c_1/c_2$ verhalten (s. Abb. \ref{fig:brechung}). Damit nehmen die Überlagerungen der Elementarwellen, welche in dem jeweiligen Medium wieder ebene Wellenfronten bilden, verschiedene Winkel zum Lot auf die Grenzfläche ein. Bezeichnet $\alpha$ den Einfallswinkel und $\beta$ den Ausfallswinkel, dann erhalten wir das allgemeingültige Brechungsgesetz nach Snellius:
\begin{equation} \label{eq:Brechungsgesetz}
\frac{\sin\alpha}{\sin\beta} = \frac{c_1}{c_2}
\end{equation}

\begin{figure}[hb]
	\centering
		\includegraphics[width=.5\textwidth]{Versuch_9-10/Abbildungen/Brechung.jpg}
	\caption{Lichtbrechung mit Huygens'schem Prinzip. Im Medium oberhalb der Grenzfläche beträgt die Lichtgeschwindigkeit $c_1$, darunter $c_2$. Quelle: Wikimedia, Autor Arne Nordmann}
	\label{fig:brechung}
\end{figure}

\subsection{Brechungsindex und Snellius'sches Brechungsgesetz}

Der Brechungsindex $n$ eines Mediums kann "uber die Geschwindigkeit der Lichtwellen in diesem Medium definiert werden als:
\begin{equation}
n = \frac{c_0}{c_m}\, ,
\end{equation}
wobei $c_0$ die Lichtgeschwindigkeit im Vakuum bezeichnet und $c_m$ die Lichtgeschwindigkeit im Medium.\\

\noindent
Mit dieser Definition wird aus Gleichung \ref{eq:Brechungsgesetz} das \textit{Snellius'sche Brechungsgesetz}
\begin{equation} \label{eq:Snellius}
\frac{\sin\alpha_1}{\sin\alpha_2} = \frac{n_2}{n_1} = \frac{c_1}{c_2}
\end{equation}

\subsection{Totalreflexion}

Aus dem Brechungsgesetz (Gl. \ref{eq:Snellius}) kann man sehen, dass es einen Einfallswinkel $\alpha_1$ gibt, bei dem Reflexion unmöglich wird.\\
Da beim Übergang vom optisch dichteren Medium (mit größerem Brechungsindex) ins optisch dünnere Medium das Licht vom Lot auf die Grenzfläche weg gebrochen wird ist der Ausfallswinkel $\alpha_2$ immer größer als der Einfallswinkel. Wird der Ausfallswinkel entsprechend Gl. \ref{eq:Snellius} gleich 90$^{\circ}$, so läuft das Licht entlang der Grenzfläche. Wird der Einfallswinkel noch größer erscheint die Grenzfläche reflektierend und das Licht wird gemäß dem Reflexionsgesetz 'Einfallswinkel gleich Ausfallswinkel' wieder in das optisch dichtere Medium zurück reflektiert.\\
Damit ist klar, dass für den Einfallswinkel, unter dem Totalreflexion auftritt, gelten muss:
\begin{equation}
	\sin\alpha_1 \geq \frac{n_2}{n_1}
\end{equation}

\subsection{Dispersion}

In vielen Medien, wie z. Bsp. Wasser und Glas, hängt der Brechungsindex des Mediums von der Wellenlänge, bzw. der Frequenz, des Lichtes ab: $n = n(\lambda)$. Diese Tatsache nennt man \textit{Dispersion}. Aus historischen Gründen bezeichnet man den Fall, dass der Brechungsindex mit abnehmender Wellenlänge zunimmt $\left( \frac{dn}{d\lambda}<0\right)$ als \textit{normale Dispersion}, den umgekehrten Fall $\left( \frac{dn}{d\lambda}>0\right)$ als \textit{anomale Dispersion} (Beachte: nicht anormal, sondern anomal).\\
In diesen Fällen gilt immer noch das Brechungsgesetz nach Snellius, so lange man für jede betrachtete Wellenlänge den korrespondierenden Brechungsindex einsetzt.\\

\noindent
Das klassische Beispiel für Dispersion in der Natur ist der Regenbogen. Wir sehen einen Regenbogen vor uns, wenn die Sonne nicht zu hoch und hinter uns steht und es vor uns regnet. In diesem Fall tritt ein weißer Lichtstrahl von der Sonne in einen Wassertropfen ein, wobei es zur Dispersion kommt. Das Licht wird an der uns abgewandten Seite des Tropfens total reflektiert und tritt in unsere Richtung wieder aus dem Tropfen aus, wobei die Dispersion wieder auftritt. Da kurzwelliges Licht (blau) stärker gebrochen wird als langwelliges (rot), kann man aus genügend großer Entfernung die verschiedenen Farben des Sonnenlichtes unter verschiedenen Winkeln, also scheinbar an verschiedenen Orten am Himmel sehen: Ein Regenbogen.

\subsection{Ein paar Worte zum Prisma}

Wenn ein Lichtstrahl symmetrisch durch das Prisma hindurchgeht, d.h. parallel zur Basis (s. Abb. \ref{fig:prisma2}), erfährt er die kleinste Ablenkung (Minimal-Ablenkungswinkel). In diesem Fall lässt sich eine geometrische Beziehung zwischen dem Brechungsindex $n$ des Prismas, dem brechenden Winkel $\epsilon$ des Prismas und dem Minimalablenkungswinkel $\delta$ aufstellen:
\begin{equation} \label{eq:Brechungsindex}
n = \frac{\sin\left(\frac{\delta + \epsilon}{2}\right)}{\sin\left(\frac{\epsilon}{2}\right)}
\end{equation}
Diese nennt man die \textit{Fraunhofer Formel}.\\

\noindent
Aufgrund der Dispersion des Glases hängt der Brechungswinkel von der Wellenlänge ab. Dieser Effekt wird beschrieben durch die \textit{Winkeldispersion} des Prismas:
\begin{equation}
	\frac{d\delta}{d\lambda} = \frac{\partial\delta}{\partial n} \frac{dn}{d\lambda}.
\end{equation}
Der erste Faktor kann aus der Fraunhoferschen Formel berechnet werden, der zweite $\frac{dn}{d\lambda}$ ist die Dispersion.
%------------------------------------------------
\section{Fragen zur Vorbereitung}
%------------------------------------------------

\begin{enumerate}
 %
% \item Was soll heute im Praktikum gemessen werden? Warum?
 %
 \item Welchen physikalischen Vorgang beschreibt das Snellius'sche Brechungsgesetz?
 %
 \item Wie ist der Brechungsindex definiert?
 %
 \item Wie hängen Wellenlänge und Frequenz einer Lichtwelle zusammen?
 %
 \item Was ist Dispersion?
 %
 \item Wann kann man Totalrefelxion beobachten?
 %
 \item Wie lässt sich der Winkel f"ur die Totalreflexion aus dem Snellius'schen Brechungsgesetz ableiten?
 %
 \item Wird im Prisma rotes oder blaues Licht stärker abgelenkt?
 %
\end{enumerate}

%------------------------------------------------
\section{Durchführung} 
%------------------------------------------------

Im Versuch wird der Brechungsindex n eines Glasprismas bestimmt, das auf einer Apparatur steht, mit der man Winkel zwischen Lichtstrahlen sehr genau messen kann.

\noindent
Das Fernrohr kann geschwenkt und die Stellung auf einer Gradeinteilung abgelesen werden. Die Einteilung der Scheibe ist auf 0,50\degree genau. Der Nonius umfasst 30 Skalenteile, so dass man auf eine Bogenminute genau ablesen kann.\\
Der Spalt sollte aus Intensitätsgründen nicht zu eng eingestellt werden. Der Spalt sollte breiter sein als das Fadenkreuz im Fernrohr.


\begin{enumerate}
 %
 \item Verschieben Sie den Tubus des Spaltes so lange, bis Sie mit dem Fernrohr ein scharfes Bild des beleuchteten Spaltes sehen.
%Herstellung von parallelem Licht\\
  %Dieser Versuchsteil muss sorgf"altig nach Anleitung durchgeführt werden, sonst werden Sie sp"ater keine vernünftigen Ergebnisse erhalten !
  %\begin{itemize}
   %\item Nehmen Sie das Okular aus dem Fernrohr und stellen Sie das Fadenkreuz bei entspanntem Auge scharf. Schauen Sie dazu am besten gegen einen strukturlosen Hintergrund.
   %\item Setzen Sie das Okular wieder ein und stellen Sie das Fernrohr am offenen Fenster auf unendlich ein. Jetzt sehen Sie gleichzeitig das Fadenkreuz und weit entfernte Gegenst"ande scharf und parallaxenfrei (bei seitlicher Bewegung des Auges verschiebt sich das Fadenkreuz nicht gegen den Hintergrund). \\
   %Mit dieser Einstellung wird paralleles Licht im Auge fokussiert.
   %\item Stellen Sie nun das Fernrohr direkt in den Strahlengang und verschieben Sie den Spalt so lange, bis Sie ihn durch das Fernrohr scharf sehen. Damit haben Sie den Spalt so eingestellt, dass er paralleles Licht emittiert.
  %\end{itemize}
 %
 %\item Messen Sie den (doppelten) brechenden Winkel $2\,\gamma$ des Prismas mit reflektiertem Licht (s. Abb.~\ref{fig:prisma1}). Wiederholen Sie die Messung drei Mal.
 %%
 %\item Messen Sie den (doppelten) minimalen Ablenkwinkel $2\,\delta$ f"ur die rote H$_2$-Linie ($\lambda = 656,7$\,nm) und die blaue H$_2$-Linie ($\lambda = 486,1$\,nm). Wiederholen Sie die Messungen jeweils drei Mal. S. Abb. \ref{fig:prisma2}.\\
 %Den minimalen Ablenkwinkel erkennt man wie folgt: Beobachtung der roten H$_2$-Linie. Das Prisma wird in eine Richtung gedreht. Dabei erkennt man, dass diese Linie über eine bestimmte Stelle nicht hinauswandert. Diese Stelle ist diejenige, die dem Winkel der Minimalablenkung zugeordnet werden kann.
 %
% \item Beobachten Sie das Spektrum der Wasserstoff-Lampe.\\
%		Die Tatsache, dass das von angeregten Gasen emittierte Licht kein kontinuierliches Spektrum aufweist, sondern aus diskreten Spektrallinien zusammengesetzt ist, ist die Grundlage der Spektroskopie, welche unter anderem in  Chemie, Physik und Astronomie eine bedeutende Rolle spielt.
%\end{enumerate}
%
%\textbf{Alternativ:}
%
%\begin{enumerate}
	\item Beleuchten Sie den Spalt mit der Hg-Cd Lampe und lesen Sie auf dem Nonius den Winkel ab, unter dem Sie direkt das Bild des Spaltes sehen. \label{task:Spaltbild}
	%
	\item Stellen Sie das Prisma so in den Strahl, dass das Licht wie in Abb. \ref{fig:prisma2} gezeigt durch das Prisma fällt. Stellen Sie durch Drehen das Prisma auf den minimalen Ablenkwinkel für die gelbe Doppellinie der Hg-Cd Lampe ein. Verkleinern Sie die Spaltöffnung, bis Sie deutlich zwei Linien sehen. Notieren Sie die Winkel.
	%
	\item Arretieren Sie den Grobtrieb des Fernrohrs. 
	%
	\item Messen Sie den Ablenkwinkel der restlichen Linien im Spektrum der Hg-Cd Lampe mit dem Feintrieb des Fernrohrs. Schätzen Sie die Unsicherheit der Winkelmessung ab und notieren Sie diese. \label{task:HgSpektrum}
	%
	%\item Ersetzen Sie nun die Hg-Cd Lampe durch die Wasserstofflampe.
	%%
	%\item Messen Sie die Winkel der sichtbaren Spektrallinien der Wasserstofflampe.
\end{enumerate}

%------------------------------------------------
\section{Auswertung} 
%------------------------------------------------

\begin{enumerate}
 %%
 %\item Berechnen Sie den Mittelwert des brechenden Winkels $\overline{\gamma}$ inklusive seines Fehlers. Beachten Sie, dass Sie für die Berechnung des Fehlers den Winkel im Bogenmaß benutzen müssen.
 %%
 %\item Berechnen Sie die Mittelwerte des minimalen Ablenkwinkels $\overline{\delta}$ f"ur die rote und die blaue Wasserstofflinie inklusive der Fehler.
 %%
 %\item Nehmen Sie an, dass der Brechungsindex von Luft $n=1,0$ betr"agt. Berechnen Sie den Brechungsindex des Glases f"ur die rote und die blaue Wasserstofflinie, inklusive des jeweiligen Fehlers. Rechnen Sie daf"ur im Bogenma{\ss}.
 %%
 \item Korrigieren Sie die gemessenen Ablenkwinkel $\varphi$ aus Aufgabe \ref{task:HgSpektrum} um den Winkel aus Aufgabe \ref{task:Spaltbild}, um die wirklichen Ablenkwinkel $\delta$ zu bekommen.
 %
 \item Der Winkel der brechenden Kante des Prismas $\epsilon$ wird vom Hersteller mit 60\degree angegeben. Berechnen Sie aus den Ablenkwinkeln $\delta$ und dem Winkel der brechenden Kante des Prismas die jeweiligen Brechungsindizes mit Hilfe von Gleichung \ref{eq:Brechungsindex}. Berechnen Sie ebenfalls die Unsicherheit des Brechungsindex.
	%
 \item Tragen Sie die für die Hg-Cd Lampe gemessenen Brechungsindizes gegen die bekannten Wellenlängen der Spektrallinien auf.
 %
 \item Lesen Sie aus dem Diagramm die Dispersion $\frac{dn}{d\lambda}$ für blaues Licht und für gelbes Licht ab. Interpolieren Sie hierzu linear zwischen zwei geeigneten Wellenlängen. Wie groß ist die Unsicherheit?\\
	Der Hersteller gibt an:\\
	$\frac{dn}{d\lambda}|_{blau} = 2365\,\mathrm{cm^{-1}}$ sowie $\frac{dn}{d\lambda}|_{gelb} = 691\,\mathrm{cm^{-1}}$. Wie gut stimmen Ihre Werte mit den Herstellerangaben überein? Diskutieren Sie mögliche Abweichungen.
\end{enumerate}


\begin{table}[hb]
	\centering
		\begin{tabular}{lll}
			\hline
			Wellenlänge & Farbe & Intensität\\
			\hline
			643,85 nm & rot & stark\\
			579,07 nm & orangegelb & sehr stark\\
			576,96 nm & orangegelb & sehr stark\\
			546,07 nm & gelbgrün & stark\\
			508,58 nm & gelb & stark\\
			479,99 nm & blaugrün & stark\\
			467,82 nm & blau & mittel\\
			441,46 nm & blau & mittel\\
			407,78 nm & violett & stark\\
			\hline
		\end{tabular}
	\caption{Wellenlängen der Emissionslinien von Quecksilber und Cadmium}
	\label{tab:Wellenlaengen}
\end{table}

\begin{figure}[ht]
	\centering
		\includegraphics[width=.75\textwidth]{Versuch_9-10/Abbildungen/Hauptschnitt_lpgoe.jpg}
	\caption{Hauptschnitt durch ein Prisma. Quelle: Lehrportal Universität Göttingen}
	\label{fig:prisma2}
\end{figure}		% Brechungsindex von Glas
\include{Versuch_9-10/Versuch-10}		% Beugung am Gitter

\include{Versuch_11-12/Versuch-11}	% Thermoelement
%\include{Versuch_11-12/Versuch-12}	% Kennlinien verschiedener Leiter
\include{Versuch_neu_1-2/Versuch_Neu-2}	% Ultraschall

\chapter{Elektrische Netzwerke}
\label{v:13}

In diesem Versuch lernen Sie grundlegende Schaltungen kennen, die die Bestimmung eines unbekannten ohmschen Widerstandes durch Strom- und Spannungsmessungen erlauben. 

%------------------------------------------------
\section{Stichworte}
%------------------------------------------------

Spannung und Strom; Spannungs- und Strommessung; Widerstand, Ohm'sches Gesetz; Kirchhoff'sche Gesetze; Wheatstone'sche Br"uckenschaltung; Innenwiderstand; Spannungsteilerschaltung.
%
%------------------------------------------------
\section{Literatur}
%------------------------------------------------

Gehrtsen, Kapitel 6.1.2, 6.3.1 - 6.3.4
%
%------------------------------------------------
\section{Anwendungsbeispiele}
%------------------------------------------------

Strom- und Spannungsmessung, Aufbau aller elektrischen Netzwerke

%------------------------------------------------
\section{Theoretischer Hintergrund}
%------------------------------------------------

\subsection{Spannung, Strom und Widerstand}

Die elektrische \textit{Spannung} ist die Differenz der Coulombpotenziale an zwei Orten, zum Beispiel zwei Punkten in einem elektrischen Schaltkreis:
\begin{equation}
	U = \Phi_A - \Phi_B
\end{equation}
Wie man an dieser Definition sieht, bezieht sich die Angabe einer Spannung immer auf ein Referenzpotenzial, welches entweder explizit angegeben wird (''Spannung zwischen Punkten A und B'') oder impliziert wird (typischerweise die sogenannte ''Erde''). Daher kann eine Spannung \textit{positiv} oder \textit{negativ} sein, je nachdem ob das Potenzial am beschriebenen Punkt höher oder niedriger als das Referenzpotenzial ist. Bei einer Spanungsquelle in einem elektrischen Schaltkreis nennt man den Anschluß mit dem höheren Potenzial den \textit{Pluspol} und den mit dem niegrigeren Potenzial den \textit{Minuspol}. Bei den Schaltsymbolen in den nachfolgenden Schaltplänen bezeichnet der längere Strich an der Spannungsquelle den Pluspol.\\
Diese Potenzialdifferenz erzeugt ein elektrisches Feld ($\vec{E} = -\mathrm{grad}\Phi$), welches auf geladene Teilchen (zum Beispiel Elektronen) eine \textit{Coulombkraft} in Richtung der Feldlinien ausübt und diese damit beschleunigt. Negativ geladene Elektronen werden dabei auf den Pluspol zu beschleunigt. Die resultierende \textit{Drift} der Ladungsträger, zum Beispiel in einem Draht, nennt man dann \textit{Strom}:
\begin{equation}
	I = \frac{dQ}{dt}
\end{equation}
Die \textit{technische Stromrichtung} ist so definiert, dass Strom vom Pluspol einer Spannungsquelle zum Minuspol fließt (also entgegen der Driftrichtung von Elektronen). In den nachfolgenden Schaltplänen ist teilweise die technische Stromrichtung durch Pfeile gekennzeichnet.\\

\noindent
Die Einheiten von Spannung, $[U]~=~1~V$, und von Strom, $[I]~=~1~A$, das Volt und das Ampere, sind SI Einheiten. \\

\noindent
Der ohmsche Widerstand eines Leiters ist definiert durch das Ohm'sche Gesetz
\begin{equation}
R:=\frac{U}{I}\, ,
\end{equation}
seine Einheit $[R]~=~1$~\Ohm\, ist das \textit{Ohm}.

\subsection{Einfache Netzwerke}

Elektrische Netzwerke bestehen aus elektrischen Bauteilen, welche miteinander verbunden werden. In diesem Versuch wollen wir uns mit dem einfachsten passiven Bauteil beschäftigen: dem rein ohmschen Widerstand.\\
Netzwerke sind aus \textit{Knoten} und \textit{Maschen} aufgebaut. Aufbauend auf der Erhaltung der elektrischen Ladung ($\frac{dQ}{dt} = 0$) kann man das Verhalten von Strom und Spannung durch die \textit{Kirchhoffschen Gesetze} beschreiben:\\

\noindent
1. Kirchhoffsches Gesetz (Knotenregel): \\
	Die Summe aller Ströme, die in eine Knoten hinein bzw. aus einem Knoten herausfließen, ist null.
 \begin{equation}
  \sum^n_{k=1}{I_k} = 0 \; .
  \label{eq:Kirchhoff1}
 \end{equation}
%
2. Kirchhoffsches Gesetz (Maschenregel): \\
	Die Summe aller Spannungen in einer Masche ist null.
\begin{equation}
 \sum^n_{k=1}{U_k} = 0 \; .
 \label{eq:Kirchoff2}
\end{equation}

\noindent
Die Kirchhoffschen Gesetze können benutzt werden, um den Strom durch ein Bauteil und die über ihm abfallende Spannung zu beschreiben. Ein Beispiel ist die Berechnung des Gesamtwiderstandes einer Parallel- oder Serienschaltung von Widerständen.
%
\subsection{Parallelschaltung}

\begin{minipage}{0.35\textwidth}
 \includegraphics[width=1.00\textwidth]{Versuch_13-14/Abbildungen/Parallel.jpg}
 \label{fig:Parallel}
\end{minipage}
%
\begin{minipage}{0.6\textwidth}
Aus der Knotenregel folgt: \hfill $I = I_1 + I_2\, .$\\
Mit dem Ohm'schen Gesetz folgt: \hfill $\frac{U_0}{R_{ges}} = \frac{U_0}{R_1} + \frac{U_0}{R_2}\, .$\\
Den Gesamtwiderstand der Parallelschaltung von Widerständen kann man also schreiben als: 
\begin{equation}
\frac{1}{R_{ges}} = \frac{1}{R_1} + \frac{1}{R_2} = \sum_i{\frac{1}{R_i}} \, .
\end{equation}
\end{minipage}
%
\subsection{Serienschaltung}

\begin{minipage}{0.35\textwidth}
 \includegraphics[width=1.00\textwidth]{Versuch_13-14/Abbildungen/Seriell.jpg}
 \label{fig:Seriell}
\end{minipage}
%
\begin{minipage}{0.6\textwidth}
Aus der Maschenregel folgt: \hfill $U_0 = U_1 + U_2\, .$\\
Mit dem Ohm'schen Gesetz folgt: \hfill $I R_{ges} = I R_1 + I R_2\, .$\\
Den Gesamtwiderstand der Serienschaltung von Widerständen kann man also schreiben als: 
\begin{equation}
R_{ges} = R_1 + R_2 = \sum_i{R_i} \, .
\end{equation}
\end{minipage}
%
\subsection{Der (unbelastete) Spannungsteiler}

In Schaltung \ref{fig:Seriell} kann man die Spannung $U_2$, die über den Widerstand $R_2$ abfällt auch als Eingangsspannung für eine daran anschliessende Schaltung benutzen. Schaltung \ref{fig:Seriell} bezeichnet man in diesem Fall als \textit{Spannungsteiler}. Die Spannung $U_2$ wird dann durch die \textit{Spannungsteilerformel} gegeben, die wir kurz herleiten wollen:\\
Fließt ein Strom $I$ durch $R_2$, so fällt über den Widerstand die Spannung ab
\begin{equation*}
U_2 = I\cdot R_2\; .
\end{equation*}
Aus der Maschenregel folgt für den Strom:
\begin{equation*}
I = \frac{U_0}{R_1 + R_2} \; .
\end{equation*}
Einsetzen ergibt die Spannungsteilerformel:
\begin{equation}
U_2 = U_0\cdot\frac{R_2}{R_1+R_2}
\end{equation}

\subsection{Die Wheatstone'sche Brückenschaltung}

Widerstände im Bereich von 0,01 $\Omega$ bis 10 M$\mathrm{\Omega}$ werden oft mit einer Meßbrückenschaltung nach Wheatstone (siehe Abbildung rechts) gemessen. Der zu messende Widerstand sei $R_3$. Er läßt sich bei \"abgeglichener\" Brücke, d.h. wenn kein Strom durch das Amperemeter fließt, aus der Beziehung
\begin{equation}
 R_3 = R_4\frac{R_1}{R_2}
 \label{eq:Wheatstone}
\end{equation}
berechnen. Durch das Amperemeter fließt nämlich nur dann kein Strom, wenn die Punkte C und D auf demselben Potenzial liegen. Es muss also gelten

\begin{minipage}[b]{0.5\textwidth}
\begin{equation}
 U_{AC} = U_{AD},
\end{equation}
woraus folgt
\begin{equation}
 R_1 I_1 = R_3 I_3.
\end{equation}
Gleichzeitig muss gelten
\begin{equation}
 U_{CB} = U_{DB},
\end{equation}
woraus folgt
\begin{equation}
 R_2 I_2 = R_4 I_4.
\end{equation}

Da bei abgeglichener Brücke $I_1 = I_2$ und $I_3 = I_4$ ist, folgt für $R_3$ Gleichung \ref{eq:Wheatstone}.
\end{minipage}
%
\begin{minipage}[b]{0.5\textwidth}
 \centering
 \includegraphics[width=0.7\textwidth]{Versuch_13-14/Abbildungen/Wheatstone_Prinzip.jpg}\\
 Brückenschaltung nach Wheatstone.
\end{minipage}


%------------------------------------------------
\section{Fragen zur Vorbereitung}
%------------------------------------------------

\begin{enumerate}
 %
 \item Was soll heute im Praktikum gemessen werden? Warum?
 %
 \item Wiederholung: Definition von Strom und Spannung
 %
 \item Wiederholung: Ohm'sches Gesetz, Schaltung von Messgeräten
 %
 \item Wie lauten die Kirchhoff'schen Gesetze? Welche Erhaltungssätze liegen ihnen zugrunde?
 %
 \item Wie berechnet man den Gesamtwiderstand bei Reihen- und Parallelschaltungen von Widerständen?
 %
 %\item Was ist ein Innenwiderstand?
 %
 \item Wie funktioniert die Spannungsteilerschaltung (Schaltskizze und Erklärung)?
 %
 \item Wie funktioniert die Wheatstone'sche Brückenschaltung (Schaltskizze und Erklärung)? Was kann man mit ihr messen?
\end{enumerate}

%------------------------------------------------
\section{Durchführung} 
%------------------------------------------------

\begin{minipage}{0.6\textwidth}
 \begin{enumerate}
  %
  \item Der unbekannte Widerstand $R_1$ soll mithilfe des Ohm'schen Gesetzes bestimmt werden.\\
   Erh"ohen Sie die Ausgangsspannung der regelbaren Spannungsquelle $U_1$ von 0,5\,V bis 6\,V in Schritten von 0,5\,V. Messen Sie f"ur jede Spannung den Gesamtstrom $I$ mit dem Amperemeter.\\
   Anmerkung: Ver"andern Sie w"ahrend der Messung den Widerstand $R_X$ nicht. Dann k"onnen $R_2$ und $R_x$ zum Innenwiderstand $R_i$ zusammengfasst werden, der hier nicht betrachtet werden muss.
  %
 \end{enumerate}
\end{minipage}
%
\begin{minipage}{0.35\textwidth}
 \includegraphics[width=1.00\textwidth]{Versuch_13-14/Abbildungen/Schaltung1.jpg}
 \label{fig:Schaltung1}
\end{minipage}

\begin{enumerate} \setcounter{enumi}{1}
 %
 \item Messen Sie die feste Ausgangsspannung $U_k$ mit dem Voltmeter.
 %
 \item Die drei unbekannten Widerstände $R_x$(1,2,3) sollen nach den beiden Schaltungen A und B gemessen werden. Messen Sie dazu in Schaltung A den Spannungsabfall $U_1$ am Widerstand $R_1$ und in Schaltung B den Gesamtstrom $I_1$.
 %
\end{enumerate}

%\begin{figure}[h]
\begin{minipage}[b]{0.5\textwidth}
 \centering
 \includegraphics[width=0.7\textwidth]{Versuch_13-14/Abbildungen/SchaltungA.jpg}\\
 Schaltung A
\end{minipage}
%
\begin{minipage}[b]{0.5\textwidth}
 \centering
 \includegraphics[width=0.7\textwidth]{Versuch_13-14/Abbildungen/SchaltungB.jpg}\\
 Schaltung B
\end{minipage}
%\end{figure}

%\begin{figure}[h]
\begin{minipage}[b]{0.6\textwidth}
\begin{enumerate} \setcounter{enumi}{3}
 %
 \item Die Innenwiderstände der Spannungsquelle und des Amperemeters sollen gemessen werden. Messen Sie dazu in Schaltung C den Gesamtstrom $I_g$ sowie die Spannung über die Lastwiderstände $U_L$. Die Größe des Widerstandes $R_3$ kann gegenüber dem großen Innenwiderstand des Voltmeters ($\approx\, 10\,M\Omega$) vernachlässigt werden. Dann berechnet sich der Innenwiderstand der Spannungsquelle nach
 \begin{equation}
  R_i = \frac{U_k - U_L}{I_g}\, .
 \end{equation}
 %
\end{enumerate}
\end{minipage}
%
\begin{minipage}[b]{0.35\textwidth}
 \centering
 \includegraphics[width=0.8\textwidth]{Versuch_13-14/Abbildungen/SchaltungC.jpg}\\
 Schaltung C
\end{minipage}
%\end{figure}

%\begin{figure}[h]
\begin{minipage}[b]{0.6\textwidth}
\begin{enumerate} \setcounter{enumi}{4}
 %
 \item Mithilfe der Wheatstone'schen Brückenschaltung sollen die drei unbekannten Widerstände $R_y$ (y=1,2,3) bestimmt werden. \\
  Bauen sie den Versuch auf der rechten Hälfte des Experimentierschaltkreises auf. Betreiben Sie hierbei das Amperemeter im empfindlichsten Bereich.\\
  Verdrehen Sie nun das Potentiometer $R_3$ so lange, bis kein Strom mehr durch das Amperemeter fließt.\\
  Die Ablesung 'X' am Potentiometer bedeutet
  \begin{equation}
   \frac{R_{3b}}{R_{3a}} = \frac{X}{100 - X}
  \end{equation}
  und ist damit nicht direkt der Widerstand.
 %
\end{enumerate}
\end{minipage}
%
\begin{minipage}[b]{0.35\textwidth}
 \centering
 \includegraphics[width=0.8\textwidth]{Versuch_13-14/Abbildungen/Wheatstone.jpg}
 \label{fig:Wheatstone}\\
 Wheatsone'sche Brückenschaltung
\end{minipage}
%\end{figure}

%\pagebreak

%------------------------------------------------
\section{Auswertung} 
%------------------------------------------------

\begin{enumerate}
 %
 \item Berechnen Sie den Mittelwert des Widerstands $R_1$ inkl. seines Fehlers.
 %
 \item Leiten Sie die Formel für den Widerstand $R_x$ in Schaltung A her.\\
  Anleitung: Berechnen Sie den Gesamtwiderstand der Parallelschaltung von $R_x$ und $R_2$. Zusammen mit der Spannungsteilerformel erhält man schließlich:
  \begin{equation}
   R_x = R_2 \frac{1-\frac{U_k}{U_1}}{\frac{U_k}{U_1}-\frac{R_2}{R_1}-1}
  \end{equation}
 %
 \item Leiten Sie die Formel für den Widerstand $R_x$ in Schaltung B her.\\
  Anleitung: Drücken Sie $U_1$ durch den Strom $I$ aus. Danach, wie oben, einfach nach $R_x$ auflösen. Dann finden Sie:
  \begin{equation}
   R_x = R_2 \frac{R_1 - \frac{U_k}{I}}{\frac{U_k}{I}-R_1-R_2}
  \end{equation}
 %
 \item Berechnen Sie nun die Widerstände $R_x$ (x=1,2,3) aus Schaltung A und B mit ihrem Fehler. 
  Benutzen Sie eine Ungenauigkeit ($\hat{=}$ Fehler) für die Messungen der Ströme und Spannungen von 1\% .
 %
 \item Berechnen Sie den Innenwiderstand der Spannungsquelle.
 %
 \item Berechnen Sie den Widerstand $R_y$ (y=1,2,3) aus der Wheatstoneschen Brückenschaltung.\\
  Hinweis: Aus der Bedingung, dass durch das Amperemeter kein Strom fließt, ergibt sich:
  \begin{equation}
   R_y = R_4\frac{R_{3b}}{R_{3a}} \; .
  \end{equation}
\end{enumerate}	% Elektrische Netzwerke
\chapter{Nichtstationäre Diffusion}
\label{v:14}

In diesem Versuch betrachten Sie, wie ein Farbstoff in Wasser diffundiert. Dabei messen Sie die momentane Konzentration des Farbstoffes mithilfe eines lichtempfindlichen Widerstandes.

%------------------------------------------------
\section{Stichworte}
%------------------------------------------------

Wheatstone'sche Brückenschaltung; Photowiderstand; Brown'sche Molekularbewegung; stationäre und nichtstationäre Diffusion; Fick'sche Gesetze; Diffusionkoeffizient; Diffusionsgleichung.
%
%------------------------------------------------
\section{Literatur}
%------------------------------------------------

Gehrtsen, Kapitel 5.2.8 und 5.4
%
%------------------------------------------------
\section{Anwendungsbeispiele}
%------------------------------------------------

Diffusion spielt sowohl in der Natur, als auch in zahlreichen technischen Anwendungen eine tragende Rolle.\\
Zum Beispiel geschieht der Gasaustausch zwischen Lungenbläschen und Blut bei der Lungenatmung durch Diffusion, genauso wie der Transport bestimmter Stoffe durch Membranen (teilweise sog. erleichterte Diffusion). \\
Beim Sintern von Werkstoffen spielt die Diffusion eine wichtige Rolle beim Zusammenwachsen der Pulverbestandteile, in der Halbleiterindustrie wird sie benutzt, um Dotierungsstoffe in das Halbleitermaterial einzubringen. In der technischen Chemie spielt Diffusion, gekoppelt mit Konvektion und chemischen Reaktionen eine zentrale Rolle im Reaktor- und Katalysatordesign.

%------------------------------------------------
\section{Theoretischer Hintergrund}
%------------------------------------------------

\subsection{Diffusion in Gasen und Lösungen}

Schichtet man Alkohol vorsichtig über Wasser oder reines Wasser über eine Salzlösung, dann wird die anfangs scharfe Trennfläche allmählich diffuser. Die steile Dichtestufe flacht sich mit der Zeit immer mehr ab. In Lösungen dauert diese Durchmischung Stunden, in Gasen nur Sekunden.\\
Diese Diffusion findet immer dann statt, wenn die Konzentration eines gelösten Stoffes, der Druck eines Gases oder der Partialdruck eines Bestandteiles eines Gasgemisches, allgemein also wenn die Teilchenzahldichte $n$ von Ort zu Ort unterschiedlich ist. Der Vorgang endet erst, wenn die Teilchendichte an allen Punkten des zur Verfügung stehenden Volumens gleich groß ist, falls Teilchen dieses nicht verlassen oder von außen hereinkommen können. Zustande kommt diese Bewegung der Teilchen durch ihre thermische Energie (Brown'sche Bewegung).\\
Der Teilchentransport in der Diffusion wird durch den Gradienten der Teilchenzahldichte, $\mathrm{grad} n$, angetrieben. Die \textit{Teilchenstromdichte} $\vec{j}_n$, ein Vektor, dessen Betrag die Anzahl von Teilchen darstellt, die pro Sekunde durch die Flächeneinheit treten, ist proportional zum Gefälle von $n$:\\

1. Ficksches Gesetz:
\begin{equation}
 \vec{j}_n = -D\,\mathrm{grad}\, n.
 \label{eq:Fick1}
\end{equation}

Der Diffusionstrom fließt immer in die Richtung, in die $n$ am schnellsten abnimmt. $D$ ist der Diffusionskoeffizient, dessen Einheit $\left[D\right] = \mathrm{m^2/s}$ ist.\\

Wenn aus einem Volumen mehr Teilchen ausströmen als hineinfließen, nimmt die Teilchenzahl dort ab:
\begin{equation}
 \dot{n} = -\mathrm{div}\,\vec{j}_n\; .
\end{equation}

Mit Gleichung \ref{eq:Fick1}) erhält man die allgemeine Diffusionsgleichung\\

2. Ficksches Gesetz:
\begin{equation}
 \dot{n} = D\,\mathrm{div}\,\mathrm{grad}\, n = D\,\Delta n\; .
\end{equation}

Diese Art der Diffusion, die durch einen Konzentrationsgradienten getrieben wird, nennt man auch \textit{nichtstationäre Diffusion}. Ihr gegenüber steht die \textit{stationäre Diffusion}, auch Selbstdiffusion genannt, die beschreibt wie sich Teilchen innerhalb derselben Substanz bewegen. Da die Teilchen prinzipiell ununterscheidbar sind, ist die stationäre Diffusion allerdings sehr schwierig zu beobachten.
%------------------------------------------------
\section{Fragen zur Vorbereitung}
%------------------------------------------------

\begin{enumerate}
 %
 %\item Was soll heute im Praktikum gemessen werden? Warum?
 %%
 \item Was ist eine Wheatstone'sche Brückenschaltung? Wie funktioniert sie?
 %
 \item Was ist ein Photowiderstand?
 %
 \item Was versteht man unter Brown'scher Molekularbewegung? Wodurch entsteht sie?
 %
 \item Welcher Effekt wird als Diffusion bezeichnet?
 %
 %\item Was ist stationäre bzw. nichtstationäre Diffusion?
 %
 \item Was ist die Bedeutung der Diffusionskonstanten D (Diffusionsgleichung für stationäre Diffusion) ?
 %
 \item Wie wird ein Photowiderstand zusammen mit einer Wheatstone Brücke zur Messung der Diffusion benutzt ?
%
\end{enumerate}

%------------------------------------------------
\section{Durchführung} 
%------------------------------------------------

\begin{minipage}{0.6\textwidth}
 \begin{enumerate}
  %
  \item Bauen Sie zunächst die Wheatstone'sche Brückenschaltung auf.\\
  
   Der Strom, den das Amperemeter zeigt, hängt bei vorgegebenem Vergleichswiderstand $R_V$ von der Potentiometereinstellung und von dem elektrischen Widerstand des Photowiderstandes $R_P$ ab. Dieser
   Widerstand ist durch die Lichtmenge bestimmt, die auf den Photowiderstand fällt.
  %
 \end{enumerate}
\end{minipage}
%
\begin{minipage}{0.35\textwidth}
 \includegraphics[width=1.00\textwidth]{Versuch_13-14/Abbildungen/Wheatstone14.jpg}
 \label{fig:Schaltung1}
\end{minipage}

\begin{enumerate} \setcounter{enumi}{1}
 %
 \item Stellen Sie den Strahlengang so ein, dass der Beleuchtungsspalt scharf auf den zweiten Spalt abgebildet wird, hinter dem sich der Photowiderstand befindet. Das Amperemeter in der Brückenschaltung zeigt dann den maximalen Strom an. Fixieren Sie den Strahlengang in dieser Konfiguration.
 %
 \item Schieben Sie den Graufilter C016 vor den Photowiderstand und gleichen Sie die Brückenschaltung ab, so dass das Amperemeter keinen Strom mehr anzeigt. Arretieren Sie das Potentiometer in dieser Stellung.
 %
 \item Füllen Sie die Diffusionsküvette zu drei Vierteln mit Wasser und bringen Sie sie anstelle des Graufilters in den Strahlengang, so dass die Wasseroberfläche sich auf gleicher Höhe mit dem Lichtstrahl befindet.
 %
 \item Geben Sie einige Tropfen Farbstofflösung der Anfangskonzentration $C_0$ auf das Wasser und starten Sie zur gleichen Zeit die Stoppuhr ($t=0$).
 %
 \item Verschieben Sie nun die Küvette mittels des Höhentriebes so weit, dass sich die Farbzone mit der Konzentration $C = 1/16\; C_0$ vor dem Spalt befindet.\\
  Diese Farbzone absorbiert Licht genauso stark wie der Graufilter, also zeigt das Amperemeter bei richtiger Einstellung wieder keinen Strom mehr an ($I = 0$). Notieren Sie den Ort $x$ der Farbzone, den man am Höhentrieb ablesen kann.
 %
 \item Wiederholen Sie diese Einstellung zu den Zeiten t = 1, 2, 4, 6, 9, 12, 16, 19, 22, 25, 27, 30, 33, 36 Minuten und notieren Sie jeweils den Ort der Farbzone.
 %
\end{enumerate}

\subsection*{Hinweise:}
\begin{itemize}
 %
 \item Halten sie während der Messung nicht die Stoppuhr an und verstellen Sie nicht das Potentiometer.
 %
 \item Verstellen Sie stets nur den Höhentrieb der Küvette. Bei Manipulation im Strahlengang stellen Sie das Amperemeter auf den Messbereich 'grob'.
 %
 \item Vermeiden Sie größere Erschütterungen des Tisches.
 %
 \item Reinigen und trocknen Sie bitte nach Ende des Versuchs die Küvette gründlich.
 %
\end{itemize}

%------------------------------------------------
\section{Auswertung} 
%------------------------------------------------

\begin{enumerate}
 %
 \item Tragen Sie den Ort $x$ als Funktion von $\sqrt{t}$ graphisch auf. \label{Aufg:1}
 %
 \item Berechnen Sie aus der Steigung der Geraden den Diffusionskoeffizienten $D$ nach:
  \begin{equation}
   D = \frac{x^2}{t}\cdot f\left(\frac{C}{C_0}\right)
  \end{equation}
	
	\noindent
	Bemerkung:\\ 
	\noindent
	$f\left(C/C_0\right) = f\left(1/16\right) = 0,212$ ist ein numerischer Faktor, der die Abhängigkeit des Diffusionskoeffizienten $D$ von der Konzentration der beobachteten Farbzone angibt.
 %
 \item Schätzen Sie den Fehler von $D$ aus den Grenzgeraden der Auftragung aus Aufgabe \ref{Aufg:1}) ab.
 %
\end{enumerate} 	% Nichtstation�re Diffusion

\include{Versuch_19-20/Versuch-19}	% K�nstliche Radioaktivit�t
\include{Versuch_19-20/Versuch-20}	% Spezifische Elektronenladung e/m

\chapter{Spezifische Wärmekapazität}
\label{v:5}

In diesem Versuch lernen Sie Messmethoden zur Bestimmung der spezifischen Wärmekapazität verschiedener Stoffe kennen.

%------------------------------------------------
\section{Stichworte}
%------------------------------------------------
Temperatur; Wärme; Wärmemenge; (spezifische) Wärmekapazität; Regel von Dulong-Petit; Kalorimeter.
%
%------------------------------------------------
\section{Literatur}
%------------------------------------------------
Gehrtsen, Kapitel 5.1.1 und 5.1.5/6
%
%------------------------------------------------
\section{Anwendungsbeispiele}
%------------------------------------------------
%
Die spezifische Wärmekapazität ist eine Stoffeigenschaft die angibt, wie viel Wärme von einem Körper aufgenommen werden muss, damit sich die Temperatur von 1~kg des Stoffes um 1\degree ändert. Die spez. Wärmekapazität spielt also immer dann eine Rolle, wenn Stoffe erwärmt oder abgekühlt werden sollen:\\
Wie lange dauert es, eine biologische Probe einzufrieren, oder einen Braten zuzubereiten? Welche Leistung muss ein Klimasystem (z. Bsp. Klimaanlage, Heizung) haben, um Stoffe auf die gewünschte Temperatur bringen zu können? Wieso ändert sich die Temperatur am Meer weniger als im Landesinneren?
%
%------------------------------------------------
\section{Theoretischer Hintergrund}
%------------------------------------------------

\subsection{Eine kurze Einführung in die Wärmelehre}

Die ganze Wärmelehre läßt sich in wenigen kurzen Sätzen zusammenfassen: 
\begin{itemize}
 %
 \item Wärme ist die ungeordnete Bewegung von Molekülen.
 %
 \item Wärmeenergie ist die Energie dieser Bewegung.
 %
 \item Temperatur ist ein Maß für den Mittelwert dieser kinetischen Energie.
\end{itemize}

Wenn man nur die kinetische Energie der Translationsbewegung der Moleküle betrachtet (Welche Bewegungen/Energien gibt es noch?), so kann man die Temperatur über deren Mittelwert definieren:
\begin{equation} \label{eq:DefT}
 \bar{E}_{trans} = \frac{1}{2}m\overline{v^2} = \frac{3}{2}kT\; .
\end{equation}
Die Konstante $k$, die sogenannte \textit{Boltzmann-Konstante}, hat den Wert $k=1,381\cdot10^{-23}\,J/K$.\\
Die Temperatur wird in K (Kelvin, nicht in Grad Kelvin) gemessen. Man erkennt in Gleichung \ref{eq:DefT}, dass es einen nichtunterschreitbaren absoluten Nullpunkt der Temperatur gibt, bei dem die Moleküle völlig ruhen, also $E$ und $T$ Null sind. Von hier aus zählt man die absolute oder Kelvin-Temperatur.\\
Ihre Einheit, 1\,K, ist ebensogroß wie $1^{\circ}$\,C, das als $\frac{1}{100}$ des Abstandes zwischen dem Gefrier- und dem Siedepunkt des Wassers bei einem Druck von 1,013\,bar definiert ist. Bei diesem Druck liegt der Gefrierpunkt des Wassers bei 273,2\,K, sein Siedepunkt bei 373,2\,K.\\

Moleküle können nicht nur Translationsenergie haben, sondern auch Rotationsenergie. Außerdem können ihre Bestandteile, Atome, Ionen und sogar Elektronen, gegeneinander schwingen. Jede solche unabhängige Bewegungsmöglichkeit nennt man einen \textit{Freiheitsgrad}.\\
Eine Translationsbewegung hat drei Freiheitsgrade, nämlich die drei unabhängigen Raumrichtungen. Auch die Rotation hat drei Freiheitsgrade, entsprechend der drei unabhängigen Rotationsachsen. Bei bestimmten Körpern kann es jedoch sein, dass einer oder mehrere solcher Rotationsbewegungen nicht zur Energiebilanz beitragen. Ein Beispiel wäre ein zweiatomiges Molekül, welches um die Achse zwischen den beiden Atomen rotiert. Dann fällt einer oder mehrere der Freiheitsgrade weg. Ein Beispiel, bei dem mehr als ein Rotationsfreiheitsgrad wegfällt, wäre ein kugelsymmetrisches Atom. Bei diesem trägt keine der Rotationsachsen, welche durch den 'geometrischen' Mittelpunkt des Atoms geht, zur Energiebilanz bei.\\
Wenn ein Atom in ein Kristallgitter eingebaut ist, kann es meist nicht mehr rotieren, aber Schwingungen in alle drei Raumrichtungen sind möglich, welche ebenfalls Freiheitsgrade darstellen.\\

Auf jeden dieser Freiheitsgrade entfällt im thermischen Gleichgewicht die gleiche mittlere Energie, und zwar für jedes Molekül $\overline{E}_{FG}= \frac{1}{2}\, kT$. Damit wird die Gesamtenergie für ein Molekül mit $f$ Freiheitsgraden
\begin{equation} \label{eq:DefW}
 \overline{E}_{mol} = \frac{f}{2}kT\; .
\end{equation}

\subsection{Wärmekapazität}

Um einen Körper von der Temperatur $T_1$ auf die Tempereatur $T_2$ zu erwärmen, muss man ihm Energie zuführen. Die benötigte Energie folgt direkt aus Gleichung \ref{eq:DefW}, wenn man weiss, wieviele Moleküle der Körper enthält. Ein homogener (aus lauter gleichen Molekülen zusammengesetzter) Körper der Masse $M$ enthält $M/m$ Moleküle der Masse $m$. Die benötigte Energie beträgt also
\begin{equation}
 \Delta E = \frac{M}{m}\frac{f}{2}k\Delta T\; .
\end{equation}

Man nennt das Verhältnis 
\begin{equation}
 C = \frac{\Delta E}{\Delta T} = \frac{M}{m}\frac{f}{2}k
\end{equation}
die \textit{Wärmekapazität} des Körpers. Bezogen auf 1~kg eines bestimmten Stoffes erhält man dessen \textit{spezifische Wärmekapazität} 
\begin{equation} \label{eq:spez_Waermekapazitaet}
 c = \frac{\Delta E}{M\,\Delta T} = \frac{fk}{2m}\; .
\end{equation}
Bezogen auf ein mol eines Stoffes erhält man genauso dessen \textit{molare Wärmekapazität}. Da die Anzahl der Moleküle in einem mol eines Stoffes bekanntermaßen immer der Avogadro-Konstanten $N_A$ entspricht, können wir die für einfach Stoffe wie Gase oder feste Metalle, bzw. allgemeine Elementkristalle mit $f=6$ die molare Wärmekapazität entsprechend der \textit{Regel von Dulong-Petit} schreiben als
\begin{equation} \label{eq:Dulong-Petit}
 C_{mol} = N_A\frac{f}{2}k = 3\,N_A\,k = 24,9\,\mathrm{J\,mol^{-1}\,K^{-1}}\; .
\end{equation}

%------------------------------------------------
\section{Fragen zur Vorbereitung}
%------------------------------------------------

\begin{enumerate} 
 \item Was soll heute im Praktikum gemessen werden? Warum?
 %
 \item Wie lautet der erste Hauptsatz der Wärmelehre und gilt für ihn die Energieerhaltung?
 %
 \item Wie ist die Wärmekapazität definiert und was beschreibt sie?
 %
 \item Wodurch unterscheiden sich spezifische und molare Wärmekapazität?
 %
 \item Welcher Zusammenhang besteht zwischen Kalorie und Joule?
 %
 \item Was ist ein Dewar-Gefäß (Kalorimeter)? Wie bestimmt man im Versuch seine Wärmekapazität?
 %
 \item Was besagt die Regel von Dulong-Petit, wie kommt sie zu Stande, welchen Zusammenhang gibt es zu den Freiheitsgraden in Festkörpern?
 %
 \item Wie funktioniert ein Druckkochtopf?
\end{enumerate} 

%------------------------------------------------
\section{Durchführung} 
%------------------------------------------------

Vor jedem Mischversuch sind die Masse und Temperatur des Wassers im Kalorimeter zu bestimmen. Um eine homogene Temperaturverteilung im Kalorimeter zu erzielen, sollte der Rührer bei jeder Temperaturmessung hinzugeschaltet werden. Schalten Sie das Rührwerk nach Ende der Messung bitte wieder ab.\\

\underline{\textbf{Achtung:}} Bitte mit den heißen Gegenständen im Versuch vorsichtig umgehen!

\begin{enumerate}
 %
 \item Bereiten Sie eine Tabelle für die folgende Messung vor.
 %
 \item Erhitzen Sie den Al-Körper in kochendem Wasser auf 100°\,C. Messen Sie währenddessen die Temperatur des isolierten Wasserbades (Kalorimeter) mit kaltem Wasser für 5 Minuten alle 20 Sekunden. Bringen Sie nun den erhitzten Metallkörper in das Wasserbad und messen Sie über die nächsten zwei Minuten die Temperatur möglichst alle 5 Sekunden, danach für drei Minuten alle 20 Sekunden.
 %
 \item Tragen Sie den Temperaturverlauf des Wassers über die gesamte Messdauer von zehn Minuten grafisch auf. Bestimmen Sie die Anfangs- und Misch-Temperatur durch Extrapolation der gemessenen Werte. Überlegen Sie sich eine sinnvolle Abschätzung des Fehlers des Temperaturunterschiedes.
 %
 \item Für den Cu und den Fe-Körper wird die Prozedur vereinfacht wiederholt. Messen Sie für diese nur die Anfangstemperatur vor dem Einbringen des 100°\,C heißen Metallkörpers und die Endtemperatur des Kalorimeters wenn dieses im thermischen Gleichgewicht ist.
 %
 \item Führen Sie einen einfachen Mischversuch durch, um die Wärmekapazität des Kalorimeters berechnen zu können. Dazu gießen Sie heißes Wasser in das kalte Wasser des Kalorimeters. Messen Sie alle benötigten Temperaturen und Massen.
 %
 \item Bestimmen Sie die Massen der drei Metallkörper mit der Waage (jedes Stück nur einmal).
 %
\end{enumerate}
%------------------------------------------------
\section{Auswertung} 
%------------------------------------------------

\begin{enumerate}
 %
 \item Bestimmen Sie die Wärmekapazität des Kalorimeters. Beachten Sie die Messfehler.
 %
 \item Bestimmen Sie die spezifische Wärmekapazität von Al, Cu und Fe inklusive ihrer Fehler.
 %
 \item Überprüfen Sie die Gültigkeit der Regel von Dulong-Petit für die drei Metalle.
\end{enumerate}			% Spezifische W�rmekapazit�t
\include{Versuch_5-6/Versuch-6}			% Molare W�rmekapazit�t von Luft

%%%%%%%%%%%%%%%%%%%%%%%%%%%%%%%%%%%%%%%%%%%%%%%%%%%%%%%%%%%
%
%\cleardoublepage
%\appendix
%%
%%%%%%%%%%%%%%%%%%%%%%%%%%%%%%%%%%%%%%%%%%%%%%%%%%%%%%%%%%%%
%
\backmatter

\renewcommand{\thechapter}{\Roman{chapter}}
\setcounter{chapter}{10}

 \include{00_einl/raumverz}

\end{document}

