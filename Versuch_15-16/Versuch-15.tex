\chapter{Wechselstrom und R-C-Kreis}
\label{v:15}

In diesem Versuch lernen Sie die Grundlagen der Funktion und Bedienung eines Digital-Speicher-Oszilloskops (DSO), sowie die Auf- und Entladekurve eines Kondesnators kennen.

%------------------------------------------------
\section{Stichworte}
%------------------------------------------------

Kathodenstrahl; Braun'sche Röhre; Oszillograph; Ablenkung im elektrischen Feld; Wechselspannung; Kondensator; Kapazität.
%
%------------------------------------------------
\section{Literatur}
%------------------------------------------------

Gehrtsen, Kapitel 7.5.2/3, 8.2.1, 8.2.3
%Teile übernommen aus der Anleitung \textit{''Oszilloskop und Funktionsgenerator''} der Universität Oldenburg
%
%------------------------------------------------
\section{Anwendungsbeispiele}
%------------------------------------------------

R ist die Abkürzung (Symbol) für den elektrischen Widerstand, C für die Kapazität eines Kondensators und L für die Induktivität einer Spule; R-C-Kreise sind Kombinationen aus Widerständen und Kondensatoren, die ein für viele physikalische Vorgänge charakteristisches exponentielles Abklingverhalten zeigen. R-L-Kreise, aus Widerstand und Spule aufgebaut, zeigen auf vereinfachte Weise, was passiert wenn bei den meisten Haushaltsgeräten die Spannung eingeschaltet wird (z. Bsp. Glühbirnen).\\

Eine Zellmembran mit geringer Leitfähigkeit als Grenzschicht zwischen gut leitenden elektrolytischen Flüssigkeiten stellt eine elektrische Parallelschaltung eines Widerstandes (Membranwiderstand) und eines Kondensators (Membrankapazität) dar. Aus der Anwendung entsprechender physikalischer Modellvorstellungen kann ein Verständnis der physiologischen Vorgänge und Messtechniken zur Untersuchung von Zelleigenschaften und interzellulären Vorgängen gewonnen werden.

%------------------------------------------------
\section{Theoretischer Hintergrund}
%------------------------------------------------

\subsection{Oszilloskop}

\begin{figure}[hb]
	\centering
		\includegraphics[width=0.5\textwidth]{Versuch_15-16/Abbildungen/TDS2000.JPG}
	\caption{Front des TDS 2001C.}
	\label{fig:TDS2000}
\end{figure}

%Das Oszilloskop ist das vielseitigste Messgerät im Bereich der Elektrik und Elektronik. Es dient der Beobachtung und Messung zeitabhängiger, schneller und wiederkehrender elektrischer Signale. Bis vor einigen Jahren waren noch hauptsächlich analoge Elektronenstrahl-Oszilloskope im Einsatz, die inzwischen jedoch weitestgehend von Digital-Speicher-Oszilloskopen verdrängt wurden. Dennoch wollen wir mit einer kurzen Beschreibung der Funktionsweise des Elektronenstrahl-Oszilloskops beginnen, diese viele grundlegende Funktionen des Oszilloskops anschaulich erklärt.
%
%\begin{figure}[h!]
	%\centering
		%\includegraphics[width=0.5\textwidth]{Versuch_15-16/Abbildungen/Elektronenstrahlroehre.jpg}
	%\caption{Aufbau der Elektronenstrahlröhre des Oszilloskops.}
	%\label{fig:Elektronenstrahlroehre}
%\end{figure}
%
%Kernstück eines Oszilloskops ist eine Elektronenstrahlröhre, die einen Strahl beschleunigter Elektronen als quasi trägheitslosen Zeiger auf einen szintillierenden Schirm schießt. Zur Auslenkung dieses Strahls in der vertikalen Y Richtung wird dabei die zu untersuchende Eingangsspannung an die Platten eines Ablenkkondensators gelegt, so dass die Elektronen im elektrischen Feld des Kondensators abgelenkt werden.\\
%Um die zeitliche Abhängigkeit der Eingangsspannung darzustellen, wird eine linear steigende (Sägezahn-)Spannung (Kippspannung) an die horizontalen Ablenkplatten gelegt, sodass der Elektronenstrahl mit gleichbleibender Geschwindigkeit die X Achse überstreicht.\\
%Verschieden schnelle Signale können über Änderungen der Frequenz der Sägezahnspannung dargestellt werden (Einstellknopf 'SEC/DIV', 2 in Abbildung \ref{fig:TDS2000}). Signale mit verschiedener Amplitude können durch Anpassung der Verstärkung der Eingangsspannung sichtbar gemacht werden (Einstellknopf 'VOLTS/DIV', 1 in Abbildung \ref{fig:TDS2000}).\\
%Ein Trigger-Netzwerk sorgt dafür, dass die Zeitbasis immer in dem Moment ''angestoßen'' wird, in dem das Signal am Y-Eingang erscheint und damit Signal und Zeitbasis synchronisiert werden. Für ein wiederkehrendes Signal erhält man auf diese Weise ein stehendes Bild auf dem Bildschirm des Oszilloskops.\\
%

%
%\noindent
%Auf der Frontseite des hier benutzten TDS2001C Oszilloskops erkennt man 3 Bedienfelder:\\
%Im linken Teil findet man die Einstellungen für die Verstärkung der Eingangssignale der beiden Kanäle (2), sowie die Knöpfe für die Kanalmenüs (1). Hier lässt sich die Einkopplung des Eingangssignals (DC), die Bandbreite des Kanals (Voll), sowie die Grob- bzw. Feineinstellung der Verstärkung einstellen. Des Weiteren findet man hier einen einstellbaren Verstärkungsfaktor für den Eingang (sollte 1X sein), sowie die Möglichkeit, das Eingangssignal zu invertieren (Aus).\\
%Im mittleren Teil finden sich die Einstellungen für die Zeitachse, welche für alle Kanäle gleich ist. Hier können Sie die Vergrößerung der Zeitachse ('SEC/DIV', Drehknopf 2) einstellen, sowie die Zeitachse hin- und herbewegen. Dabei gibt der vertikale weiße Pfeil den Zeitpunkt an, an dem die Triggerbedingung erfüllt wurde.\\
%Im rechten Teil können Sie den Spannungspegel einstellen, 
%Rechts oben im Feld befinden sich der Einschaltknopf (1), die anderen Knöpfe bestimmen die Zeitablenkung des Oszilloskops, d.h. die x-Achse des Bildes und die sog. Triggerung, d.h. den Beginn der Ablenkung des Elektronenstrahls von links nach rechts.\\
%Im unteren rechten Feld werden die zu untersuchenden Spannungen eingegeben. Dieses Gerät kann zwei Signale gleichzeitig anzeigen, daher sind viele Funktionen doppelt. Es regelt also die Y-Achse.\\
%Im schmalen linken Feld unter dem Bildschirm gibt es die Helligkeits- (18) und Schärfe-regelung (20) des Bildes, sowie weitere Testfunktionen des Gerätes.

Will man schnell laufende Vorgänge, z.B. Wechselspannungen oder Impulse, stetig messen und in Abhängigkeit von der Zeit registrieren, so verwendet man ein Oszilloskop. Dieses stellt Spannungen direkt als Funktion der Zeit dar und ermöglicht so die Beobachtung von Vorgängen mit hoher Frequenz.

  Meist benutzt man das Oszilloskop im sogenannten \textit{YT-Modus}, bei dem auf der horizontalen X-Achse des Displays die Zeit $t$ dargestellt wird. Die \textit{Zeitbasis}, d.h. die Skalierung der Zeitachse, kann durch den horizontalen Wahlschalter zwischen \unit{5}{\nano\second} pro Kästchen und \unit{50}{\second} pro Kästchen einstellen. 
  Die K\"astchen werden am Oszilloskop als DIV (f\"ur Division) bezeichnet. 
  Weitere Eigenschaften der Zeitachse können in dem Menü eingestellt werden, welches nach Drücken der 
  Taste HORIZ MENU angezeigt wird. Für die richtige Wahl der Zeitbasis sollten Sie sich klar machen, mit welcher Frequenz die Spannung am Eingang des Oszilloskops sich verändert.

  Auf der vertikalen Y-Achse wird die am Eingang gemessene Spannung dargestellt. Die Skala der Y-Achse kann über den vertikalen Wahlschalter für jeden der vier Kanäle des Oszilloskops getrennt zwischen \unit{20}{\milli\volt} pro Kästchen und \unit{50}{\volt} pro Kästchen eingestellt werden. Zur richtigen Wahl des Y-Verstärkungsfaktors sollten Sie sich klarmachen, welche Spannung Sie an der zu messenden Stelle Ihrer Schaltung erwarten. 

  \paragraph{Kanaleigenschaften}
  Weitere Eigenschaften der Y-Achse können im sogenannten \textit{Kanalmenü} eingestellt werden, welches nach Drücken der Taste CH MENU für den entsprechenden Kanal angezeigt wird. Diese Eigenschaften können für die beiden Kanäle unabhängig von einander eingestellt werden. 
  \begin{itemize}
    \item KOPPLUNG: Es ist möglich, über eine zum Eingang in Reihe geschaltete Kapazität, den Gleichspannungsanteil der Eingangsspannung zu unterdrücken. Dies geschieht, wenn die Kopplung des Kanals auf AC gestellt wird. Dieser Modus ist nützlich, um Spannungsänderungen zu untersuchen und beschleunigt die Arbeit mit dem Oszilloskop erheblich,
      falls der Gleichspannungsanteil tats\"achlich irrelevant ist.
      \begin{important}
        Während des Versuchs stellen Sie die Kopplung bitte auf DC.
      \end{important}

    \item BANDBREITE: Die Bandbreiteneinstellung bestimmt die frequenzabhängig Unterdrückung von Eingangssignalen. Wenn Sie Signale in der Gößenordnung MHz darstellen wollen, sollten Sie darauf achten, dass die Bandbreite auf den maximalen Wert für den Oszilloskoptyp eingestellt ist, da ansonsten die Signalform stark verzerrt dargestellt wird.
    \item TASTKOPF: Die Tastköpfe, mit denen Sie Ihre Schaltung untersuchen, können die Eingangsspannung über einen einstellbaren Spannungsteiler um den Faktor 10 unterdrücken. Dies dient dazu, größere Spannungen auf dem Oszilloskop darstellen zu können, als man sicher an den Eingang anschliessen könnte ohne Bauteile im Oszilloskop zu zerstören. 
    Stellen Sie sicher, dass die Unterdrückung des Tastkopfes und die im Kanalmenü eingestellte Unterdrückung übereinstimmen, da Sie ansonsten andere Spannungswerte am Oszilloskop ablesen, als wirklich in der Schaltung anliegen.
			\begin{important}
				Im Versuch stellen Sie bitte sicher, dass keine Unterdrückung (1X) eingestellt ist.
			\end{important}
    \item INVERTIERUNG: Hiermit können Sie anstelle der Spannung $U$ die invertierte Spannung $-U$ auf dem Oszilloskop darstellen. 
			\begin{important}
				Im Versuch stellen Sie bitte sicher, dass keine Invertierung eingestellt ist.
			\end{important}
  \end{itemize}

  \paragraph{Trigger} 
  Das Oszilloskop stellt die Eingangsspannung auf dem Display dar, wenn die sogenannte \textit{Triggerbedingung} erfüllt ist. Diese stellt man im Trigger Menü ein, welches nach Drücken der Taste TRIG MENU angezeigt wird. 
  Das Oszilloskop stellt mehrere verschiedene Arten von Triggerbedingungen zur Verfügung. So kann, unter anderem auf Pulse bestimmter Breite getriggert werden. Im Praktikum ist allerdings der meistbenutzte Triggertyp der sogenannte \textit{Flankentrigger}. Eine typische Triggerbedinung lautet: Triggere, wenn die Spannung am Kanal 1 einen Wert von \unit{0.1}{\volt} überschreitet.\\
  Diese Bedingung besteht aus drei separat einstellbaren Teilen:
  \begin{enumerate}
    \item Die Triggerquelle, d.h. die Spannung welches Kanals soll betrachtet werden? Wird im Triggermenü über den Punkt QUELLE eingestellt.
    \item Der Triggertyp, d.h. Eingangsspannung soll den Schwellenwert von unten überschreiten (positive oder steigende Flanke) oder von oben unterschreiten (negative oder fallende Flanke). Wird im Triggermenü über den Punkt FLANKE eingestellt.
    \item Der Schwellenwert, d.h. wie groß ist die Spannung, mit der die Engangsspannung verglichen werden soll? Wird über den Drehknopf LEVEL eingestellt.
  \end{enumerate}
  Eine weitere wichtige Eigenschaft des Triggers ist der \textit{Triggermodus} (Triggermenü, Punkt MODUS):
  \begin{itemize}
    \item Triggermodus AUTO: Das Display wird in regelmäßigen Abständen neu aufgebaut, unabhängig davon, ob die Triggerbedingung erfüllt ist. Dieser Modus eignet sich dafür, eine erste Vorstellung des Signals zu bekommen, w\"ahrend noch kein korrekter Trigger eingestellt ist.
    \item Triggermodus NORMAL: Das Display wird nur dann neu aufgebaut, wenn die Triggerbedingung erfüllt ist. Dieser Modus eignet sich, um schnelle Veränderungen der Eingangsspannung, wie zum Beispiel logische Signale, zu untersuchen.
      Ist die Triggerbedingung in kurzen Abst\"anden verl\"asslich erf\"ullt,
      verhalen sich NORMAL und AUTO identisch. Sie k\"onnen also h\"aufig im
      Modus AUTO verbleiben und nur zu NORMAL wechseln, wenn zwischen
      zwei Triggern zu viel Zeit vergeht.
    \item
      Triggermodus STOP: Das Oszilloskop beh\"alt die letzte Aufnahme.
    \item
      Triggermodus SINGLE: Falls Sie eine funktionierende Triggerbedingung haben
      und sich eine einzige Aufnahme des Eingangssignals anschauen m\"ochten,
      k\"onnen Sie auf SINGLE wechseln. Das Oszilloskop wartet dann auf eine
      Triggerbedingung, nimmt genau eine Aufnahme auf und wechselt zu STOP.
  \end{itemize}
	
\subsubsection{Betrieb}

Da wir nur auf Pos. I messen, wird der Tastkopf bei CH 1 eingesteckt, die zugehörige Massenverbindung erfolgt an der Bananenbuchse daneben.\\
Mit dem Drehknopf (2) wird die Empfindlichkeit der Y-Achse eingestellt: Stellung 2~V bedeutet, dass jedes Kästchen auf dem Bildschirm 2~V hoch ist.\\

\noindent
Alle Oszilloskope (dieser Welt ?) werden nach diesem Schema bedient. Die vielen Knöpfe verführen zum Spielen, und wir möchten alle ermutigen, zu probieren, was die einzelnen Schalter bewirken. Diese Anleitung führt - hoffentlich - wieder zu einer Schalterstellung zurück, die eine richtige Messung ermöglicht.

\subsection{Wechselspannung und Wechselstrom}

Als Wechselspannung bezeichnet man eine periodische Spannung mit sinus- oder kosinusförmigem Verlauf:
\begin{equation}
 U(t) = U_0 \sin\omega t
\end{equation}
%
Die Wechselspannung verursacht an einem Bauteil einen Wechselstrom, der eine zeitliche Versetzung $t'$ (Phasenverschiebung $\delta$) gegenüber der Spannung haben kann:
\begin{equation}
 I(t) = I_0 \sin\omega(t-t') = I_0 \sin(\omega t - \delta)
\end{equation}

\noindent
Wechselspannungen werden technisch meist durch Induktion erzeugt (Dynamogenerator). Wird eine Leiterschleife mit konstanter Winkelgeschwindigkeit in einem homogenen Magnetfeld gedreht, so wird in dieser auf Grund des Induktionsgesetzes eine Wechselspannung induziert.\\

\noindent
In Wechselstromkreisen gibt es unterschiedliche Angaben zur Charakterisierung der Größe von Spannung und Strom: die Amplituden oder Scheitelwerte ($U_0$, $I_0$), die Effektivwerte ($U_{eff}$, $I_{eff}$) und die Spitze-Spitze-Werte ($U_{SS}$):
\begin{itemize}
 %
 \item \underline{Amplitude (Scheitelwert):} Die Amplituden geben die Maxima von Spannung oder Strom an; sie entsprechen dem Amplitudenbegriff der trigonometrischen Funktionen.
 %
 \item \underline{Effektivwert:} Die Effektivwerte von Spannung und Strom sind charakteristische Werte, deren Produkt (wie im Fall des Gleichstroms) die (mittlere) joulesche Wärmeleistung ergibt:
  \begin{equation}
   \bar{P} = U_{eff}\, I_{eff} = \frac{U_0}{\sqrt{2}}\,\frac{I_0}{\sqrt{2}}
  \end{equation}
 %
 \item \underline{Spitze-Spitze-Wert:} In besonderen Fällen wird für Spannungen die Differenz zwischen größtem und kleinstem Wert als sogenannter Spitze-Spitze-Wert $U_{SS}$ angegeben.
 %
\end{itemize}

Die Amplituden und $U_{SS}$ können auf dem Oszilloskop direkt beobachtet werden. Die Effektivwerte werden mit Multimetern gemessen, die für Wechselspannungen und -ströme einen eingebauten Gleichrichter enthalten und für diese Messbereiche in Effektivwerten kalibriert sind.\\

Analog zum ohmschen Widerstand R wird als Wechselstromwiderstand Z (Scheinwiderstand oder Impedanz) das Verhältnis der Amplituden von Spannung und Strom definiert:
\begin{equation}
 Z = \frac{U_0}{I_0} = \frac{U_{eff}}{I_{eff}}
 \label{eq:Z}
\end{equation}
Bei ohmschen Widerständen stimmen Gleich- und Wechselstromwiderstand überein. Bei anderen Bauteilen, wie Kondensatoren und Spulen, ist dies nicht der Fall.

\subsection{Kondensator und R-C-Kreis}

Ein Kondensator ist ein Speicher für elektrische Ladung. Die in einem Kondensator befindliche Ladung $Q$ ($Q$ auf der einen Platte und $-Q$ auf der anderen) ist proportional zur ''Größe'' des Kondensators (Kapazität $C$) und zur Spannung $U$ (die Kapazität $C$ ist definiert als Verhältnis von Ladung zu Spannung):
\begin{equation}
 Q = C\cdot U
 \label{eq:Q}
\end{equation}

\noindent
Die Einheit der Kapazität ist:
\begin{equation}
 \mathrm{\left[C\right] = 1\frac{As}{V} = 1 F \quad (Farad)}
\end{equation}

\noindent
Der Strom $I_C$ durch den Kondensator ist die zeitliche Ableitung der Ladung, mit Gleichung (\ref{eq:Q}):
\begin{equation}
 I_C = \frac{dQ_C}{dt} = C\,\frac{dU_C}{dt}\; .
 \label{eq:I_C}
\end{equation}
Wie man sieht ist der Strom groß, wenn sich die Spannung schnell ändert.\\
Stellt man diese Gleichung nach der Spannung frei
\begin{equation}
 U_C = \frac{1}{C}\int{I_C\, dt}
\end{equation}
so sieht man, dass sich die Spannung beim Aufladen eines Kondensators verhält wie das Integral über den Strom.

Nun werde ein (geladener) Kondensator der Kapazität $C$ mit einem Widerstand $R$ zu einem geschlossenen Stromkreis verschaltet.
\begin{figure}[h]
	\centering
		\includegraphics[width=.1\textwidth]{Versuch_15-16/Abbildungen/RC-Kreis.jpg}
	\caption{R-C-Kreis}
	\label{fig:RC-Kreis}
\end{figure}
Die Spannung am Kondensator ($U_C$) ergibt sich aus Gleichung (\ref{eq:I_C}), die Spannung am Widerstand aus der Definition
\begin{equation}
 U_R = R\, I_R\; .
\end{equation}
Nach der Maschenregel muss die Summe aller Ströme in der Masche Null sein. Mit $I = I_C = I_R$ folgt:
\begin{equation}
 U_C + U_R = \frac{1}{C}\int{I\, dt} + R I = 0\; .
\end{equation}
Durch Ableitung nach der Zeit erhält man eine Differentialgleichung für den Strom als Funktion der Zeit:
\begin{equation}
 \frac{dI}{dt} + \frac{1}{RC}\,I = 0\; .
\end{equation}
Diese Differentialgleichung wird gelöst durch die Funktion
\begin{equation}
 I(t) = I_0\, e^{-t/RC}
 \label{eq:Entladekurve}
\end{equation}
Beim Entladen entwickelt sich ein exponentiell mit der Zeit abklingender Strom (ebenso beim Aufladen). Das Produkt $RC$ im Exponenten von Gleichung (\ref{eq:Entladekurve}) bestimmt quantitativ die Abnahme des Stromes und wird als Zeitkonstante bezeichnet.\\

Betrachten wir nun den Wechselstromwiderstand des Kondensators.\\
Mit einer Wechselspannung der Form $U(t) =  U_0\cos\omega t$ finden wir den Strom durch den Kondensator gemäß Gleichung (\ref{eq:I_C}). Mit der Definition des Wechselstromwiderstandes (Gleichung (\ref{eq:Z}) ) berechnet man die Impedanz des Kondensators zu:
\begin{equation}
 Z_C = \frac{U_0}{I_0} = \frac{1}{\omega\, C}\; .
\end{equation}
Ursache des Widerstandes ist die sich aufbauende Gegenspannung am Kondensator. Die dabei umgesetzte Energie bleibt jedoch als elektrische Feldenergie im Kondensator gespeichert und wird während der Entladephase an den Kreis zurückgegeben. Ein (idealer) Kondensator setzt dem Strom einen Widerstand entgegen, der ohne Energieabgabe nach außen bleibt und deshalb als Blindwiderstand bezeichnet wird.\\

Wegen der Frequenzabhängigkeit des Wechselstromwiderstandes von Kondensatoren (und von Spulen) können mit diesen Bauteilen sogenannte Filter gebaut werden, die aus einem Wechselspannungsspektrum bestimmte Frequenzbereiche heraussieben. Solche Filter werden oft benötigt, um in elektrischen Mess- und Steuerkreisen die interessierenden Signale auszuwählen und Störsignale mit anderen Frequenzen abzutrennen. \\
Ein einfaches Beispiel ist die Reihenschaltung eines Kondensators mit einem Widerstand als Spannungsteiler:
\begin{figure}[ht]
	\centering
		\includegraphics[width=0.5\textwidth]{Versuch_15-16/Abbildungen/Paesse_gross.jpg}
	\caption{Hoch- und Tiefpassschaltungen}
	\label{fig:Hoch-Tiefpass}
\end{figure}
Eine Teilspannung ist proportional dem Teilwiderstand, über dem sie abgegriffen wird. Für tiefe Frequenzen ist der Widerstand des Kondensators groß, so dass hier der Hauptanteil der Eingangsspannung abfällt. Der Abgriff über dem Kondensator stellt einen \textit{Tiefpass} dar. Für hohe Frequenzen ist umgekehrt der Widerstand des Kondensators klein und der des Widerstandes vergleichsweise groß, so dass der Abgriff über dem Widerstand als \textit{Hochpass} wirkt.
%------------------------------------------------
\section{Fragen zur Vorbereitung}
%------------------------------------------------

\begin{enumerate}
 %
 \item Was soll heute im Praktikum untersucht werden? 
 %
 \item Was ist eine Wechselspannung? Wie kann man sie mathematisch beschreiben (Beispiel)? 
 %
 \item Was versteht man unter der Schwingungsdauer/Periode einer Wechselspannung? Wie lautet der Zusammenhang zwischen Periode und Frequenz?
 %
 \item Was versteht man unter einer Effektivspannung/einem Effektivstrom? Wie lautet der entsprechende Zusammenhang für eine sinusförmige Wechselspannung?
 %
 \item Was ist ein Kondensator? Wie sieht ein Plattenkondensator aus?
 %
 \item Was ist die Kapazität eines Kondensators? Wie hängt sie mit Ladung und Spannung zusammen?
 %
 \item Beschreiben Sie anhand einer kleinen Skizze (Spannung in Abhängigkeit der Zeit) die Auf- und die Entladung eines Kondensators über einen ohmschen Widerstand! Wie hängt die Auf-/Entladung des Kondensators von der Größe des Widerstandes ab?
 %
% \item Was ist eine Braun'sche Röhre (Skizze und kurze Erklärung)? Stichworte: Wie wird der Elektronenstrahl erzeugt? Wie wird er abgelenkt?
 %
 \item Wozu dient ein Oszillograph?
 %
 \item Was ist eine Kippspannung? Wozu wird sie im Oszillograph gebraucht? Was wäre auf dem Bildschirm eines Oszillographen zu sehen, wenn man nur eine Wechselspannung anlegt, aber keine Kippspannung?
 %
 \item Wie stellt man eine Wechselspannung auf einem Oszillograph dar? (Stichwort: Überlagerung von zu messender Wechselspannung und Kippspannung)
 %
 \item Wozu dient der Trigger? Stichworte: Was sind Trigger-Level und Triggerflanke?
 %
\end{enumerate}

%------------------------------------------------
\section{Durchführung} 
%------------------------------------------------

\begin{enumerate}
 %
 \item In diesem ersten Versuchsteil sollen Sie sich mit dem Oszilloskop vertraut machen. Stellen Sie eine sinusförmige Wechselspannung auf dem Bildschirm des Oszilloskops dar und beobachten die Abbildung bei verschiedenen Einstellungen des Oszilloskops.\\
  (Anmerkung: In diesem Versuchsteil werden keine für die Auswertung relevanten Messungen durchgeführt.)\\
  
  \noindent
  %Einschalten des Oszilloskops (1), Regeln von Helligkeit (18) und Schärfe (20). Man vergewissere sich, dass folgende Schalter im oberen Bedienungsfeld die richtige Position haben (sie werden nicht 
  %gebraucht!): TV Set: off; Delay: off; Hold-off: rechter Anschlag; Trig: AT.\\
	Man wähle den größten Messbereich für die Eingangsspannung (2). Man verbinde den Eingang des Oszilloskops mit dem Ausgang des Netzgeräts mit Hilfe des Anschlusskabels und einer Massenleitung. Man
	verändere die Zeitauflösung (3) und den Trigger-Level (4) und beobachte das Bild auf dem Bildschirm. Man ändere die Polarität der Triggerflanke (Trig Menu). Man ändere die Ablenkempfindlichkeit (2). Überzeugen Sie sich davon, dass eine Änderung der Ablenkempfindlichkeit bzw. der Zeitauflösung zwar die Auflösung ändert, nicht aber die Spannungsamplitude (in Volt) bzw. die Periode (in Sekunden) der Wechselspannung.
 %
 \item Messen Sie mit Hilfe des Vielfachmessgerätes die Werte der Widerstände $R_C$ und $R_L$ in der Versuchsbox ''Wechselspannungsversuch''. Dazu verbinden Sie die Ausgänge des Vielfachmessgerätes mit 
 	den Anschlussbuchsen ober- und unterhalb des Widerstandes $R_C$ (bzw. $R_L$). Legen Sie keine äußere Spannung an! Diese verfälscht die Messung und kann das Vielfachmessgerät beschädigen.\\
  Bitte notieren Sie die Nummer der Versuchsbox. Anmerkung: Der Widerstand $R_L$ (und die Nummer der Versuchsbox dienen lediglich dem Assistenten als Referenz und werden in der Auswertung nicht 
  benötigt).
 %
	
	\begin{figure}[t]
		\centering
			\includegraphics[width=\textwidth]{Versuch_15-16/Abbildungen/Schaltung.jpg}
		\caption{Schaltskizze für den weiteren Versuchsverlauf.}
		\label{fig:Schaltung}
	\end{figure}
	
 \item Verbinden Sie das Netzteil mit dem Eingang des Frequenzgenerators. \\
	\begin{minipage}{0.45\textwidth}
		Verbinden Sie das Oszilloskop mit dem Ausgang des Frequenzgenerators. Die Ausgangsspannung des Frequenzgenerators wird auf dem Bildschirm dargestellt (siehe Abbildung \ref{fig:Schaltung}). Messen Sie die Maximalamplitude 
		$U_0$ sowie die Schwingungsdauer und die Frequenz des Ausgangssignals. Schätzen Sie jeweils Ihre Ablesefehler ab.
	\end{minipage} 
	\hfill
	%
	\begin{minipage}{0.45\textwidth}
		\raggedright
			\includegraphics[width=0.7\textwidth]{Versuch_15-16/Abbildungen/Oszi.jpg}
			\label{fig:Oszi}
	\end{minipage}
 %
 \item Auf- und Entladung eines Kondensators über einen Widerstand:\\
	Verbinden Sie den regelbaren Widerstand über die Leiterbrücke mit dem Kondensator. Messen Sie nun mit dem zweiten Kanal des Oszilloskops den Spannungsabfall am Kondensator. Dazu schalten Sie den
	zweiten Kanal am Oszilloskop zu (1). Wählen Sie für beide Eingänge des Oszilloskops die gleiche Empfindlichkeit und bringen Sie die Nulllinien der beiden Eingänge im unteren Bildbereich auf eine Linie. Beachten Sie, dass Sie dazu nur ein Massekabel benötigen (bei korrekter Schaltung)!\\
	Verändern Sie den regelbaren Widerstand und beobachten Sie die Veränderung der Auf- und Entladekurve des Kondensators. Notieren Sie kurz - in Stichworten - Ihre Beobachtung (in der Auswertung ausführlich formulieren)!
 %
 \item Messung der Kapazität eines Kondensators mit dem Oszilloskop:\\
	Verbinden Sie nun den Festwiderstand mit dem Kondensator. Bringen Sie die Aufladungskurve des Kondensators mit bestmöglicher Zeitauflösung auf den Bildschirm und messen Sie die Spannung am 	Kondensator als Funktion der Zeit. Legen Sie dazu eine Folie (liegt im Praktikum aus) über den Bildschirm und pausen sie den Kurvenverlauf ab. (Skalen und Einstellungen mit aufschreiben !!)\\
	Ebenso messe man die Entladung des Kondensators (dazu Triggerflanke (Trig Menu) ändern!). Kann man sich durch geschicktes Drehen/Spiegeln der Folie ggf. ein weiteres 
	Abpausen der Entladekurve sparen? Bitte begründen Sie Ihre Antwort.
 %
 \item Hochpass-Charakteristik des R-L-Kreises:\\
  Entfernen Sie die Leiterbrücke zwischen $R_C$ und dem Kondensator und verbinden stattdessen $R_L$ mit der Spule. Betrachten und skizzieren Sie den Verlauf der Spannung über der Spule. 
 %
\end{enumerate}

%------------------------------------------------
\section{Auswertung} 
%------------------------------------------------

\begin{enumerate}
 %
 \item Beschreiben Sie den Einfluss verschiedener Widerstände bei der Auf- und Entladung eines Kondensators über einen Widerstand (R-C-Kreis).Beachten Sie dabei den Zusammenhang $\tau = R\,C$ und ggf. die Ladung auf dem Kondensator.
 %
 \item Bestimmung der Kapazität $C$ des Kondensators:
  \begin{enumerate}
   %
   \item Übertragen Sie den Kurvenverlauf der Aufladung des Kondensators auf Millimeterpapier. Einheiten und Achsenbeschriftung nicht vergessen!
   %
   \item Für die Aufladung eines Kondensators gilt:
    \begin{equation}
     U_C(t) = U_0\, \left(1-e^{-t/\tau}\right), \quad \tau = R\, C
    \end{equation}
		Lösen Sie diese Gleichung nach $e^{-t/\tau}$ auf und berechnen Sie den natürlichen Logarithmus. Schreiben Sie die neue Gleichung hin. Berechnen Sie die Unsicherheit des Logarithmus.
	 %
   \item Tragen Sie $\ln\left(1-U_C(t)/U_0\right)$ gegen $t$ auf. 
   %
   \item Bestimmen Sie aus der Steigung der Geraden die Kapazität $C$ des Kondensators inklusive Fehler.
   %
  \end{enumerate}
 %
 \item Erläutern Sie kurz, warum man einen R-L-Kreis auch Hochpass nennt.
 %
\end{enumerate}